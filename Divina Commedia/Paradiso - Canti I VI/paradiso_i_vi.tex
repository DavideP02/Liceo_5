\documentclass[a4paper, twoside, titlepage]{book}
\usepackage[utf8]{inputenc}
\usepackage[italian]{babel}
\usepackage[T1]{fontenc}

\usepackage{microtype}
\usepackage{mparhack}
\usepackage{xcolor}
\usepackage{multicol}

\usepackage{quoting}
\quotingsetup{font=small}

\usepackage{verse}
\usepackage{framed}

\usepackage[nouppercase]{frontespizio}
\newcommand{\omissis}{[\textellipsis\unkern]}

\newcommand{\straniero}[1]{\textit{#1}}
\newcommand{\titolo}[1]{\textsc{#1}}
\newcommand{\evid}[1]{\textbf{\textcolor{blue}{#1}}}

%creazione del comando per le note a margine
\newcounter{mar}
\newcommand{\mar}[2]{
\addtocounter{mar}{1}
\hspace{-0.73em}\textsuperscript{\hyperref[\thechapter.\themar]{\themar}}\marginpar{\textbf{\themar}\label{\thechapter.\themar}. #2}\hspace{-0.4em}
}
\newcommand{\mat}[1]{\mar{gg}{#1}}

\usepackage{chngcntr} %pacchetto per la numerazione delle note che partono ogni capitolo
\counterwithin*{mar}{chapter}
%fine comando per le note a margine

%comando per la creazione dei puntini di sospensione tra quadre
\newcommand{\salt}{\hspace{1em}[...]}

\usepackage{ulem}
\usepackage{verse}

\usepackage[colorlinks]{hyperref}
\definecolor{RoyalBlue}{rgb}{0.0, 0.14, 0.4}

\newlength\sidebar
\newlength\envrule
\newlength\envborder
\setlength\sidebar{1.5mm}
\setlength\envrule{0.4pt}
\setlength\envborder{2.5mm}

\hypersetup{
     colorlinks=true,
     linkcolor=blue,
     filecolor=blue,
     citecolor = black,      
     urlcolor=cyan,}

\makeatletter
\long\def\fboxs#1{%
  \leavevmode
  \setbox\@tempboxa\hbox{%
    \color@begingroup
      \kern\fboxsep{#1}\kern\fboxsep
    \color@endgroup}%
  \@frames@x\relax}
\def\frameboxs{%
  \@ifnextchar(%)
    \@framepicbox{\@ifnextchar[\@frameboxs\fboxs}}
\def\@frameboxs[#1]{%
  \@ifnextchar[%]
    {\@iframeboxs[#1]}%
    {\@iframeboxs[#1][c]}}
\long\def\@iframeboxs[#1][#2]#3{%
  \leavevmode
  \@begin@tempboxa\hbox{#3}%
    \setlength\@tempdima{#1}%
    \setbox\@tempboxa\hb@xt@\@tempdima
         {\kern\fboxsep\csname bm@#2\endcsname\kern\fboxsep}%
    \@frames@x{\kern-\fboxrule}%
  \@end@tempboxa}
\def\@frames@x#1{%
  \@tempdima\fboxrule
  \advance\@tempdima\fboxsep
  \advance\@tempdima\dp\@tempboxa
  \hbox{%
    \lower\@tempdima\hbox{%
      \vbox{%
        %\hrule\@height\fboxrule
        \hbox{%
         \vrule\@width\fboxrule
          #1%
          \vbox{%
            \vskip\fboxsep
            \box\@tempboxa
            \vskip\fboxsep}%
          #1%
          }%\vrule\@width\fboxrule}%
        }%\hrule\@height\fboxrule}%
                          }%
        }%
}
\def\esefcolorbox#1#{\esecolor@fbox{#1}}
\def\esecolor@fbox#1#2#3{%
  \color@b@x{\fboxsep\z@\color#1{#2}\fboxs}{\color#1{#3}}}
\makeatother

\definecolor{exampleborder}{rgb}{0.5, 0.5, 0.5}
\definecolor{examplebg}{rgb}{0.89, 0.89, 0.89}
\definecolor{statementborder}{rgb}{.9,0,0}
\definecolor{statementbg}{rgb}{1,.9,.9}

\newenvironment{eseframed}{%
  \def\FrameCommand{\fboxrule=\the\sidebar  \fboxsep=\the\envborder%
  \esefcolorbox{exampleborder}{examplebg}}%
  \MakeFramed{\FrameRestore}}%
 {\endMakeFramed}

\newenvironment{nota}[1]
{\par\medskip%\refstepcounter{esempio}%
\hbox{%
\fboxsep=\the\sidebar\hspace{-\envborder}\hspace{-.5\sidebar}%
\colorbox{exampleborder}{%
\hspace{\envborder}\footnotesize\sffamily\bfseries%
\textcolor{white}{{{\large\textsc{
#1}}}\hspace{\envborder}}
}
}
\nointerlineskip\vspace{-\topsep}%
\begin{eseframed}\noindent\ignorespaces%
}
{\end{eseframed}\vspace{-\baselineskip}\medskip}

%modifica la posizione del numero di verso
\setlength{\vrightskip}{-30pt}%regola tu
 \verselinenumbersleft

\definecolor{Black}{rgb}{0, 0, 0}

%\usepackage[eulerchapternumbers,beramono,pdfspacing]{classicthesis}
%\usepackage{arsclassica}

\makeatletter
\def\@makechapterhead#1{%
  \vspace*{50\p@}%
  {\parindent \z@ \raggedright \normalfont
    \vskip 20\p@
    \interlinepenalty\@M
    \Huge \bfseries #1\par\nobreak
    \vskip 40\p@
  }}
\makeatother

\addto\captionsitalian{\renewcommand{\chaptername}{}}

\usepackage{fancyhdr}
\pagestyle{fancy}
\makeatletter
\def\cleardoublepage{\clearpage\if@twoside \ifodd\c@page\else
    \hbox{}
    \vspace*{\fill}
    \vspace{\fill}
    \thispagestyle{empty}
    \newpage
    \if@twocolumn\hbox{}\newpage\fi\fi\fi}
\makeatother
\renewcommand{\chaptermark}[1]{\markboth{#1}{}}
\renewcommand{\sectionmark}[1]{\markright{\thesection\ #1}}
\fancyhf{}
\fancyhead[LE,RO]{\scshape\thepage}
\fancyhead[LO]{\scshape\footnotesize\nouppercase{\rightmark}} 
\fancyhead[RE]{\scshape\footnotesize\nouppercase{\leftmark}} 

\usepackage{graphicx}

\begin{document}
	
	\begin{frontespizio}
		\Istituzione{Liceo Classico Scientifico Musicale "I. Newton"}
		\Divisione{Dante}
		\Scuola{Liceo delle Scienze Applicate}
		\Titoletto{Paradiso, I-VI}
		\Piede{Anno scolastico 2020-2021}
		\Titolo{La Divina Commedia}
		\Sottotitolo{Dante Alighieri}
		\NCandidato{Autore}
		\Candidato{Davide Peccioli}
		
		\NRelatore{Appunti basati sulle lezioni di}{Appunti basati sulle lezioni di}
		\Relatore{Professoressa Mistero}
	\end{frontespizio}
	
% # Come scrivere questo file
% 
% Bisogna utilizzare ``\paragraph{title}'' per scrivere riguardo a determinati argomenti, monografie
%
% Per fare commenti generali sul testo interrompere semplicemente l'ambiente 'verse' e scrivere, per poi riprenere l'uso di 'verse' con la dicitura '\setverselinenums{first verse}{first labelled verse}'. Per commenti più brevi sono sufficienti le note, con l'ambiente personalizzato \mat{NOTA}
%
% \end{document}
	
	\chapter*{Introduzione}
	
	\begin{center}
		\includegraphics[height=43em]{Struttura_Paradiso_di_Dante-removebg}
	\end{center}
	
	Il \titolo{Paradiso} è la terza e ultima cantica de \titolo{La Divina Commedia}, e racconta l'ultima parte del viaggio di Dante.
	
	I beati, in condizioni normali, sono nell'\textbf{Empireo}, ma in occasione del suo viaggio Dante immagina che le anime si siano trasferite nei vari cieli, secondo un certo criterio. La ragione di ciò è dare una certa simmetria all'opera.
	
	In tutto questo sembra quasi che ci sia una qualche differenza tra la anime, ma in realtà non è così. La \textbf{distanza} da Dio non comporta sofferenza, poiché gli spiriti si beano della volontà di Dio stesso.
	
	La missione di Dante si viene a conoscere solo nel \titolo{Paradiso}: egli dopo questa esperienza dovreò scrivere un'opera che varrà agli altri uomini come a lui è valso il viaggio, in una sorta di \textbf{funzione catartica}. Dante riceve questo compito durante l'incontro con Cacciaguida, il suo trisavolo.
	
	Cacciaguida, infatti, assume il simbolo di colui che l'ha aspettato per assegnargli il suo compito. Le ragioni che hanno portato Dante a scegliere proprio Cacciaguida sono legate al fatto che egli abbia vissuto in un tempo sufficientemente lontano da poter parlare di un'epoca \textbf{senza corruzione}.
	
	\chapter{Canto I}
	
	Il proemio di questa cantica è più ampio rispetto agli altri: Dante si rivolge addirittura ad Apollo.
	
	\begin{verse}
	\poemlines{3}	
	La gloria di colui che tutto move\mat{Dio}\\
	per l’universo penetra, e risplende\\
	in una parte più e meno altrove.\mat{solo la capacità degli uomini di assimilare la grandezza di Dio determina questa differenza}\\!
	Nel ciel\mat{Empireo} che più de la sua luce prende\\
	fu’ io, e vidi cose che ridire\\
	né sa né può chi di là sù discende;\mat{inizia il tema dell'ineffabile: in questi versi si fa riferimento all'estasi mistica o \straniero{excessus mentis}, cioè l'uscita dell'anima da sé, quando, lasciato il proprio corpo, è rapita nella contemplazione di Dio. Dante spiega il fenomeno nella \titolo{Epistola a Cangrande}}\\!
	perché appressando sé al suo disire,\\
	nostro intelletto si profonda tanto,\\
	che dietro la memoria non può ire.\mat{spiega il perché}\\!
	Veramente quant’io del regno santo\\
	ne la mia mente potei far tesoro,\\
	sarà ora materia del mio canto.\\!
	O buono Appollo, a l’ultimo lavoro\\
	fammi del tuo valor sì fatto vaso,\\
	come dimandi a dar l’amato alloro.\mat{"rendimi vaso per la tua arte (la poesia) così da meritarmi l'alloro}\\!
	Infino a qui l’un giogo di Parnaso\mat{monte delle muse}\\
	assai mi fu; ma or con amendue\\
	m’è uopo intrar ne l’aringo rimaso.\mat{l'aringo è il campo di gioco, e si riferisce alla parte di opera ancora da scrivere}\\!
	Entra nel petto mio, e spira tue\mat{ispirami}\\
	sì come quando Marsia traesti\\
	de la vagina de le membra sue.\mat{mito del satiro Marsia: egli aveva sfidato Apollo, che poi lo aveva scuoiato}\\!
	O divina virtù, se mi ti presti\\
	tanto che l’ombra del beato regno\\
	segnata nel mio capo io manifesti,\\!
	vedra’mi al piè del tuo diletto legno\mat{lauro, alloro}\\
	venire, e coronarmi de le foglie\\
	che la materia e tu mi farai degno.\mat{mito di Dafne e Apollo}\\!
	Sì rade volte, padre\mat{Apollo}, se ne coglie\mat{di alloro}\\
	per triunfare o cesare o poeta,\\
	colpa e vergogna de l’umane voglie,\mat{per colpa delle umane voglie}\\!
	che\mat{eppure} parturir letizia in su la lieta\\
	delfica deità\mat{Apollo} dovria la fronda\mat{soggetto: Alloro (Dafne era figlia di Peneo)}\\
	peneia, quando alcun di sé asseta.\mat{quando porta a qualcuno desiderio di sé}\\!
	\end{verse}
	\paragraph{Alloro} Le ultime due terzine sono una critica ai contemporanei che non bramano l'alloro, ovvero la gloria poetica o militare, perché attratti dai beni terreni e non dalla gloria. Le umane voglie, infatti, non rendono gli uomini degni.
	\begin{verse}
	\poemlines{3}	
	\setverselinenums{34}{36}
	Poca favilla\mat{piccola scintilla: si riferisce alle opere di Dante stesso} gran fiamma seconda:\mat{favorisce}\\
	forse di retro a me con miglior voci\\
	si pregherà perché Cirra\mat{Apollo} risponda.\\!
	Surge ai mortali per diverse foci\mat{diversi punti dell'orizzonte}\\
	la lucerna del mondo\mat{sole}; ma da quella\\
	che quattro cerchi giugne con tre croci,\mat{grande perifrasi per alludere alla primavera, ma potrebbe anche alludere alle 3 virtù teologali e alle 4 virtù cardinali}\\!
	con miglior corso e con migliore stella\mat{in modo più propizio}\\
	esce congiunta, e la mondana cera\mat{materia del mondo}\\
	più a suo modo tempera e suggella.\mat{da la sua impronta in modo migiore: porta influssi migliori}\\!
	Fatto avea di là\mat{purgatorio} mane e di qua\mat{terra} sera\\
	tal foce, e quasi tutto era là bianco\\
	quello emisperio, e l’altra parte nera,\\!
	quando Beatrice in sul sinistro fianco\\
	vidi rivolta e riguardar nel sole\mat{Beatrice guarda il sole senza protezioni senza problemi, e Dante è sbalordito}:\\
	aguglia sì non li s’affisse unquanco.\mat{nemmeno un acquila riuscirebbe}\\!
	E sì come secondo raggio\mat{raggio riflesso} suole\\
	uscir del primo e risalire in suso,\\
	pur come pelegrin che tornar vuole,\mat{sta indicando il movimento degli occhi di Dante: guarda gli occhi di Beatrice, che gli danno la forza di guardare il sole}\\!
	così de l’atto suo, per li occhi infuso\\
	ne l’imagine mia, il mio si fece,\\
	e fissi li occhi al sole oltre nostr’uso.\\!
	\end{verse}
	\paragraph{Motivo della luce} È introdotto in questo punto il motivo della luce, che sarà importantissimo per tutta la cantica
	\begin{verse}
	\poemlines{3}
	\setverselinenums{55}{57}
	Molto è licito là, che qui non lece\\
	a le nostre virtù, mercé\mat{grazie} del loco\\
	fatto per proprio de\mat{appositamente per} l’umana spece.\mat{questa terzina è un discorso generale sul paradiso terrestre}\\!
	Io nol soffersi\mat{sopportai} molto, né sì poco,\\
	ch’io nol vedessi sfavillar dintorno,\mat{(ma non così poco da non vedere che)}\\
	com’ferro che bogliente esce del foco;\\!
	e di sùbito parve giorno a giorno\\
	essere aggiunto, come quei che puote\\
	avesse il ciel d’un altro sole addorno.\mat{aggiunto}\\!
	Beatrice tutta ne l’etterne rote\mat{i cieli}\\
	fissa con li occhi stava; e io in lei\\
	le luci fissi,\mat{fissai} di là sù rimote.\mat{rimossi}\\!
	Nel suo aspetto tal dentro mi fei,\\
	qual si fé Glauco nel gustar de l’erba\\
	che ‘l fé consorto\mat{con la stessa sorte} in mar de li altri dèi.\\!
	\end{verse}
	\paragraph{Mito di Glauco} È un mito tratto da Ovidio, che permette a Dante di spiegare che ha acquisito capacità sovraumane. Secondo la leggenda, Glauco nacque mortale e faceva il pescaatore. Un giorno appoggiò la rete da pesca contentente il pescato su un prato, ed i pesci, mangiando quell'erba, tornavano in vita e si rigettavano in mare. Glauco, incuriosito, assaggiò quell'erba e, grazie alle sue proprietà magiche, divenne immortale e divino; inoltre le sue gambe si tramutarono nella coda di un pesce.
	\begin{verse}
	\poemlines{3}
	\setverselinenums{70}{72}
	Trasumanar\mat{andare oltre alle possibilità} significar \straniero{per verba}\\
	non si poria; però\mat{per cui} l’essemplo basti\\
	a cui esperienza grazia serba.\mat{tema dell'ineffabile}\\!
	S’i’ era sol di me quel che creasti\\
	novellamente,\mat{anima} amor che ‘l ciel governi,\mat{Dio}\\
	tu ‘l sai, che col tuo lume mi levasti.\mat{Dante non riesce a capire se nel volo fosse sol anima o anche corpo}\\!
	Quando la rota\mat{movimento rotatorio} che tu sempiterni\mat{fai ruotare}\\
	desiderato,\mat{i cieli si muovono perché attivati dal desiderio di Dio} a sé mi fece atteso\mat{attirò la mia attenzione}\\
	con l’armonia che temperi e discerni,\\!
	parvemi tanto allor del cielo acceso\\
	de la fiamma del sol,\mat{la luce divenne più forte} che pioggia o fiume\\
	lago non fece alcun tanto disteso.\\!
	La novità del suono e ‘l grande lume\\
	di lor cagion m’accesero un disio\\
	mai non sentito di cotanto acume.\mat{intensità}\\!
	Ond’ella, che vedea me sì com’io,\mat{leggeva i miei pensieri come lo facevo io}\\
	a quietarmi l’animo commosso,\\
	pria ch’io a dimandar, la bocca aprio,\\!
	e cominciò: «Tu stesso ti fai grosso\mat{ottuso}\\
	col falso imaginar,\mat{di essere ancora sulla terra} sì che non vedi\\
	ciò che vedresti se l’avessi scosso.\mat{rimosso}\\!
	Tu non se’ in terra, sì come tu credi;\\
	ma folgore, fuggendo il proprio sito,\\
	non corse come tu ch’ad esso riedi\mat{ritorni alla sede che ti è propria}».\\!
	S’io fui del primo dubbio disvestito\\
	per le sorrise parolette brevi,\\
	dentro ad un nuovo più fu’ inretito,\\!
	e dissi: «Già contento \straniero{requievi}\mat{restare tranquillo}\\
	di grande ammirazion; ma ora ammiro\mat{mi stupisco}\\
	com’io trascenda questi corpi levi\mat{leggeri: aria e fuoco}».\\!
	Ond’ella, appresso d’un pio sospiro,\\
	li occhi drizzò ver’ me con quel sembiante\\
	che madre fa sovra figlio deliro,\\!
	e cominciò: «Le cose tutte quante\\
	hanno ordine tra loro, e questo è forma\\
	che l’universo a Dio fa simigliante.\\!
	Qui veggion l’alte creature\mat{esseri razionali: uomini e angeli} l’orma\\
	de l’etterno valore, il qual è fine\\
	al quale è fatta la toccata norma.\mat{ordine}\\!
	Ne l’ordine ch’io dico sono accline\\
	tutte nature, per diverse sorti,\\
	più al principio loro e men vicine\mat{più o meno vicine al principio loro};\\!
	onde si muovono a diversi porti\\
	per lo gran mar de l’essere, e ciascuna\\
	con istinto\mat{dettato da Dio} a lei dato che la porti.\\!
	Questi\mat{l'istinto} ne porta il foco inver’ la luna;\\
	questi ne’ cor mortali è permotore;\\
	questi la terra in sé stringe e aduna;\\!
	né pur le creature che son fore\\
	d’intelligenza quest’arco saetta\mat{l'arco dell'istinto}\\
	ma quelle c’hanno intelletto e amore.\\!
	La provedenza, che cotanto assetta,\\
	del suo lume fa ‘l ciel sempre quieto\mat{Empireo}\\
	nel qual si volge\mat{ruota} quel c’ha maggior fretta\mat{primo cielo mobile};\\!
	e ora lì,\mat{Empireo} come a sito decreto,\\
	cen porta la virtù di quella corda\\
	che ciò che scocca drizza in segno lieto.\\!
	Vero è che, come forma non s’accorda\\
	molte fiate a l’intenzion de l’arte,\mat{artista}\\
	perch’a risponder la materia è sorda,\\!
	così da questo corso si diparte\\
	talor la creatura, c’ha podere\\
	di piegar, così pinta, in altra parte;\\!
	e sì come veder si può cadere\\
	foco di nube, sì l’impeto primo\\
	l’atterra torto\mat{deviato} da falso piacere.\mat{beni materiali}\\!
	Non dei più ammirar, se bene stimo,\\
	lo tuo salir, se non come d’un rivo\\
	se d’alto monte scende giuso ad imo.\mat{verso terra}\\!
	Maraviglia sarebbe in te se, privo\\
	d’impedimento, giù ti fossi assiso,\\
	com’a terra quiete\mat{immobilità} in foco vivo».\\!
	Quinci rivolse inver’ lo cielo il viso.\\
	\end{verse}
	
\chapter{Canto II}

\paragraph{Riassunto} Il canto si apre con un appello del poeta ai lettori per metterli in guardia dal seguirlo in una materia
tanto complessa, per cui è richiesta un’alta preparazione filosofica e teologica, pena lo smarrirsi con
facilità. Dante e Beatrice si alzano intanto velocemente fino al primo cielo, quello della Luna, nel quale
Dante penetra col proprio corpo, nonostante tale cielo sia costituito di materia solida e trasparente.
Beatrice, a cui il poeta chiede una spiegazione relativa all’origine delle macchie lunari, confuta le tesi
dell’interrogante, fintate sulla maggiore o minore densità della materia di cui i corpi celesti sono
costituiti. Poi spiega che l’intensità luminosa di tali corpi è da porre in relazione alle intelligenze
angeliche che li fanno muovere, o meglio alla loro virtù. Nell’Empireo si muove, per impulso divino, il
Primo Mobile, il cielo che trasmette il moto a quello delle Stelle Fisse, il quale a sua volta lo
trasmette ai rimanenti sette, che esercitano così via i loro influssi. Il cielo della Luna, essendo il più
lontano da Dio, ha luce minore: da ciò derivano le macchie lunari.

\chapter{Canto III}

\paragraph{Piccarda Donati} Era una monaca colta da \textbf{reale vocazione}, rapita dal fratello per un matrimonio politico. Sorella di Forese e Corso Donati, che andranno rispettivamente nel purgatorio e nell'inferno.

\begin{verse}
\poemlines{3}
Quel sol\mat{Beatrice} che pria d’amor mi scaldò ‘l petto,\\
di bella verità m’avea scoverto,\mat{domanda sulle macchie lunari del canto precedente}\\
provando e riprovando, il dolce aspetto;\\!
e io, per confessar corretto e certo\\
me stesso, tanto quanto si convenne\\
leva’ il capo a proferer più erto;\\!
ma visione\mat{la prima schiera di anime del paradiso. Sono le ultime che \textbf{vede} tutte le altre saranno solo \textit{percepite}} apparve che ritenne\\
a sé me tanto stretto, per vedersi,\\
che di mia confession\mat{ciò che volevo dire} non mi sovvenne.\\!
Quali per\mat{attraverso} vetri trasparenti e tersi,\\
o ver per acque nitide e tranquille,\\
non sì profonde che i fondi sien persi,\\!
tornan d’i nostri visi le postille\mat{i contorni: le postille in un testo sono le note che, una volta, per risparmiare spazio, si mettevano tutto attorno al testo}\\
debili sì, che perla in bianca fronte\\
non vien men forte a le nostre pupille;\\!
tali vid’io più facce a parlar pronte;\\
per ch’io dentro a l’error contrario corsi\\
a quel ch’accese amor tra l’omo e ‘l fonte.\mat{si allude al mito del bellissimo Narciso, il quale, specchiatosi nelle acque di una fontana, si innamorò della propria immagine, ritenendola vera e appartente ad un'altra persona.}\\!
Sùbito sì com’io di lor m’accorsi,\\
quelle stimando specchiati sembianti,\\
per veder di cui fosser, li occhi torsi;\\!
e nulla vidi, e ritorsili avanti\\
dritti nel lume de la dolce guida,\mat{Beatrice}\\
che, sorridendo, ardea ne li occhi santi.\\!
«Non ti maravigliar perch’io sorrida»,\\
mi disse, «appresso il tuo pueril coto,\mat{pensiero}\\
poi\mat{poiché} sopra ‘l vero ancor lo piè non fida,\mat{poggia}\\!
ma te rivolve, come suole, a vòto\mat{errore; da origine ad una rima imperfetta}:\\
vere sustanze son ciò che tu vedi,\\
qui rilegate per manco di voto.\mat{siamo nel cielo della luna; \textit{lunatico} deriva da questo}\\!
Però parla con esse e odi e credi;\\
ché la verace luce che li appaga\\
da sé non lascia lor torcer li piedi».\\!
E io a l’ombra che parea più vaga\\
di ragionar, drizza’mi, e cominciai,\\
quasi com’uom cui troppa voglia smaga\mat{togliere le forze}:\\!
«O ben creato spirito, che a’ rai\\
di vita etterna la dolcezza senti\\
che, non gustata, non s’intende mai,\\!
grazioso mi fia se mi contenti\\
del nome tuo e de la vostra sorte».\\
Ond’ella, pronta e con occhi ridenti:\\!
«La nostra carità non serra porte\\
a giusta voglia, se non come\mat{così come} quella\mat{divina}\\
che vuol simile a sé tutta sua corte.\\!
I’ fui nel mondo vergine sorella;\\
e se la mente tua ben sé riguarda,\\
non mi ti celerà l’esser più bella,\\!
ma riconoscerai ch’i’ son Piccarda,\\
che, posta qui con questi altri beati,\\
beata sono in la spera più tarda.\\!
Li nostri affetti, che solo infiammati\\
son nel piacer de lo Spirito Santo,\\
letizian del suo ordine formati.\\!
E questa sorte che par giù\mat{umile} cotanto,\\
però n’è data, perché fuor negletti\\
li nostri voti, e vòti\mat{manchevoli} in alcun canto».\\!
Ond’io a lei: «Ne’ mirabili aspetti\\
vostri risplende non so che divino\\
che vi trasmuta da’ primi concetti\mat{sembianze precedenti}:\\!
però non fui a rimembrar festino;\\
ma or m’aiuta ciò che tu mi dici,\\
sì che raffigurar m’è più latino.\\!
Ma dimmi: voi che siete qui felici,\\
disiderate voi più alto loco\\
per più vedere e per più farvi amici?».\\!
Con quelle altr’ombre pria sorrise un poco;\\
da indi mi rispuose tanto lieta,\\
ch’arder parea d’amor nel primo foco\mat{fuoco divino, amore per Dio}:\\!
«Frate, la nostra volontà quieta\\
virtù di carità,\mat{soggetto} che fa volerne\\
sol quel ch’avemo, e d’altro non ci asseta.\\!
Se disiassimo esser più superne,\\
foran discordi li nostri disiri\\
dal voler di colui che qui ne cerne;\\!
che vedrai non capere\mat{non può avere luogo} in questi giri,\\
s’essere in carità è qui necesse,\\
e se la sua natura ben rimiri.\mat{qui lo stile si innalza}\\!
Anzi è formale ad esto beato esse\\
tenersi dentro a la divina voglia,\\
per ch’una fansi nostre voglie stesse;\\!
sì che, come noi sem di soglia in soglia\\
per questo regno, a tutto il regno piace\\
com’a lo re che ‘n suo voler ne ‘nvoglia.\\!
E ‘n la sua volontade è nostra pace:\\
ell’è quel mare al qual tutto si move\mat{tende}\\
ciò ch’ella cria o che natura face».\\!
Chiaro mi fu allor come ogne dove\\
in cielo è paradiso, etsi la grazia\\
del sommo ben d’un modo non vi piove.\\!
Ma sì com’elli avvien, s’un cibo sazia\\
e d’un altro rimane ancor la gola,\\
che quel si chere e di quel si ringrazia,\\!
così fec’io con atto e con parola,\\
per apprender da lei qual fu la tela\\
onde non trasse infino a co la spuola.\\!
«Perfetta vita e alto merto inciela\\
donna più sù\mat{Santa Chiara}», mi disse, «a la cui norma\\
nel vostro mondo giù si veste e vela,\\!
perché fino al morir si vegghi e dorma\\
con quello sposo\mat{Gesù Cristo: il linguaggio del canto di San Francesco} ch’ogne voto accetta\\
che caritate a suo piacer conforma.\\!
Dal mondo, per seguirla, giovinetta\\
fuggi’mi, e nel suo abito mi chiusi\\
e promisi la via de la sua setta.\\!
Uomini poi, a mal più ch’a bene usi,\\
fuor mi rapiron de la dolce chiostra\mat{quasi ossimorica: il chiostro è spesso simbolo di monacazione forzata}:\\
Iddio si sa qual poi mia vita fusi.\\!
E quest’altro splendor che ti si mostra\\
da la mia destra parte e che s’accende\\
di tutto il lume de la spera nostra,\\!
ciò ch’io dico di me, di sé intende;\\
sorella fu, e così le fu tolta\\
di capo l’ombra de le sacre bende.\\!
Ma poi che pur al mondo fu rivolta\\
contra suo grado e contra buona usanza,\\
non fu dal vel del cor già mai disciolta.\\!
Quest’è la luce de la gran Costanza\\
che del secondo vento\mat{potenza} di Soave\mat{casata di Svevia}\\
generò ‘l terzo e l’ultima possanza».\\!
\end{verse}
\paragraph{Costanza} Fu la madre di Federico II di Svevia, ultimo imperatore che risiederà in Italia. Dante accoglie una leggenda secondo la quale Costanza, monaca del monastero di Palermo, all'età di cinquantadue anni fu fatta sposare a Enrico VI. Da tale matrimonio nacque poi Federico II, nato da una ex suora in una età molto avanzata.

La realtà storica è diversa: Costanza non fu mai suora, sebbene è probabile abbia compiuto degli studi in un qualche monastero, ed ebbe Federico in tarda età (42 anni). Inoltre, ella pretese di partorire in mezzo alla gente per avere dei testimoni.
\begin{verse}
\poemlines{3}
\setverselinenums{121}{123}
Così parlommi, e poi cominciò ‘Ave,\\
Maria’ cantando, e cantando vanio\mat{svanì}\\
come per acqua cupa cosa grave.\\!
La vista mia, che tanto lei seguio\\
quanto possibil fu, poi che la perse,\\
volsesi al segno di maggior disio,\\!
e a Beatrice tutta si converse;\\
ma quella folgorò nel mio sguardo\\
sì che da prima il viso non sofferse\mat{non riuscì a sostenerlo};\\!
e ciò mi fece a dimandar più tardo.\\
\end{verse}

\chapter{Canto IV}

\paragraph{Riassunto} Beatrice legge sul volto di Dante il desiderio che gli siano sciolti due pressanti dubbi: come la violenza altrui possa far diminuire i nostri meriti, nel caso che persista la buona volontà di compiere il bene, e come sia possibile che le anime, dopo essere discese dalle stelle per introdursi nei corpi, vi facciano ritorno, secondo quanto afferma Platone. Beatrice risponde prima al secondo dubbio. Le anime risiedono stabilmente nell'Empireo e si mostrano nei singoli cieli solo per far comprendere, in termini sensibili e quindi umani, il loro grado di beatitudine che deriva loro dalla maggiore o minore capacità di sentire l'amore di Dio. L'affermazione di Platone sul ritorno alle stelle da parte delle anime è errata, ci può essere del vero se si allude all'influsso su queste ultime a opera delle prime. Poi passa a chiarire il primo dubbio. Se Piccarda  e Costanza si fossero opposte con tutte le loro forze alla violenza subita, non sarebbero venute meno al loro voto, invece vi cedettero per evitare conseguenze peggiori. Dante manifesta anche un terzo dubbio (se l'uomo può compensare il fatto di non aver mantenuto i voti con delle opere buone) a cui sarà risposto nel canto seguente.

\chapter{Canto V}

\paragraph{Riassunto} Beatrice risponde al dubbio che Dante ha manifestato nel canto precedente (se sia possibile porre rimedio al mancato mantenimento di un voto con delle buone azioni) e afferma che ciò non è possibile, in quanto il voto ha un altissimo valore derivatogli dal sacrificare liberamente la propria volontà che è peculiare delle creature intelligenti. Però è vero che la Chiesa può concedere la dispensa dall'osservanza di un voto, in quanto esso si compone di due elementi, la <<convenza>> o patto con Dio, a cui non si può venire meno, e la materia o oggetto, che può essere cambiata con un'altr di valore maggiore. Terminate le spiegazioni, i due salgono velocemente al secondo cielo, quello di Mercurio, dove sono avvicinati da una moltitudine di spiriti resi irriconoscibili dal loro stesso splendore. Invitato a parlare da una di queste anime, Dante le chiede chi sia e per quale motivo si trovi nel cielo di Mercurio.

\chapter{Canto VI}

\paragraph{Canto politico} Questo è un canto politico, così come tutti i sesti canti:
\begin{itemize}
\item nell'\titolo{Inferno} si era parlato della città di Firenze;
\item nel \titolo{Purgatorio} si era parlato dell'Italia;
\item nel \titolo{Paradiso} si parla dell'\textbf{Impero}.
\end{itemize}

\paragraph{Soggetto} Il soggetto grammaticale dell'intero canto è l'\textbf{aquila}, simbolo del \textit{vessillo imperiale}.

\paragraph{Giustiniano} Sarà Giustiniano a parlare dall'inizio alla fine del canto, raccontando tutta la storia dell'Impero Romano. Egli fu imperatore d'Oriente, marito di una ballerina; commissionò il codice di leggi (\titolo{Codex Iustinianus}); aveva cercato di riedificare l'Impero con la riconquista dell'Italia, per mezzo della guerra longobarda.

Le ragioni che hanno spinto Dante a scegliere di far raccontare la storia dell'Impero a Giustiniano piuttosto che a Costantino sono:
\begin{itemize}
\item Costantino è in paradiso e sarà incontrato più avanti;
\item probabilmente c'era qualcosa che aveva fatto Costantino che dava fastidio a Dante: in particolar modo si tratta della \textbf{donazione di Costantino}, che aveva dato il potere temporale alla Chiesa;
\item alcuni critici propongono che la visione dei mosaici di Ravenna, ritraenti Giustiniano, abbiano suggestionato Dante nella scelta;
\item il motivo più probabile, comunque, resta il suo \textbf{corpo di leggi}.
\end{itemize}

Giustiniano immagina un percorso dell'aquila da Oriente a Occidente, a partire da Enea che, partito da Troia, darà vita all'Impero.

\begin{verse}
\poemlines{3}
«Poscia che Costantin l’aquila\mat{soggetto} volse\\
contr’al corso del ciel, ch’ella seguio\\
dietro a l’antico\mat{Enea} che Lavina tolse,\\!
cento e cent’anni e più l’uccel di Dio\\
ne lo stremo d’Europa si ritenne,\\
vicino a’ monti de’ quai prima uscìo;\\!
e sotto l’ombra de le sacre penne\\
governò ‘l mondo lì di mano in mano,\\
e, sì cangiando, in su la mia pervenne.\\!
Cesare fui e son Iustiniano,\mat{questo è un \textbf{chiasmo}. ``fui'' indica il distacco con il ruolo terreno, legato al fatto di essere in Paradiso. Qui infatti ogni legame terreno sparisce}\\
che, per voler del primo amor ch’i’ sento,\\
d’entro le leggi trassi il troppo e ‘l vano.\\!
E prima ch’io a l’ovra fossi attento,\\
una natura in Cristo esser, non piùe,\\
credea,\mat{eresia monofisita} e di tal fede era contento;\\!
ma ‘l benedetto Agapito,\mat{sfasamento cronologico} che fue\\
sommo pastore, a la fede sincera\\
mi dirizzò con le parole sue.\\!
Io li credetti; e ciò che ’n sua fede era,\\
vegg’io or chiaro sì, come tu vedi\\
ogni contradizione e falsa e vera.\mat{linguaggio filosofico}\\!
Tosto che con la Chiesa mossi i piedi,\\
a Dio per grazia piacque di spirarmi\\
l’alto lavoro, e tutto ‘n lui mi diedi;\\!
e al mio Belisar commendai l’armi,\\
cui la destra del ciel fu sì congiunta,\\
che segno fu ch’i’ dovessi posarmi.\\!
\end{verse}
\paragraph{Belisario} Fu un generale di Giustiniano, che ad un certo punto cadde in disgrazia. Visto che qui Giustiniano ne parla molto bene, le ipotesi sono:
\begin{enumerate}
\item Dante non sapeva che Belisario fosse caduto in disgrazia;
\item Dante sapeva ciò, e quindi questa terzina sarebbe una sorta di \textit{palinodia}.
\end{enumerate}
\begin{verse}
\poemlines{3}
\setverselinenums{28}{123}
Or qui a la question prima s’appunta\mat{termina}\\
la mia risposta; ma sua condizione\\
mi stringe a seguitare alcuna giunta,\\!
perché tu veggi con quanta ragione\\
si move contr’al sacrosanto segno\\
e chi ‘l s’appropria e chi a lui s’oppone.\mat{Parla rispettivamente di Ghibellini e Guelfi, e sbagliano entrambi}\\!
Vedi quanta virtù l’ha fatto degno\\
di reverenza; e cominciò da l’ora\\
che Pallante\mat{alleato di Enea, ucciso barbareamente da Turno} morì per darli regno.\\!
Tu \colorbox{yellow}{sai} ch’el fece in Alba sua dimora\\
per trecento anni e oltre, infino al fine\\
che i tre a’ tre\mat{Orazi e Curiazi} pugnar per lui ancora.\\!
E \colorbox{yellow}{sai} ch’el fé dal mal de le Sabine\mat{ratto delle Sabine, Romolo}\\
al dolor di Lucrezia in sette regi,\mat{fine del periodo regio}\\
vincendo intorno le genti vicine.\\!
\colorbox{yellow}{Sai}\mat{anafora} quel ch’el fé portato da li egregi\\
Romani incontro a Brenno,\mat{re dei Galli} incontro a Pirro,\mat{alleato di Taranto nella guerra contro Roma}\\
incontro a li altri principi e collegi;\\!
onde Torquato e Quinzio,\mat{Cincinnato} che dal cirro\\
negletto\mat{ciuffo arruffato} fu nomato, i Deci e ‘ Fabi\\
ebber la fama che volontier mirro.\mat{onoro volentieri}\\!
\end{verse}
\paragraph{Roma} In questi versi si riassume la storia della Roma repubblicana. Vengono ricordate le imprese contro Brenno, il capo dei Galli, contro Pirro, re dell'Epiro, alleato dei Tarentini, le vittorie di Tito Manlio Torquato contro i Galli e i Latini, quelle di Lucio Quinzio Cincinnato contro gli Equi. I tre Deci si sacrificarono in battaglia per la vittoria dei Romani. I Fabi morirono in più di trecento nella guerra contro Veio.
\begin{verse}
\poemlines{3}
\setverselinenums{49}{51}
Esso atterrò l’orgoglio de li Aràbi\mat{Cartaginesi}\\
che di retro ad Annibale passaro\\
l’alpestre rocce, Po, di che tu\mat{fa riferimento al Po} labi.\\!
Sott’esso giovanetti triunfaro\\
Scipione e Pompeo; e a quel colle\\
sotto ’l qual tu nascesti\mat{Firenze; il <<colle>>, situato nei pressi di Firenze, è quello dove sorge Fiesole che, secondo la leggenda, fu distrutta dai romani per aver aiutato Catilina, accusato di aver congiurato contro Roma} parve amaro.\\!
Poi, presso al tempo che tutto ’l ciel volle\\
redur lo mondo a suo modo sereno,\\
Cesare per voler di Roma il tolle.\\!
\end{verse}
\paragraph{Cesare} Secondo Dante, Cesare ha dato il via all'impero voluto da Dio per accogliere Gesù Cristo. I sentimenti di Dante, però, come al solito non sono così chiari:
\begin{itemize}
\item i suoi traditori sono paragonati a Giuda;
\item Catone, suo nemico, è messo a guardia del purgatorio nonostante sia suicida;
\item in questo punto Cesare viene nuovamente visto come un eroe.
\end{itemize}
\begin{verse}
\poemlines{3}
\setverselinenums{58}{60}
E quel che fé da Varo infino a Reno,\\
Isara vide ed Era e vide Senna\mat{affluenti del Rodano}\\
e ogne valle onde Rodano è pieno.\\!
Quel che fé poi ch’elli uscì di Ravenna\\
e saltò Rubicon, fu di tal volo,\\
che nol seguiteria lingua né penna.\\!
\end{verse}
\paragraph{Rubicone} Passare il Rubicone era una dichiarazione di guerra a Roma. Egli dirà ``Il dado è tratta'' (``\straniero{Alea Iacta est}''), proclamando la guerra civile contro Pompeo

\paragraph{Fulmineità} In queste ultime due terzine viene sottolineata la fulmineità delle azioni di Cesare. Da qui il famoso passo del \titolo{5 maggio} di Manzoni.
\begin{verse}
\poemlines{3}
\setverselinenums{64}{66}
Inver’ la Spagna\mat{i pompeiani stavano in Spagna} rivolse lo stuolo,\\
poi ver’ Durazzo, e Farsalia percosse\\
sì ch’al Nil caldo si sentì del duolo.\mat{scontro tra Cesare e Pompeo}\\!
Antandro e Simeonta, onde si mosse,\\
rivide e là dov’Ettore si cuba;\mat{secondo Lucano, Cesare si sarebbe fermato a visitare la Troade}\\
e mal per Tolomeo\textsuperscript{\hyperref[6.23]{23}} poscia si scosse.\mat{riprese il volo}\\!
Da indi scese folgorando a Iuba\mat{Re della Mauritania};\\
onde si volse nel vostro occidente,\mat{Spagna}\\
ove sentia la pompeana tuba.\\!
Di quel che fé col baiulo seguente,\mat{imperatore seguente: Augusto}\\
Bruto con Cassiom\mat{puniti dal II triumvirato: Augusto, Marcantonio e Lepido. I primi due diventeranno nemici} ne l’inferno latra,\\
e Modena\mat{dove fu sconfitto Marcantonio} e Perugia fu dolente.\\!
Piangene ancor la trista Cleopatra\mat{precedentemente amante di Cesare, diventa amante di Marcantonio},\\
che, fuggendoli innanzi, dal colubro\mat{serpente con cui si uccide Cleopatra}\\
la morte prese subitana e atra.\\!
Con costui corse infino al lito rubro\mat{mar Rosso};\\
con costui puose il mondo in tanta pace,\mat{\straniero{pax augustea}}\\
che fu serrato a Giano il suo delubro.\mat{vennero chiuse le porte del tempio di Giano, come succedeva solo in tempo di pace}\\!
Ma ciò che ‘l segno che parlar mi face\\
fatto avea prima e poi era fatturo\\
per lo regno mortal ch’a lui soggiace,\\!
diventa in apparenza poco e scuro,\\
se in mano al terzo Cesare\mat{visto che Dante considera Cesare come primo imperatore, si tratta di Tiberio} si mira\\
con occhio chiaro e con affetto puro\mat{riferimento alla predicazione e alla Crocifissione di Cristo};\\!
ché la viva giustizia che mi spira,\\
li concedette, in mano a quel ch’i’ dico,\\
gloria di far vendetta a la sua ira.\\!
Or qui t’ammira in ciò ch’io ti replìco:\\
poscia con Tito a far vendetta corse\\
de la vendetta del peccato antico.\mat{peccato originale}\\!
\end{verse}
\paragraph{Gerusalemme} Le ultime due righe fanno riferimento a come Tito vendichi la morte di Cristo (con la quali si vendica il peccato originale): la conquista di Gerusalemme.
\begin{verse}
\poemlines{3}
\setverselinenums{94}{96}
E quando il dente longobardo morse\\
la Santa Chiesa, sotto le sue ali\\
Carlo Magno, vincendo, la soccorse.\\!
Omai puoi giudicar di quei cotali\mat{Guelfi e Ghibellini di Firenze}\\
ch’io accusai di sopra e di lor falli,\\
che son cagion di tutti vostri mali.\mat{ambo le parti sono giudicabili}\\!
L’uno al pubblico segno i gigli gialli\mat{corona di Francia (Guelfi)}\\
oppone, e l’altro appropria quello a parte,\mat{i Ghibellini appoggiano l'Impero solo per interessi di partito}\\
sì ch’è forte a veder chi più si falli.\\!
\end{verse}
\paragraph{Apostrofe} Inizia l'apostrofe contro Guelfi e Ghibellini.
\begin{verse}
\poemlines{3}
\setverselinenums{103}{105}
Faccian li Ghibellin, faccian lor arte\\
sott’altro segno; ché mal segue quello\\
sempre chi la giustizia e lui diparte;\\!
e non l’abbatta esto Carlo novello\mat{Carlo il Giovane, re dei francesi}\\
coi Guelfi suoi, ma tema de li artigli\\
ch’a più alto leon trasser lo vello.\\!
Molte fiate già pianser li figli\mat{sembra alludere alla situazione personale di Dante, in quanto i suoi figli lo avevano seguito nell'esilio}\\
per la colpa del padre, e non si creda\\
che Dio trasmuti l’arme\mat{l'aquila} per suoi gigli\mat{emblema della corona di Francia}!\\!
Questa picciola stella si correda\\
di buoni spirti che son stati attivi\\
perché onore e fama li succeda\mat{per conseguire la gloria terrena}:\\!
e quando li disiri poggian quivi,\\
sì disviando, pur convien che i raggi\\
del vero amore in sù poggin men vivi.\mat{è necessario che i raggi del vero amore diminuiscano di intensità}\\!
Ma nel commensurar d’i nostri gaggi\\
col merto è parte di nostra letizia,\\
perché non li vedem minor né maggi.\mat{stesso concetto espresso da Piccarda: le anime sono felici di soddisfare la volontà di Dio}\\!
\end{verse}

\salt
	
\end{document}