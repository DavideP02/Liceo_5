\documentclass{book}
\usepackage[italian]{babel}
\usepackage[utf8]{inputenc}
\usepackage[T1]{fontenc}
\usepackage{microtype}
\usepackage{xcolor}
\usepackage{xargs}
\usepackage{multicol}
\usepackage{graphicx}

\newcommand{\pagine}[1]{\colorbox{yellow}{#1}}
\newcommand{\evidenziatore}[1]{\textbf{#1}}

%definire header e footer
\usepackage{fancyhdr}
\pagestyle{fancy}

  %alternativa 1
  %\renewcommand{\headrulewidth}{0pt}
  %\renewcommand{\footrulewidth}{0pt}
  %\renewcommand{\chaptermark}[1]{\markboth{#1}{}}
  %\fancypagestyle{plain}{
  %\fancyhf{}}
  %\fancyhf{}

  %alternativa 2
  \renewcommand{\chaptermark}[1]{\markboth{#1}{}}
  \fancyhf{}
  \fancyhead[LE,RO]{\scshape\thepage}
  \fancyhead[RE]{\scshape\footnotesize\nouppercase{\leftmark}}
%

%rimuovere header e footer dalle pagine vuote
\makeatletter
\def\cleardoublepage{\clearpage\if@twoside \ifodd\c@page\else
    \hbox{}
    \vspace*{\fill}
    \vspace{\fill}
    \thispagestyle{empty}
    \newpage
    \if@twocolumn\hbox{}\newpage\fi\fi\fi}
\makeatother
%

%rimuovere il titolo del capitolo & impaginazione titolo capitolo
\makeatletter
\def\@makechapterhead#1{%
  \vspace*{30\p@}%
  {\parindent \z@ \raggedright \normalfont
    \vskip 20\p@
    \interlinepenalty\@M
    \Huge \bfseries #1\par\nobreak
    \vskip 40\p@
  }% mettere qui i contatori da annullare per ogni capitolo
}
\makeatother
\addto\captionsitalian{\renewcommand{\chaptername}{}}
%

\usepackage{verse} % utilizzo dell'ambiente per la scrittura in versi
%modifica la posizione del numero di verso
\setlength{\vrightskip}{-30pt}%regola tu
 \verselinenumbersleft
%

%creazione del comando per le note a margine
\newcounter{mar}
\newcommand{\mar}[2]{
\addtocounter{mar}{1}
\hspace{-0.73em}\textsuperscript{\hyperref[\thechapter.\themar]{\themar}}\marginpar{\footnotesize\textbf{\themar}\label{\thechapter.\themar}. #2}\hspace{-0.4em}
}
\newcommand{\mat}[1]{\mar{gg}{#1}}
%fine comando per le note a margine

\newcommand{\finepoesia}{\vspace*{1em}\hrule\vspace*{1em}}


%%%%%% HYPERREF VA CARICATO SEMPRE PER ULTIMO, DEMENTE!

\usepackage[colorlinks]{hyperref}
\definecolor{RoyalBlue}{rgb}{0.0, 0.14, 0.4}
\hypersetup{
     colorlinks=true,
     linkcolor=blue,
     filecolor=blue,
     citecolor = black,
     urlcolor=cyan}

\title{Italia postunitaria}
\date{Interrogazione 8 mar 2021}
\author{Davide Peccioli}


\begin{document}

\maketitle

\tableofcontents
\setcounter{secnumdepth}{-1}

\chapter{Introduzione}

\section{Politica e società}

\begin{itemize}
\item
  \pagine{p. 4-7}: Le strutture politiche, economiche e sociali
\end{itemize}

Stiamo parlando di quella letteratura che si sviluppa a ridosso della
unità d'Italia fino alla fine dell'800.

Uno dei temi portanti di questa letteratura è il \textbf{progresso}, che
benché sia chiaro esista, non deve essere identificato come la felicità
umana. Questo è un figlio diretto del ``fallimento'' dell'Illuminismo.

L'unità d'Italia ha avuto in impatto enorme sull'opera letteraria
dell'epoca.

In Italia il romanticismo è diverso da quello d'oltralpe, molto più
positivo, ed è impregnato di valori positivo quali l'\textbf{amor
patrio}. Alla conclusione di questo processo romantico, che culmina con
l'unità d'Italia, inizia la delusione di tutti quegli intellettuali che
vedevano nel Romanticismo un modo per cambiare la nazione; la differenza
tra nord e sud, che sembra incolmabile, rappresenta il nodo principale
di questa delusione.

Tra le ragioni, vi è principalmente l'inettitudine del governo sabaudo,
che, senza considerare le differenze tra nord e sud, ampliarono
semplicemente la loro legislazione su tutto il territorio italiano: -
\textbf{leva obbligatoria} - \textbf{tassazione} - etc etc etc

Lo stesso Manzoni, con la sua ricerca di una lingua comune, rappresenta
questo desiderio di creare una Italia realmente unita, senza differenze
o spaccature al suo interno.

Iniziarono diverse \textbf{inchieste}: la più famosa è quella di
Franchetti e Sonnino, \textsc{Inchiesta in Sicilia}, che mise in luce le
condizioni di bambini nel sud Italia. Altro fattore è
l'alfabetizzazione, minima nel sud più che nel nord, che si cercò di
migliorare per mezzo dell'istituzione della scuola obbligatoria; questo
causò ulteriori problemi, poiché insegnanti del nord andavano al sud e
viceversa, non capendo i propri studenti.

In definitiva,
\evidenziatore{questo enorme squilibrio dettato dalle differenze tra il nord e il sud}
porta gli intellettuali a vivere \textbf{questa enorme delusione}:
l'idea dell'unità d'Italia aveva, durante il romanticismo, affascinato
moltissimi intellettuali, che vedevano in essa la risoluzione dei
problemi italiani, quali la dominazione degli stranieri, etc etc etc;
invece l'unità reale porterà un enorme delusione.

Questo fenomeno è molto simile a quello che abbiamo visto con la
Rivoluzione Francese: inizialmente vi era una grande aspettativa, che
poi viene completamente delusa dal corso degli eventi.

Da un punto di vista politica, vi fu un certo impegno per quanto
riguarda le infrastrutture, quali le ferrovie. Si ha un fenomeno
d'\textbf{industrializzazione} che nel resto di Europa è molto più
avanti: quando in Italia si inizia, in paesi quali Francia e Inghilterra
si vivono e si vedono già i suoi aspetti negativi. 

\paragraph{ex.}Se il naturalismo di Zola ha come protagonista il proletariato urbano,
il naturalismo di Verga (che pure si ispira a Zola) ha come protagonisti
i minatori, i contadini, etc etc etc

\begin{itemize}

\item
  \emph{1851}: I esposizione universale a Londra
\item
  \emph{1881}: esposizione a Milano: preparazione del \textsc{Gran Ballo
  Excelsior}
\item
  \emph{1889}: esposizione a Parigi, per cui viene costruita la Tour
  Eiffel, molto criticata a causa dell'utilizzo del Ferro, materiale
  considerato poco nobile
\end{itemize}

Tutte queste fiere universali sono la testimonianza di questo grande
processo di \textbf{industrializzazione}; questo argomento è trattato in
un romanzo di \textbf{Hemile Zola}: l'industrializzazione ha causato la
morte delle piccole imprese e della vendita al dettaglio.

In questo periodo nasce la \textbf{pubblicità}, nonché il
\textbf{cinema}.

\section{Invenzioni}\

\begin{itemize}

\item
  \textbf{Palazzo di Cristallo}: Fu costruito in onore dell'esposizione
  universale, e inaugurato dalla Regina Vittoria.
\end{itemize}

L'epoca fu caratterizzata da molte altre invenzioni, come
l'illuminazione elettrica, le macchine, etc etc etc. Tutte queste
invenzioni cambiarono proprio lo stile di vita delle persone, e di
conseguenza l'arte.

\subsection{Illuminazione}

\begin{itemize}

\item
  Lampada ad Arco di Giacomo Balla
\item
  Notturno su un Voulvar Parigino di Stein
\end{itemize}

L'illuminazione è stata una rivoluzione; se prima le strade di notte
erano poco sicure e poco raccomandabili, adesso sono illuminate. Diventa
fondamentale la figura del \textbf{lampionaio}.

Cambia la vita degli uomini e degli individui. La ritroveremo in molte
opere

\section{Positivismo}

\begin{itemize}

\item
  \pagine{p. 8}: Il Positivismo, Il mito del progresso
\end{itemize}

È una ripresa di quella filosofia settecentesca dell'Illuminismo, che
riservava un'illimitata fiducia nella ragione.

\textbf{Compte} riteneva che fosse assolutamente possibile leggere e
interpretare la società attraverso un \textbf{approccio scientifico}, e
anche migliorare la società attraverso lo stesso. Riteneva che la classe
dirigente dovesse avere delle competenze tecnico scientifiche. Questo
concetto sarà fondamentale per \textbf{Emile Zola}: per lui l'opera
teorica di Compte sarà fondamentale.

Nel 1830 Compte scrive un \textsc{Corso di filosofia Positiva}. - La
scienza è lo strumento fondamentale per conoscere la realtà - La storia
dell'uomo è una lunga storia di una evoluzione passata attraverso varie
fasi (Darwin sta lavorando nella stessa direzione da un punto di vista
più scientifico che filosofico) - Attraverso il metodo positivo (metodo
scientifico) si indaga la realtà, riuscendo a distinguere i
\textbf{nessi causa effetto}, che sono alla base di tutto -
\evidenziatore{pertanto la condizione umana esclude qualsiasi possibilità di spiegazioni metafisiche e non oggettive}

\section{Darwinismo}

In quegli stessi anni \textbf{Darwin} stava sviluppando le sue
\evidenziatore{teorie sull’\textit{evoluzionismo}}: l'uomo non è stato creato,
ma è giunto al livello in cui si trova attualmente grazie ad una
evoluzione. Egli parte da un metodo scientifico, ovvero con
l'osservazione del dato reale, per poi arrivare ad una teoria; egli è
convinto che ci sia una lotta per la sopravvivenza (secondo la
\textbf{legge del più forte}), che porta ad una selezione naturale per
cui sopravvive solo il più adattato. Questo porta alla continua
evoluzione delle specie. Questi concetti rivestiranno una importanza
fondamentale in letteratura: questo concetto di evoluzione può essere
associato al concetto di progresso; secondo Verga, ad esempio, il
progresso si lascia alle spalle una serie di cadaveri.

\section{Determinismo sociale}

\textbf{Hippolyte Taine} teorizza il
\evidenziatore{determinismo sociale}: secondo lui il carattere di ogni
individuo è determinato dalla società, dalla storia, dall'ambiente e da
tutta una serie di fattori esterni. Questo è molto importanti per autori
come Zola, che studia gli effetti dell'ambiente industriale sulla
società: i suoi personaggi sono della classe industriale, e si studiano
tutti gli effetti negativi del processo di industrializzazione e del
capitalismo.

Secondo queste teorie, tutti i problemi che attanagliavano quella
società, come l'\textbf{alcolismo}, erano dettati da problemi concreti
dell'ambiente industrializzato.

\section{Letteratura e cultura}

\begin{itemize}

\item
  \pagine{p. 18-22}: Fenomeni letterari e generi - Non tutto
\end{itemize}

Non c'è una grande varietà di nuovi generi letterari, ma si afferma il
\textbf{romanzo}: dopo la morte di Manzoni i \textsc{Promessi Sposi}
diventano parte del programma ministeriale in Italia.

Nasce la \textbf{critica letteraria}: De Santis scrive la \textsc{Storia
della Letteratura italiana}

Abbiamo le diverse riforme scolastiche: - \textsc{Legge Casati} (1859):
impone l'obbligo scolastico dei primi due anni della scuola elementare -
\textsc{Legge Coppino} (1877): impone l'obbligo scolastico fino a nove
anni

Nel meridione, dove vi era uno sfruttamento minorile, queste leggi non
venivano rispettate; alla fine del secolo, quindi, vi era una forte
analfabetizzazione.

Come già detto, l'unità d'Italia aveva sottolineato come l'Italia non
fosse affatto unita: si cerca quindi una matrice comune nella cultura.
La \textsc{Storia della Letteratura Italiana} andava proprio in questo
verso. Iniziano ad apparire statue e vie dei grandi autori italiani.

Il romanzo si afferma, le cronache e la memorialistica pure. Si ha una
riscoperta del melodramma, come la \textsc{Traviata} di Giuseppe Verdi
(\pagine{p. 23}), che è la trasposizione in melodramma de \textsc{La
signora delle Camelie} di Dumas

In generale si possono riscontrare negli intellettuali di quest'epoca
degli atteggiamenti diversi rispetto alle novità trattate, rispetto alla
modernità e al progresso che avanzano con una velocità incredibile. 1.
Alcuni hanno delle reazioni assolutamente entusiaste ed encomiastiche 2.
Altri hanno atteggiamenti di assoluto rifiuto, un rifiuto a priori del
progresso, con un atteggiamento di tipo Romantico che vede nel passato
una sorta di rifugio 3. Altri autori invece hanno un atteggiamento
sereno e di accettazione della modernità, magari ignorando questi
aspetti.

\section{Il problema della lingua}

\begin{itemize}
\item
  \pagine{p. 16-17}
\end{itemize}

In questo periodo il problema della lingua è solo la punta dell'iceberg
che dimostra e rende evidente le differenze e la frammentazione
dell'Italia unita.

Manzoni si occupa di questo problema già prima dell'unificazione
italiana; interviene subito, fin dall'inizio della sua carriera, con la
sua opera più famosa: \textsc{I promessi sposi}. Manzoni sceglie il
Fiorentino colto parlato, come lingua unitaria. Nel \textbf{1868} egli
viene nominato presidente per la commissione della lingua, e in quest
occasione fa una relazione intitolata \textsc{Dell'unità della lingua e
dei mezzi per diffonderla}. Qui continua ad indicare il fiorentino come
lingua che avrebbe potuto creare il contesto unitario, ma propone
addirittura l'utilizzo di maestri fiorentini nelle scuole italiane.

Nel \textbf{1872} è pubblicato il \textsc{Vocabolario della lingua italiana},
di stampo fiorentino; si tenga conto che all'epoca c'era dal 70 al 90\%
di analfabetismo.

\section{Obbligo scolastico}

\begin{itemize}

\item
  \pagine{p. 10-11}: La scuola
\end{itemize}

La situazione all'indomani dell'unificazione era molto diversa rispetto
alle aspettative e rispetto alla coscienza dei singoli italiani rispetto
alla situazione: l'unione politica mette in luce la divisione sotto ogni
aspetto.

La scuola, come la cultura, cerca di assumere una funzione unificatrice:
le fratture si cercano di ricomporre sulla base di una unione culturale
e linguistica. La scuola quindi cerca non solo di istruire le masse, ma
anche di amalgamare la popolazione. Questo aspetto sarà poi ben
analizzato da Verga, che con novelle come \textsc{Rosso Malpelo} studia
l'aspetto parallelo alla scuola: il lavoro minorile. Pochi bambini,
infatti, solvevano a quest'obbligo, poiché in molti luoghi si sfruttano
i bambini.

\section{Intellettuale e società}

\begin{itemize}

\item
  \pagine{p. 12-13}: Gli intellettuali
\end{itemize}

Nonostante questo vi era il fenomeno della \textbf{disoccupazione
intellettuale}. Di fatto, sebbene gli studenti universitari non fossero
tanti come oggi, capitava che il sistema economico produttivo italiano
fosse così arretrato che questi laureati non trovavano lavoro, né un
impiego inerente al proprio status. Questo problema impensieriva le
classi dirigenti, in quanto questi intellettuali disoccupati potevano
dimostrare il proprio sentimento di fastidio nei confronti della
situazione.

Questo fenomeno è strettamente collegato ad al tema del \textbf{rapporto
tra intellettuale e società}. Col romanticismo, in Italia vi erano
grosse differenze rispetto al resto dell'Europa. L'intellettuale,
infatti, all'epoca aveva ancora un ruolo importante, e non si erano
diffuse le tematiche negative proprie del romanticismo europeo Con
questo processo della \textbf{disoccupazione intellettuale} inizia ad
avvertirsi, anche in Italia, lo scollamento dell'intellettuale dalla
società: con un ritardo di qualche decennio, nella letteratura del
secondo ottocento, si profila questo scollamento tra i valori
dell'intellettuale con quelli della società borghese; il primo esempio è
quello della corrente della \textbf{scapigliatura}.


\section{Letteratura per l'infanzia}

\begin{itemize}

\item
  \pagine{p. 20-21}: Il romanzo di consumo e la letteratura per
  l'infanzia
\end{itemize}

È una letteratura didascalica che si occupa di sancire l'importanza
della scuola per le ragioni sopra citate. I testi trattati saranno
\textsc{Pinocchio} e \textsc{Libro Cuore}

Questa letteratura ha uno scopo prevalentemente di istruire

\subsection{Libro Cuore}

Il \textsc{libro Cuore}, ad esempio, è il diario di Enrico, intervallate
da 9 racconti e da lettere dei genitori che Enrico riporta nelle pagine
di diario.

Questo libro è il documento di un epoca che dimostra quanto si volesse
scegliere una funzione ben specifica per la scuola, ovvero di amalgamare
una nazione che era prevalentemente disunita: è una scuola statale,
pubblica e obbligatoria.

Questo libro pone al primo posto la scuola, con quasi una sorta di
adorazione, e poi al secondo posto i genitori (ed in particolare la
mamma di Enrico).

\textbf{Leggere una pagina di diario, uno dei nove racconti e una
lettera}

\subsection{Pinocchio}

Ha avuto decisamente più fortuna, e nasce come ``una bambinata'': nasce
come un libro per bambini, ma ci sono delle simbologie e dei topoi
letterari che vanno ben oltre al pubblico infantile. Il motivo per cui
nasce come una bambinata è che l'autore lo scrive così, quasi per gioco,
e poi chiamando un direttore di una rivista per bambini, gli chiede di
pubblicarlo.

Lo scopo è educativo, allora come adesso. La scuola è al centro
dell'attenzione anche in questo romanzo, tanto che Pinocchio, che ha
all'inizio è una testa di legno, decide di abbandonare la scuola:
durante tutto il suo viaggio sono presenti delle metamorfosi e dei topos
molto emblematici, e alla fine andrà anche a scuola

Un messaggio importantissimo è che Pinocchio cresce e impara grazie alla
prassi, ovvero sbatte il naso contro le cose prima di capirle;

Nella dichiarazione dei diritti umani c'è il diritto all'istruzione, che
a nostro parere oggi è portato a compimento, specie in Italia.

\chapter{La scapigliatura}

\begin{itemize}
\item
  \pagine{p. 27-30 - Un crocevia intellettuale}
\end{itemize}

Questo movimento è molto importante; anche autori che invece sono
decisamente più famosi, come Verga si era avvicinato molto a questo
movimento.

La scapigliatura nasce come un gruppo di intellettuali accomunati da una
esperienza biografica; si definisce a seguito della pubblicazione de
\textsc{La scapigliatura e il sei febbraio}, un romanzo di Cletto Arrighi;
è la prima volta che compare questo termine, ed è ambientato in un clima
rinascimentale..

Questo movimento nasce nell'Italia Settentrionale, soprattutto a Torino,
Genova e Milano. Il nome fa riferimento ad un gruppo di Intellettuali
che vivono in un certo modo; è una sorta di libera traduzione di
\emph{Boheme}: letteralmente significa ``vita da Zingaro'', e questa
traduzione veniva dal fatto che si pensava che gli zingari venissero
dalla Boemia. Questo termine a Parigi venne usato per identificare degli
intellettuali: classici artisti disadattati, che vivono al limite della
povertà. Questo movimento è reso famoso in Italia da un testo di
Puccini, \emph{La Boheme},

Questi artisti della scapigliatura non si riconoscono nella società
attuale, vivono con disprezzo il capitalismo, la nascente borghesia e i
suoi valori, e quindi si riducono ai margini della società. Le biografie
di questi scrittori, presentano vicende analoghe tra di loro: la più
parte di questi autori vissero vicende molto simili a quelle dei loro
protagonisti; si tratta molto spesso di giovani nati da ottime famiglie,
che poi vivono o un dissesto finanziario o si separano dalla famiglia,
riducendosi a vivere in povertà, abusando a volte di alcol e droghe; la
maggior parte muore giovanissimi

I romanzieri si trovano in un clima molto simile a quello che nel resto
di europa era stato vissuto qualche decennio prima: nasce quella crisi
di identità dell'intellettuale, che non trova più una collocazione
all'interno della società; tutto il processo risorgimentale aveva ancora
ricavato un ruolo per il poeta, dal momento che era stato accompagnato
ed esaltato nei temi patriottici dai poeti e dai letterati italiani. Ora
invece anche l'intellettuale vive questo disagio, proprio dell'Italia e
dell'intellettuale stesso.

Gli scapigliati, anche se con qualche decennio di ritardo, guardano a
Baudelaire come esempio, dal momento che egli scrive del male di vivere.
Questi intellettuali, quindi, riprendono tutta una serie di motivi del
romanticismo europeo, che in Italia erano stati completamente saltati:
Manzoni (unico autore importante dichiaratamente romantico) aveva avuto
un approccio totalmente diverso al romanticismo rispetto al resto
dell'Europa.

Gli scapigliati riprendono tutte quelle tematiche fosche e buie, come il
buio, la paura, tipiche del romanticismo europeo dei primi anni
dell'800.

\emph{Come si giunge a questo?} Questi intellettuali vivono in modo
decisamente negativo il rapporto con il progresso e con la società;
esprimono questo loro approccio nelle loro opere. Loro non si ritrovano
nei valori espressi dalla borghesia e dal capitalismo, e allo stesso
tempo non riescono più a recuperare i valori che si sono lasciati alle
spalle. Nella loro letteratura c'è un autore che viene sempre pieno di
mira: è Manzoni; è poeta preso di mira perché rappresenta quegli ideali
in cui loro non possono più credere: si riconoscerebbero ma non possono
più attuarli nella società contemporanea. La fede è un esempio di questi
valori non più attuabili. C'è una sorta di vagheggiamento di quelli
ideali come la bellezza, l'arte, la poesia, ma allo stesso tempo la
consapevolezza di non poterli più perseguire; la decisione, quindi, è
quella di descrivere, accogliendo la lezione dei naturalisti (quali
Emile Zola), anche il brutto.

Esempio è il romanzo \textsc{Fosca} (\pagine{trama p. 45}) di Igino Ugo
Tarchetti: ci sono due sorelle, Fosca e Clara: la prima è scura, mentre
la seconda è un chiaro riferimento alla donna angelo. Il protagonista si
innamora di Fosca, che è descritto come brutta dal protagonista stesso.
È il primo esempio di \emph{Femme Fatale}; la femme fatale è
diametralmente opposta alla donna angelo. Lei trascina il protagonista
nella sua malattia, sia fisica che mentale, e il loro rapporto è malato

Questi scapigliati si pongono in un momento particolarmente emblematico
della storia letteraria italiana, e recuperano delle tematiche
romantiche che nel resto di europa si erano sviluppate prima, e che
saranno poi mantenute in auge fino ai primi del novecento. Il
romanticismo, in effetti, non muore per ancora molto molto tempo, dal
momento che addirittura il decadentismo è originato dal pensiero
romantico. La scapigliatura, quindi, partendo da tematiche romantiche e
naturalistiche, avvierà molti autori verso il decadentismo.

\section{T: \scshape{Preludio}}

Questa poesia contiene tutti gli spunti tematici per comprendere questo
movimento.

\begin{verse}
\poemlines{5}
\textit{Noi}\textsuperscript{1} siamo i figli dei \textit{padri ammalati}\textsuperscript{2};\\
\textit{Aquile}\textsuperscript{3} al tempo di mutar le piume,\\
\textit{Svolazziam}\textsuperscript{4} muti, attoniti, affamati,\\
Sull’agonia di un \textit{nume}\textsuperscript{5}.\\!
Nebbia remota è \textit{lo splendor dell’arca}\textsuperscript{6},\\
E già all’\textit{idolo d’or}\textsuperscript{7} torna l’umano,\\
E dal vertice sacro il patriarca\\
S’attende invano;\\!
S’attende invano dalla \textit{musa bianca}\textsuperscript{8}\\
che abitò venti secoli il Calvario,\\
E invan l’esausta vergine s’abbranca\\
Ai lembi del Sudario...\\!
\textit{Casto poeta}\textsuperscript{9} che l’Italia adora,\\
\textit{Vegliardo}\textsuperscript{10} in sante visioni assorto,\\
Tu puoi morir!... Degli antecristi è l’ora!\\
Cristo è rimorto!\\!
O \textit{nemico lettor}\textsuperscript{11}, canto \textit{la Noia}\textsuperscript{12},\\
L’eredità del dubbio e dell’ignoto,\\
Il tuo re, il tuo pontefice, il tuo boia,\\
Il tuo cielo, e il tuo loto!\\!
Canto litane di martire e d’empio;\\
Canto gli amori dei sette peccati\\
Che mi stanno nel cor, come in un tempio,\\
Inginocchiati.\\!
Canto le ebbrezze dei bagni d’azzurro,\\
E l’Ideale che annega nel fango...\\
Non irrider, \textit{fratello}\textsuperscript{13}, al mio sussurro,\\
Se qualche volta piango:\textsuperscript{14}\\!
Giacchè più del mio pallido demone,\\
Odio il minio e la maschera al pensiero,\\
Giacchè canto una misera canzone,\\
Ma canto il vero!
\end{verse}

\vspace*{1em}\hrule\vspace*{1em}

\begin{enumerate}
\item fa riferimento ai poeti scapigliati
\item  fa riferimento ai poeti
  romantici della generazione precedente, ovvero principalmente a
  Manzoni: sono i Romantici italiani; gli scapigliati riprendono molti
  temi tipici del romanticismo, ma quello europeo d'inizio secolo.
  Questi poeti attaccano ciò che non può più essere ripreso; Manzoni
  rappresenta il poeta dai forti ideale, ma costoro non possono più.
\item  riferimento al fatto che le aquile,
  quando cambiano le piume, non riescono più a volare come prima: i
  poeti sono paragonati alle aquile
\item è un verbo che sta a metà, e
  rappresenta l'idea di un movimento privo di meta precisa e di punti di
  riferimento certi
\item chiaro riferimento a Manzoni
\item sono i valori religiosi, che
  non si riescono più a vedere chiaramente; viene meno il sentimento di
  fede nei confronti della religione
\item riferimento all'adorazione
  dell'idolo d'oro nel deserto da parte degli Ebrei, stufi di aspettare
  Mosè. Riferimenti fittizi. Questi versi sono l'emblema della mancanza
  di fiducia in valori in cui poter credere
\item fa riferimento alla poesia
  cristiana, che non ha più senso: non ci sono più i valori religiosi e
  la fede, quindi questa poesia non è più percorribile
\item riferimento a Manzoni; non gli
  augurano la morte, ma fanno riferimento alla sua mentalità
\item è un termine ``cattivo'',
  ``arrabbiato''
\item il poeta identifica il lettore
  con la classe borghese, i cui ideali non sono condivisi.
\item tema poi molto presente nel
  decadentismo
  \item il lettore è diventato fratello perché diventa uomo: siamo tutti accomunati dallo stesso destino
\item c'è il desiderio degli ideali, di cantare ``bagni
  d'azzurro'', ma questi ideali ``annegano nel fango''. C'è questa
  duplice attrazione verso il reale e verso il brutto e il concreto
\end{enumerate}

La poesia è una sorta di manifesto della Scapigliatura, in cui Praga descrive la condizione spiri tuale che è propria di un'intera generazione intellettuale (si noti il «noi» collettivo), quella suc cessiva al Romanticismo.

\paragraph{La perdita dei valori}La prima parte (strofe 1-4) è negativa e mira a definire ciò che quella generazione non può più essere: essenzialmente non ha più la fede religiosa, fonte di tutti i valori. Per questo Praga esprime un duro rifiuto nei confronti di Manzoni, che rappresenta appunto lo scrittore che ispira tutta la sua vita a quei valori, fede religiosa, integrità morale, vita casta. Nei confronti di Manzoni gli scapigliati hanno un atteggiamento ambivalente di odio-amore, ripul sä-ammirazione. Egli costituisce come una figura paterna, a cui sentono la necessità di ribellarsi, ma di cui non riescono a liberarsi, perché ne avvertono la grandezza ineguagliabile, che li schiac cia. Il compiacimento "maledetto" del vizio e della bestemmia è anche un modo per negare una presenza incombente e condizionante, per "uccidere" simbolicamente il padre. Il rovesciamento dell'estetica e dell'etica manzoniane assume toni oltranzistici («tu puoi morir!... Degli antecristi è l'ora!»): in realtà proprio il tono truculento, nella sua esagerazione, tradisce una disperata no stalgia della fede. Come ha ben indicato Tessari a proposito di questa poesia, «la bestemmia è una preghiera capovolta che conferma la fede in Dio». La tematica baudelairiana La seconda parte (strofe 5-8) definisce invece ciò che quella gene razione intellettuale è realmente (o crede, o pretende di essere) dopo la perdita delle certezze. Si delinea chiaramente la tematica baudelairiana: la noia, rappresentata come carnefice della tor mentata anima moderna; la tensione verso l'ideale e la perdizione nel vizio e nel male; gli atteggiamenti blasfemi, ma che imitano la devozione religiosa («litane di martire e d'empio», v. 21); la “malattia" interiore che porta alla distruzione («pallido demone», v. 29).

\paragraph{Una dichiarazione di poetica} L'ultimo verso («canto il vero») è una dichiarazione di poetica. Non si riferisce tanto al vero scientifico, positivisticamente inteso (di cui non si ravvisa traccia nella poesia) ma, come si ricava dal contesto, la realtà desolata della vita moderna, privata di fedi e di ideali, che la poesia deve rivelare nel suo volto brutale, senza mascheramenti ipocriti: il vi zio, l'abiezione, la malattia interiore, lo spleen. Per questo la canzone è «misera», perché dipinge senza finzioni la miseria della vita moderna.

\section{T: \textsc{Dualismo}}

\paragraph{Dualismo e vita moderna} È una poesia "manifesto", che definisce la condizione spirituale dell'avanguardia scapigliata. Essa riprende motivi baudelairiani, ma, ancora più indietro, del Romanticismo tedesco: la lacerazione tra due opposti inconciliabili, la tendenza alla sublimazione nell'idea le e la caduta nel vizio e nel male.
Questo dualismo è evidentemente il riflesso di una condizione di crisi: nasce dalla consapevolezza di vivere un'età che nega i valori ideali, dominata dal criterio della pura economicità, dalla cancellazione di ogni bellezza nello squallore del nascente industrialismo moderno. La vita modema può solo essere sofferenza, bruttezza, turpitudine. Questo, nella visione di Boito, diventa un pessimismo metafisico: l'uomo è la creatura di un ``buio'' dio del male, che l'ha creata per compiacersi della sua sofferenza. Da questo rifiuto della vita moderna nasce il rimpianto dell'ideale, che è una condizione ormai irraggiungibile di purezza morale e di bellezza estetica. Dalla definizione di una condizione spirituale deriva una dichiarazione di poetica. Il poeta ispira ad un'arte che realizzi la bellezza ideale, assoluta, contrapposta alla bruttezza moder na (l' ``Arte eterea'', v. 71), ma poiché questa bellezza è impossibile, non resta che cantare il ``vero", la squallida e prosaica realtà presente, che è la negazione del ``Vero'' ideale (``un ver che mente al Vero''. v. 95): ne può nascere solo una poesia aspra, sgradevole, urtante.

\paragraph{Il linguaggio poetico} Boito si compiace di sviscerare con minuzia analitica i temi, di ampliare retoricamente i concetti, moltiplicando i paragoni e le metafore, in una serie studiata di parallelismi e di antitesi. Prevale lo studio formalistico, che lo avvicina ai parnassiani francesi, cultori del la forma perfetta (Gioanola). Si può notare tuttavia come questi poeti scapigliati non operino una rivoluzione del linguaggio poetico, nell'uso della parola, della sintassi, delle immagini, dei metri. La parola è ben lontana dal caricarsi di valori suggestivi ed evocativi, che agiscano a un livello più profondo di quello della comunicazione razionale (come avverrà pochi anni più tardi con d'Annunzio e Pascoli). Le immagini hanno ancora un impianto retorico, non rendono il con senso di misteriose «corrispondenze» tra le cose. Anche i metri sono quelli fortemente ritmati della tradizione romantica italiana (mentre d'Annunzio e Pascoli, in forme diverse, dissolveranno la parola in musica, grazie a nuove modulazioni metriche).

\section{T: \textsc{L'attrazione della morte}}

\begin{quote}
	Dio! Come esprimere colle parole la bruttezza orrenda di quella donna! Come vi sono beltà di cui è impossibile il dare una idea, così vi sono bruttezze che sfuggono ad ogni manifesta zione, e tale era la sua. Né tanto era brutta per difetti di natura, per disarmonia di fattezze, - ché anzi erano in parte regolari, -quanto per una magrezza eccessiva, direi quasi incon- cepibile a chi non la vide; per la rovina che il dolore fisico e le malattie avevano prodotto sulla sua persona ancora così giovine. Un lieve sforzo d'immaginazione poteva lasciarne travedere lo scheletro, gli zigomi e le ossa delle tempie avevano una sporgenza spaventosa, l'esiguità del suo collo formava un contrasto vivissimo colla grossezza della sua testa, di cui un ricco volume di capelli neri, folti, lunghissimi, quali non vidi mai in altra donna, aumen tava ancora la sproporzione. Tutta la sua vita era ne' suoi occhi che erano nerissimi, grandi, velati – occhi d'una beltà sorprendente. Non era possibile credere che ella avesse mai potu to essere stata bella, ma era evidente che la sua bruttezza era per la massima parte effetto della malattia, e che, giovinetta, aveva po tuto forse esser piaciuta. La sua persona era alta e giusta; v'era ancora qualche cosa di quella pieghevolezza', di quella grazia, di quella flessibilità che hanno le donne di sentimento e di nascita distinta; i suoi modi erano così naturalmente dolci, così spontaneamente cortesi che pareva no attinti dalla natura più che dall'educa zione: vestiva colla massima eleganza, e veduta un poco da lontano, poteva trarre ancora in inganno. Tutta la sua orribilità era nel suo viso.
\end{quote}

Questa è la descrizione di Fosca, che corrisponde un po' alla
descrizione di Femme Fatale. Di solito è bella, di bellezza diversa
rispetto alla donna angelica. Qui è brutta, ma ci sono caratteristiche,
come gli occhi e i capelli, che sono tipici della Femme Fatale

\paragraph{Trama} Il protagonista, Giorgio, ufficiale dell'esercito (che racconta in prima persona), è diviso tra due immagini femminili: Clara, donna bella e serena, con cui ha una relazione felice e gioio sa, e Fosca, bruttissima, isterica, dalla sensibilità acuta e pato logica. Fosca è la cugina del colonnello comandante la guar nigione della piccola città dove Giorgio, come ufficiale, è destinato. Nella cornice grigia della provincia si svolge il nucleo della vicenda, che è povera di eventi esteriori e tutta giocata a livello psicologico. A poco a poco Giorgio subisce il fascino morboso di Fosca, senza potersene più liberare. La donna muore dopo una spaventosa notte d'amore con lui, e il protagonista resta come contaminato dalla malattia della donna.
\paragraph{Tra studio naturalistico e simbolismo ``nero''} il romanzo si muove su due piani, tra loro strettamente congiunti. Da un lato rivela un interesse per l'analisi del caso patologi co, sulla linea del romanzo naturalistico (nel 1865 i de Goncourt avevano già studiato un caso di isteria nel romanzo \textsc{Germinie Lacerteux}); dall'altro lato pre senta una struttura simbolica, che si colloca nella dimensione del "nero" e dell'orrore. Fosca è la donna fatale, la "donna vampiro" che succhia la vita dell'uomo che cade vittima del suo fascino e gli trasmette il suo morbo. Ma nella sua magrezza estrema, che evoca costantemente il teschio e lo scheletro, è anche un'immagine della morte.
Il protagonista subisce dunque il fascino tenebroso della mor te e sprofonda nel piacere dell'autodistruzione. Si crea così un'opposizione simbolica tra Fosca e Clara (i due nomi han no un evidente valore allusivo), la quale rappresenta al con trario l'attrazione della vita. In questa esplorazione dei «misteri del sottosuolo psichico» (Gioanola), in cui Eros e Thanatos, Amore e Morte si mescolano inestricabilmente, il romanzo è un'interessante anticipazione di tematiche decadenti (lo vedremo in d'Annunzio). Come ha scritto Italo Calvino, «Fosca è un personaggio tra liberty e dannunziano saltato fuori con un anticipo di almeno vent'anni in un mondo che non è ancora) il suo>>

\chapter{Carducci}

\begin{itemize}
\item
  \pagine{p. 92-93-96-97 - In sintesi: studiare solo quello}
\end{itemize}

Carducci cerca di traslare la verseggiatura classica alla poesia
italiana, con successo ma con un risultato poco gradevole

Carducci è il classico poeta, in completa opposizione rispetto agli
scapigliati.

Tanta critica dice che Carducci era l'unico sano in mezzo agli ammalati

Egli nasce nel 1835 in maremma, paesaggio selvaggio che ha influenzato
moltissimo la sua produzione, è un poeta che si distingue dagli altri
del suo tempo per la sua sanità. L'intellettuale in questo periodo è in
crisi, ma Carducci, con vicende alterne, con le sue rievocazioni
storiche solide, divenne l'emblema del poeta vate dello stato italiano,
al punto che si metteva in evidenza la sua sanità rispetto al contesto
in cui agiva

Le sue rievocazioni storiche, che poggiano sui classici, riprendono
momenti dell'età classica e del medioevo; addirittura sperimentò modi di
fare poesia utilizzando i metri classici; esempio sono le \textbf{Odi
Barbare}.

Successivamente la critica su Carducci tendeva a sottolineare come le
sue fughe verso i mondi lontani dei classici, era sostanzialmente delle
fughe vere e proprie, ovvero il suo modo per recuperare dei valori che
si stavano diffondendo: era un modo diverso per cercare di arginare
questo dissidio che viveva all'interno della società.

Oggi la critica su Carducci è piuttosto negativa: il poeta appare un po'
tronfio, e il suo tentativo di riprendere quello stile classico un po'
datato; inoltre la sua vicenda subisce delle variazioni che ce lo
mostrano poco coerente. Abbiamo un poeta che in età giovanile segue con
entusiasmo il processo rinascimentale, e come tanti altri poeti alla
fine di questo processo vive quella delusione data dal risultato
dell'unificazione. Segue un periodo estremamente violento nei toni:
inizia una poesia di invettiva contro tutto e contro tutti, conto lo
stato neonato, contro la chiesta. Ad un certo punto, complice anche un
incontro con la regina, inizia a smorzare i toni della sua poesia, e
nasce quella figura di poeta di maniera: diventerà senatore dello stato,
e quindi probabilmente questa trasformazione risulta poco apprezzabile
dal punto di vista ideologico.

\section{T: \textsc{Pianto antico}}

Dalle rime nuove, è poesia molto semplice ma che nasce da un sentimento
molto puro di dolore per la morte del figlioletto Dante

Il testo è una \emph{odicina anacreontica}: Anacreonte era un poeta
greco che per primo aveva utilizzato questo verso

Qui abbiamo un contrasto tra la vita e la morte: è rappresentato dal
giardino in cui ritornerà la primavera, mentre la primavera non tornerà
né per il bambino né per il poeta. Questo contrasto si gioca anche sul
piano dei colori: nella prima parte abbiamo verde e rosso, mentre nella
seconda compaiono delle tinte decisamente più fosche, come la
\emph{terra negra}. Anche sul piano lessicale, troviamo \emph{rinverdì},
che con il prefisso ``ri'' indica rinascita, ma è affiancato a
\emph{orto solingo}.

C'è una metafora tipica, quella che deriva dal fiore reciso di Virgilio.

\begin{verse}
\poemlines{5}
L’albero a cui tendevi\\
La pargoletta mano,\\
Il verde melograno\\
a’ bei vermigli fior,\\!
Nel muto orto solingo\\
Rinverdí tutto or ora\\
E giugno lo ristora\\
Di luce e di calor.\\!
Tu fior de la mia pianta\\
Percossa e inaridita,\\
Tu de l’\textit{inutil}\textsuperscript{1} vita\\
Estremo unico fior,\\!
Sei ne la terra fredda,\\
Sei ne la terra negra;\\
Né il sol piú ti rallegra\\
Né ti risveglia amor.
\end{verse}

\vspace*{1em}\hrule\vspace*{1em}

\begin{enumerate}
\item si contrappone a quel \emph{rinverdì}
  di prima, in quanto indica negazione.
\end{enumerate}

Questo componimento presenta la tematica centrale della poesia carducciana, l'opposizione luce/ombra, vita/morte. Le due polarità in opposizione sono nettamente ripartite tra le prime due strofe e le ultime due. Nelle prime due dominano immagini di luce e di calore, con intense note coloristiche («verde», «vermigli», «rinverdì»), e rendono il senso della vitalità prorompente della natura primaverile. A questi motivi, nelle ultime due si contrappone il motivo dell'aridità, del freddo, del buio, dell'assenza di gioia vitale e d'amore. La serie delle opposizioni si può così ricostruire sulla base della trama delle parole chiave:
\begin{itemize}
\item rinvesì \textit{vs} inaridita
\item luce \textit{vs} terra negra
\item calor \textit{vs} terra fredda
\item amore \textit{vs} inutil vita
\end{itemize}

Ma già nella prima parte, pur dominata dalla solarità, è presente una nota cupa, che anticipa il clima della seconda parte: il «muto orto solingo» (v. 5). È un'immagine di morte: il giardino è muto perché non risuona più dei giochi del bambino. L'io lirico si protende disperatamente, ma vanamente, verso immagini di solare vitalità, per scacciare l'immagine della morte che l'ossessiona. È una tematica strettamente affine a quella del componimento \textsc{San Martino}.

\section{T: \textsc{Alla stazione una mattina di autunno}}

Questa poesia è interessante perché in essa si trovano alcuni elementi
che saranno poi tipici della poesia decadente, e che di fatto mostrano
un ripiego umano nel poeta Carducci: nascono temi cupi, del novembre,
che proiettano Carducci con i temi che seguiranno.

Si crede che sia stata composta in due momenti diversi, a distanza di
anni, perché c'è una parte della poesia che potrebbe essere stata
composta prima. C'è una scena insolita, che si rappresenta in una
stazione; è un luogo estremamente poco poetico. È presente il treno, che
è l'emblema della modernità; qualche anno prima Carducci aveva scritto
\textsc{L'inno a Satana}, in cui Satana era simboleggiato dal treno: era
quindi un inno alla modernità, che fa pensare che Carducci accolga la
modernità, ma vi è sempre un po' il timore del nuovo (rappresentato
dalla scelta di Satana, ovvero di una figura diabolica). Nel '76, quando
viene composta questa poesia, ecco che il treno osannato nell'\textsc{Inno
a Satana} diventa un mostro, perché porta via la donna amata.

La poesia è dedicata ad una donna, sua amante.

Lo scenario è infernale: abbiamo delle notazioni che ci ricordano molto
l'inferno dantesco: è buio perché è mattina presto e c'è la nebbia;
inoltre vi sono bagliori improvvisi; il tutto dona una ambientazione
piuttosto fosca.

\begin{verse}
\poemlines{5}
Oh quei \textit{fanali come s’inseguono}\textsuperscript{1}\\
\textit{accidïosi}\textsuperscript{2} là dietro gli alberi,\\
tra i rami stillanti di pioggia\\
\textit{sbadigliando}\textsuperscript{3} la luce su ’l fango!\\!
\textit{Flebile, acuta, stridula}\textsuperscript{4} fischia\\
la vaporiera da presso. Plumbeo\\
il cielo e il mattino d’autunno\\
come un grande fantasma n’è intorno.\\!
Dove e a che move questa, che affrettasi\\
a’ carri foschi, ravvolta e tacita\\
gente? a che ignoti dolori\\
o tormenti di speme lontana?\textsuperscript{5}\\!
Tu pur pensosa, Lidia, la tessera\\
al secco taglio dài de la guardia,\\
e al tempo incalzante i begli anni\\
dài, gl’istanti gioiti e i ricordi.\textsuperscript{6}\\!
Van lungo il nero convoglio e vengono\\
incappucciati di nero i \textit{vigili},\textsuperscript{7}\\
com’ombre; una fioca lanterna\\
hanno, e mazze di ferro: ed i ferrei\\!
freni \textit{tentati}\textsuperscript{8} rendono un \textit{lugubre}\\
\textit{rintócco lungo}\textsuperscript{9}: di fondo a l’anima\\
un’eco di tedio risponde\\
doloroso, che spasimo pare.\textsuperscript{10}\\!
E gli sportelli sbattuti al chiudere\\
paion \textit{oltraggi}\textsuperscript{11}: scherno par l’ultimo\\
appello che rapido suona:\\
\textit{grossa scroscia su’ vetri la pioggia.}\textsuperscript{12}\\!
Già il \textit{mostro},\textsuperscript{13} conscio di sua metallica\\
anima, sbuffa, crolla, ansa, \textit{i fiammei}\\
\textit{occhi sbarra}\textsuperscript{14}; immane pe ’l buio\\
gitta il fischio che sfida lo spazio.\\!
Va l’empio mostro; con traino orribile\\
sbattendo l’\textit{ale}\textsuperscript{15} gli amor miei portasi.\\
Ahi, l\textit{a bianca faccia e ’l bel velo}\textsuperscript{16}\\
salutando scompar ne la tenebra.\textsuperscript{17}\\!
\end{verse}

Si immmagina che la seconda parte sia stata scritta in un altro momento: probabilmente nel 1875; è un ricordo.

\begin{verse}
\poemlines{5}
\setverselinenums{36}{40}
O viso dolce di pallor roseo,\\
o stellanti occhi di pace, o candida\\
tra’ floridi ricci inchinata\\
pura fronte con atto soave!\textsuperscript{18}\\!
Fremea la vita nel tepid’ aere,\textsuperscript{19}\\
fremea l’estate quando mi arrisero:\\
e il giovine sole di giugno\\
si piacea di baciar luminoso\\!
in tra i riflessi del crin castanei\\
la molle guancia: come un’aureola\\
piú belli del sole i miei sogni\\
ricingean la persona gentile.\textsuperscript{20}\\!
Sotto la pioggia, tra la caligine\textsuperscript{21}\\
torno ora, e ad esse vorrei confondermi;\\
barcollo com’ebro, e mi tócco,\\
non anch’io fossi dunque un fantasma.\textsuperscript{22}\\!
Oh qual \textit{caduta di foglie},\textsuperscript{23} gelida,\\
continua, muta, greve, su l’anima!\textsuperscript{24}\\
io credo che solo, che eterno,\\
che per tutto \textit{nel mondo è novembre}.\textsuperscript{25}\\!
Meglio a chi ’l senso smarrí de l’essere,\textsuperscript{26}\\
meglio quest’ombra, questa \textit{caligine}\textsuperscript{27}:\\
io voglio io voglio adagiarmi\\
in un tedio che duri infinito.\\
\end{verse}

\finepoesia

\begin{enumerate}

\item I fanali che si inseguonoa sono i lampioni
\item assolutamente emblematico, in quanto
  la noia caratterizzerà i poeti delle generazioni successive
\item significativo in quanto in
  generale è un verbo intransitivo, utilizzato con forma transitiva
  senza un oggetto interno: capiamo il tutto solo per mezzo della scena;
  altro indice di noia.
\item è una climax, con suoni quasi onomatopeici Prima parlava
  della nebbia, e ora abbiamo l'autunno che ci da l'idea di un grigio
  opaco tutto intorno.
\item Abbiamo una inversione: infatti il soggetto è
  \emph{gente}; vi è inoltre una iperbato, ovvero la separazione di due
  termini che dovrebbero stare vicini
\item Particolari prosaici. Già la situazione e il luogo non sono tra i più poetici; una grandissima distanza separa questa poesia da una poesia petrarchesca come “Chiare fresche et dolci acque”: questo è tutt’altro che un locus amenus. Anche i gesti sono prosaici: la donna è rappresentata mentre sporge la tessera al controllore.
L’immagine prosaica diventa immagine del tempo incalzante, che porta via dei pezzi: come il controllore si porta via una parte del biglietto, così il tempo strappa via i begli anni.
\item soggetto postposto
\item latinismo: picchiati
\item ogni gesto, azione e suono si riflette o evoca un sentimento dell'animo
\item Altre immagini piuttosto prosaiche: tutte figure che ruotano attorno all’ambiente della stazione, che servono a sottolineare lo squallore della vita moderna; i vigili sono tutti incappucciati di nero, e la loro immagine lugubre serve a completare quell’atmosfera infernale: non è un locus amenus.
\item oltraggio al sentimento del poeta
\item allitterazione della S e scontri tra vocali dure
\item treno: si personifica e ci permette di capire quanta importanza avesse questo soggetto
\item Caronte occhi di bragia: diventa palese il riferimento alll'inferno Dantesco
\item gli stantuffi del treno
\item immagine simbolica: sono state abbandonate le immagini prosaiche; sono un passaggio alle strofe successive
\item il linguaggio prosegue la metafora del cando precedente
\item classica immagine della \textbf{donna angelo}
\item è cambiata la stagione, ora è estate
\item il poeta coglie l'immagine di una aureola
\item nebbia
\item Con questa strofa ritorniamo all’atmosfera dei primi versi. Il novembre che all’inizio era soltanto una condizione atmosferica, adesso diventa una condizione dell’animo.
\item emblema della caduta delle emozioni
\item l'immagine è di un mondo triste e privo di valori positivi
\item l'atmosfera novembrina può essere estesa all'animo del poeta: la nebbia è il tedio. Secondo alcuni, Carducci maschera un disagio dietro alla sua poesia.
\item è più fortunato colui per cui l'esistenza ha perso di significato: tipico del movimento sucecessivo di poeti
\item simbolo del tedio
\end{enumerate}

\paragraph{Realtà e simbolo nella rappresentazione del paesaggio urbano} Torna in questa poesia la struttura che abbiamo visto ricorrente nella lirica carducciana: in apertura un dato reale e presente, da cui prende avvio la "fantasia" (in questo caso il ricor do), poi il ritorno al presente. Qui però l'indugio sul presente ha di gran lunga maggior estensione che la rievocazione del passato. Colpisce innanzitutto il paesaggio d'apertura, contrassegnato da pioggia, fango, cielo plumbeo, con le metafore che lo accompagnano: i fanali si inseguono «accidiosi» e «sbadigliando» la luce, il mattino d'autunno è come «un grande fantasma»: è un paesaggio urbano di gusto inconfondibilmente baudelairiano.
Ma colpisce soprattutto il luogo dove la poesia è ambientata: una stazione ferroviaria, luogo per nulla “poetico", secondo il gusto tradizionale. Non solo, ma il discorso insiste su tutta una serie di particolari quotidiani banali e prosaici (il biglietto forato dal controllore, i frenatori che percuotono i freni con sbarre di ferro, il rumore degli sportelli sbattuti, l'ultimo invito a salire in carrozza). La poesia del secondo Ottocento ama indugiare su particolari realistici di questo tipo. Ma ciò in Carducci è inconsueto: il poeta predilige usualmente un discorso aulico e sublime. L'indugio sui particolari prosaico possiede quindi una funzione particolare: sottolineare lo squallore avvilente della vita moderna, che nega ogni aspirazione alla bellezza. La stazione ferroviaria, nella cultura del tempo, è infatti il luogo emblematico della modernità, rappresenta gli scambi, il dinamismo, la vita attiva, gli interessi positivi. Essa ha però qui come due facce: la prima e più evidente è quella realistica e quotidiana, che si è sottolineata; ma su di essa si sovrappone un'altra dimensione. Si noti il ricorrere di colori foschi (il nero in particolare) che è troppo insistito per non essere voluto e denso di significati: «carri foschi», «nero convoglio», vigili incappucciati di «nero», «com'ombre», «buio», «tènebra». Il nero è colore altamente simbolico, richiama essenzialmente la morte. Questa stazione ferroviaria è infatti una sorta di regno dei morti: le figure che in essa si muovono silenziose, oppresse da «ignoti dolori» e «tormenti», richiamano le ombre dei dannati che si aggirano negl'Inferi; tutto è avvolto come da un'atmosfera spettrale, lugubre, su cui aleggia un senso di pena e sofferenza. Non manca neppure, in questo scenario infernale, la figura di Satana: il treno, un «empio mostro contrassegnato da un tratto tipico delle figure demoniache, i «fiammei / occhi».
\paragraph{Il treno e la realtà moderna} Poiché il treno, nella cultura dell'Ottocento, è il simbolo per eccellenza della modernità, tutto ciò rivela nel poeta paura e orrore per la vita moderna. Per comprenderne il motivo, occorre richiamare brevemente il contesto storico-culturale. La macchina è fatta segno, nella cultura di questo periodo, di una forte ambivalenza: è esaltata entusiasticamente dalla retorica corrente, come indizio del progresso inarrestabile, del trionfo della scienza sulle tenebre dell'ignoranza, ma appare anche come un'immagine mostruosa, inquietante. La macchina fa paura perché distrugge dalle radici tutto un mondo e un sistema di vita, capovolge i valori tradizionali, è un mostro che minaccia di sfuggire al controllo dell'uomo e di rivolgersi maleficamente contro il suo creatore (si ricordi il \textsc{Frankenstein} della Shelley, dove si esprimono, nell'Inghilterra della Rivoluzione industriale, questi segreti terrori). La paura della macchina è propria soprattutto degli artisti: essi sentono, magari senza averne chiara coscienza, che la macchina, cioè la nuova civiltà industriale, distrugge la loro posizione sociale tradizionale, li declassa a produttori di merci per il mercato )\textsc{La strada ferrata} di Emilio Praga). In Carducci l'ambivalenza è particolarmente evidente. Già dodici anni prima, nell'\textsc{Inno a Satana}, aveva affrontato il tema del treno. Si trattava di un inno al progresso e all'avanzata della scienza, perciò il treno era presentato in chiave positiva. Ma si ricordi come veniva designato: «Un bello e orribile / mostro». La contraddittorietà dei due aggettivi impiegati denunciava in modo trasparente l'ambivalenza. Carducci, democratico e progressista, si proponeva come cantore della modernità, della scienza, della macchina, ma il simbolo del progresso suscitava in lui un oscuro timore. Qui, in \textsc{Alla stazione}, prevale decisamente il polo negativo. Il «mostro» non è più «bello», ma «empio»: la vita moderna, rappresentata dalla stazione e dal treno, suscita ormai nel poeta solo angoscia e «tedio», perché è bruttezza irredimibile, \textit{spleen}, negazione della gioia e dell'amore.

\paragraph{Il sogno e lo \textit{spleen} come reazione alla modernità} A questa realtà negativa si contrappone la "fantasia", il sogno del poeta, che, per vincere lo spleen e l'orrore, richiama un'immagine lieta del passato, quella della donna amata. E puntualmente ricorre la costellazione di temi che Carducci ama evocare per esorcizzare le forze oscure che lo angosciano, la luce, il calore, l'estate, la gioia vitale, la bellezza, l'amo re. Il tutto assume le inevitabili vesti classiche: il sole che bacia la molle guancia di Lidia ha le sembianze apollinee di un dio greco (Apollo era appunto il dio del sole). Dopo la parentesi del "sogno", il poeta torna al presente. Si riaffaccia il paesaggio iniziale, contrassegnato dalla pioggia, dalla «caligine», dal freddo, e diviene quasi esplicitamente l'oggettivazione simbolica di uno spleen di tipo baudelairiano. Il poeta è preso come da un'ebbrezza di annientamento: anch'egli si sente come gli altri un'ombra che vaga negl'Inferi della città moderna («non anch'io fossi dunque un fantasma», 52). Nel "classico" Carducci, il cantore di tutte le forze sane e vitali, dell'operosità umana e del progresso, si annida, come si è potuto constatare più vol te, la tipica "malattia" romantica. Ma l'importante è che qui, accanto al motivo della pena esistenziale, compare anche in piena evidenza quella che ne è la causa profonda, l'angoscia della realtà moderna.

% sono arrivato qui (6 mar 2021 - 23:25)

\chapter{Naturalismo francese} 

\begin{itemize}
\item
  \pagine{p. 98-101}
\end{itemize}

Noi facciamo questa rapida carrellata sugli autori del naturalismo, e in
particolare su Emile Zola, perché, oltre al fatto che è un momento
importante della letteratura francese, anche perché la stessa è il
nostro termine di confonto. Ad esempio, è importante conoscere Zola per
comprendere e confrontare Verga.

In Francia il naturalismo nasce tra gli anni '70 e '80. La produzione di
Zola è al massimo livello. Sono gli anni in cui iniziano a evidenziarsi
gli effetti, anche negativi, della rivoluzione industriale; quella
rivoluzione che ha creato delle classi sociali, come il
\textbf{proletariato urbano}, la \textbf{classe degli operai}. Questo è
l'ambiente privileggiato dagli autori del naturalismo: un ambiente
\emph{cittadino} e \emph{degradato}; proprio qui si differenzia dal
\textbf{verismo italiano}: la fotografia di Zola è quella di un ambiente
cittadino, mentre Verga, ad esempio, prediligerà gli ambienti rurali e
contadini.

Questi autori ebbero come punto di riferimento \textbf{Hippolyte Taine}:
egli è un teorico del naturalismo; egli con il suo \emph{determinismo}
pone le basi del naturalismo francese. Ci sono tre principi di base: 1.
razza 2. ambiente 3. contesto storico.

Partendo dal presupposto che \textbf{tutta} la realtà è analizzabile con
metodo scientifico, questi tre elementi sono fondamentali a determinare
qualsiasi cosa che riguardi l'individuo: le sue scelte, le sue pulsioni;
da questo nasce la scrittura naturalista. Taine propone degli esempi e
dei modelli letterari: - Balzac - Flaubert

\textsc{Madame Bovary}, opera di Flaubert, è fondamentale, specie per
quanto riguarda la posizione dell'autore e del narratore all'interno
della narrazione. Tutti i romanzi \textbf{precedenti} presentano un
narratore esplicito, che interviene, e che si pone da un punto di vista
\emph{superiore} rispetto ai personaggi. Il narratore è il portavoce
dell'autore. Flaubert invece introduce la \textbf{teoria
dell'impersonalità} \textgreater{} L'artista deve essere nella sua opera
come Dio nella creazione, invisibile e onnipotente, sì che lo si senta
ovunque {[}\ldots{]} (aggiungere da p.~98)

Flaubert pone l'obiettivo di un approccio di tipo scientifico nel
romanzo. L'immagine dell'artista da adito ad un nuovo tipo di narratore.
Vi è una scomparsa del narratore alla Manzoni. Nella letteratura
naturalista il narratore subirà delle modifiche fondamentali, e ciò
influenzerà moltissimo Verga.

L'\textbf{approccio scientifico}, già testimoniato da Flaubert, si
ritrova nei Fratelli \textbf{de Goncourt}: la loro opera \textsc{Germinie
Lacerteux} è importante perché preceduta da una \textsc{Prefazione}, di
fatto un manifesto della nuova tendenza. Con i loro romanzi iniziamo a
vedere la descrizione di \textbf{personaggi brutti}, malati: vi è il
coraggio di guardare una realtà degradata e malata con \textbf{occhio
scientifico}. Tuttavia vi è un piccolo scarto tra gli aspetti teorici e
il romanzo vero e proprio: la produzione immaginata basandosi sulla
teoria degli autori è diversa da quella effettiva.

\section{T: \textsc{Un manifesto del Naturalismo}}

Questa è la \emph{Prefazione} al \textsc{Germinie Lacerteux} dei fratelli
de Gouncourt.

\begin{quote}
Dobbiamo chiedere scusa al pubblico per questo libro che gli offriamo e avvertirlo di quanto vi troverà. Il pubblico ama i romanzi falsi: questo romanzo è un romanzo vero. Ama i romanzi che dànno l'illusione di essere introdotti nel gran mondo: questo libro viene dalla strada.

Ama le operette maliziose, le memorie di fanciulle, le confessioni d'alcova', le sudicerie erotiche, lo scandalo racchiuso in un'illustrazione nelle vetrine di librai: il libro che sta per leggere è severo e puro. Che il pubblico non si aspetti la fotografia licenziosa del Piacere: lo studio che segue è la clinica dell'Amore? 

Il pubblico apprezza ancora le letture anodine e consolanti, le avventure che finiscono bene, le fantasie che non sconvolgono la sua digestione né la sua serenità: questo libro, con la sua triste e violenta novità, è fatto per contrariare le abitudini del pubblico, per nuocere alla sua igiene.

Perché mai dunque l'abbiamo scritto? Proprio solo per offendere il lettore e scandalizzare i suoi gusti? 

No.

Vivendo nel diciannovesimo secolo, in un'epoca di suffragio universale, di democrazia, di liberalismo, ci siamo chiesti se le cosiddette «classi inferiori» non abbiano diritto al Romanzo; se questo mondo sotto un mondo, il popolo, debba restare sotto il peso del «vietato» letterario e del disdegno degli autori che sino ad ora non hanno mai parlato dell'anima e del cuore che il popolo può avere. Ci siamo chiesti se possano ancora esistere, per lo scrittore e per il lettore, in questi anni d'uguaglianza che viviamo, classi indegne, infelicità troppo terrene, drammi troppo mal recitati, catastrofi d'un terrore troppo poco nobile. Ci ha presi la curiosità di sapere se questa forma convenzionale di una letteratura dimenticata e di una società scomparsa, la Tragedia, sia definitivamen te morta; se, in un paese senza caste e senza aristocrazia legale, le miserie degli umili e dei poveri possano parlare all'interesse, all'emozione, alla pietà, tanto quanto le mise rie dei grandi e dei ricchi; se, in una parola, le lacrime che si piangono in basso possano far piangere come quelle che si piangono in alto.
\end{quote}

Questa prefazione è importante per capire quali sono gli intenti, al
punto che viene quasi vista come \textbf{manifesto}

\paragraph{La poetica del naturalismo} Da questo manifesto emergono alcuni punti essenziali della poetica del Naturalismo: 
\begin{itemize}
\item il rifiuto della narrativa di consumo, convenzionale, di evasione, ed il perseguimento di finalità serie, la ricostruzione della «Storia morale contemporanea»; 
\item il proposito di non curarsi, in questa ricostruzione, dei gusti del pubblico, anzi, di andare provocatoriamente contro le sue abitudini più consolidate; 
\item l'acquisizione alla letteratura di una nuova zona del reale, esclusa dalla narrativa tradizio nale: le classi inferiori, le loro miserie ei loro drammi, che vengono trattati in chiave seria; 
\item l'attribuzione alla letteratura del rigore metodologico e dei fini della scienza, come prescriveva la contemporanea mentalità positivistica; viene dato per scontato che la forma per eccellenza di questa nuova letteratura è il romanzo;
\item l'intento dello studio sociale, dell'analisi di miserie della società, in nome di una visione umanitaria.
\end{itemize}

\paragraph{L'attrazione morbosa per il patologico} È necessario però distinguere queste enunciazioni teoriche dalla realtà effettiva delle opere dei Goncourt. Ciò che spingeva i due scrittori a rappresentare il popolo era soprattutto la ricerca del nuovo e del raro, di sapori inediti e forti, propria di un gusto ormai sazio ed annoiato. Lo stesso Edmond annotava nel Diario il 3 dicembre 1871: «Il popolo, la cana glia, se volete, ha per me l'attrazione delle popolazioni sconosciute e non ancora esplorate, qualche cosa di quell'esotico che ricercano i viaggiatori».

In particolare in \textsc{Germinie Lacerteu} ciò che muove i due scrittori è un'attrazione sensuale, morbosa, per il brutto, il repellente, il patologico, qualche cosa che non ha nulla a che vedere con l'obiettività scientifica e documentaria, ma rivela semmai quelle tendenze de cadenti di cui si diceva. E non vi è veramente la volontà di inserire in una rappresentazione letteraria il proletariato operaio, la classe nuova che si affacciava alla ribalta sociale nella nuova organizzazione industriale: il romanzo tratta di una serva, cioè ancora di un'appendice della nobiltà e della borghesia (su questi aspetti, si veda il saggio di Erich Auer- bach, \textsc{Mimesis, Il realismo nella letteratura occidentale} del 1956).

% sono arrivato qui: 7 mar 2021 - 12:41

\chapter{Emile Zola}

\begin{itemize}
\item
  \pagine{p. 116-117}
\end{itemize}

È il principale autore del naturalismo francese. E' un contemporaneo di
Verga. Nasce da padre italiano e madre francese, è un giornalista e si
dedica alla letteratura: è teorico e interprete del naturalismo. Scrive
un ciclo di 20 romanzi, \textsc{Il ciclo dei Rougon-Macquart}.

Due concetti stanno alla base della teoria di Zola: 
\begin{enumerate}
\item eredità biologica; 
\item influsso dell'ambiente.
\end{enumerate}

Zola si allinea con Taine, pensando alla realtà con occhio scientifico.
Egli va oltre, dicendo che la letteratura è uno strumento, con la
funzione sociale e politica, di cui devono servirsi gli scrittori e i
politici, per studiare la verità, la realtà, l'analisi dell'individuo e
delle sue malattie, dello stato in cui vive. Lui vuole descrivere tutta
la realtà circostante, ma è chiaro che si sofferma molto sulla realtà
degradata parigina. I politici possono guardare queste opere come ad un
tentativo di denuncia, da cui si possono cogliere dei suggerimenti per
eventuali miglioramenti.

In questo Zola è diverso da Verga: pur mostrando una realtà negativa,
brutta, vera, il suo scopo è \textbf{positivo}: egli crede che ci sia la
possibilità di migliorare la società; invece il pessimismo di Verga è
\textbf{assoluto}.

Per quanto riguarda il suo impegno sociale, egli come giornalista aveva
dato origine ad un circolo letterario, in cui si svolgevano le
\emph{serate di Medain}, da cui nasce una raccolta di novelle, che
costituisce una sorta di poetica del naturalismo. In questo ambiente si
ritrovavano diversi artisti che condividevano gli ideali di Zola.

Inoltre vi fu l'\textbf{affare Dreyfus}: un generale ebreo accusato
ingiustamente di spionaggio: fu un caso eclatante di antisemitismo; in
Francia vi furono discussioni accesi, e lo stesso Zola scrisse un
articolo, \textsc{J'accuse}, che gli costò un anno di prigione (tramutato
in esilio).

Lo scrittore morì nel 1902, asfissiato dalle esalazioni di una stufa; le
circostanze della morte restano poco chiare, e si sospetta un attentato
per vendetta, in conseguenza delle posizioni assunte da Zola nell'affare
Dreyfus.

I punti fondamentali del programma proposto da Zola, che si concretizza
nel \textbf{Romanzo sperimentale}: - il romanzo deve essere uno
strumento scientifico che si deve basare sul metodo sperimentale delle
scienze, ovvero sulla teorizzazione a partire dall'osservazione diretta
e concreta. - l'uomo deve impadronirsi di questi meccanismi psicologici
per poi dirigerli; - si occupa in particolar modo di \emph{tare
ereditarie}, come ad esempio l'alcolismo: è presente ne
\textsc{L'Assommoir} (riassunto \pagine{p. 117}); partendo dalla
traduzione di questo romanzo in Italia parte la riflessione sulla
letteratura verista.

\paragraph{Riassunto \textsc{L'Assommoir}} Il romanzo, pubblicato nel 1877, è ambientato nella Parigi operaia e narra una storia di alcolismo, di miseria e di degradazione umana. È anche un esperimento stilistico, poiché Zola vuol riprodurre il caratteristico gergo dell'am biente proletario. Come afferma nella Prefazione, lo scrittore intende «colare in uno stampo molto elaborato la lin gua del popolo». Il titolo deriva dal nome dato in gergo alla bettola dove si beve acquavite. \textit{Assommoir} significa propriamente "mattatoio": la bettola è così chiamata perché l'acquavite porta rapidamente all'abbrutimento e alla morte gli operai che contraggono il vizio del bere.

Gervaise, venuta a Parigi giovanissima dalla provincia me ridionale con l'amante Lantier, è da questi abbandonata con due figli piccoli e vive stentatamente facendo la lavandaia. Conosce Coupeau, un operaio lattoniere onesto e laborioso, e lo sposa. La famiglia prospera, sinché Coupeau cade dal tetto dove lavora ad una grondaia. Dopo l'incidente, trascura il lavoro e si dà al bere; la famiglia sopravvive grazie al duro lavoro di Gervaise, che ha aperto una lavanderia. Ritorna Lantier, e riallaccia la relazione con Gervaise, mentre Coupeau si degrada sempre più. La figlia Anna (la futura protagonista del romanzo Nana) comincia a corrompersi nell'ambiente sordido dei sobborghi proletari. Anche Gervaise cade preda dell'alcolismo e muore in conseguenza di esso, dopo aver sperimentato la miseria più atroce e l'abbrutimento totale.
 
\chapter{Letteratura russa}

\section{Delitto e castigo}

\begin{itemize}
\item
  \pagine{p. 132: trama}
\item
  \pagine{p. 133: T7 - I labirinti della coscienza: la confessione di Raskolnikov}
\end{itemize}

\paragraph{Trama} Il protagonista, Raskolnikov, un giovane provinciale povero, studente a Pietroburgo, uccide una vecchia usuraia e la sorella. La motivazione del delitto è nelle idee del giovane, che ritiene che vi siano uomini superiori che hanno il diritto di violare le leggi morali, arrivando sino al delitto, se le grandi idee che essi aspirano a realizzare lo esigono. Raskolnikov uccide appunto per provare a se stesso di essere uno di questi uomini superiori; ma è assalito da un rimorso violento e dall'ossessione di essere scoperto. L'ossessione lo spinge a sfidare la polizia stessa. Il giudice Porfirij ha intuito la sua colpevolezza, ma indugia ad arrestarlo perché ha letto nel suo animo e sa che presto o tardi egli stesso si costituirà.

Si aggiungono all'omicidio altri motivi d'angoscia: la sorel la Dunija, insidiata da un essere perverso, Svidrigailov, giunge a Pietroburgo a chiedere aiuto e decide di sposare Lužin, uomo ricco ma sordido e meschino. Raskolnikov viene anche a contatto con altri relitti umani: l'ubriacone Marmeladov, la moglie Katerina Ivanovna, isterica e semi folle, la figlia Sònja, che si prostituisce per sfamare i suoi. Proprio Sònja, creatura dolce e pura, raccoglie da Raskolnikov la confessione del delitto e lo spinge a costituirsi. Ma nell'intimo il giovane è ribelle ad un autentico penti mento. Solo in Siberia, dove Sònja l'ha seguito, si profila, appena accennata al termine del romanzo, la sua autentica redenzione.

\subsection{T: \textsc{I labirinti della coscienza: la confessione di Raskolnikov}}

\begin{quote}
«E com'è stato in realtà?» disse, come dopo profonda riflessione «è stato proprio così! Ecco:
volevo diventare un Napoleone, è per questo che ho ucciso... Su, puoi capire adesso?».
«No, no» sussurrò ingenuamente e timidamente Sonja «ma... parla, parla! Io capirò, dentro di me
capirò tutto!» lo supplicò.

Egli tacque e rifletté a lungo.

«Capirai? Bene, vedremo. Si tratta di questo: una volta mi ero proposto un quesito: se, per
esempio, al mio posto si fosse trovato Napoleone e non avesse avuto, per cominciare la sua
carriera, né Tolone, né l'Egitto, né il passaggio del Monte Bianco, ma, invece di tutte queste
belle e monumentali imprese, gli si fosse trovata dinanzi nient'altro che una spregevole
vecchierella, vedova di un impiegato del registro, che per giunta si dovesse uccidere per rubarle
i denari nel baule (per far carriera, capisci?), ebbene, si sarebbe egli deciso a farlo, non avendo
altra via di uscita? Non si sarebbe inalberato al pensiero di un'azione così poco monumentale
e... e delittuosa? Ebbene, io ti dico che con un simile «quesito», mi torturai per lunghissimo
tempo, tanto che mi prese una gran vergogna quando alla fine intuii (d'un tratto) che non
soltanto egli non si sarebbe inalberato, ma non gli sarebbe neppure venuta in mente l'idea che
la cosa non fosse monumentale... e anzi non avrebbe capito affatto che motivo ci fosse lì di
inalberarsi. E purché non avesse avuto altra strada, e poi l'avrebbe soffocata senza lasciarle dire
né ahi né bai, e senza pensarci più che tanto! Ebbene anch'io... sono uscito dalle mie
meditazioni... e l'ho soffocata... seguendo l'autorevole esempio... Ed è stato così punto per
punto! Ti viene da ridere? Ma qui, la cosa più buffa è che forse è stato proprio così...».

Sonja non aveva nessuna voglia di ridere.

«Parlatemi piuttosto chiaramente... senza esempi» ella pregò ancora più timida e con voce
appena udibile.

Egli si voltò verso di lei, la guardò con tristezza e la prese per le mani.

«Hai di nuovo ragione tu, Sonja. Tutte queste sono scempiaggini, è quasi una vuota cicalata!
Vedi: tu sai pure che mia madre non possiede quasi nulla. Mia sorella ha ricevuto per caso
un'educazione ed è condannata ad andare di qua e di là come istitutrice. Tutte le loro speranze
non erano riposte che in me. Io ho studiato, ma non potevo mantenermi all'università e sono
stato costretto a lasciarla per un certo tempo. Ma anche se si fosse andati avanti a quel modo, tra una decina, una quindicina d'anni (sempre che le cose si fossero messe bene), avrei potuto
sperare di diventare insegnante o impiegato, con mille rubli di stipendio...». Pareva che dicesse
cose imparate a memoria. «E intanto mia madre si sarebbe rinsecchita dai crucci e dagli affanni,
senza che tuttavia mi riuscisse di darle la tranquillità, e mia sorella... be’, a mia sorella sarebbe
potuto capitare anche di peggio!... E che gusto, per la vita intera, passare dinanzi a tutto e
rinunciare a tutto, dimenticarsi della madre e sopportare umilmente, per esempio, la vergogna
di una sorella! E perché? Forse soltanto per metter su, dopo averle sotterrate, una nuova
famiglia, moglie e figli, e lasciar poi anche loro senza un soldo e senza un boccon di pane?
Ebbene... ebbene, ecco io decisi che, dopo essermi impadronito dei denari della vecchia, li avrei
impiegati, nei primi anni, per mantenermi all'università, senza tormentare mia madre, e per i
primi passi da fare dopo l'università; e avrei fatto tutto questo con larghezza, radicalmente, in
modo da prepararmi tutta una nuova carriera e mettermi su di una strada nuova,
indipendente... Ebbene, ebbene, ecco tutto... Già, si capisce, quanto a uccidere la vecchia, in
questo ho fatto male... e adesso basta!».

Come spossato si trascinò sino alla fine del racconto e chinò il capo.

«Oh, non è quello, non è quello...» esclamò Sonja angosciata «e forse che si può così... no, non
è così, non è così».

«Lo vedi anche tu che non è così... Eppure ti ho fatto un racconto sincero; è la verità!».

«Ma che verità è mai questa! O Signore!».

«Io non ho ucciso che un pidocchio, Sonja, inutile, schifoso, nocivo».

«Ma è una creatura umana quel pidocchio!».

«Ma sì, lo so anch'io che non è un pidocchio» egli rispose guardandola stranamente. «Però io
dico degli spropositi, Sonja» aggiunse «è già un pezzo che ne dico... Tutto questo è un'altra
cosa; tu dici giusto. Qui ci sono altre cause, ben diverse!... Era già tanto che non parlavo con
nessuno, Sonja. Adesso ho un gran mal di capo».ù

I suoi occhi ardevano di un fuoco febbrile. Cominciava quasi a delirare; un sorriso inquieto
errava sulle sue labbra. Attraverso l'eccitazione del suo spirito faceva capolino una tremenda
spossatezza Sonja capì quanto egli si straziasse. Anche a lei cominciava a girar la testa. Egli
parlava in un modo così strano: le pareva di capire qualcosa, ma... «ma come mai! Come mai! O
Signore». Ed ella si torceva le mani disperata.

«No, Sonja, non è quello» egli rispose, sollevando d'un tratto il capo, come se un improvviso
nuovo giro di pensieri lo avesse colpito e di nuovo eccitato «non è quello! Ma piuttosto... supponi (sì! così infatti è meglio!) supponi che io sia egoista, invidioso, malvagio, abietto,
vendicativo, e... magari anche incline alla pazzia. (Tutto questo insieme! Della pazzia si parlava
già prima, me n'ero accorto!) Dunque ti ho detto poc'anzi che non potevo mantenermi
all'università. Ma sai tu che forse lo potevo anche? Mia madre mi avrebbe mandato di che
pagare quel che occorreva, e quanto alle scarpe, ai vestiti e al pane, avrei provveduto col mio
lavoro, di sicuro! Lezioni se ne presentavano; mi si offriva mezzo rublo per ciascuna. Lavora
pure Razumíchin! Ma io m'incattivii e non volli. Per l'appunto mi incattivii (ecco una bella
parola!). Allora, come un ragno, mi ficcai nel mio cantuccio. Tu sei stata nel mi canile, hai
veduto... E sai, Sonja, che i soffitti bassi e le camere strette opprimono l'anima e l'intelligenza?
Oh, quanto odiavo quel canile! E tuttavia non ne volevo uscire. Apposta non lo volevo! Per
interi giorni non ne uscivo e non volevo lavorare, e non volevo neppur mangiare, stavo sempre
disteso. Se Nastàsja me ne portava, mangiavo; se non me ne portava, la giornata passava cosi;
apposta, per rabbia, non ne chiedevo! Di notte non avevo lume, stavo coricato al buio, non
volevo lavorare per comprarmi delle candele! Bisognava studiare e io avevo venduto tutti i
libri; e sulla mia tavola, sugli appunti e sui quaderni, c'è anche adesso un dito di polvere.
Preferivo stare sdraiato e pensare. E pensavo sempre. E facevo sempre certi sogni, una
quantità di sogni strani, non è il caso di dir quali! Solo che allora cominciò anche a sembrarmi
che... No, non è così! Di nuovo non racconto bene! Vedi, allora mi domandavo sempre: perché
son così stupido da non voler essere più intelligente degli altri, se quelli sono sciocchi, e se io
so con certezza che lo sono? Poi ho capito, Sonja, che, a voler attendere che tutti fossero
diventati intelligenti, sarebbe stato troppo lungo... Poi ho capito ancora che questo non sarebbe
stato mai, che gli uomini non cambieranno e che nessuno li può trasformare, e che non val la
pena di sprecar fatica! Sì, è così! È la loro legge... Una legge, Sonja! È così!... E ora io so, Sonja,
che chi è vigoroso e forte di mente e di spirito, quello è il loro dominatore! Chi molto oserà,
avrà ragione di loro. Chi è capace di disprezzare più cose, quello è il legislatore, e chi più di tutti
è capace di osare, quello ha più ragione di tutti! Cosi è andato finora e cosi sarà sempre!
Soltanto un cieco non lo vedrebbe!»

Raskolnikov, dicendo questo, benché guardasse Sonja, non si preoccupava più s'ella capisse o
no. Una febbre l'aveva preso. Egli era in preda a una specie di cupo entusiasmo. (Era
veramente troppo tempo che non parlava con nessuno!) Sonja capì che quel tetro catechismo
era diventato la sua fede e la sua legge.

«Io indovinai allora, Sonja» egli seguitò con fervore «che la potenza si dà solo a chi osa chinarsi
a prenderla. Qui non ci vuole che una cosa, una sola: basta osare! Mi venne allora, per la prima
volta in vita mia, un pensiero che nessuno mai aveva avuto prima di me! Nessuno! D'un tratto
mi si presentò chiaro come il sole questo pensiero: come mai neppure uno finora aveva osato
né osava, passando dinanzi a tutta questa assurdità, prendere il tutto puramente e
semplicemente per la coda e scaraventarlo al diavolo! Io... io ho voluto osare, e ho ucciso... ho
voluto soltanto osare, Sonja, ecco l'unica causa!».

«Oh, tacete, tacete!» gridò Sonja, giungendo le mani. «Vi siete allontanato da Dio, e Dio vi ha
colpito, vi ha abbandonato al demonio!...».

«A proposito, Sonja, quando io stavo coricato al buio e mi venivano tutti quei pensieri, era il
demonio che mi tentava? Eh?».

«Tacete! Non ridete, bestemmiatore, nulla, nulla voi comprendete. O Signore! Nulla, nulla egli
comprenderà!».

«Taci, Sonja, io non rido affatto; lo so anch’io che era il demonio a trascinarmi. Taci, Sonja,
taci!» ripeté con cupa insistenza «Io so tutto. Tutto questo me lo sono già ruminato e ripetuto
da me, quando stavo disteso nell’oscurità... Tutto questo l' ho vagliato con me stesso, sino
all'ultima minuzia, e so tutto, tutto! E tutte queste ciance mi avevano allora tanto, tanto
annoiato! Volevo dimenticar tutto e cominciare daccapo, Sonja, e smetterla di cianciare. E tu
pensi forse che io ci sia andato come uno scemo, a rotta di collo? Ci sono andato da persona di
senno, ed è stato questo a perdermi. E tu pensi forse che io non sapessi, per esempio, almeno
questo, che, avendo cominciato a interrogarmi e a domandarmi: ho io il diritto di possedere la
potenza? ciò voleva dire che io non avevo il diritto di possederla? Oppure che, se mi ponevo il
quesito: è un pidocchio quella persona? ciò voleva dire che quella persona non era già un
pidocchio per me, ma era un pidocchio per quello a cui questo pensiero non fosse nemmeno
venuto in mente e che fosse andato difilato, senza domandarsi nulla... Se per tanti giorni mi son
tormentato a pensare se Napoleone ci sarebbe andato o no, è che sentivo già chiaramente di
non essere un Napoleone... Tutta, tutta la tortura di quelle lunghe ciance io sopportai, Sonja, e
mi venne il desiderio di sbarazzarmene di colpo: io volli, Sonja, uccidere senza tante casistiche,
uccidere per me, per me solo! Non volevo mentire a quel riguardo neppure a me stesso! Non
per aiutare mia madre ho ucciso, sciocchezze! Non ho ucciso per farmi, acquistata ricchezza e
potenza, il benefattore dell'umanità. Sciocchezze! Ho ucciso semplicemente; per me stesso ho
ucciso, per me solo, e che poi avrei beneficato qualcuno, o per la vita intera, come un ragno, avrei acchiappato tutti quanti nella mia ragnatela e a tutti avrei succhiato il sangue, questo a me,
in quel momento, doveva essere indifferente!... E non il denaro, soprattutto, mi occorreva,
Sonja, quando ho ucciso; non tanto il denaro quanto un'altra cosa... Tutto questo ora lo so...
Comprendimi: forse, pur andando per quella medesima strada, non avrei mai più commesso un
assassinio. Altro avevo bisogno di sapere, altro mi spingeva: avevo allora bisogno di sapere, e
di sapere al più presto, se io fossi un pidocchio, come tutti, o un uomo. Avrei potuto passar
oltre o non avrei potuto? Avrei osato chinarmi e prendere, o no? Ero una creatura tremante o
avevo il diritto...».

«Di uccidere? Se avete il diritto di uccidere?» e Sonja giunse le mani.

«E-eh, Sonja!» egli gridò irritato, e voleva già replicare, ma tacque sprezzantemente. «Non
interrompermi, Sonja! Volevo soltanto dimostrarti una cosa: che allora fu il diavolo a
trascinarmi, ma poi mi spiegò che io non avevo il diritto di andar là, perché anch'io ero un
pidocchio così come tutti! Si fece beffe di me, ed ecco che ora son venuto qui! Accogli il tuo
ospite! Se non fossi un pidocchio, sarei venuto da te? Ascolta: quando andai dalla vecchia, vi
andai soltanto per provare... Sappilo dunque!».

«E avete ucciso! Avete ucciso!».

«Ma come ho ucciso? Forse è così che si uccide? Forse è così che si va ad uccidere, come ci
sono andato io?... Ti racconterò un giorno o l'altro come ci sono andato... Ho forse ucciso la
vecchia? Me stesso ho ucciso, e non la vecchia! Mi sono bravamente accoppato da me, per
sempre!... E quella vecchietta l' ha uccisa il diavolo, e non io... Basta, basta, Sonja, basta!
Lasciami» esclamò a un tratto, in una spasmodica angoscia «lasciami!».
\end{quote}

\paragraph{Le giustificazioni del delitto} Nel suo lungo monologo, appena inframmezzato dalle scarne battute di Sonja, Raskolnikov tenta di ricostruire i processi interiori che lo hanno portato al delitto: ma è confuso, si con traddice, si corregge continuamente, non riuscendo a spiegarsi come vorrebbe, anzi a chia rire a se stesso i propri impulsi. Il giovane studente è come in preda a un «cupo entusia smo», il suo discorso appare convulso e febbrile, quasi un delirio, per cui risulta incoerente e oscillante fra diverse spiegazioni.

Prima Raskolnikov, per giustificare il suo gesto, si appiglia al modello di Napoleone: il grande uomo, se la vecchia fosse stata un ostacolo alla sua ascesa, non avrebbe esitato dinanzi al delitto. Ma poi rinuncia a questa giustificazione, ricorrendo a motivazioni più meschine e materiali, la necessità di pagarsi gli studi, di uscire dallo squallore della sua vita misera, preparandosi a una carriera. Il suo giudizio sul delitto muta nel giro di poche parole: prima egli riconosce di aver fatto male a uccidere la vecchia, ma subito dopo, in modo contraddittorio, ribadisce la liceità del suo gesto, adducendo come motivazione il fatto che la vittima era «un pidocchio», un essere «inutile, schifoso, nocivo», per poi di
nuovo smentirsi alle rimostranze sdegnate di Sonja. 

Ritorna allora ad analizzare le condizioni che l'hanno portato al delitto, il suo degradarsi nella miseria, il suo «incattivirsi», che lo aveva fatto precipitare in una sorta di cupa depressione e lo aveva indotto a chiudersi in un feroce isolamento nella propria topaia. È da questa profonda frustrazione che sorgono le sue concezioni superomistiche (già peraltro intuibili dal riferimento al modello napoleonico): il disprezzo per l'umanità comune e l'esaltazione di colui che è capace di «osare», quindi di diventarne il «dominatore» e il «legislatore». Così l'intellettuale, relegato a una vita miserabile, tenta di riscattare la sua rea le condizione attraverso un delirio di onnipotenza, trasfigurando uno squallido gesto delittuoso in un'azione eroica, che pone chi la compie al di sopra degli uomini meschini.

\paragraph{La presa di coscienza} Ai deliri dell'intellettuale frustrato si contrappone, in una netta antitesi, la fede semplice dalla creatura ingenua e pura, Sonja, che mette in crisi le mistificazioni dell'interlocutore. Allora il giovane è indotto a confessare i propri dubbi tormentosi, da cui trapela la consapevolezza di non essere un «Napoleone», quell'essere superiore e dominatore con cui si compiaceva di identificarsi. Si rende conto che la ragione vera che l'ha spinto al delitto era la necessità di tentare una prova, cioè di sapere se era un «pidocchio» come tutti, oppure «un uomo»: e ha scoperto appunto di essere «un pidocchio» come tutti gli altri. Per cui in realtà non ha ucciso la vecchia, ma se stesso. Quindi non c'è in lui, per ora, un vero pentimento per ciò che ha fatto, solo un senso di sconfitta delle sue smisurate illusioni e un disprezzo di sé. E cerca di scacciare il senso di colpa accusando il diavolo di averlo trascinato al delitto, per cui sostiene che è stato il diavolo a uccidere la vecchia, non lui. 

Questa lacerazione devastante della coscienza è una condizione ricorrente negli eroi dostoievskiani, e si traduce spesso in forme narrative come questa, monologhi dei personaggi, confessioni in cui essi scavano nei recessi più segreti e più torbidi della loro psiche, nonrisparmiando nulla a se stessi, per cercare di portare alla luce e districare il groviglio degli impulsi più vergognosi (il «sottosuolo» a cui allude uno dei romanzi più famosi dello scrittore, \textsc{Memorie del sottosuolo}).

\chapter{Verismo}

\begin{itemize}
\item
  \pagine{p. 153-156: Gli scrittori italiani nell'età del verismo}
\end{itemize}

Vi è uno stretto collegamento con il naturalismo francese

Le riflessioni in Italia iniziano dall'\textsc{Assomoir}. Gli italiani
sono molto attratti dai francesi.

Il naturalismo giunge prevalentemente nell'Italia settentrionale, che
era l'area geografica più aperta alle innovazioni francesi, ed ecco che
l'eco del successo della letteratura naturalista arriva in un ambiente
di sinistra, in cui la produzione viene accolta bene. C'è un interesse
per questo tipo di letteratura, ma non vi è la capacità di creare una
vera scuola; resta una infatuazione che però non ottiene nulla.

Saranno due autori siciliani, di ambiente non esattamente di sinistra,
ad accogliere l'influenza naturalista e sviluppare una vera e propria
letteratura: Capuana e Verga

\evidenziatore{Capuana} fu il vero e proprio teorico del Verismo. Il suo
apporto più importante è in termini teorici. Il suo lavoro inizia come
una presentazione in Italia de \textsc{L'Assommoir}.

La produzione di \evidenziatore{Verga} si divide in due parti. Con la
sua seconda produzione egli darà poi origine a quella che sarà la
\textbf{scuola verista italiana}, nonostante sia lui il massimo
esponente. Verga fa una rivoluzione per quanto riguarda le tecniche
espressive, e questo crea un modello che sarà variamente ripreso da
altri autori.

Capuana fa delle riflessioni importanti nella prefazione ai
\textsc{Malavoglia} di Verga, da cui si estrapolano diverse informazioni
sulla sua concezione del Verismo.

Egli si occupa principalmente di \textbf{giornalismo} e fa una
riflessione sul Corriere della Sera, sull'\textsc{Assomoir}.

Sarà poi \evidenziatore{Verga} a dare origine alla scuola verista, anche
se fu solo lui l'unico personaggio di spicco. - rivoluziona le tecniche
espressive: narrazione e focalizzazione - sarà modello

Nell'articolo di presentazione di un romanzo spiega il suo punto di
vista, commentando il \textbf{naturalismo francese} - i francesi hanno
approcci troppo formali - egli non vuole intendere l'opera letteraria
come utile per l'insegnamento e come funzionale

\begin{quote}
Senza dubbio l'elemento scientifico s'infiltra nel romanzo contemporaneo
{[}\ldots{]}; ma la vera novità non istà in questo. Né stà nella pretesa
di un \emph{romanzo sperimentale}, bandiera che lo Zola inalbera
arditamente, a sonori colpi di grancassa {[}\ldots{]}. Un'opera d'arte
non può assimilarsi un concetto scientifico che alla propria maniera,
secondo la sua natura d'opera d'arte. Se il romanzo non dovesse far
altro che della fisiologia o della patologia, o della psciologia
comparata in azione, {[}\ldots{]} il guadagno non sarebbe né grande né
bello. Il positivismo, il naturalismo esercitano una vera e radicale
influenza nel romanzo contemporaneo, ma \emph{soltanto nella forma}, e
tal influenza si traduce nella \emph{perfetta impersonalità} di questa
opera d'arte.
\end{quote}

\chapter{Giovanni Verga}

Autore di una rivoluzione nel modo di narrare, contemporaneo a Zola.

\section{Vita}

\begin{itemize}

\item
  \pagine{p. 186-189}: Vita, romanzi preveristi e svolta verista
\end{itemize}

Nasce nel 1840 a Catania, negli stessi anni di Zola, da una famiglia
nobile e benestante. Compie studi privati, in cui riveste una grande
importanza il suo abate.

Inizia l'università, per poi abbandonarla in favore della letteratura:
si appassiona della letterautra francese.

Verga sente la necessità di sprovincializzarsi, così abbandona la
Sicilia e si reca a Firenze, dove compone \textsc{Storia di una capinera}.

Nel \textbf{1878} ha una svolta verista, che avviene con la
pubblicazione di \textsc{Rosso Malpelo}. Avviene a Milano, dove in un
primo momento ha frequentato l'ambiente della scapigliatura. La sua
produzione sarà affine ai gusti del movimento, con ambienti
aristocratici, romanticismo, Femme Fatale.

A Milano: 
\begin{itemize}
\item \textbf{1878}: \textsc{Rosso Malpelo}, una novella
completamente verista
\item \textbf{1883}: \textsc{I Malavoglia}
\item \textbf{1889}: \textsc{Mastro Don Gesualdo}: stesso anno del \textsc{Piacere} di D'Annunzio.
\end{itemize}

Abbandona la letteratura: 
\begin{itemize}
\item \textbf{1903}: ritorna definitivamente in
Sicilia; da questo anno in avanti avrà un periodo buio, con problemi
economico finanziari: abbandona la letteratura. 
\item \textbf{1922}: muore; questo è l'anno della marcia su Roma e dell'avvento del Fascismo.
\end{itemize}

Il verismo ha dato una grande svolta al \textbf{narratore}. È d'obbligo
il confronto con Manzoni: in Manzoni il narratore è onnisciente, e
presenta i personaggi umili; per far parlare i suoi personaggi, egli
utilizza il discorso diretto, oppure l'indiretto libero.

\subsection{Passione per la fotografia}

Negli ultimi anni della sua vita, Verga scopre una passione per la
fotografia, che lo porta a fare moltissime foto. Questo ha una relazione
con la sua tecnica del verismo, in quanto è la realtà immortalata così
com'è.

\section{Darwinismo sociale}

Verga rappresenta la lotta per la vita; egli presenta un pessimismo
assoluto: a qualsiasi livello, per la lotta per la vita i più deboli
soccombono.

Verga pensa che il \pagine{progresso} lascia i suoi cadaveri, distrugge
i più deboli, nonostante da lontano sembri positivo.

Nelle classi sociali più alte la lotta è mimetizzata.

Verga non riesce a chiudere il ciclo dei vinti.

\section{Poetica e tematiche}

\subsection{Regressione del narratore}

\begin{itemize}

\item
  \pagine{p. 190-191}: Poetica e tecnica narrativa del Verga verista
\end{itemize}

Rappresenta l'\textbf{eclissi dell'autore}. L'autore non si serve più
del narratore per parlare, quindi bisogna effettuare un cambiamento al
narratore stesso: dal punto di vista ideologico e culturale è diverso
dall'autore.

Verga non si identifica mai con il narratore.

Non abbiamo un'unica focalizzazione, e non è sempre quella del
protagonista.

In Verga il narratore si confonde con il popolo e con i personaggi. Il
punto di vista dell'autore si capisce da alcuni aggettivi giudicanti.

L'opera sembra fatta da sé. Questo porta al \textbf{disorientamento del
lettore}

\subsection{T: \textsc{Impersonalità e regressione}}

Farina è un amico letterario e un critico: Verga gli racconta l'idea di
opera fatta da sé.

Si presenta il tema del \evidenziatore{\textbf{narratore inattendibile}} . Il
narratore, intervenendo con dei commenti, si identifica con la mentalità
dei personaggi, dando spesso interpretazioni scorrette.

Verga non utilizza il dialetto, ma fa molto uso di proverbi e modi di
dire, utilizzando espressioni dialettali.

Inoltre, nella struttura sintattica, viene evidenziato l'uso del
\textbf{che} polivalente.

Nella descrizione delle classi sociali elevate, il narratore è diverso.

\begin{quote}
Caro Farina, eccoti non un racconto, ma l'abbozzo di un racconto. Esso almeno avrà il merito di essere brevissimo, e di esser storico - un documento umano, come dicono oggi - interessante forse per te, e per tutti coloro che studiano nel gran libro del cuore. Io te lo ripeterò così come l'ho raccolto pei viottoli dei campi, press'a poco colle medesime parole semplici e pittoresche della narrazione popolare, e tu veramente preferirai di trovarti faccia a faccia col fatto nudo e schietto, senza stare a cercarlo fra le linee del libro, attraverso la lente dello scrittore. Il semplice fatto umano farà pensare sempre; avrà sempre l'efficacia dell'essere stato, delle lagrime vere, delle febbri e delle sensazioni che sono passate per la carne. Il misterioso processo per cui le passioni si annodano, si intrecciano, maturano, si svolgono nel loro cammino sotterraneo, nei loro andirivieni che spesso sembrano contradditori, costituirà per lungo tempo ancora la possente attrattiva di quel fenomeno psicologico che forma l'argomento di un racconto, e che l'analisi moderna si studia di seguire con scrupolo scientifico. Di questo che ti narro oggi, ti dirò soltanto il punto di partenza e quello d'arrivo; e per te basterà, - e un giorno forse basterà per tutti.

  Noi rifacciamo il processo artistico al quale dobbiamo tanti monumenti gloriosi, con metodo diverso, più minuzioso e più intimo. Sacrifichiamo volentieri l'effetto della catastrofe, allo sviluppo logico, necessario delle passioni e dei fatti verso la catastrofe resa meno impreveduta, meno drammatica forse, ma non meno fatale. Siamo più modesti, se non più umili; ma la dimostrazione di cotesto legame oscuro tra cause ed effetti non sarà certo meno utile all'arte dell'avvenire. Si arriverà mai a tal perfezionamento nello studio delle passioni, che diventerà inutile il proseguire in cotesto studio dell'uomo interiore? La scienza del cuore umano, che sarà il frutto della nuova arte, svilupperà talmente e così generalmente tutte le virtù dell'immaginazione, che nell'avvenire i soli romanzi che si scriveranno saranno i fatti diversi?
  
  Quando nel romanzo l'affinità e la coesione di ogni sua parte sarà così completa, che il processo della creazione rimarrà un mistero, come lo svolgersi delle passioni umane, e l'armonia delle sue forme sarà così perfetta, la sincerità della sua realtà così evidente, il suo modo e la sua ragione di essere così necessarie, che la mano dell'artista rimarrà assolutamente invisibile, allora avrà l'impronta dell'avvenimento reale, l'opera d'arte sembrerà essersi fatta da sé, aver maturato ed esser sòrta spontanea, come un fatto naturale, senza serbare alcun punto di contatto col suo autore, alcuna macchia del peccato d'origine.
\end{quote}

\paragraph{La consapevolezza teorica di Verga} È, con la \textsc{Prefazione ai Vinti}, l'unico testo teorico che Verga abbia pubblicato; tutto il resto della sua riflessione sulla scrittura letteraria è affidato a lettere private, non destinate alla stampa, o è ricostruibile da interviste giornalistiche. Ciò testimonia il pudo re e il rigore di Verga: egli voleva parlare attraverso le sue realizzazioni artistiche concrete, senza sbandierare sulla scena letteraria teorie più o meno suggestive o provocatorie. Per questo è durato a lungo il pregiudizio di un Verga “debole di idee", sprovveduto dal punto
di vista teorico. Questo testo lo smentisce, dimostrando come Verga avesse ben chiari problemi teorici che erano alla base del suo lavoro.

\paragraph{I punti chiave della poetica Verghiana} Si possono desumere da questa pagina alcuni punti essenziali della poetica di Verga.
\begin{itemize}
\item L'impersonalità: essa viene intesa come «eclisse» dell'autore, che deve sparire dal nar rato, non deve filtrare i fatti attraverso la sua «lente», ma deve mettere il lettore «faccia a faccia» con il fatto «nudo e schietto». Il lettore deve seguire lo sviluppo di certe passioni come se non fossero raccontate, ma si svolgessero di fronte a lui, drammaticamente. L'opera, pertanto, deve sembrare «essersi fatta da sé».
\item In relazione all'impersonalità e all'«eclisse» dell'autore, si delinea anche la teoria della "regressione" del punto di vista narrativo entro il mondo rappresentato: i fatti saranno riferiti «colle medesime parole semplici e pittoresche della narrazione popolare». Deve cioè scomparire il narratore tradizionale, portavoce dell'autore, e deve essere sostituito da un'anonima voce narrante che ha la visione del mondo e il modo di esprimersi dei personaggi stessi.
\item L'«eclisse» dell'autore porta con sé un processo di scarnificazione del racconto, di riduzione all'essenziale. Vengono eliminate le minute analisi psicologiche della narrativa romantica. Il processo delle passioni è ricostruito solo da pochi punti indispensabili (il «punto di partenza» e «d'arrivo»). In un'altra lettera, a Felice Cameroni, Verga chiarirà che la psicologia si deve ricavare non dai profili dei personaggi costruiti dal narratore, ma dai loro semplici comportamenti, dai gesti e dalle parole, persino «dal modo di soffiarsi il naso».
\item Di qui deriva il rifiuto di una facile drammaticità, degli effetti romanzeschi plateali, «il pepe della scena drammatica», come Verga lo definisce in una lettera a Capuana.
\item Agli effetti romanzeschi si sostituisce una ricostruzione scientifica dei processi psicologici, fondata su una rigorosa consequenzialità logica e su rapporti necessari di causa e di effetto. Questa fiducia nelle leggi di causa ed effetto che regolano la vita interiore è un tratto tipico della narrativa di impianto naturalistico. È una concezione che rimanda alla mentalità positivistica, alla convinzione che tutta la realtà umana, sociale e psicologica, sia retta da leggi ferree e precise, che la ragione può ricostruire e dominare. È un principio che sarà poi messo in crisi dal romanzo novecentesco, soprattutto da Pirandello e da Svevo, che insisteranno sull’incoerenza e la discontinuità della psiche, sui suoi processi oscuri, ambi gui e indecifrabili.
\end{itemize}

\subsection{T: \textsc{``Sanità'' rusticana e ``malattia'' cittadina}}

\begin{quote}
La prima edizione della tua \textsc{Giacinta} andrà in sei mesi, e al più tardi dentro l'anno, senti quel che ti dico; e allora sarai in condizione di aver meglio retribuito questo lavo ro, e di ottenere altre condizioni per quel che scriverai. Anch'io faccio assegnamento su \textit{Padron 'Ntoni}, e avrei voluto, se la \textit{disgrazia}\textsuperscript{1} non mi avesse perseguitato sì accanita mente e spietatamente darvi \emph{quell'impronta di fresco e sereno raccoglimento che avrebbe dovuto fare un immenso contrasto con le passioni turbinose e incessanti delle gran di città}\textsuperscript{2}, con quei bisogni fittizii, e quell'altra prospettiva delle idee o direi anche dei sentimenti. Perciò avrei desiderato andarmi a rintanare in campagna, sulla riva del mare, fra quei pescatori e coglierli vivi come Dio li ha fatti. Ma forse non sarà male dall'altro canto che io li consideri da una certa distanza in mezzo all'attività di una città come Milano o Firenze. Non ti pare che per noi l'aspetto di certe cose non ha risalto che visto sotto un dato angolo visuale? e che mai riusciremo ad essere tanto schiettamente ed efficacemente veri che allorquando facciamo un lavoro di ricostruzione intel lettuale e sostituiamo la nostra mente ai nostri occhi?
\end{quote}

\finepoesia

\begin{enumerate}
\item fa riferimento alla morte
della madre
\item Si è discusso a lungo la posizione di Verga
sulla campagna: fin da Virgilio era stata un \textit{locus amenus}, ma si
ha l'impressione che Verga non consideri positiva neanche quella; questa
espressione può far pensare che egli abbia una immagine positiva, ma in
realtà non è questione di luogo o meno: le dinamiche negative Verga le
vede anche in ambienti distanti dalla città. Rimangono comunque alcune
reminescenze romantiche, per cui la campagna rimane un luogo pacifico.
\end{enumerate}

Verga ci spiega come la rappresentazione della realtà non debba essere
per forza mimetica, ma può essere anche una ricostruzione intellettuale.

\paragraph{Una visione ancora romantica della campagna} La lettera è un documento importante della visione di Verga e della sua poetica. Vi si coglie innanzitutto l'opposizione campagna/città: il mondo popolare e rurale è concepito come un mondo di «fresco e sereno raccoglimento», antidoto alle «passioni turbinose e incessanti» e ai «bisogni fittizii» della vita cittadina e borghese. L'opposizione si specifica dunque come serenità vs turbamento, autenticità vs inautenticità. Ciò testimonia la presenza, ancora in una fase avanzata della stesura del romanzo, di una visione romantica del mondo della campagna, come nella novella \textsc{Fantasticheria}.

\paragraph{Il filtro intellettuale e la posizione straniata} Ma con questa visione romantica e mitizzante contrasta il proposito affermato subito dopo: Verga non vuole un'immersione nostalgica in quel mondo (che pure sarebbe la sua prima tentazione), ma una rappresentazione a distanza, attraverso un filtro intellettuale. È questo che preserva Verga, nel romanzo, dall'idillio campagnolo e dalla mitizzazione idealizzante del mondo rurale. Ma lo preserva anche da una forma di “verismo" come riproduzione puramente mimetica, “fotografica", che annullerebbe ogni distacco critico dall'oggetto. Verga vuole dare una «ricostruzione intellettuale», a distanza, della vita popolare siciliana, per mantenersi in una posizione straniata, criticamente vigile. È un principio fondamentale per capire la rappresentazione verghiana della realtà.

\subsection{T: \textsc{L'«eclissi» dell'autore e la regressione nel mondo rappresentato}}

I personaggi si fanno conoscere con le azioni.

Verga in questo periodo ('78) si interessa dei documentari e dei saggi
giornalistici che mettevano in luce la \textbf{questione meridionale},
specie per l'aspetto delle condizioni sociali.

Il testo \textsc{Inchiesta in Sicilia} mette in luce la mafia, la violenza
e il lavoro minorile (estratto \pagine{p. 224} )

\textsc{Lettere meridionali} sono di Vivari, uno storico del meridione.
Sono considerate il manifesto del movimento \textbf{meridionalista}:
sono articoli che l'autore invia ad un amico che lavorava per un
giornale; mettono per la prima volta in luce il problema. Sono accolte
con molta criticità a causa dei toni esagerati.

\subsubsection{Passo A}

\begin{quote}
Avevo un bel dirmi che quella semplicità di linee, quell'uniformità di toni, quella certa fusione dell'insieme che doveva servirmi a dare nel risultato l'effetto più vigoroso che potessi, quella tal cura di smussare gli angoli, di dissimulare quasi il dramma sotto gli avvenimenti più umani, erano tutte cose che avevo volute e cercate apposta e non erano certo fatte per destare l'interesse ad ogni pagina del racconto, ma l'interesse doveva risultare dall'insieme, a libro chiuso, quando tutti quei personaggi si fossero affermati sì schiettamente da riapparirvi come persone conosciute, ciascuno nella sua azione. Che la confusione che dovevano produrvi in mente alle prime pagine tutti quei personaggi messivi faccia a faccia senza nessuna presentazione, come li aveste conosciuti sempre, e foste nato e vissuto in mezzo a loro, doveva scomparire mano mano col pro gredire nella lettura, a misura che essi vi tornavano davanti, e vi si affermavano con nuove azioni ma senza messa in scena, semplicemente, naturalmente, era artificio voluto e cercato anch'esso, per evitare, perdonami il bisticcio, ogni artificio letterario, per darvi l'illusione completa della realtà. Tutte buone ragioni, o scuse di chi non si sente sicuro del fatto suo; e sai che l'inferno è lastricato di buone intenzioni. Capirai dunque com'ero inquieto non solo sul valore che avrebbe accordato il pubblico a queste inten zioni artistiche, giacché le intenzioni non valgono nulla, ma sul risultato che avrei saputo cavarne nell'ottenere dal lettore l'impressione che volevo.
\end{quote}

\subsubsection{Passo B}

\begin{quote}
Caro Pessimista, con me tu non sei tale, anzi temo che la tua benevolenza non ti faccia essere assolutamente il contrario.

Ho letto il giudizio che dai nel \textsc{Sole} dei miei \textsc{Malavoglia} e mi ha fatto un gran piacere il vedere quello che tu pensi del mio libro, e l'essere riuscito in parte ad incarnare il mio concetto agli occhi di un critico fine e imparziale come te. So anch'io che il mio lavoro non avrà un successo di lettura, e lo sapevo quando mi son messo a disegnare le mie figure col proposito artistico che tu approvi. Il mio solo merito sta forse nell'avere avuto il coraggio e la coscienza di rinunziare ad un successo più generale e più facile, per non tradire quella forma che sembrami assolutamente necessaria'. [...] Io mi son messo in pieno, e fin dal principio, in mezzo ai miei personaggi e ci ho condotto il lettore, come ei li avesse tutti conosciuti diggià, e più vissuto con loro e in quell'ambien te sempre. Parmi questo il modo migliore per darci completa l'illusione della realtà; ecco perché ho evitato studiatamente quella specie di profilo che tu mi suggerivi pei personaggi principali. Certamente non mi dissimulavo che una certa confusione non dovesse farsi nella mente del lettore alle prime pagine; però man mano che i miei attori si fossero affermati colla loro azione essi avrebbero acquistato maggior rilievo, si sarebbero fatti conoscere più intimamente e senza artificio, come persone vive, il libro tutto ci avrebbe guadagnato nell'impronta di cosa avvenuta. Ecco la mia ambizione e il peccato che mi rimproveri. D'esserci riuscito non mi lusingo, ma lasciami pensare ancora che il concetto è perfettamente coerente ai nostri criteri artistici, e non mi dire che sono più realista del re.
\end{quote}

\subsubsection{Passo D}

\begin{quote}
Devo a Lei il più bello ed importante articolo critico che sia stato scritto sui \textsc{Malavoglia}. Io non avrei potuto augurarmi encomio maggiore di quello che Ella mi fa dicendo co testo romanzo perfettamente obbiettivo ed impersonale. Sì, il mio ideale artistico è che l'autore s'immedesimi talmente nell'opera d'arte da scomparire in essa. Vorrei quasi che un romanzo arrivasse a non portare il nome del suo autore, si affermasse da sé, come vivente per un organismo proprio e necessario, producesse quell'illusione potente dell'essere stato, che hanno le epopee dei rapsodi e tutte le figure schiette della poesia popolare. E in questa obbiettività efficacissima della rappresentazione artistica, Zola istesso, così grande e possente, ha ancora della debolezza pel gusto colorista della nuova scuola letteraria francese, per la sua mirabile abilità di descrizioni [...]. A me è parso che la descrizione nei \textsc{Malavoglia} doveva essere tanto più sobria, quanto meno è il sen timento della natura in quegli uomini primitivi, e del resto la più rigorosa efficacia parmi stia sempre nella sobrietà. Quegli uomini io ho cercato di riprodurli nella loro genuina originalità mettendomi completamente nel loro ambiente, il più che ho potuto, rendendoli tali quali senza farli passare per nessuna preoccupazione artistica. Sono lietissimo di vedere che negli occhi di Lei ci sono riuscito, almeno in gran parte, e che Ella mi dia ragione in cotesto primo tentativo, che in Italia può passare per disperato, di farli parlare con la loro lingua inintellegibile a gran parte degli Italiani, almeno di dare la fisionomia del loro intelletto alla lingua che essi parlano. Certuni mi addebitano di non aver separato in questo metodo la parte dello scrittore da quella dei suoi personaggi; e se arrivano a concedermi venia per l'ardimento in questo, avrebbero voluto che per la diversa intonazione dello stile lo scrittore avesse fatto sentire ogni venti linee: ora son io che parlo. La questione [...] si riannoda a quel che ho detto in principio, e parmi che non possa sussistere un momento l'illusione della completa immedesima zione col soggetto senza dare un'uniforme intonazione a tutta l'opera, senza eclissare completamente lo scrittore.
\end{quote}

\subsubsection{Passo E}

\begin{quote}
Se dovessi fare a voi, amico, e non pel pubblico le mie confessioni letterarie, direi soltan to questo: – che ho cercato di mettermi nella pelle dei miei personaggi, vedere le cose coi loro occhi ed esprimerle colle loro parole – ecco tutto. Questo ho cercato di fare nei Malavoglia e questo cerco di fare nella \textsc{Duchessa} in altro tono, con altri colori, in diverso ambiente. E qui cade in acconcio quel che disse Goncourt che le scene e le persone del popolo sono più facili a ritrarsi, perché più caratteristici e semplici - quanto complicati e tutti esprimentisi per sottintesi sono le classi più elevate, massime se si deve tener conto di quella specie di maschera e di sordina“ che l'educazione impone alla manifestazione degli stessi sentimenti, e alla vernice quasi uniforme che gli usi, la moda, il linguaggio quasi uniforme nella stessa società tendono a rendere pressoché internazionale in una data società. E massime nel mio metodo - che Dio m'assista per questa  \textsc{Duchessa}!
\end{quote}

\paragraph{Eclisse dell'autore e regressione} I cinque passi contengono la descrizione più chiara dei procedimenti in cui, secondo Verga, doveva tradursi il principio dell'impersonalità: soprattutto l'«eclisse» dello scrittore e la regressione del punto di vista narrativo entro la realtà rappresentata. L'obiettivo primario di Verga è eliminare ogni senso di artificiosità letteraria, dare «completa l'illusione della realtà», l'impressione di assistere direttamente ai fatti, senza alcun intermediario. Per questo elimina la presenza dell'«autore», più propriamente quella voce del narratore onnisciente che interviene costantemente a descrivere gli ambienti, a delineare profili dei personaggi, a spiegare psicologie, a commentare e a giudicare. Il punto di osservazione si trasferisce all'interno dell'ambiente rappresentato, il narratore assume la mentalità e il linguaggio che sono propri dei personaggi («ho cercato di mettermi nella pelle dei miei personaggi, vedere le cose coi loro occhi ed esprimerle colle loro parole»).

\paragraph{La narrazione si avvicina al dramma} Ciò dà origine ad un impianto narrativo molto nuovo, persino sconcertante: il lettore si trova di fronte ad una folla di personaggi, senza che essi gli vengano presentati e descritti, e deve imparare a conoscerli dall'azione stessa, man mano che essa progredisce. E li conosce attraverso le loro parole e i loro gesti, anche quelli apparentemente più insignificanti («da dieci parole e dal modo di soffiarsi il naso»). È un procedimento che avvicina la narrativa al dramma. Nel teatro realistico ottocentesco, infatti, al levarsi del sipario lo spettatore si trova di fronte a dei personaggi che parlano e agiscono, senza saper nulla di loro e degli antefatti, e impara a conoscerli solo attraverso l'azione stessa. Questo avvicinarsi del romanzo al dramma è un tratto caratteristico della narrativa del secondo Ottocento.

\section{T: \textsc{Rosso Malpelo}}

Scritta alla fine degli anni '70, quando il dibattito sulla condizione
della Sicilia è piuttosto acceso; cominciano ad uscire i primi testi che
denunciano la situazione: questi testi Verga li conosceva molto bene.

Il \textbf{narratore inattendibile} è presente in questa novella. Il
narratore fa parte, dal punto di vista degli ideali, della morale e del
linguaggio, dello stesso ambiente dei personaggi che compaiono nella
novella o nel romanzo. Il narratore si trova allo stesso livello dei
personaggi raccontati, ma può non condividere determinate idee e
determinati stili di vita. Il narratore è inattendibile, e giudica Rosso
Malpelo con lo spirito critico caratteristico degli altri lavoranti.

Verga non ha la speranza che da qualche parte ci siano sentimenti
migliori rispetto che da qualche altra parte; l'umiltà non è un pregio,
e Verga ritiene che la lotta per la vita, con tutte le dinamiche
collegate ad essa, si possa verificare e si possa constatare in
qualsiasi classe sociale e in qualsiasi ambiente. Ciò non toglie che nei
vari ambienti vi siano alcuni personaggi che rispecchiano dei valori
positivi; generalmente questi personaggi non sono compresi, soprattutto
dal \textbf{narratore}, che quindi diventa inattendibile.

L'autore non giudica Rosso Malpelo, ma certamente ci da gli strumenti
per capire le dinamiche del suo comportamento e della sua psicologia, e
in qualche modo ci fa capire che il narratore ci dice delle cose
scorrette: non capisce, per la sua grettezza, la psicologia di Rosso
Malpelo, così come non lo capiscono i compaesani e la sua stessa
famiglia.

Rosso Malpelo è un soprannome: il vero nome del bambino non viene fatto
per tutta la novella. Il narratore ci dice che nemmeno la madre si
ricordava come si chiamasse il figlio. Questo è molto significativo: il
nome di un individuo è ciò che determina la sua identità. Il fatto che
questo fanciullo non abbia un nome, significa che per la società non è
niente e nessuno.

Verga non usa toni commossi e non giudica, differentemente da quanto succede nell'\textsc{Inchiesta in Sicilia}

\begin{quote}
 Malpelo\mat{vuol dire ``cattivo''} si chiamava così perchè aveva i capelli rossi; ed \textit{aveva i capelli rossi perchè era un ragazzo malizioso e cattivo}\mat{viene fornito come dato oggettivo}, che prometteva di riescire un fior di \textit{birbone}\mat{malvagio}. Sicchè tutti alla cava della rena rossa lo chiamavano Malpelo; e persino sua madre, col sentirgli dir sempre a quel modo, aveva quasi dimenticato il suo nome di battesimo.
\end{quote}

Secondo la visione delle persone ignoranti, il narratore crede ad alcune superstizioni (narratore inattendibile): questo prova una sorta di \textbf{straniamento} alientante. Il lettore è stimolato a produrre un giudizio, in quanto il narratore da una spiegazione diversa e ``sbagliata'' a fatti oggettivi.
 
\begin{quote} 
 Del resto, ella lo vedeva soltanto il sabato sera, quando tornava a casa con quei pochi soldi della settimana; e \textit{siccome era malpelo}\mat{conseguenza logica basata sul nulla} c’era anche a temere che ne sottraesse un paio, di quei soldi: nel dubbio, per non sbagliare, la sorella maggiore gli faceva la ricevuta a scapaccioni.
 
 Però il padrone della cava aveva confermato che i soldi erano tanti e non più; e in coscienza erano anche troppi per Malpelo, un monellaccio che nessuno avrebbe voluto vederselo davanti, e che tutti schivavano come un can rognoso, e lo accarezzavano coi piedi, allorchè se lo trovavano a tiro.
 
 Egli era davvero un brutto ceffo, torvo, ringhioso, e selvatico. Al mezzogiorno, mentre tutti gli altri operai della cava si mangiavano in crocchio la loro minestra, e facevano un po’ di ricreazione, egli andava a rincantucciarsi col suo corbello fra le gambe, per rosicchiarsi quel po’ di pane bigio, come fanno le bestie sue pari, e ciascuno gli diceva la sua, motteggiandolo, e gli tiravan dei sassi, finchè il soprastante lo rimandava al lavoro con una pedata. Ei c’ingrassava, fra i calci, e si lasciava caricare meglio dell’asino grigio, senza osar di lagnarsi. Era sempre cencioso e sporco di rena rossa, chè la sua sorella s’era fatta sposa, e aveva altro pel capo che pensare a ripulirlo la domenica. Nondimeno ora conosciuto come la bettonica per tutto Monserrato e la Carvana, tanto che la cava dove lavorava la chiamavano ``la cava di Malpelo'', e cotesto al padrone gli seccava assai. Insomma lo tenevano addirittura per carità, e perchè mastro Misciu, suo padre, \colorbox{green}{era morto} in quella stessa cava.
 \end{quote}
 
 Rosso Malpelo non mette in discussione l'ordine, non si lamenta mai. Da un certo punto, Rosso prende coscienza della situazione e sviluppa una sua filosofia di vita. Poco a poco, nel corso della novella, la focalizzazione si focalizzerà su di lui
 
 \colorbox{green}{Qui} possiamo vedere un procedimento tipico della narrazione popolare, spesso orale: talvolta delle sequenze riprendono la fine di quelle precedenti
 
 \begin{quote}
 \colorbox{green}{Era morto così}, che un sabato aveva voluto terminare certo lavoro preso a \textit{cottimo}\mat{a tempo}, di un pilastro lasciato altra volta per sostegno dell’ingrottato, e dacchè non serviva più, s’era calcolato, cosi ad occhio col padrone, per 35 o 40 carra di rena. Invece mastro Misciu sterrava da tre giorni, e ne avanzava ancora per la mezza giornata del lunedì. Era stato un magro affare e solo un \textit{minchione}\mat{Verga non considera Mastro Misciu un minchione. È un uomo buono, che vuole bene al figilio; il giudizio del narratore è inattendibile. Mastro Misciu è buono, e non mena le mani. Malpelo fa le occhiatacce ai molestatori} come mastro Misciu aveva potuto lasciarsi gabbare a questo modo dal padrone; perciò appunto lo chiamavano mastro Misciu Bestia, ed era l’asino da basto di tutta la cava. Ei, povero diavolaccio, lasciava dire, e si contentava di buscarsi il pane colle sue braccia, invece di menarle addosso ai compagni, e attaccar brighe. Malpelo faceva un visaccio, come se quelle soperchierie cascassero sulle sue spalle, e così piccolo com’era aveva di quelle occhiate che facevano dire agli altri: — Va là, che tu non ci morrai nel tuo letto, come tuo padre.
 
 Invece nemmen suo padre ci morì, nel suo letto, tuttochè fosse una buona bestia. Zio Mommu lo sciancato, aveva detto che quel pilastro lì ei non l’avrebbe tolto per venti onze, tanto era pericoloso; ma d’altra parte tutto è pericolo nelle cave, e se si sta a badare a tutte le sciocchezze che si dicono, è meglio andare a fare l’avvocato.
 
 Dunque il sabato sera mastro Misciu raschiava ancora il suo pilastro \textit{che}\mat{che polivalente, tipico del dialetto siciliano} l’avemaria era suonata da un pezzo, e tutti i suoi compagni avevano accesa la pipa e se n’erano andati dicendogli di divertirsi a grattar la rena per amor del padrone, o raccomandandogli di non fare la morte del sorcio. Ei, che c’era avvezzo alle beffe, non dava retta, e rispondeva soltanto cogli “ah! ah!'' dei suoi bei colpi di zappa in pieno, e intanto borbottava:
 
 — Questo è per il pane! Questo pel vino! Questo per la gonnella di Nunziata! — e così andava facendo il conto del come avrebbe speso i denari del suo appalto, il cottimante!
 
 Fuori della cava il cielo formicolava di stelle, e laggiù la lanterna fumava e girava al pari di un arcolaio. Il grosso pilastro rosso, sventrato a colpi di zappa, contorcevasi e si piegava in arco, come se avesse il mal di pancia, e dicesse ohi! anch’esso. Malpelo andava sgomberando il terreno, e metteva al sicuro il piccone, il sacco vuoto ed il fiasco del vino. Il padre, che gli voleva bene, poveretto, andava dicendogli: “Tirati in là!'' oppure “Sta attento! Bada se cascano dall’alto dei sassolini o della rena grossa, e scappa!'' Tutt’a un tratto, punf! Malpelo, che si era voltato a riporre i ferri nel corbello, udì un tonfo sordo, come fa la rena traditora allorchè fa pancia e si sventra tutta in una volta, ed il lume si spense.
 
 L’ingegnere che dirigeva i lavori della cava, si trovava a teatro quella sera, e non avrebbe cambiato la sua poltrona con un trono, quando vennero a cercarlo per il babbo di Malpelo che aveva fatto la morte del sorcio. Tutte le femminucce di Monserrato, strillavano e si picchiavano il petto per annunziare la gran disgrazia ch’era toccata a comare Santa, la sola, poveretta, che non dicesse nulla, e sbatteva i denti invece, quasi avesse la terzana. L’ingegnere, quando gli ebbero detto il come e il quando, che la disgrazia era accaduta da circa tre ore, e Misciu Bestia doveva già essere bell’e arrivato in Paradiso, andò proprio per scarico di coscienza, con scale e corde, a fare il buco nella rena. Altro che quaranta carra! Lo sciancato disse che a sgomberare il sotterraneo ci voleva almeno una settimana. Della rena ne era caduta una montagna, tutta fina e ben bruciata dalla lava, che si sarebbe impastata colle mani, e dovea prendere il doppio di calce. Ce n’era da riempire delle carra per delle settimane. Il bell’affare di mastro Bestia!
 
 Nessuno badava al ragazzo che si graffiava la faccia ed urlava, come una bestia davvero.
 
 — To’! — disse infine uno. — E Malpelo! Di dove è saltato fuori, adesso?
 
 — Se non fosse stato Malpelo non se la sarebbe passata liscia....
 
 Malpelo non rispondeva nulla, non piangeva nemmeno, scavava colle unghie colà, nella rena, dentro la buca, sicchè nessuno s’era accorto di lui; e quando si accostarono col lume, gli videro tal viso stravolto, e tali occhiacci invetrati, e la schiuma alla bocca da far paura; le unghie gli si erano strappate e gli pendevano dalle mani tutte in sangue. Poi quando vollero toglierlo di là fu un affar serio; non potendo più graffiare, mordeva come un cane arrabbiato, e dovettero afferrarlo pei capelli, per tirarlo via a viva forza.
 
 Però infine tornò alla cava dopo qualche giorno, quando sua madre piagnucolando ve lo condusse per mano; giacchè, alle volte, il pane che si mangia non si può andare a cercarlo di qua e di là. Lui non volle più allontanarsi da quella galleria, e sterrava con accanimento, quasi ogni corbello di rena lo levasse di sul petto a suo padre. \textit{Spesso, mentre scavava, si fermava bruscamente, colla zappa in aria, il viso torvo e gli occhi stralunati, e sembrava che stesse ad ascoltare qualche cosa che il suo diavolo gli susurrasse nelle orecchie}\mat{ha ancora speranza che il padre sia vivo}, dall’altra parte della montagna di rena caduta. In quei giorni era più tristo e cattivo del solito, talmente che non mangiava quasi, e il pane lo buttava al cane, quasi non fosse grazia di Dio. Il cane gli voleva bene, perchè i cani non guardano altro che la mano che gli dà il pane, e le botte, magari. Ma l’asino, povera bestia, sbilenco e macilento, sopportava tutto lo sfogo della cattiveria di Malpelo; ei lo picchiava senza pietà, col manico della zappa, e borbottava:
 
 — Così creperai più presto!
 
 Dopo la morte del babbo pareva che gli fosse entrato il \textit{diavolo in corpo}\mat{A causa della reazione di Malpelo, egli viene paragonato al diavolo}, e lavorava al pari di quei bufali feroci che si tengono coll’anello di ferro al naso. Sapendo che era malpelo, ei si acconciava ad esserlo il peggio che fosse possibile, e se accadeva una disgrazia, o che un operaio smarriva i ferri, o che un asino si rompeva una gamba, e che crollava un tratto di galleria, si sapeva sempre che era stato lui; e infatti ci si pigliava le busse senza protestare, proprio come se le pigliano gli asini che curvano la schiena, ma seguitano a fare a modo loro. \textit{Cogli altri ragazzi poi era addirittura crudele},\mat{da questo momento Malpelo si separa dalla società, e inizia a sviluppare una sua filosofia di vita} e sembrava che si volesse vendicare sui deboli di tutto il male che s’immaginava gli avessero fatto gli altri, a lui e al suo babbo. Certo ei provava uno strano diletto a rammentare ad uno ad uno\textit{ tutti i maltrattamenti ed i soprusi che avevano fatto subire a suo padre}\mat{Qui Rosso Malpelo sta acquisendo una certa lucidità,
   denunciando e accorgendosi di una certa irregolarità: sviluppa un
   sentimento malevolo nei confronti di tutti coloro che non hanno fatto
   nulla per il padre}, e del modo in cui l’avevano lasciato crepare. E quando era solo borbottava: “Anche con me fanno così! e a mio padre gli dicevano Bestia, perchè egli non faceva così!'' E una volta che passava il padrone, accompagnandolo con un’occhiata torva: “È stato lui! per trentacinque tarì!'' E un’altra volta, dietro allo Sciancato: “E anche lui! e si metteva a ridere! Io l’ho udito, quella sera!''
 
 \textit{Per un raffinamento di malignità}\mat{il narratore
   inattendibile non comprende le ragioni di Rosso Malpelo, e da la
   ragione delle sue azioni alla malignità; vi sono continui cambi di
   focalizzazione, da Rosso Malpelo ad agente esterno: questo pone il
   lettore in una posizione di critica attenta, in quanto deve capire.} sembrava aver preso a proteggere un povero ragazzetto, venuto a lavorare da poco tempo nella cava, il quale per una caduta da un ponte s’era lussato il femore, e non poteva far più il manovale. Il poveretto, quando portava il suo corbello di rena in spalla, arrancava in modo che gli avevano messo nome Ranocchio; ma lavorando sotterra, così ranocchio com’era, il suo pane se lo buscava. Malpelo gliene dava anche del suo, per prendersi il gusto di tiranneggiarlo, dicevano.
 
 Infatti egli lo tormentava in cento modi. Ora lo batteva senza un motivo e senza misericordia, e se Ranocchio non si difendeva, lo picchiava più forte, con maggiore accanimento, dicendogli: — To’, bestia! Bestia sei! Se non ti senti l’animo di difenderti da me che \textit{non ti voglio male}\mat{dice esattamente che gli vuole bene, nonostante i
   metodi siano discutibili}, vuol dire che ti lascerai pestare il viso da questo e da quello!
 
 O se Ranocchio si asciugava il sangue che gli usciva dalla bocca e dalle narici: — Così, come ti cuocerà il dolore delle busse, imparerai a darne anche tu!\mat{Verga non vuole dare una immagine confezionata del
   ragazzo, ma semplicemente raffigurare una realtà in cui anche i poveri
   sono violenti, non beatificati: è una visione estremamente lucida e
   pessimista.} — Quando cacciava un asino carico per la ripida salita del sotterraneo, e lo vedeva puntare gli zoccoli, rifinito, curvo sotto il peso, ansante e coll’occhio spento, ei lo batteva senza misericordia, col manico della zappa, e i colpi suonavano secchi sugli stinchi e sulle costole scoperte. Alle volte la bestia si piegava in due per le battiture, ma stremo di forze, non poteva fare un passo, e cadeva sui ginocchi, e ce n’era uno il quale era caduto tante volte, che ci aveva due piaghe alle gambe. Malpelo soleva dire a Ranocchio: — L’asino va picchiato, perchè non può picchiar lui; e s’ei potesse picchiare, ci pesterebbe sotto i piedi e ci strapperebbe la carne a morsi.
 
 Oppure: — Se ti accade di dar delle busse, procura di darle più forte che puoi; così gli altri ti terranno da conto, e ne avrai tanti di meno addosso.\mat{codice comportamentale di un ambiente disumanizzato
   come quello delle cave}
 
 Lavorando di piccone o di zappa poi menava le mani con accanimento, a mo’di uno che l’avesse con la rena, e batteva e ribatteva coi denti stretti, e con quegli ah! ah! che aveva suo padre. — La rena è traditora, — diceva a Ranocchio sottovoce; — somiglia a tutti gli altri, che se sei più debole ti pestano la faccia, e se sei più forte, o siete in molti, come fa lo Sciancato, allora si lascia vincere. Mio padre la batteva sempre, ed egli non batteva altro che la rena, perciò lo chiamavano Bestia, e la rena se lo mangiò a tradimento, perchè era più forte di lui.
 
 Ogni volta che a Ranocchio toccava un lavoro troppo pesante, e il ragazzo piagnucolava a guisa di una femminuccia, Malpelo lo picchiava sul dorso, e lo sgridava: — Taci, pulcino! — e se Ranocchio non la finiva più, ei gli dava una mano, dicendo con un certo orgoglio: — Lasciami fare; io sono più forte di te. — Oppure gli dava la sua mezza cipolla, e si contentava di mangiarsi il pane asciutto, e si stringeva nelle spalle, aggiungendo: — Io ci sono avvezzo.
 
 Era avvezzo a tutto lui, agli scapaccioni, alle pedate, ai colpi di manico di badile, o di cinghia da basto, a vedersi ingiuriato e beffato da tutti, a dormire sui sassi, colle braccia e la schiena rotta da quattordici ore di lavoro; anche a digiunare era avvezzo, allorchè il padrone lo puniva levandogli il pane o la minestra. Ei diceva che la razione di busse non gliel’aveva levata mai, il padrone; ma le busse non costavano nulla\mat{Malpelo non si ribella, ma il narratore non è
   onnisciente, e si chiede il perché di determinati comportamenti del
   ragazzo}. Non si lamentava però, e si vendicava di soppiatto, a tradimento, con qualche tiro di quelli che sembrava ci avesse messo la coda il diavolo: perciò ei si pigliava sempre i castighi, anche quando il colpevole non era stato lui. Già se non era stato lui sarebbe stato capace di esserlo, e non si giustificava mai: per altro sarebbe stato inutile. E qualche volta, come Ranocchio spaventato lo scongiurava piangendo di dire la verità, e di scolparsi, ei ripeteva: — A che giova? Sono malpelo! — e nessuno avrebbe potuto dire se quel curvare il capo e le spalle sempre fosse effetto di fiero orgoglio o di disperata rassegnazione, e non si sapeva nemmeno se la sua fosse salvatichezza o timidità. Il certo era che nemmeno sua madre aveva avuta mai una carezza da lui, e quindi non gliene faceva mai.
 
 Il sabato sera, appena arrivava a casa con quel suo visaccio imbrattato di lentiggini e di rena rossa, e quei cenci che gli piangevano addosso da ogni parte, la sorella afferrava il manico della scopa, scoprendolo sull’uscio in quell’arnese, chè avrebbe fatto scappare il suo damo se vedeva con qual gente gli toccava imparentarsi\mat{si nota lo straniamento: ci parla di cose stranissime
   facendole passare per normali}; la madre era sempre da questa o da quella vicina, e quindi egli andava a rannicchiarsi sul suo saccone come un cane malato. Per questo, la domenica, in cui tutti gli altri ragazzi del vicinato si mettevano la camicia pulita per andare a messa o per ruzzare nel cortile, ei sembrava non avesse altro spasso che di andar randagio per le vie degli orti, a dar la caccia alle lucertole e alle altre povere bestie che non gli avevano fatto nulla, oppure a sforacchiare le siepi dei fichidindia. Per altro le beffe e le sassate degli altri fanciulli non gli piacevano.
 
 La vedova di mastro Misciu era disperata di aver per figlio quel malarnese, come dicevano tutti, ed egli era ridotto veramente come quei cani, che a furia di buscarsi dei calci e delle sassate da questo e da quello, finiscono col mettersi la coda tra le gambe e scappare alla prima anima viva che vedono, e diventano affamati, spelati e selvatici come lupi. Almeno sottoterra, nella cava della rena, brutto, cencioso e lercio com’era, non lo beffavano più, e sembrava fatto apposta per quel mestiere persin nel colore dei capelli, e in quegli occhiacci di gatto che ammiccavano se vedevano il sole. Così ci sono degli asini che lavorano nelle cave per anni ed anni senza uscirne mai più, ed in quei sotterranei, dove il pozzo d’ingresso è a picco, ci si calan colle funi, e ci restano finchè vivono. Sono asini vecchi, è vero, comprati dodici o tredici lire, quando stanno per portarli alla Plaja, a strangolarli; ma pel lavoro che hanno da fare laggiù sono ancora buoni; e Malpelo, certo, non valeva di più; se veniva fuori dalla cava il sabato sera, era perchè aveva anche le mani per aiutarsi colla fune, e doveva andare a portare a sua madre la paga della settimana.
 
 Certamente egli avrebbe preferito di fare il \textit{manovale}\mat{si vede l'ideale di vita di Malpelo: \emph{non
   sottoterra}}, come Ranocchio, e lavorare cantando sui ponti, in alto, in mezzo all’azzurro del cielo, col sole sulla schiena, — o il carrettiere, come compare Gaspare, che veniva a prendersi la rena della cava, dondolandosi sonnacchioso sulle stanghe, colla pipa in bocca, e andava tutto il giorno per le belle strade di campagna; — o meglio ancora, avrebbe voluto fare il contadino, che passa la vita fra i campi, in mezzo al verde, sotto i folti carrubbi, e il mare turchino là in fondo, e il canto degli uccelli sulla testa. Ma quello era stato il mestiere di suo padre, e in quel mestiere era nato lui. E pensando a lutto ciò, narrava a Ranocchio del pilastro che era caduto addosso al genitore, e dava ancora della rena fina e bruciata che il carrettiere veniva a caricare colla pipa in bocca, e dondolandosi sulle stanghe, e gli diceva che quando avrebbero finito di sterrare si sarebbe trovato il cadavere del babbo, il quale doveva avere dei calzoni di fustagno quasi nuovi. Ranocchio aveva paura, ma egli no. Ei pensava che era stato sempre là, da bambino, e aveva sempre visto quel buco nero, che si sprofondava sotterra, dove il padre soleva condurlo per mano. Allora stendeva le braccia a destra e a sinistra, e descriveva come l’intricato laberinto delle gallerie si stendesse sotto i loro piedi all’infinito, di qua e di là, sin dove potevano vedere la sciara nera e desolata, sporca di ginestre riarse, e come degli uomini ce n’erano rimasti tanti, o schiacciati, o smarriti nel buio, e che camminano da anni e camminano ancora, senza poter scorgere lo spiraglio del pozzo pel quale sono entrati, e senza poter udire le strida disperate dei figli, i quali li cercano inutilmente.
 
 Ma una volta in cui riempiendo i corbelli si rinvenne una delle scarpe di mastro Misciu, ei fu colto da tal tremito che dovettero tirarlo all’aria aperta colle funi, proprio come un asino che stesse per dar dei calci al vento. Però non si poterono trovare nè i calzoni quasi nuovi, nè il rimanente di mastro Misciu; sebbene i pratici affermarono che quello dovea essere il luogo preciso dove il pilastro gli si era rovesciato addosso; e qualche operaio, nuovo al mestiere, osservava curiosamente come fosse capricciosa la rena, che aveva sbatacchiato il Bestia di qua e di là, le scarpe da una parte e i piedi dall’altra.
 
 Dacchè poi fu trovata quella scarpa, Malpelo fu colto da tal paura di veder comparire fra la rena anche il piede nudo del babbo, che non volle mai più darvi un colpo di zappa, gliela dessero a lui sul capo, la zappa. Egli andò a lavorare in un altro punto della galleria, e non volle più tornare da quelle parti. Due o tre giorni dopo scopersero infatti il cadavere di mastro Misciu, coi calzoni indosso, e steso bocconi che sembrava imbalsamato. Lo zio Mommu osservò che aveva dovuto penar molto a finire, perchè il pilastro gli si era piegato proprio addosso, e l’aveva sepolto vivo: si poteva persino vedere tutt’ora che mastro Bestia avea tentato istintivamente di liberarsi scavando nella rena, e avea le mani lacerate e le unghie rotte. “Proprio come suo figlio Malpelo! — ripeteva lo sciancato — ei scavava di qua, mentre suo figlio scavava di là.'' Però non dissero nulla al ragazzo, per la ragione che \textit{lo sapevano maligno e vendicativo}\mat{rinterpretazione della realtà del narratore}.
 
 Il carrettiere si portò via il cadavere di mastro Misciu al modo istesso che caricava la rena caduta e gli asini morti, chè stavolta, oltre al lezzo del carcame, trattavasi di un compagno, e di carne battezzata. La vedova rimpiccolì i calzoni e la camicia, e li adattò a Malpelo, il quale così fu vestito quasi a nuovo per la prima volta. Solo le scarpe furono messe in serbo per quando ei fosse cresciuto, giacchè rimpiccolire le scarpe non si potevano, e il fidanzato della sorella non le aveva volute le scarpe del morto.\mat{focalizzazione di Malpelo, fino alla fine di questo
   brano, in cui ritorna lo straniamento (\emph{cervellaccio})}
 
 Malpelo se li lisciava sulle gambe, quei calzoni di fustagno quasi nuovi, gli pareva che fossero dolci e lisci come le mani del babbo, che solevano accarezzargli i capelli, quantunque fossero così ruvide e callose. Le scarpe poi, le teneva appese a un chiodo, sul saccone, quasi fossero state le pantofole del papa, e la domenica se le pigliava in mano, le lustrava e se le provava; poi le metteva per terra, l’una accanto all’altra, e stava a guardarle, coi gomiti sui ginocchi, e il mento nelle palme, per delle ore intere, rimuginando chi sa quali idee in quel cervellaccio.
 
 Ei possedeva delle idee strane, Malpelo\mat{Malpelo ha dei bei sentimenti nei confronti del
   padre, ma il narratore non riesce a concepirli, in quanto è permeato
   di una visione utilitarista e materialista, caratteristica di tutta la
   società}! Siccome aveva ereditato anche il piccone e la zappa del ladre, se ne serviva, quantunque fossero troppo pesanti per l’età sua; e quando gli aveano chiesto se voleva venderli, che glieli avrebbero pagati come nuovi, egli aveva risposto di no. Suo padre li aveva resi così lisci e lucenti nel manico colle sue mani, ed ei non avrebbe potuto farsene degli altri più lisci e lucenti di quelli, se ci avesse lavorato cento e poi cento anni.
 
 In quel tempo era crepato di stenti e di vecchiaia l’asino grigio; e il carrettiere era andato a buttarlo lontano nella sciara.
 
 — Così si fa, — brontolava Malpelo; — gli arnesi che non servono più, si buttano lontano.
 
 Egli andava a visitare il carcame del grigio in fondo al burrone, e vi conduceva a forza anche Ranocchio, il quale non avrebbe voluto andarci; e Malpelo gli diceva che a questo mondo bisogna avvezzarsi a vedere in faccia ogni cosa, bella o brutta; e stava a considerare con l’avida curiosità di un monellaccio i cani che accorrevano da tutte le fattorie dei dintorni a disputarsi le carni del grigio. I cani scappavano guaendo, come comparivano i ragazzi, e si aggiravano ustolando sui greppi dirimpetto, ma il Rosso non lasciava che Ranocchio li scacciasse a sassate. — Vedi quella cagna nera, — gli diceva, che non ha paura delle tue sassate? Non ha paura perchè ha più fame degli altri\mat{vige la legge del più forte}. Gliele vedi quelle costole al grigio? Adesso non soffre più. L’asino grigio se ne stava tranquillo, colle quattro zampe distese, e lasciava che i cani si divertissero a vuotargli le occhiaie profonde, e a spolpargli le ossa bianche; i denti che gli laceravano le viscere non lo avrebbero fatto piegare di un pelo, come quando gli accarezzavano la schiena a badilate, per mettergli in corpo un po’ di vigore nel salire la ripida viuzza. — Ecco come vanno le cose\mat{è molto
   significativa, perché è la dichiarazione di Malpelo: è l'espressione
   della \colorbox{green}{sua filosofia}}! Anche il grigio ha avuto dei colpi di zappa e delle guidalesche; anch’esso quando piegava sotto il peso, o gli mancava il fiato per andare innanzi, aveva di quelle occhiate, mentre lo battevano, che sembrava dicesse: — Non più! non più! — Ma ora gli occhi se li mangiano i cani, ed esso se ne ride dei colpi e delle guidalesche, con quella bocca spolpata e tutta denti. \colorbox{green}{Ma se non fosse mai nato sarebbe stato meglio}.
 
 La sciara si stendeva malinconica e deserta, fin dove giungeva la vista, e saliva e scendeva in picchi e burroni, nera e rugosa, senza un grillo che vi trillasse, o un uccello che venisse a cantarci. Non si udiva nulla, nemmeno i colpi di piccone di coloro che lavoravano sotterra. E ogni volta Malpelo ripeteva che la terra lì sotto era tutta vuota dalle gallerie, per ogni dove, verso il monte e verso la valle; tanto che una volta un minatore c’era entrato da giovane, e n’era uscito coi capelli bianchi, e un altro, cui s’era spenta la candela, aveva invano gridato aiuto per anni ed anni.
 
 — Egli solo ode le sue stesse grida! — diceva, — e a quell’idea, sebbene avesse il cuore più duro della sciara, trasaliva.
 
 — Il padrone mi manda spesso lontano, dove gli altri hanno paura d’andare. Ma io sono Malpelo, e se non torno più, nessuno mi cercherà.
 
 Pure, durante le belle notti d’estate, le stelle splendevano lucenti anche sulla sciara, e la campagna circostante era nera anch’essa, come la lava, ma Malpelo, stanco della lunga giornata di lavoro, si sdraiava sul sacco, col viso verso il cielo, a godersi quella quiete e quella luminaria dell’alto; perciò odiava le notti di luna, in cui il mare formicola di scintille, e la campagna si disegna qua e là vagamente — perchè allora la sciara sembra più brulla e desolata.
 
 — Per noi che siamo fatti per vivere sotterra, — pensava Malpelo, — dovrebbe essere buio sem pre e da per tutto.
 
 La civetta strideva sulla sciara, e ramingava di qua e di là; ei pensava:
 
 — Anche la civetta sente i morti che son qua sotterra, e si dispera perchè non può andare a trovarli.
 
 Ranocchio aveva paura delle civette e dei pipistrelli; ma il Rosso lo sgridava, perchè chi è costretto a star solo non deve aver paura di nulla, e nemmeno l’asino grigio aveva paura dei cani che se lo spolpavano, ora che le sue carni non sentivano più il dolore di esser mangiate.
 
 — Tu eri avvezzo a lavorar sui tetti come i gatti, — gli diceva, — e allora era tutt’altra cosa. Ma adesso che ti tocca a viver sotterra, come i topi, non bisogna più aver paura dei topi, nè dei pipistrelli, che son topi vecchi con le ali; quelli ci stanno volentieri in compagnia dei morti.
 
 \colorbox{green}{Ranocchio} invece provava una tale compiacenza a spiegargli quel che ci stessero a far le stelle lassù in alto; e gli raccontava che lassù c’era il paradiso, dove vanno a stare i morti che sono stati buoni, e non hanno dato dispiaceri ai loro genitori. “Chi te l’ha detto?'' domandava Malpelo, e Ranocchio rispondeva che glielo aveva detto la mamma.
 
 Allora Malpelo si grattava il capo, e sorridendo gli faceva un certo verso da monellaccio malizioso che la sa lunga. “Tua madre ti dice così perchè, invece dei calzoni, tu dovresti portar la gonnella.'' E dopo averci pensato su un po’:
 
 — Mio padre era buono, e non faceva male a nessuno, tanto che lo chiamavano Bestia. Invece è là sotto, ed hanno persino trovato i ferri, le scarpe e questi calzoni qui che ho \colorbox{green}{indosso io}.
 
 Da lì a poco, Ranocchio, il quale deperiva da qualche tempo, si ammalò in modo che la sera dovevano portarlo fuori dalla cava sull’asino, disteso fra le corbe, tremante di febbre come un pulcin bagnato. Un operaio disse che quel ragazzo non ne avrebbe fatto osso duro a quel mestiere, e che per lavorare in una miniera, senza lasciarvi la pelle, bisognava nascervi. Malpelo allora si sentiva orgoglioso di esserci nato, e di mantenersi così sano e 
 
 vigoroso in quell’aria malsana, e con tutti quegli stenti. Ei si caricava Ranocchio sulle spalle, e gli faceva animo alla sua maniera, sgridandolo e picchiandolo. Ma una volta, nel picchiarlo sul dorso, Ranocchio fu colto da uno sbocco di sangue; allora Malpelo spaventato si affannò a cercargli nel naso e dentro la bocca cosa gli avesse fatto, e giurava che non avea potuto fargli poi gran male, così come l’aveva battuto, e a dimostrarglielo, si dava dei gran pugni sul petto e sulla schiena, con un sasso; anzi un operaio, lì presente, gli sferrò un gran calcio sulle spalle: un calcio che risuonò come su di un tamburo, eppure Malpelo non si mosse, e soltanto dopo che l’operaio se ne fu andato, aggiunse:
 
 — Lo vedi? Non mi ha fatto nulla! E ha picchiato più forte di me, ti giuro!
 
 Intanto Ranocchio non guariva, e seguitava a sputar sangue, e ad aver la febbre tutti i giorni. Allora Malpelo prese dei soldi della paga della settimana, per comperargli del vino e della minestra calda, e gli diede i suoi calzoni quasi nuovi, che lo coprivano meglio. Ma Ranocchio tossiva sempre, e alcune volte sembrava soffocasse; la sera poi non c’era modo di vincere il ribrezzo della febbre, nè con sacchi, nè coprendolo di paglia, nè mettendolo dinanzi alla fiammata. Malpelo se ne stava zitto ed immobile, chino su di lui, colle mani sui ginocchi, fissandolo con quei suoi occhiacci spalancati, quasi volesse fargli il ritratto, e allorchè lo udiva gemere sottovoce, e gli vedeva il viso trafelato e l’occhio spento, preciso come quello dell’asino grigio allorchè ansava rifinito sotto il carico nel salire la viottola, egli borbottava:
 
 È meglio che tu crepi presto! Se devi soffrire a quel modo, è meglio che tu crepi!
 
 E il padrone diceva che Malpelo era capace di schiacciargli il capo, a quel ragazzo, e bisognava sorvegliarlo.
 
 Finalmente un lunedì Ranocchio non venne più alla cava, e il padrone se ne lavò le mani, perchè allo stato in cui era ridotto oramai era più di impiccio che altro. Malpelo si informò dove stesse di casa, e il sabato andò a trovarlo. Il povero Ranocchio era più di là che di qua; sua madre piangeva e si disperava come se il figliuolo fosse di quelli che guadagnano dieci lire la settimana.
 
 Cotesto non arrivava a comprenderlo Malpelo, e domandò a Ranocchio perchè sua madre strillasse a quel modo, mentre che da due mesi ei non guadagnava nemmeno quel che si mangiava. Ma il povero Ranocchio non gli dava retta; sembrava che badasse a contare quanti travicelli c’erano sul tetto. Allora il Rosso si diede ad almanaccare che la madre di Ranocchio strillasse a quel modo perchè il suo figliuolo era sempre stato debole e malaticcio, e l’aveva tenuto come quei marmocchi che non si slattano mai. Egli invece era stato sano e robusto, ed era malpelo, e sua madre non aveva mai pianto per lui, perchè non aveva mai avuto timore di perderlo.
 
 Poco dopo, alla cava dissero che Ranocchio era morto, ed ei pensò che la civetta adesso strideva anche per lui la notte, e tornò a visitare le ossa spolpate del grigio, nel burrone dove solevano andare insieme con Ranocchio. Ora del grigio non rimanevano più che le ossa sgangherate, ed anche di Ranocchio sarebbe stato così. Sua madre si sarebbe asciugati gli occhi, poiché anche la madre di Malpelo s’era asciugati i suoi, dopo che mastro Misciu era morto, e adesso si era maritata un’altra volta, ed era andata a stare a Cifali colla figliuola maritata, e avevano chiusa la porta di casa. D’ora in poi, se lo battevano, a loro non importava più nulla, e a lui nemmeno, chè quando sarebbe divenuto come il grigio o come Ranocchio, non avrebbe sentito più nulla.
 
 Verso quell’epoca venne a lavorare nella cava uno che non s’era mai visto, e si teneva nascosto il più che poteva. Gii altri operai dicevano fra di loro che era scappato dalla prigione, e se lo pigliavano ce lo tornavano a chiudere per anni ed anni. Malpelo seppe in quell’occasione che la prigione era un luogo dove si mettevano i ladri, e i malarnesi come lui, e si tenevano sempre chiusi là dentro e guardati a vista.
 
 Da quel momento provò una malsana curiosità per quell’uomo che aveva provata la prigione e ne era scappato. Dopo poche settimane però il fuggitivo dichiarò chiaro e tondo che era stanco di quella vitaccia da talpa, e piuttosto si contentava di stare in galera tutta la vita, chè la prigione, in confronto, era un paradiso, e preferiva tornarci coi suoi piedi.
 
 — Allora perchè tutti quelli che lavorano nella cava non si fanno mettere in prigione? — domandò Malpelo.
 
 — Perchè non sono malpelo come te! — rispose lo Sciancato. — Ma non temere, che tu ci andrai! e ci lascerai le ossa!
 
 Invece le ossa le lasciò nella cava, Malpelo come suo padre, ma in modo diverso. Una volta si doveva esplorare un passaggio che doveva comunicare col pozzo grande a sinistra, verso la valle, e se la cosa andava bene, si sarebbe risparmiata una buona metà di mano d’opera nel cavar fuori la rena. Ma a ogni modo, però, c’era il pericolo di smarrirsi e di non tornare mai più. 
 
 Sicchè nessun padre di famiglia voleva avventurarcisi, nè avrebbe permesso che ci si arrischiasse il sangue suo, per tutto l’oro del mondo.
 
 Malpelo, invece, non aveva nemmeno chi si prendesse tutto l’oro del mondo per la sua pelle, se pure la sua pelle valeva tanto: sicchè pensarono a lui. Allora, nel partire, si risovvenne del minatore, il quale si era smarrito, da anni ed anni, e cammina e cammina ancora al buio, gridando aiuto, senza che nessuno possa udirlo. Ma non disse nulla. Del resto a che sarebbe giovato? Prese gli arnesi di suo padre, il piccone, la zappa, la lanterna, il sacco col pane, il fiasco del vino, e se ne andò: nè più si seppe nulla di lui.
 
 Così si persero persin le ossa di Malpelo, e i ragazzi della cava abbassano la voce quando parlano di lui nel sotterraneo, chè hanno paura di vederselo comparire dinanzi, coi capelli rossi e gli occhiacci grigi.
\end{quote}

\paragraph{L'impostazione narrativa rivoluzionaria} Il racconto occupa una posizione fondamentale nell'arco dell'opera verghiana: è infatti il testo che dà inizio alla fase "verista" dello scrittore. Subito la frase iniziale evidenzia la rivoluzionaria novità dell'impostazione narrativa verghiana: affermare che Malpelo ha i capelli rossi «perché era un ragazzo malizioso e cattivo» è una stortura logica, che rivela un pregiudizio superstizioso, proprio di una mentalità primitiva. La voce che racconta non è dunque al livello dell'autore reale, non è portavoce della sua visione del mondo, ma è al livello dei personaggi, è interna al mondo rappresentato. L'apertura del racconto presenta immediatamente il procedimento della "regressione", mediante cui si attua il basilare principio dell'impersonalità. Scompare il narratore onnisciente, portavoce dello scrittore stesso, che era l'elemento caratterizzante della narrativa del primo Ottocento, in Manzoni, Scott, Balzac. Non essendo onnisciente, ma portavoce di un ambiente popolare primitivo e rozzo, il narratore di \textsc{Rosso Malpelo} non è depositario della verità, come è proprio dei narratori tradizionali. Difatti ciò che ci dice del protagonista non è attendibile: il narratore non capisce le motivazioni dell'agire di Malpelo, le deforma sistematicamente. Alcuni esempi sono molto evidenti. Dopo la morte del padre nel crollo della galleria Rosso scava con accanimento, ed ogni tanto si ferma, ascoltando. È facile intuire che scava nella speranza di riuscire ancora a salvare il padre e si ferma cercando di udire la sua voce al di là della parete di sabbia; ma il narratore non capisce questi suoi sentimenti filiali e attri buisce il suo comportamento, in base al pregiudizio del "Malpelo", alla sua strana cattiveria («sembrava che stesse ad ascoltare qualche cosa che il suo diavolo gli sussurrasse negli orecchi»). Più avanti, Malpelo tributa un vero e proprio culto alle reliquie del padre morto, gli strumenti di lavoro, i calzoni, le scarpe: ciò dimostra in lui un attaccamento profondo, una pietà filiale per l'unica persona che gli voleva bene. Anche qui è facile intuire che cosa si muova nel suo animo, dolore, rimpianto. Ma ancora una volta il comportamento del personaggio resta impenetrabile al narratore, che riflette la visione ottusa e disumanizzata di un ambiente duro come quello della cava («rimuginando chi sa quali idee in quel cervellaccio»). Infine, Rosso prende a ben volere Ranocchio, lo protegge, gli vuole insegnare le leggi brutali che regolano la vita e si toglie il pane di bocca per darlo all'amico. Il narratore interpreta, riproducendo evidentemente l'opinione corrente nella cava: «per prendersi il gusto di tiranneggiarlo».

\paragraph{La funzione delle soluzioni narrative} Qual è la funzione di questo sistematico stravolgimento della figura del protagonista? È evidente dal montaggio del racconto che Rosso, pur essendosi formato nell'ambiente disumano della cava, ha conservato alcuni valori autentici, disinteressati: la pietà filiale, il senso della giustizia (si sdegna contro il padrone, responsabile dell'omicidio bianco” di cui il padre è stato vittima), l'amicizia, la solidarietà altruistica. Il punto di vista del narratore "basso", con le sue deformazioni e incomprensioni, esercita su questi valori un processo di straniamento; fa apparire strano, incomprensibile, ciò che dovrebbe essere normale, i sentimenti autentici, i valori. Ciò deriva dal fatto che il narratore è il portavoce della visione di un mondo disumano, che ignora i valori e conosce solo l'interesse e la forza.

Lo straniamento che scaturisce dall'accettazione del punto di vista che domina la realtà oggettiva ha così la funzione di negare i valori, di mostrarne l'impraticabilità in un mondo dominato dal meccanismo brutale della lotta per la vita, che non lascia alcuno spazio ai sentimenti disinteressati. L'artificio narrativo è quindi gravido di significato: con la scelta di narrare dal punto di vista degli operai della cava, Verga esprime tutto il suo pessimismo. Ma si verifica anche uno straniamento in senso inverso, nei confronti del narratore: poiché chi conduce il racconto è proprio chi è portatore di quella visione disumana, ciò che do vrebbe essere strano, l'insensibilità totale ai valori, finisce per apparire normale: ciò de nuncia con incisiva evidenza, sempre senza interventi giudicanti dal punto di vista dell'autore, lasciando parlare le cose stesse, lo stravolgimento profondo che domina nella visione del mondo di quell'ambiente e nei rapporti sociali che lo regolano (Luperini). Non va dimenticato che si tratta di un ambiente popolare: ciò dimostra come qui il mondo rurale non sia affatto mitizzato nostalgicamente come paradiso di innocenza e autenticità, ma sia dominato dalle stesse leggi che regolano anche gli strati più evoluti, la società delle «Banche» e delle «Imprese industriali». Questa soluzione costituisce la smentita amara delle tendenze romantiche che erano presenti in Verga nei confronti del mondo popolare

\paragraph{Il punto di vista di Malpelo} Non tutto il racconto è però impostato sull'effetto di deformazione e straniamento della fi gura del protagonista. Se nella prima parte Malpelo è visto solo dall'esterno, dal punto di vista ottuso e malevolo del suo ambiente e le motivazioni dei suoi atti restano incompren sibili al narratore (sicché solo per induzione il lettore le può ricostruire), nella seconda parte emerge il punto di vista del protagonista stesso, e possiamo allora sapere che cosa pensa e che cosa sente.

Affiora così la visione cupa e pessimistica del ragazzo indurito dalla disumanità di quella vita di fatiche, patimenti e angherie. Rosso ha colto perfettamente l'essenza della legge che regola tutta la realtà, quella sociale come quella naturale: la lotta per la vita, in cui prevale il più forte e il più debole rimane schiacciato. E su questa presa di coscienza regola tutta la sua condotta. Nelle vesti del povero garzone di una cava si delinea perciò la figura di un eroe intellettuale, portatore di una consapevolezza lucida dei meccanismi di una realtà tragica quanto immodificabile.  In lui si proietta evidentemente il pessimismo dello scrittore stesso, la sua visione lucida ma disperatamente rassegnata della negatività di tutta la realtà, sociale e naturale. Verga non sa proporre alternative, però conserva un distacco conoscitivo che gli consente di rappresentare con straordinaria acutezza quella negatività. Si può cogliere allora l'importanza dell'impostazione narrativa della novella, che inaugura tutto il modo di narrare del Verga verista: la materia in astratto (i patimenti di un povero orfano incompreso e maltrattato) potrebbe essere quella di un racconto umanitario, edificante e patetico, teso a suscitare facile commozione, come ce ne sono tanti nella letteratura ottocentesca. Ma il modo in cui viene raccontata trasforma \textsc{Rosso Malpelo} in un'analisi dura e impietosa delle leggi sociali, dotata di altissimo valore conoscitivo e critico.

\section{I Malavoglia}

\begin{itemize}
\item
  \pagine{p. 228 - Il ciclo dei vinti}
\item
  \pagine{p. 233-236}
\end{itemize}

Fa parte del \evidenziatore{ciclo dei vinti} . Ad un certo punto, Verga,
sul modello di Zola decide di scrivere un ciclo di romanzi,
originariamente 5. I primi due saranno completati, mentre il terzo
rimarrà a metà. Quarto e quinto non saranno mai scritti.

Il tema centrale dei Malavoglia è la \evidenziatore{lotta per la vita} .
Verga decide che vuole esprimere attraverso questo ciclo di romanzi
questa sorta di darwinismo sociale che è la lotta per l'esistenza, e che
secondo lui si evince in qualsiasi classe sociale. Nelle classi sociali
più basse è una lotta per la sussistenza, mentre nelle classi più alte
diventa una lotta per il potere, etc etc. Ad ogni livello questa lotta
si conclude con la sconfitta dei più deboli.

Verga propone l'immagine de \evidenziatore{la fiumana del progresso} .
Il tema del progresso è un concetto molto significativo per Verga. Verga
non nega che ci sia un progresso, ma lascia delle vittime col suo
passaggio: le vittime sono i più deboli. Visto da lontano è
\textbf{progresso}, ma quando poi ci si avvicina e si vedono i cadaveri
è difficile vederlo come qualcosa di positivo.

Il suo intento è di fotografare nelle varie classi sociali questa lotta.

\subsection{Costruzione bipolare del romanzo}

Questa espressione significa che ci sono due visioni differenti: una è
quella dei Malavoglia (intesi in senso generale) e l'altra è quella del
villaggio. Questa differenziazione provoca lo straniamento.

Ne risulta una particolare configurazione della struttura romanzesca, una costru zione bipolare. Si tratta di un romanzo corale, fittamente popolato di personaggi, senza che spicchi un protagonista. Ma questo "coro" si divide nettamente in due: da un lato si collocano i Malavoglia, con alcuni personaggi a loro collegati (Alfio, Nunziata, la cugina Anna), che sono caratterizzati dalla fedeltà ai valori puri; dall'altro la comunità del paese, pettegola, cinica, mossa solo dall'interesse, insensibile sino alla disumanità. Si alternano quindi costantemente, nella narrazione, due punti di vista opposti, quello nobile e disinteressato dei Malavoglia e quello gretto e ottuso degli altri abitanti del villaggio.

Questo gioco di punti di vista ha una funzione importantissima. L'ottica del paese ha il compito di “straniare” sistematicamente i valori ideali proposti dai Malavoglia. Quei valori, onestà, disinteresse, altruismo, visti con gli occhi della collettività appaiono "strani", non vengono compresi, anzi, vengono stravolti e deformati: padron 'Ntoni che rinuncia alla casa per onorare il debito non è ammirato per il suo gesto nobile, ma giudicato un «minchione», perché non ha applicato la legge dell'interesse; l'angoscia del vecchio patriarca per il figlio in mare durante la tempesta è attribuita dal villaggio essenzialmente al timore carico di lupini in pericolo, cioè a ragioni economiche. Lo straniamento operato sui per il valori dal punto di vista del paese vale a denunciarne l'impraticabilità in un mondo dominato dalla lotta per la vita». Se quindi Verga non sa rinunciare al vagheggiamento dei valori autentici e li sovrappone ancora in parte alla realtà popolare, è anche tanto lucido da rendersi conto che essi non trovano posto nella realtà e da prendere distanze dalle sue tendenze all'idealizzazione, negandole con l'analisi impietosa della realtà di fatto. D'altro lato però il punto di vista ideale dei Malavoglia vale a fornire un metro di giudizio dei meccanismi spietati che dominano l'ambiente del villaggio, facendo emergere dalle cose stesse, senza interventi giudicanti del narratore, la disumanità della logica dell'interesse e della forza e consentendo di rappresentarla in una luce critica. Il romanzo, come si vede, ha una costruzione estremamente problematica: le due componenti della visione verghiana, l'idealizzazione romantica della realtà arcaica e il verismo pessimistico reagiscono l'una contro l'altra, in un gioco dialettico.

\subsection{Microsaggio: Lotte per la vita e ``darwinismo
sociale''}

\paragraph{La ripresa delle teorie di Darwin} La nozione di «lotta per l'esistenza» impiegata da Verga nella \textsc{Prefazione ai Vinti} (ma già presente nel primo progetto del ciclo di romanzi, la lettera al Paola del 1878, dove si parla di una «fantasmagoria della lotta per la vita»), proviene dall'opera di Charles Darwin (1809-82), lo scienziato che con la teoria della selezione naturale (\textsc{L'origine della specie}, 1859) rivoluzionò la concezione tradizionale dell'origine delle specie viventi e diede un assetto organico e definitivo alla concezione evoluzionistica.

Darwin sosteneva che il numero degli organismi viventi che nasce è superiore a quello che può vivere con le risorse disponibili. Quindi esiste tra i vari individui una continua lotta per poter sopravvivere. In questa lotta prevalgono i più adatti alle condizioni di vita in cui si trovano e trasmettono i loro caratteri ai loro discendenti. Questa sopravvivenza del più adatto è la «selezione naturale>>: come l'uomo seleziona artificialmente le specie animali e vegetali più utili ai suoi bisogni, modificandone le caratte ristiche, così opera la natura, scegliendo per la riproduzione gli individui che nella lotta per l'esistenza hanno dei vantaggi sopra i concorrenti.

\paragraph{L'antagonismo sociale} La dottrina darwiniana ebbe un'influenza enorme su tutto lo sviluppo scientifico e filosofico del secondo Ottocento ed ebbe un peso notevole anche nelle scienze sociali, dando origine a quel filone del pensiero sociologico che si definisce appunto "darwinismo sociale". Tale dottrina tende a vedere la società umana regolata dalle stesse leggi del mondo animale e naturale, quindi dominata anch'essa dalla lotta per la vita, che assicura la sopravvivenza e il dominio al più forte. In effetti la società umana nella sua storia millenaria è sempre stata caratterizzata da conflitti tra le varie classi sociali. Tuttavia il darwinismo sociale non analizza la lotta per la vita come un dato legato a forme specifiche, storicamente definite di società, ma la pone come legge assoluta di ogni forma di società possibile. Le tendenze di pensiero più reazionarie ne ricavano la conclusione che l'assetto sociale vigente, fondato sul dominio di una classe sulle altre, corrisponde alle leggi stesse di natura e non potrà mai essere modificato, o addirittura affermano la legittimità e la necessità del predominio del più forte sui più deboli, respingendo quelle nozioni di eguaglianza e di democrazia maturate nel corso moderno della storia borghese, dall'illuminismo e dalla Rivoluzione francese in poi.

Queste teorie sono la manifestazione della profonda crisi attraversata dalla coscienza borghese nella seconda metà dell'Ottocento: viene meno la sicurezza di poter dominare concettualmente e praticamente tutta la realtà, la serena certezza in un futuro di pace, di equilibrio senza conflitti e sconvolgimenti, di giustizia, di benessere illimitato, che erano i punti fondamentali della concezione della borghesia nel periodo eroico della sua ascesa. L'ideologia borghese perde quindi quei caratteri progressivi, tesi all'emancipazione dell'umanità intera, che possedeva durante le lotte rivoluzionarie contro il regime assolutistico-feudale, si chiude a difesa del dominio della classe egemone da ogni forza che possa contrastarlo e si riduce ad essere una semplice giustificazione dell'ordine vigente, o addirittura un'esaltazione dei suoi aspetti più negativi, la diseguaglianza, il trionfo della forza sul diritto, l'oppres sione, lo sfruttamento, non più mascherati e taciuti, ma accettati apertamente come dati "naturali" e necessari a chi detiene il potere.

Per molti aspetti la concezione della società che si può ricavare dalle affermazioni teoriche di Verga e soprattutto dalla sua rappresentazione della realtà rientra nell'ambito culturale del "darwinismo sociale". Per Verga la società a tutti i suoi livelli, dai più elevati ai più bassi, è dominata da uno spietato antagonismo tra individui, gruppi e classi: le leggi che la regolano sono la sopraffazione del più forte sul più debole e l'interesse individuale. E questa condizione è un dato di natura, sostanzialmente eguale in tutti i tempi e in tutti i luoghi.

\subsection{Microsaggio: Lo straniamento}

\paragraph{Una definizione} Nelle opere veriste di Verga troviamo abbondantemente usato il procedimento narrativo dello straniamento. Esso fu definito teoricamente dai formalisti russi degli anni Venti, una corrente 
critica i cui rappresentanti principali erano Viktor Šklovskij, Boris Ejchenbaum, Jurij Tynjanov, Boris Tomaševskij e che insisteva sugli aspetti tecnici e formali dell'arte, giungendo addirittura a identificare l'arte con l'«artificio, cioè con i procedimenti tecnici mediante cui si costruisce il discorso letterario. Lo straniamento consiste nell'adottare, per narrare un fatto e descrivere una persona, un punto di vista completamente estraneo all'oggetto. Famoso ad esempio è un racconto di Tolstoj, \textsc{Cholstomer}, in cui i rapporti umani sono riflessi nell'ipotetica psicologia di un cavallo. Il risultato è che le cose più abituali, "normali", presentate attraverso un punto di vista estraneo, appaiono insolite, strane, incomprensibili.

\paragraph{Lo straniamento nell'opera verghiana} Questo avviene frequentemente nei racconti e nei romanzi verghiani. Nei \textsc{Malavoglia} ad esempio i sentimenti autentici e disinteressati che sono propri dei protagonisti vengono spesso filtrati attraverso il punto di vista della collettività del villaggio che a quei valori è completamente insensibile e che giudica solo in base al principio dell'interesse economico e del diritto del più forte. Di conseguenza ciò che è "normale", secondo la scala di valori universalmente accettata e partecipata dal lettore, finisce per apparire "strano", subisce una deformazione che ne stravolge la fisionomia. Ad esempio l'onestà di padron 'Ntoni, che pur di non mancar di parola riguardo al debito lascia che la sua casa venga pignorata, si trasforma in una vera e propria truffa nell'ottica stravolta di padron Cipolla, che accettava per nuora Mena Malavoglia solo se portava in dote delle proprietà; e sempre per lo stesso motivo padron 'Ntoni viene giudicato «minchione>> dalla comunità, perché incapace di fare i suoi affari; cosi pure la purezza dei sentimenti che uniscono Alfio e Mena viene deformata dall'ottica grosso lana di zio Crocifisso in una «rabbia di maritarsi>>; e gli esempi potrebbero continuare all'infinito.

Questo tipo di straniamento compare quando sono in scena personaggi "ideali", come i Malavoglia, che sono l'antitesi del punto di vista dominante della narrazione. Ma quando sono in scena i loro antagonisti, i personaggi gretti, meschini e insensibili sino alla crudeltà che compongono il "coro" del villaggio, si verifica una forma di straniamento per così dire "rovesciata": infatti, siccome il punto di vista di chi racconta è perfettamente in armonia con quello dei personaggi, il loro comportamento ottuso e crudele, invece di apparire nella sua vera luce, viene presentato come se fosse normale, o addirittura degno di approvazione. Come si vede è questo l'esatto rovescio del procedimento abituale dello straniamento, che abbiamo prima indicato: là ciò che era "normale" appariva "strano", qui ciò che è "strano" appare "normale" (tale procedimento è stato individuato in Verga da R. Luperini, \textsc{L'orgoglio e la disperata rassegnazione}, Savelli, Roma 1974).

Si veda ad esempio l'episodio già citato del pignoramento della casa del nespolo: il comportamento abietto di Piedipapera, che fa da prestanome a zio Crocifisso per spogliare i Malavoglia e va in giro dicendo che essi sono «una manica di carogne», disonesti, avari e prepotenti, è guardato dal "narratore" popolare come se fosse cosa ovvia e giusta, senza il minimo moto di ripugnanza e di critica. Questa connivenza tra il "narratore" e la crudeltà o l'avidità interessata di un personaggio è forse esemplificata nella maniera più chiara e persuasiva nella novella \textsc{La roba}: qui il "narratore" non dimostra mai riprovazione nei confronti di Mazzarò e dei metodi da lui usati per arricchire la sua avarizia, la sordità ad ogni affetto familiare, la brutalità nei confronti dei lavoranti, la disumanità verso i fittavoli rovinati e ridotti alla fame dal suo contegno di usuraio, le malversazioni e i raggiri; anzi il comportamento di Mazzarò non appare solo "normale", ma addirittura eroico e degno di encomio.

\subsection{Trama}

Nel romanzo c'è una famiglia di pescatori, composta da un nonno, suo
figlio e i nipoti. La famiglia Malavoglia, ed in particolare il nonno, è
il depositario di quei \evidenziatore{valori che Verga ritiene positivi}
. È uno dei pochissimi personaggi di Verga che ci permette di
comprendere la posizione di Verga rispetto a determinati valori: la
correttezza, l'onestà, il senso della famiglia, del lavoro, il rispetto
della parola data.

Questo non significa che Verga proponga i valori positivi del nonno come
valori accettabili e proponibili: sarà proprio lui, infatti, a
\evidenziatore{distruggere la famiglia} . Il suo tentativo di migliorare
la condizione della famiglia, andando \textbf{volontario} a fare
servizio di leva, inizieranno i problemi, dovuti principalmente alla
carenza di braccia e di forza-lavoro.

Padron 'Ntoni (il nonno) per far del bene decide di lasciarsi tentare
dal commercio dei
\href{https://it.wikipedia.org/wiki/Lupinus_albus}{lupini}. Questa è
l'inizio della rovina. Dovranno poi vendere la casa e la barca perché
vuole \textbf{mantenere la parola data}. La famiglia subisce un
declassamento sociale ed economico. Egli morirà, alla fine, in quel
famoso ospedale in cui non avrebbe mai voluto andare.

La visione di Verga è estremamente pessimistica.

All'interno della famiglia i \evidenziatore{valori opposti} sono
rappresentati dal nonno 'Ntoni e dal nipote 'Ntoni, attratto dalla
modernità. All'interno del villaggio, invece, la famiglia Malavoglia
rappresenta i valori positivi, in contrasto con i valori negativi
diffusi nel villaggio. Con lo straniamento, infatti, il narratore,
descrivendo i Malavoglia negativamente (padron 'Ntoni viene definito
\emph{minchione}), assume la focalizzazione corale del resto del
villaggio.

\subsection{T: \textsc{Il mondo arcaico e l'irruzione della storia}}

Prima dell'unificazione italiana questo mondo arcaico sembra quasi
immobile, mentre dopo tutto cambia

Lo stesso Verga aveva detto di non voler usare un narratore onnisciente
che parli e racconti, ed infatti in questa introduzione vediamo come la
famiglia sia presentata quasi da un passante. Successivamente i
personaggi si introdurranno da soli
\setcounter{mar}{0}

\begin{quote}
Un tempo i Malavoglia erano stati numerosi come i sassi della strada vecchia di Trezza; ce n’erano persino ad Ognina, e ad Aci Castello, tutti buona e brava gente di mare, proprio all’opposto di quel che sembrava dal nomignolo, come dev’essere. Veramente nel libro della parrocchia si chiamavano Toscano, ma questo non voleva dir nulla, poichè da che il mondo era mondo, all’Ognina, a Trezza e ad Aci Castello, li avevano sempre conosciuti per Malavoglia, di padre in figlio, che avevano sempre avuto delle barche sull’acqua, e delle tegole al sole. Adesso a Trezza non rimanevano che i Malavoglia di padron ’Ntoni, quelli della casa del nespolo, e della Provvidenza ch’era ammarrata sul greto, sotto il lavatoio, accanto alla Concetta dello zio Cola, e alla paranza di padron Fortunato Cipolla.

[...]

Diceva pure: —\textit{ Gli uomini son fatti come le dita della mano: il dito grosso deve far da dito grosso, e il dito piccolo deve far da dito piccolo.}\mat{Uno dei momenti in cui Verga descrive la mentalità e il pensiero di Padron 'Ntoni} —

E la famigliuola di padron ’Ntoni era realmente disposta come le dita della mano. Prima veniva lui, il dito grosso, che comandava le feste e le quarant’ore; poi suo figlio Bastiano, Bastianazzo, perchè era grande e grosso quanto il San Cristoforo che c’era dipinto sotto l’arco della pescheria della città; e così grande e grosso com’era filava diritto alla manovra comandata, e non si sarebbe soffiato il naso se suo padre non gli avesse detto «sóffiati il naso» tanto che s’era tolta in moglie la Longa quando gli avevano detto «pigliatela». Poi veniva la Longa, una piccina che badava a tessere, salare le acciughe, e far figliuoli, da buona massaia; infine i nipoti, in ordine di anzianità: ’Ntoni il maggiore, un bighellone di vent’anni, che si buscava tutt’ora qualche scappellotto dal nonno, e qualche pedata più giù per rimettere l’equilibrio, quando lo scappellotto era stato troppo forte; Luca, «che aveva più giudizio del grande» ripeteva il nonno; Mena (Filomena) soprannominata «Sant’Agata» perchè stava sempre al telaio, e si suol dire «donna di telaio, gallina di pollaio, e triglia di gennaio»; Alessi (Alessio) un moccioso tutto suo nonno colui!; e Lia (Rosalia) ancora nè carne nè pesce. — Alla domenica, quando entravano in chiesa, l’uno dietro l’altro, pareva una processione.

Padron ’Ntoni sapeva anche certi motti e proverbi che aveva sentito dagli antichi: «Perchè il motto degli antichi mai mentì»: — «Senza pilota barca non cammina» — «Per far da papa bisogna saper far da sagrestano» — oppure — «Fa il mestiere che sai, che se non arricchisci camperai» — «Contentati di quel che t’ha fatto tuo padre; se non altro non sarai un birbante» ed altre sentenze giudiziose.

Ecco perchè la casa del nespolo prosperava, e padron ’Ntoni passava per testa quadra, al punto che a Trezza l’avrebbero fatto consigliere comunale, se don Silvestro, il segretario, il quale la sapeva lunga, non avesse predicato che era un codino marcio, un reazionario di quelli che proteggono i Borboni, e che cospirava pel ritorno di Franceschello, onde poter spadroneggiare nel villaggio, come spadroneggiava in casa propria.

Padron ’Ntoni invece non lo conosceva neanche di vista Franceschello, e badava agli affari suoi, e soleva dire: «Chi ha carico di casa non può dormire quando vuole» perchè «chi comanda ha da dar conto».

Nel dicembre 1863, ’Ntoni, il maggiore dei nipoti, era stato chiamato per la leva di mare. Padron ’Ntoni allora era corso dai pezzi grossi del paese, che son quelli che possono aiutarci.\mat{Abbiamo l'esordio del romanzo. L'irruzione della storia rompe l'equilibrio e l'immobilismo all'interno di questo semplice villaggio}
\end{quote}
\setcounter{mar}{0}

\paragraph{Le tecniche narrative} Come in \textsc{Rosso Malpelo} anche nel romanzo si può cogliere subito dalle prime
righe la novità radicale introdotta da Verga nell'impostazione narrativa. Il racconto non è condotto dal narratore onnisciente proprio del modello manzoniano, che dà ampie informazioni sull'ambiente, sugli antefatti della vicenda, sui personaggi che entrano in scena; la voce che narra proviene chiaramente dall'interno del mondo rappresentato e si colloca al livello culturale dei personaggi stessi. Numerosi indizi ce ne avvertono: il narratore dà per scontato che il soprannome «Malavoglia» sia giusto l'opposto rispetto alle caratteristiche effettive della famiglia («come dev'essere»), e basta questo a rivelare che è dentro l'orizzonte culturale di quella comunità di pescatori e contadini del villaggio siciliano, dove l'uso della 'ngiuria è comune. Ancora più esplicita è l'affermazione sui «pezzi grossi» che «son quelli che possono aiutarci», dove il pronome «ci» testimonia come chi narra faccia parte delle <<basse sfere». Si possono poi notare modi di dire tipici e riferimenti a particolari legati a un determinato ambiente («comandava le feste e le quarant'ore», «come i sassi della strada vecchia di Trezza», «grande e grosso quanto il San Cristoforo che c'era dipinto sotto l'arco della pescheria della città», «festa di Trecastagni»), mosse colloquiali («un moccioso tutto suo nonno colui!»). Non solo, ma si parla di certi personaggi, padron Ntoni, zio Cola, padron Fortunato Cipolla, Sara di comare Tudda, comare Venera «la Zuppidda», come se fossero perfettamente noti e non ci fosse bisogno di fornire ragguagli su di essi: è la conferma migliore del fatto che il narratore fa parte di quel piccolo mondo paesano. Troviamo così perfettamente realizzato il principio che Verga enuncia in una lettera a Capuana, di poco posteriore all'uscita del romanzo: «La confusione che dovevano produrvi in mente alle prime pagine tutti quei personaggi messivi faccia a faccia senza nessuna presentazione, come li aveste conosciuti sempre, e foste nato e vissuto in mezzo a loro, [...] era artificio voluto e cercato anch'esso, per evitare, perdonami il bisticcio, ogni artificio letterario, per darvi l'illusione completa della realtà», concetto ribadito poco dopo in una lettera al critico Felice Cameroni: «Io mi son messo in pieno, e fin dal principio, in mezzo ai miei personaggi e ci ho condotto il lettore, come ei li avesse tutti conosciuti diggià, e più vissuto con loro e in quell'ambiente sempre. Parmi questo il modo migliore per darci completa l'illusione della realtà». Come si vede, Verga è uno scrittore che ha un'avvertita coscienza critica dei procedimenti che mette in atto nel costruire i suoi testi. E sappiamo quali siano le vaste implica zioni di questo eclissarsi dell'autore, di questo suo regredire nella realtà rappresentata: la teoria dell'impersonalità, il rifiuto del giudizio, il pessimismo materialistico.

\paragraph{Le tematiche} Da queste prime pagine si delineano altresì le tematiche centrali del romanzo. Assume innanzitutto rilievo la figura di padron 'Ntoni, il vecchio patriarca, depositario dei valori di una società arcaica, in particolare quello dell'unità della famiglia («bisogna che le cinque dita s'aiutino l'un l'altro»). Padron 'Ntoni è portavoce di una mentalità tradizionalista, che concepisce la società come qualche cosa di immobile, in cui ciascuno è in chiodato alla sua condizione come da un destino immutabile. Questa mentalità immobilista si esprime attraverso la sapienza secolare dei proverbi, in cui il vecchio ha una fede incrollabile («Fa il mestiere che sai, che se non arricchisci camperai»; «Contentati di quel che t'ha fatto tuo padre; se non altro non sarai un birbante»). È una mentalità che il narratore condivide in pieno, poiché definisce «giudiziose» tali «sentenze». In questo mondo che sembrerebbe così immobile irrompe però la storia, che porta trasformazioni sconvolgenti. La storia si presenta in queste prime pagine nella forma della leva militare, introdotta dal nuovo Stato unitario, che sottrae alla famiglia, fondamentale unità produttiva in quella società arcaica, le braccia indispensabili per il lavoro. Di qui avrà origine l'impulso che spingerà il vecchio patriarca a trasgredire i principi secolari in cui crede e a mutare «mestiere», intraprendendo quel commercio dei lupini da cui scaturirà tutta la serie di sventure che colpirà la famiglia.

Si delinea anche da queste pagine l'opposizione costitutiva su cui si fonda la struttura romanzesca, quella che si determina tra il nucleo dei Malavoglia, depositari dei valori, e l'ambiente del villaggio, malevolo e pettegolo, sordo ai valori e ispirato solo ai principi dell'interesse egoistico. Le figure rappresentative di questo sfondo negativo sono don Silvestro, con le sue insinuazioni su padron 'Ntoni «codino marcio», il vicario, con le sue ottuse idee reazionarie, il farmacista, con le sue velleitarie e fasulle posizioni democratiche, la Zuppidda, con i suoi pettegolezzi velenosi sulla relazione fra ’Ntoni e Sara.

Questa sordità umana e questa meschinità soffocante risaltano soprattutto a contrasto con la delicatezza d'animo di padron 'Ntoni, che capisce immediatamente il dolore di Maruzza (la Longa) per il distacco dal figlio e invita perciò il marito a consolarla («Va a dirle qualche cosa, a quella poveretta; non ne può più»), poi, il giorno della partenza del giovane, le paga due centesimi di acqua col limone, «indovinando che la nuora dovesse avere la bocca amara». Dietro l'asciutto pudore di questi gesti si coglie la nobiltà e la profondità dei sentimenti del vecchio patriarca. Parimenti ricca di umanità è Maruzza, con il suo dolore di madre, e significativo è il dolente silenzio di Bastianazzo, che non ha l'animo di aprir bocca dinanzi alla sventura.

\subsection{T: \textsc{I Malavoglia e la comunità del villaggio: valori ideali e interesse economico}}

Qui c'è lo \textbf{straniamento rovesciato}: Zio crocifisso appare come
personaggio positivo, a causa della focalizzazione. Egli è un usuraio.

\begin{quote}
Il peggio era che i lupini li avevano presi a credenza, e lo zio Crocifisso non si contentava di «buone parole e mele fradicie», per questo lo chiamavano Campana di legno, perchè non ci sentiva di quell’orecchio, quando lo volevano pagare con delle chiacchiere, e’ diceva che «alla credenza ci si pensa». Egli era un buon diavolaccio, e viveva imprestando agli amici, non faceva altro mestiere, che per questo stava in piazza tutto il giorno, colle mani nelle tasche, o addossato al muro della chiesa, con quel giubbone tutto lacero che non gli avreste dato un baiocco; ma aveva denari sin che ne volevano, e se qualcheduno andava a chiedergli dodici tarì glieli prestava subito, col pegno, perchè «chi fa credenza senza pegno, perde l’amico, la roba e l’ingegno» a patto di averli restituiti la domenica, d’argento e colle colonne, che ci era un carlino dippiù, com’era giusto, perchè «coll’interesse non c’è amicizia». Comprava anche la pesca tutta in una volta, con ribasso, e quando il povero diavolo che l’aveva fatta aveva bisogno subito di denari, ma dovevano pesargliela colle sue bilancie, le quali erano false come Giuda, dicevano quelli che non erano mai contenti, ed hanno un braccio lungo e l’altro corto, come san Francesco; \textit{e anticipava anche la spesa per la ciurma, se volevano, e prendeva soltanto il denaro anticipato, e un rotolo di pane a testa, e mezzo quartuccio di vino, e non voleva altro}\mat{si prendeva tutto: enumerazioni e congiunzioni
  ripetitive, con forte uso dell'ironia}, chè era cristiano e di quel che faceva in questo mondo avrebbe dovuto dar conto a Dio. Insomma era la provvidenza per quelli che erano in angustie, e aveva anche inventato cento modi di render servigio al \textit{prossimo}\mat{termine religioso; particolarmente
    significativo il fatto che avesse materiale per andare per mare senza
    andarci davvero: andare per mare era pericoloso}, e senza essere uomo di mare aveva barche, e attrezzi, e ogni cosa, per quelli che non ne avevano, e li prestava, contentandosi di prendere un terzo della pesca, più la parte della barca, che contava come un uomo della ciurma, e quella degli attrezzi, se volevano prestati anche gli attrezzi, e finiva che la barca si mangiava tutto il guadagno, tanto che la chiamavano la barca del diavolo — e quando gli dicevano perchè non ci andasse lui a rischiare la pelle come tutti gli altri, che si pappava il meglio della pesca senza pericolo, rispondeva: — Bravo! e se in mare mi capita una disgrazia, Dio liberi, che ci lascio le ossa, chi me li fa gli affari miei? — Egli badava agli affari suoi, ed avrebbe prestato anche la camicia; ma poi voleva esser pagato, senza tanti cristi; ed era inutile stargli a contare ragioni, perchè era sordo, e per di più era scarso di cervello, e non sapeva dir altro che «Quel che è di patto non è d’inganno», oppure «Al giorno che promise si conosce il buon pagatore».

Ora i suoi nemici gli ridevano sotto il naso, a motivo di quei lupini che se l’era mangiati il diavolo; e gli toccava anche recitare il deprofundis per l’anima di Bastianazzo, quando si facevano le esequie, insieme con gli altri confratelli della Buona Morte, colla testa nel sacco.

[...]

La casa del nespolo era piena di gente; e il proverbio dice: «triste quella casa dove ci è la visita pel marito!» Ognuno che passava, al vedere sull’uscio quei piccoli Malavoglia col viso sudicio e le mani nelle tasche, scrollava il capo e diceva:

— Povera comare Maruzza! ora cominciano i guai per la sua casa!

Gli amici portavano qualche cosa\mat{si parla della giornata del consolo; il narratore ha
  appena descritto la situazione della famiglia, disperata per la morte
  del figlio; all'esterno della casa dialoghi mostrano la grettezza e
  l'insensibilità del resto della comunità.}, com’è l’uso, pasta, ova, vino e ogni ben di Dio, che ci avrebbe voluto il cor contento per mangiarsi tutto, e perfino compar Alfio Mosca era venuto con una gallina per mano. — Prendete queste qua, gnà Mena, — diceva, — che avrei voluto trovarmici io al posto di vostro padre, vi giuro. Almeno non avrei fatto danno a nessuno, e nessuno avrebbe pianto.

La Mena, appoggiata alla porta della cucina, colla faccia nel grembiule, si sentiva il cuore che gli sbatteva e gli voleva scappare dal petto, come quelle povere bestie che teneva in mano. La dote di Sant’Agata se n’era andata colla Provvidenza, e quelli che erano a visita nella casa del nespolo, pensavano che lo zio Crocifisso ci avrebbe messo le unghie addosso.

Alcuni se ne stavano appollaiati sulle scranne, e ripartivano senza aver aperto bocca, da veri baccalà che erano; ma chi sapeva dir quattro parole, cercava di tenere uno scampolo di conversazione, per scacciare la malinconia, e distrarre un po’ quei poveri Malavoglia i quali piangevano da due giorni come fontane\mat{giorno del consolo: cerimonia che si svolge quando
  c'è lutto in casa di defunti, amici portano cibo}. Compare Cipolla raccontava che sulle acciughe c’era un aumento di due tarì per barile, questo poteva interessargli a padron ’Ntoni, se ci aveva ancora delle acciughe da vendere; lui a buon conto se n’era riserbati un centinaio di barili; e parlavano pure di compare Bastianazzo, buon’anima, che nessuno se lo sarebbe aspettato, un uomo nel fiore dell’età, e che crepava di salute, poveretto!\mat{rapporto ossimorico tra i Malavoglia e il resto
    della comunità; \textbf{crepava di salute}: espressione fuori luogo}

[...]

Don Silvestro per far ridere un po’ tirò il discorso sulla tassa di successione di compar Bastianazzo, e ci ficcò così una barzelletta che aveva raccolta dal suo avvocato, e gli era piaciuta tanto, quando gliel’avevano spiegata bene, che non mancava di farla cascare nel discorso ogniqualvolta si trovava a visita da morto.

— Almeno avete il piacere di essere parenti di Vittorio Emanuele, giacchè dovete dar la sua parte anche a lui!

E tutti si tenevano la pancia dalle risate, chè il proverbio dice: «Nè visita di morto senza riso, nè sposalizio senza pianto».\mat{battuta che gira il coltello nella piaga: i
  Malavoglia sono senza soldi: esempio di assoluta insensibilità}

[...]

Don Silvestro faceva il gallo colle donne, e si muoveva ogni momento col pretesto di offrire le scranne ai nuovi arrivati, per far scricchiolare le sue scarpe verniciate. — Li dovrebbero abbruciare, tutti quelli delle tasse! — brontolava comare Zuppidda, gialla come se avesse mangiato dei limoni, e glielo diceva in faccia a don Silvestro, quasi ei fosse quello delle tasse. — Ella lo sapeva benissimo quello che volevano certi mangiacarte che non avevano calze sotto gli stivali inverniciati, e cercavano di ficcarsi in casa della gente per papparsi la dote e la figliuola: «Bella, non voglio te, voglio i tuoi soldi». Per questo aveva lasciata a casa sua figlia Barbara. — Quelle facce lì non mi piacciono.

— A chi lo dite! — esclamò padron Cipolla; — a me mi scorticano vivo come san Bartolomeo.

— Benedetto Dio! — esclamò mastro Turi Zuppiddo, minacciando col pugno che pareva la malabestia del suo mestiere. — Va a finire brutta, va a finire, con questi italiani!

— Voi state zitto! — gli diede sulla voce comare Venera, — chè non sapete nulla.

— Io dico quel che hai detto tu, che ci levano la camicia di dosso, ci levano! — borbottò compare Turi, mogio mogio.

Allora Piedipapera, per tagliar corto, disse piano a padron Cipolla: — Dovreste pigliarvela voi, comare Barbara, per consolarvi; così la mamma e la figliuola non si darebbero più l’anima al diavolo.

— È una vera porcheria! — esclamava donna Rosolina, la sorella del curato, rossa come un tacchino, e facendosi vento col fazzoletto; e se la prendeva con Garibaldi che metteva le tasse, e al giorno d’oggi non si poteva più vivere, e nessuno si maritava più. — O a donna Rosolina cosa gliene importa oramai? — susurrava Piedipapera. Donna Rosolina intanto raccontava a don Silvestro le grosse faccende che ci aveva per le mani: dieci canne di ordito sul telaio, i legumi da seccare per l’inverno, la conserva dei pomidoro da fare, che lei ci aveva un segreto tutto suo per avere la conserva dei pomidoro fresca tutto l’inverno. — Una casa senza donna non poteva andare; ma la donna bisognava che avesse il giudizio nelle mani, come s’intendeva lei; e non fosse di quelle fraschette che pensano a lisciarsi e nient’altro, «coi capelli lunghi e il cervello corto», chè allora un povero marito se ne va sott’acqua come compare Bastianazzo, buon’anima. — Beato lui! — sospirava la Santuzza, — è morto in un giorno segnalato, la vigilia dei Dolori di Maria Vergine, e prega lassù per noi peccatori, fra gli angeli e i santi del paradiso. «A chi vuol bene Dio manda pene». Egli era un bravo uomo, di quelli che badano ai fatti loro, e non a dir male di questo e di quello, e peccare contro il prossimo, come tanti ce ne sono.

Maruzza allora, seduta ai piedi del letto, pallida e disfatta come un cencio messo al bucato, che pareva la Madonna Addolorata, si metteva a piangere più forte, col viso nel guanciale, e padron ’Ntoni, piegato in due, più vecchio di cent’anni, la guardava, e la guardava, scrollando il capo, e non sapeva che dire, per quella grossa spina di Bastianazzo che ci aveva in cuore, come se lo rosicasse un pescecane.

— La Santuzza ci ha il miele in bocca! osservava comare Grazia Piedipapera.

— Per fare l’ostessa, rispose la Zuppidda, — e’ s’ha ad essere così. «Chi non sa l’arte chiuda bottega, e chi non sa nuotare che si anneghi».

[...]

— Metteranno pure la tassa sul sale!\mat{la storia è presente anche nel paese} — aggiunse compare Mangiacarrubbe. — L’ha detto lo speziale che è stampato nel giornale. Allora di acciughe salate non se ne faranno più, e le barche potremo bruciarle nel focolare.

Mastro Turi il calafato stava per levare il pugno e incominciare: — Benedetto Dio! — ; ma guardò sua moglie e si tacque mangiandosi fra i denti quel che voleva dire.

— Colla malannata che si prepara, — aggiunse padron Cipolla, che non pioveva da santa Chiara, — e se non fosse stato per l’ultimo temporale in cui si è persa la Provvidenza, che è stata una vera grazia di Dio, la fame quest’inverno si sarebbe tagliata col coltello!\mat{si evidenzia lo scontro tra ideali delle due
  parti: la considerazione può avere senso, ma è totalmente insensibile}

Ognuno raccontava i suoi guai, anche per conforto dei Malavoglia, che non erano poi i soli ad averne. «Il mondo è pieno di guai, chi ne ha pochi e chi ne ha assai», e quelli che stavano fuori nel cortile guardavano il cielo, perchè un’altra pioggerella ci sarebbe voluta come il pane. Padron Cipolla lo sapeva lui perchè non pioveva più come prima. — Non piove più perchè hanno messo quel maledetto filo del telegrafo, che si tira tutta la pioggia, e se la porta via. — Compare Mangiacarrubbe allora, e Tino Piedipapera rimasero a bocca aperta, perchè giusto sulla strada di Trezza c’erano i pali del telegrafo; ma siccome don Silvestro cominciava a ridere, e a fare ah! ah! ah! come una gallina, padron Cipolla si alzò dal muricciuolo infuriato e se la prese con gli ignoranti, che avevano le orecchie lunghe come gli asini. — Che non lo sapevano che il telegrafo portava le notizie da un luogo all’altro; questo succedeva perchè dentro il filo ci era un certo succo come nel tralcio della vite, e allo stesso modo si tirava la pioggia dalle nuvole, e se la portava lontano, dove ce n’era più di bisogno\mat{irruzione del progresso}; potevano andare a domandarlo allo speziale che l’aveva detta; e per questo ci avevano messa la legge che chi rompe il filo del telegrafo va in prigione. Allora anche don Silvestro non seppe più che dire, e si mise la lingua in tasca.

— Santi del Paradiso! si avrebbero a tagliarli tutti quei pali del telegrafo, e buttarli nel fuoco! — incominciò compare Zuppiddo, ma nessuno gli dava retta, e guardavano nell’orto, per mutar discorso.

— Un bel pezzo di terra! — diceva compare Mangiacarrubbe; — quando è ben coltivato dà la minestra per tutto l’anno.

La casa dei Malavoglia era sempre stata una delle prime a Trezza; ma adesso colla morte di Bastianazzo, e ’Ntoni soldato, e Mena da maritare, e tutti quei mangiapane pei piedi, era una casa che faceva acqua da tutte le parti.

Infine cosa poteva valere la casa? Ognuno allungava il collo sul muro dell’orto, e ci dava una occhiata, per stimarla così a colpo. Don Silvestro sapeva meglio di ogni altro come andassero le cose, perchè le carte le aveva lui, alla segreteria di Aci-Castello.

— Volevo scommettere dodici tarì che non è tutt’oro quello che luccica, andava dicendo; e mostrava ad ognuno il pezzo da cinque lire nuovo.

Ei sapeva che sulla casa c’era un censo di cinque tarì all’anno. Allora si misero a fare il conto sulle dita di quel che avrebbe potuto vendersi la casa, coll’orto, e tutto.\mat{ accenno al declassamento dei Malavoglia; il
  comportamento sublime dei Malavoglia è distrutto dalle chiacchiere
  della gente: Verga ci vuole comunicare che in questo mondo
  utilitaristico non c'è più spazio per questo.}

— Nè la casa nè la barca si possono vendere perchè ci è su la dote di Maruzza, — diceva qualchedun altro, e la gente si scaldava tanto che potevano udirli dalla camera dove stavano a piangere il morto. — Sicuro! — lasciò andare alfine don Silvestro come una bomba; — c’è l’ipoteca dotale.

Padron Cipolla, il quale aveva scambiato qualche parola con padron ’Ntoni per maritare Mena con suo figlio Brasi, scrollava il capo e non diceva altro.

— Allora, — aggiunse compare Cola, — il vero disgraziato è lo zio Crocifisso che ci perde il credito dei suoi lupini.

Tutti si voltarono verso Campana di legno il quale era venuto anche lui, per politica, e stava zitto, in un cantuccio, a veder quello che dicevano, colla bocca aperta e il naso in aria, che sembrava stesse contando quante tegole e quanti travicelli c’erano sul tetto, e volesse stimare la casa. I più curiosi allungavano il collo dall’uscio, e si ammiccavano l’un l’altro per mostrarselo a vicenda. — E’ pare l’usciere che fa il pignoramento! — sghignazzavano.

Le comari che sapevano delle chiacchiere fra padron ’Ntoni e compare Cipolla, dicevano che adesso bisognava passare la doglia, a comare Maruzza, e conchiudere quel matrimonio della Mena. Ma la Longa in quel momento ci aveva altro pel capo, poveretta.

Padron Cipolla voltò le spalle freddo freddo, senza dir nulla; e dopo che tutti se ne furono andati, i Malavoglia rimasero soli nel cortile. — Ora, — disse padron ’Ntoni, siamo rovinati, ed è meglio per Bastianazzo che non ne sa nulla.

A quelle parole, prima Maruzza, e poi tutti gli altri tornarono a piangere di nuovo, e i ragazzi, vedendo piangere i grandi, si misero a piangere anche loro, sebbene il babbo fosse morto da tre giorni. Il vecchio andava di qua e di là, senza sapere che facesse; Maruzza invece non si muoveva dai piedi del letto, quasi non avesse più nulla da fare. Quando diceva qualche parola, ripeteva sempre, cogli occhi fissi, e pareva che non ci avesse altro in testa. — Ora non ho più niente da fare!

— No! — rispose padron ’Ntoni, no! chè bisogna pagare il debito allo zio Crocifisso, e non si deve dire di noi che «il galantuomo come impoverisce diventa birbante».\mat{codice comportamentale moralmente giusto di
  Padron 'Ntoni}

E il pensiero dei lupini gli ficcava più dentro nel cuore la spina di Bastianazzo. Il nespolo lasciava cadere le foglie vizze, e il vento le spingeva di qua e di là pel cortile.

— Egli è andato perchè ce l’ho mandato io, — ripeteva padron ’Ntoni, — come il vento porta quelle foglie di qua e di là, e se gli avessi detto di buttarsi dal fariglione con una pietra al collo, l’avrebbe fatto senza dir nulla. Almeno è morto che la casa e il nespolo sino all’ultima foglia erano ancora suoi; ed io che son vecchio sono ancora qua. «Uomo povero ha i giorni lunghi».

Maruzza non diceva nulla, ma nella testa ci aveva un pensiero fisso, che la martellava, e le rosicava il cuore, di sapere cos’era successo in quella notte, che l’aveva sempre dinanzi agli occhi, e se li chiudeva le sembrava di vedere ancora la Provvidenza, là verso il Capo dei Mulini, dove il mare era liscio e turchino, e seminato di barche, che sembravano tanti gabbiani al sole, e si potevano contare ad una ad una, quella dello zio Crocifisso, l’altra di compare Barabba, la Concetta dello zio Cola, e la paranza di padron Fortunato, che stringevano il cuore; e si udiva mastro Cola Zuppiddo il quale cantava a squarciagola, con quei suoi polmoni di bue, mentre picchiava colla malabestia, e l’odore del catrame che veniva dal greto, e la tela che batteva la cugina Anna sulle pietre del lavatoio, e si udiva pure Mena a piangere cheta cheta in cucina.

— Poveretta! — mormorava il nonno, — anche a te è crollata la casa sul capo, e compare Fortunato se ne è andato freddo freddo, senza dir nulla.

E andava toccando ad uno ad uno gli arnesi che erano in mucchio in un cantuccio, colle mani tremanti, come fanno i vecchi; e vedendo Luca lì davanti, che gli avevano messo il giubbone del babbo, e gli arrivava alle calcagna, gli diceva: — Questo ti terrà caldo, quando verrai a lavorare; perchè adesso bisogna aiutarci tutti per pagare il debito dei lupini.

Maruzza si tappava le orecchie colle mani per non sentire la Locca che si era appollaiata sul ballatoio, dietro l’uscio, e strillava dalla mattina, con quella voce fessa di pazza, e pretendeva che le restituissero loro il suo figliuolo, e non voleva sentir ragione.

— Fa così perchè ha fame, — disse infine la cugina Anna; adesso lo zio Crocifisso ce l’ha con tutti loro per quell’affare dei lupini, e non vuol darle più nulla. Ora vo a portarle qualche cosa, e allora se ne andrà.

La cugina Anna, poveretta, aveva lasciato la sua tela e le sue ragazze per venire a dare una mano a comare Maruzza, la quale era come se fosse malata, e se l’avessero lasciata sola non avrebbe pensato più ad accendere il fuoco, e a mettere la pentola, che sarebbero tutti morti di fame. «I vicini devono fare come le tegole del tetto, a darsi l’acqua l’un l’altro». Intanto quei ragazzi avevano le labbra pallide dalla fame. La Nunziata aiutava anche lei, e Alessi, col viso sudicio dal gran piangere che aveva fatto vedendo piangere la mamma, teneva a bada i piccini, perchè non le stessero sempre fra i piedi, come una nidiata di pulcini, chè la Nunziata voleva averle libere le mani, lei.

— Tu sai il fatto tuo! — le diceva la cugina Anna; — e la tua dote ce l’hai nelle mani, quando sarai grande.
\end{quote}

\paragraph{L'osmosi tra narratore e personaggio} Il ritratto di zio Crocifisso che apre il capitolo offre un bell'esempio dell'originalissima impostazione narrativa del romanzo. Il personaggio è presentato dall'ottica di un narratore che condivide del tutto la visione di un ambiente dominato solo dalla logica dell'interesse; oppure (e l'incertezza testimonia quanto la costruzione della pagina sia complessa e sfumata), si può dire che la voce narrante è il riflesso del punto di vista di Crocifisso stesso, riecheggia il suo modo di vedere le cose e il suo modo di esprimersi. È un procedimento abituale nei \textsc{Malavoglia}: siccome narratore e personaggi hanno la stessa mentalità e lo stesso linguaggio, spesso è difficile stabilire se il discorso appartiene all'uno o agli altri. Nel caso di zio Crocifisso, il risultato di questa ambigua osmosi è che il ritratto dell'usuraio avido e disumano risulta molto benevolo («era la provvidenza per quelli che erano in angustie»), la mancanza di scrupoli con cui strappa i suoi profitti appare perfettamente naturale («ci era un carlino dippiù, com'era giusto»), o si rovescia addirittura in comportamento benefico e meritorio («aveva anche inventato cento modi di rendere servigio al prossimo»); se si affaccia qualche aspetto negativo (le bilance «false come Giuda»), esso è attribuito alla malevolenza di quelli che <<non erano mai contenti». Si verifica qui il tipico procedimento di straniamento "rovesciato" che si è già individuato in \textsc{Rosso Malpelo}: ciò che è strano, abnorme e ripugnante (l'avidità spietata dell'usuraio), venendo filtrato da un punto di vista che condivide la visione del personaggio stesso appare normale, giusto e perfino lodevole. Come di consueto, questo tipo di straniamento fa risaltare lo stravolgimento profondo dei valori che si verifica in quella piccola comunità rurale; uno stravolgimento che la rende in tutto equivalente alla società evoluta, borghese e cittadina

\paragraph{Il coro del paese} Il fitto chiacchierio che percorre tutta la scena successiva della visita del consòlo non è una colorita scena di commedia, costruita per suscitare il sorriso indulgente sull'ingenuità primitiva di quei popolani, come è stato detto da taluni critici. Emergono al contrario la chiusura mentale, la grettezza interessata, l'insensibilità ai limiti della crudeltà che sono proprie della comunità paesana, e che lasciano un'impressione cupa, desolata, soffocante. Si pensi solo all'agghiacciante battuta di padron Cipolla, sull'«ultimo temporale in cui si è persa la Provvidenza», che «è stato una vera grazia di Dio» per le sue colture agricole. La comicità di Verga non è mai serena e liberatoria, ma sempre amara, sarcastica, intrisa del suo totale pessimismo sugli uomini e sui moventi delle loro azioni.

\paragraph{Il punto di vista dei Malavoglia} Se nella prima parte del capitolo la scena è occupata dal “coro” del paese, nella seconda parte emergono in primo piano i Malavoglia, che finora sono comparsi solo indirettamente, attraverso i discorsi degli altri personaggi. Si determina così uno stacco fortissimo rispetto alla precedente sequenza narrativa: alla squallida commedia dell'interesse e dell'egoismo si contrappone una prospettiva tragica, la rovina della famiglia che è la rovina di tutto un mondo. I Malavoglia, insieme con la cugina Anna e la Nunziata, contro la grettezza ottusa del paese si propongono come portatori di alti valori etici, gli affetti familiari, l'onestà, il rispetto per la parola data, l'altruismo e la solidarietà disinteressata. Muta anche la tecnica narrativa: nella sequenza precedente gli abitanti del villaggio sono sempre presentati solo dall'esterno, attraverso le loro parole e i loro gesti; i Malavoglia invece sono visti anche dall'interno, e il lettore è ammesso a conoscere la loro vita interiore. È un privilegio che, nel corso del romanzo, tocca solo ai Malavoglia, ed è il segno inequivocabile di un privilegio spirituale, che li distingue dalla meschinità del paese. Il capitolo esemplifica quindi perfettamente la presenza di due polarità opposte, che è caratteristica dei Malavoglia: quella della comunità del villaggio, che conosce solo la logica dell'interesse e della forza ed è il semplice riflesso di un mondo regolato dal meccanismo della lotta per la vita, e quella della famiglia Malavoglia, che si ispira invece a valori etici puri e ideali.

\subsection{T: \textsc{I ``vinti'' e la ``fiumana del progresso''}}

Anche nel momento in cui Verga fotografa la classe più bassa, come
nell'ambiente dei Malavoglia (piccolo villaggio di pescatori), c'è
comunque una \evidenziatore{gerarchia} all'interno di questa società.
Insieme a questa gerarchia c'è anche il desiderio di cercare di
sopravanzare, a partire dal momento in cui la storia (\textsc{I
Malavoglia}) irrompe in un mondo praticamente immobile.

Anche il mondo arcaico non è idilliaco.

È proprio che quando i Malavoglia cercheranno di fare una scalata
sociale che inizierà il loro declassamento: all'inizio possiedono una
casa e una barca, mentre alla fine perderanno entrambe.

Quello che ci descrive Verga è una sorta di immobilismo:
l'\evidenziatore{ideale dell'ostrica}; l'ostrica deve rimanere immobile,
e quando cerca di staccarsi viene mangiata

Nelle \evidenziatore{sfere sociali più alte} è tutto più difficile
perché subentra la finzione, l'artificio e l'\textbf{educazione}

Il concetto di \pagine{progresso} qui ha una accezione negativa.

\begin{quote}
Il cammino fatale, incessante, spesso faticoso e febbrile che segue
l'umanità per raggiungere la conquista del progresso, è grandioso nel
suo risultato, visto nell'insieme, da lontano
\end{quote}

La variazione nella \evidenziatore{forma narrativa} ha uno scopo
ideologico:

\begin{quote}
Solo l'osservatore, travolto anch'esso dalla fiumana, guardandosi
intorno, ha il diritto d'interessarsi ai deboli che restano per via, ai
fiacchi che si lasciano sorpassare dall'onda per finire più presto, ai
vinti che levano le braccia disperate, e piegano il capo sotto il piede
brutale deo sopravvegnenti, i vincitori d'oggi, affrettanti anch'essi,
avidi anch'essi d'arrivare, che saranno sorpassati domani.
\end{quote}

Egli infatti non vuole commentare, ma solamente \emph{mostrare} e
\emph{interessarsi}. Questa è la grande differenza rispetto a Zola: Zola
scrive perché è convinto che dalla sua osservazione possa nascere un
suggerimento per chi governa: una volta che chi governa conoscerà queste
realtà potrà fare qualcosa di buono.

\begin{quote}
Chi osserva questo spettacolo non ha il diritto di giudicarlo; è già
molto se riesce a trarsi un istante fuori del campo della lotta per
studiarla senza passione, e rendere la scena nettamente, coi colori
adatti, tale da dare la rappresentazione della realtà com'è stata, o
come avrebbe dovuto essere
\end{quote}

\paragraph{Il tema di fondo del ciclo} È il documento teorico più articolato e approfondito che Verga ci abbia lasciato. Il primo paragrafo è dedicato specificamente al primo romanzo del ciclo, \textsc{I Malavoglia}, ed indica con chiarezza sintetica qual è il tema di fondo dell'opera: la rottura dell'equilibrio di un mondo tradizionale e immobile, quello di una famiglia di un piccolo villaggio di pescatori, «vissuta sino allora relativamente felice», per l'irrompere di forze nuove, l'insoddisfazione dello stato attuale, il bisogno di migliorare le proprie condizioni di vita, che sono l'indizio dell'affacciarsi della modernità in quel sistema arcaico. Nel paragrafo successivo lo sguardo si allarga al complesso dei romanzi del ciclo. Anche qui al centro dell'attenzione si pone la «fiumana del progresso», cioè il grande processo di trasformazione della realtà contemporanea, in particolare dell'Italia, che si sta avviando, dopo l'Unità, ad un'organizzazione economica e sociale moderna. La forza motrice di questo processo è identificata negli appetiti, da quelli più elementari, la lotta per i bisogni materiali dell'esistenza, a quelli più complessi e raffinati, via via che si sale nella scala sociale. È evidente qui un'impostazione duramente materialistica, che esclude i moventi “ideali” dall'agire dell'uomo, o comunque li considera subordinati a quelli materiali. Tipicamente naturalistico è anche vedere i processi sociali e psicologici come un «meccanismo». Tale meccanismo sarà semplice e facile da studiare nelle «basse sfere»; lo studio diverrà invece sempre più difficile man mano che il meccanismo si complica, nelle sfere superiori della società.

\paragraph{Il problema formale} Verga tocca poi anche rapidamente il problema formale. Perché l'analisi sia «esatta» e di mostri la «verità», occorre che essa segua scrupolosamente determinate norme. Noi sappiamo da altri testi che queste «norme» si compendiano nel principio dell'impersonalità. Lo scrittore sottolinea anche come la «forma» sia strettamente inerente al «soggetto»: cioè la «forma» è un fattore indispensabile perché l'osservazione sia esatta e raggiunga la verità. Ciò dimostra quanto Verga fosse consapevole del fatto che a caratterizzare la nuova arte non bastassero i contenuti in astratto, ma condizione fondamentale fosse la forma. Alla fine della Prefazione si aggiungerà poi un'altra importante precisazione: ogni scena va rappresentata con i colori adatti»; vale a dire che in ogni romanzo del ciclo occorre fare uso di una forma che risponda al livello sociale rappresentato. A questo principio in effetti, nei due romanzi scritti, Verga si attiene scrupolosamente. Nei \textsc{Malavoglia}, che rappresentano le «basse sfere», il narratore si adegua alle categorie mentali e al linguaggio dell'ambiente popolare, mentre nel \textsc{Gesualdo} si innalza, in corrispondenza di ambienti sociali più elevati.

\paragraph{Le posizioni ideologiche} Il terzo paragrafo contiene invece le fondamentali prese di posizione ideologiche dello scrittore di fronte all'oggetto del suo ciclo, la «fiumana del progresso». Verga non partecipa a quella mitologia del progresso che era dominante nell'opinione comune della sua epoca. Egli esprime bensì la sua ammirazione per la grandiosità del processo in atto, che ha qualcosa di epico: arriva persino a ripetere uno dei princìpi basilari dell'ideologia borghese moderna, quello formulato nel Settecento dall'economista Adam Smith, il fondatore del pensiero economico liberista, secondo cui l'individuo, perseguendo il suo interesse personale, coopera inconsapevolmente al benessere di tutti. Però non si assume il ruolo di celebratore del progresso, a differenza di tanti scrittori contemporanei, maggiori o minori (si pensi a Carducci).
Lungi dal levare inni, Verga insiste proprio sui suoi aspetti negativi, «irrequietudini», «avidità», «egoismo», «vizi», «contraddizioni», quanto c'è di «meschino negli interessi particolari». E poi, nel concreto dell'opera, non assume come oggetto della rappresentazione gli aspetti epici e trionfali del progresso, bensì proprio il suo rovescio negativo: sceglie di soffermarsi sui «vinti», quelli che sono schiacciati dalle leggi inesorabili dello sviluppo moderno. I protagonisti dei cinque romanzi progettati sono appunto dei «vinti >>.

\subsection{T: \textsc{La conclusione del romanzo}}

\paragraph{Le interpretazioni di Russo e Bàrberi Squarotti} Al termine della vicenda Alessi ricompra la casa del nespolo e vi ricostituisce un nucleo familiare. La conclusione del romanzo è da vedere come un "lieto fine", la vittoria dei va lori ideali sulla pessimistica analisi di una realtà dominata solo dalla lotta per la vita? Si tratta di pagine molto problematiche, che hanno sollecitato letture diverse. Luigi Russo, che ha dominato per oltre un quarantennio la critica verghiana, le interpreta va come una celebrazione della sacralità della casa e della famiglia, una «cerimonia religiosa», in cui il «tempio» che era stato violato «viene riconsacrato» (anche se non si tratta di un approdo idillico ad un porto di quiete e di benessere). Oggi la critica ha indicato altre direzioni di lettura. Bàrberi Squarotti osserva che la conclusione del romanzo non è un ritorno esatto al punto di partenza: la famiglia è dispersa, un mondo è scomparso definitivamente. La casa di Alessi non appartiene più ad una civiltà arcaica, ma è la casa dei «tempi nuovi», il mondo del vapore, della rivoluzione del 1860. Ed in effetti la nota dominante in queste pagine finali è quella del rimpianto, della mancanza, della nostalgia, da parte dei personaggi, di un passato ormai irrecuperabile, non il senso di pienezza tranquilla di un restaurato equilibrio.

\paragraph{L'interpretazione di Luperini} Luperini, dal canto suo, osserva che il romanzo non si conclude propriamente con la ricostruzione del «nido» familiare, a cui è dedicata solo una riga frettolosa, ma con la partenza definitiva di 'Ntoni. La conclusione ha dunque il senso di un distacco definitivo, di un addio amaro a quel mondo arcaico, a quello spazio chiuso e mitico, a quel tempo circolare e immobile (si pensi alla ripresa dei ritmi ciclici della vita del villaggio, che è la nota dominante dell'ultima pagina). Verga sa che quel mondo è scomparso, ed è ormai impossibile recuperarlo. Ad esso si contrappone l'eroe che parte per il mondo del moderno, verso la storia: la «fiumana del progresso» è inarrestabile. Secondo Luperini la conclusione, lungi dall'essere la celebrazione del mondo arcaico e dei suoi valori, è il distacco definitivo di Verga da quell'atteggiamento romantico che lo aveva indotto a cercare nella realtà rurale un «fresco e sereno raccoglimento» e un'alternativa al «progresso». Il paese è allontanato nel passato; e comunque era già un mondo lacerato al suo interno da forti tensioni sociali ed economi che. Continuando su questa linea, nel romanzo successivo del cielo, il \textsc{Gesualdo}, Verga seguirà il suo eroe, una sorta di prosecuzione del personaggio di 'Ntoni, nel suo viaggio attraverso la realtà moderna, in cui la logica della «roba» domina senza contrasti e i valori sono impossibili.

\section{Le Novelle Rusticane}

\begin{itemize}
\item
  \pagine{p. 264}
\end{itemize}

Sono scritte tra la stesura de \textsc{I Malavoglia} e \textsc{Mastro Don
Gesualdo}. Contengono alcuni testi che sono estremamente importanti
perché costituiscono un anello di congiunzione tra i due romanzi; una di
queste è \textsc{La roba}

\subsection{T: \textsc{La roba}}

Questa novella ci servirà da termine di confronto con \textsc{Mastro Don
Gesualdo}.

Il protagonista della novella cerca per tutta la vita di
\evidenziatore{mettere da parte la roba}: vive come un pezzente per
tutta la vita, ma ha tanta roba. Quando muore si rende conto di non
potersi portare dietro tutto ciò, e impazzisce. È un personaggio piatto,
ma non tragico come sarà Mastro Don Gesualdo. È un personaggio gretto,
ma non sa di esserlo. Fino all'ultimo istante non si rende conto del suo
modo di essere, di come la sua vita sia priva di affetti e di ogni altro
valore importante. Mastro Don Gesualdo, invece, che da povero diventa
molto molto ricco, si rende conto dello stato spirituale in cui vive,
abbandonato da moglie e figlia, ed è un personaggio estremamente
tragico.

Anche qui c'è lo \evidenziatore{straniamento} , ma è un personaggio che
un po' è interno, un po' guarda dall'esterno, un po' fa parte della
comunità. Verga varia molto la focalizzazione, e gli permette di
esprimere tanto nonstante la teoria dell'impersonalità.

Il \pagine{primo periodo} è lungo e faticoso, e sembra quasi alludere
all'ampiezza dei campi e dei possedimenti di Mazzarò (protagonista della
novella).

Il narratore si concede ancora qualche vezzo di maniera.

Sono presenti tempistiche tipiche delle fiabe, a significare un tempo
sospeso.

Il tema dominante è quello del \pagine{possesso} .

\setcounter{mar}{0}

\begin{quote}
Il viandante che andava lungo il Biviere di Lentini, steso là come un pezzo di mare morto, e le stoppie riarse della Piana di Catania, e gli aranci sempre verdi di Francofonte, e i sugheri grigi di Resecone, e i pascoli deserti di Passaneto e di Passanitello, se domandava, per ingannare la noia della lunga strada polverosa, sotto il cielo fosco dal caldo, nell'ora in cui i campanelli della lettiga suonano tristamente nell'immensa campagna, e i muli lasciano ciondolare il capo e la coda, e il lettighiere canta la sua canzone malinconica per non lasciarsi vincere dal sonno della malaria: - Qui di chi è? - sentiva rispondersi: - Di Mazzarò -. E passando vicino a una fattoria grande quanto un paese, coi magazzini che sembrano chiese, e le galline a stormi accoccolate all'ombra del pozzo, e le donne che si mettevano la mano sugli occhi per vedere chi passava: - E qui? - Di Mazzarò -. E cammina e cammina, mentre la malaria vi pesava sugli occhi, e vi scuoteva all'improvviso l'abbaiare di un cane, passando per una vigna che non finiva più, e si allargava sul colle e sul piano, immobile, come gli pesasse addosso la polvere, e il guardiano sdraiato bocconi sullo schioppo, accanto al vallone, levava il capo sonnacchioso, e apriva un occhio per vedere chi fosse: - Di Mazzarò -. Poi veniva un uliveto folto come un bosco, dove l'erba non spuntava mai, e la raccolta durava fino a marzo. Erano gli ulivi di Mazzarò. E verso sera, allorché il sole tramontava rosso come il fuoco, e la campagna si velava di tristezza, si incontravano le lunghe file degli aratri di Mazzarò che tornavano adagio adagio dal maggese, e i buoi che passavano il guado lentamente, col muso nell'acqua scura; e si vedevano nei pascoli lontani della Canziria, sulla pendice brulla, le immense macchie biancastre delle mandre di Mazzarò; e si udiva il fischio del pastore echeggiare nelle gole, e il campanaccio che risuonava ora sì ed ora no, e il canto solitario perduto nella valle. - Tutta roba di Mazzarò. Pareva che fosse di Mazzarò perfino il sole che tramontava, e le cicale che ronzavano, e gli uccelli che andavano a rannicchiarsi col volo breve dietro le zolle, e il sibilo dell'assiolo nel bosco. Pareva che Mazzarò fosse disteso tutto grande per quanto era grande la terra, e che gli si camminasse sulla pancia. - \textit{Invece egli era un omiciattolo, diceva il lettighiere, che non gli avreste dato un baiocco, a vederlo; e di grasso non aveva altro che la pancia, e non si sapeva come facesse a riempirla, perché non mangiava altro che due soldi di pane; e sì ch'era ricco come un maiale}\mat{insignificanza di Mazzarò, anche fisica}; ma aveva la testa ch'era un brillante, quell'uomo.

  Infatti, colla testa come un brillante, aveva accumulato tutta quella roba\mat{scalata sociale}, dove prima veniva da mattina a sera a zappare, a potare, a mietere; col sole, coll'acqua, col vento; senza scarpe ai piedi, e senza uno straccio di cappotto; che tutti si rammentavano di avergli dato dei calci nel di dietro, quelli che ora gli davano dell'eccellenza, e gli parlavano col berretto in mano. Né per questo egli era montato in superbia, adesso che tutte le eccellenze del paese erano suoi debitori; e diceva che eccellenza vuol dire povero diavolo e cattivo pagatore; ma egli portava ancora il berretto, soltanto lo portava di seta nera, era la sua sola grandezza, e da ultimo era anche arrivato a mettere il cappello di feltro, perché costava meno del berretto di seta. Della roba ne possedeva fin dove arrivava la vista, ed egli aveva la vista lunga - dappertutto, a destra e a sinistra, davanti e di dietro, nel monte e nella pianura. Più di cinquemila bocche, senza contare gli uccelli del cielo e gli animali della terra, che mangiavano sulla sua terra, e senza contare la sua bocca la quale mangiava meno di tutte, e si contentava di due soldi di pane e un pezzo di formaggio, ingozzato in fretta e in furia, all'impiedi, in un cantuccio del magazzino grande come una chiesa, in mezzo alla polvere del grano, che non ci si vedeva, mentre i contadini scaricavano i sacchi, o a ridosso di un pagliaio, quando il vento spazzava la campagna gelata, al tempo del seminare, o colla testa dentro un corbello, nelle calde giornate della mèsse. Egli non beveva vino, non fumava, non usava tabacco, e sì che del tabacco ne producevano i suoi orti lungo il fiume, colle foglie larghe ed alte come un fanciullo, di quelle che si vendevano a 95 lire. Non aveva il vizio del giuoco, né quello delle donne. Di donne non aveva mai avuto sulle spalle che sua madre, la quale gli era costata anche 12 tarì, quando aveva dovuto farla portare al camposanto\mat{si mostra la grettezza del personaggio, e la mentalità utilitaristica del narratore}.
  
  Era che ci aveva pensato e ripensato tanto a quel che vuol dire la roba, quando andava senza scarpe a lavorare nella terra che adesso era sua, ed aveva provato quel che ci vuole a fare i tre tarì della giornata, nel mese di luglio, a star colla schiena curva 14 ore, col soprastante a cavallo dietro, che vi piglia a nerbate se fate di rizzarvi un momento. Per questo non aveva lasciato passare un minuto della sua vita che non fosse stato impiegato a fare della roba; e adesso i suoi aratri erano numerosi come le lunghe file dei corvi che arrivavano in novembre; e altre file di muli, che non finivano più, portavano le sementi; le donne che stavano accoccolate nel fango, da ottobre a marzo, per raccogliere le sue olive, non si potevano contare, come non si possono contare le gazze che vengono a rubarle; e al tempo della vendemmia accorrevano dei villaggi interi alle sue vigne, e fin dove sentivasi cantare, nella campagna, era per la vendemmia di Mazzarò. Alla mèsse poi i mietitori di Mazzarò sembravano un esercito di soldati, che per mantenere tutta quella gente, col biscotto alla mattina e il pane e l'arancia amara a colazione, e la merenda, e le lasagne alla sera, ci volevano dei denari a manate, e le lasagne si scodellavano nelle madie larghe come tinozze. Perciò adesso, quando andava a cavallo dietro la fila dei suoi mietitori, col nerbo in mano, non ne perdeva d'occhio uno solo, e badava a ripetere: - Curviamoci, ragazzi! - Egli era tutto l'anno colle mani in tasca a spendere, e per la sola fondiaria\mat{irruzione della storia} il re si pigliava tanto che a Mazzarò gli veniva la febbre, ogni volta.
  
  Però ciascun anno tutti quei magazzini grandi come chiese si riempivano di grano che bisognava scoperchiare il tetto per farcelo capire tutto; e ogni volta che Mazzarò vendeva il vino, ci voleva più di un giorno per contare il denaro, tutto di 12 tarì d'argento, ché lui non ne voleva di carta sudicia per la sua roba, e andava a comprare la carta sudicia soltanto quando aveva da pagare il re, o gli altri; e alle fiere gli armenti di Mazzarò coprivano tutto il campo, e ingombravano le strade, che ci voleva mezza giornata per lasciarli sfilare, e il santo, colla banda, alle volte dovevano mutar strada, e cedere il passo.
  
  Tutta quella roba se l'era fatta lui, colle sue mani e colla sua testa, col non dormire la notte, col prendere la febbre dal batticuore o dalla malaria, coll'affaticarsi dall'alba a sera, e andare in giro, sotto il sole e sotto la pioggia, col logorare i suoi stivali e le sue mule - egli solo non si logorava, pensando alla sua roba, ch'era tutto quello ch'ei avesse al mondo; perché non aveva né figli, né nipoti, né parenti; non aveva altro che la sua roba. Quando uno è fatto così, vuol dire che è fatto per la roba.
  
  Ed anche la roba era fatta per lui, che pareva ci avesse la calamita, perché la roba vuol stare con chi sa tenerla, e non la sciupa come quel barone che prima era stato il padrone di Mazzarò, e l'aveva raccolto per carità nudo e crudo ne' suoi campi, ed era stato il padrone di tutti quei prati, e di tutti quei boschi, e di tutte quelle vigne e tutti quegli armenti, che quando veniva nelle sue terre a cavallo coi campieri dietro, pareva il re, e gli preparavano anche l'alloggio e il pranzo, al minchione, sicché ognuno sapeva l'ora e il momento in cui doveva arrivare, e non si faceva sorprendere colle mani nel sacco. - Costui vuol essere rubato per forza! - diceva Mazzarò, e schiattava dalle risa quando il barone gli dava dei calci nel di dietro, e si fregava la schiena colle mani, borbottando: - Chi è minchione se ne stia a casa, - la roba non è di chi l'ha, ma di chi la sa fare -. Invece egli, dopo che ebbe fatta la sua roba, non mandava certo a dire se veniva a sorvegliare la messe, o la vendemmia, e quando, e come; ma capitava all'improvviso, a piedi o a cavallo alla mula, senza campieri, con un pezzo di pane in tasca; e dormiva accanto ai suoi covoni, cogli occhi aperti, e lo schioppo fra le gambe.
  
  In tal modo a poco a poco Mazzarò divenne il padrone di tutta la roba del barone; e costui uscì prima dall'uliveto, e poi dalle vigne, e poi dai pascoli, e poi dalle fattorie e infine dal suo palazzo istesso, che non passava giorno che non firmasse delle carte bollate, e Mazzarò ci metteva sotto la sua brava croce. Al barone non era rimasto altro che lo scudo di pietra ch'era prima sul portone, ed era la sola cosa che non avesse voluto vendere, dicendo a Mazzarò: - Questo solo, di tutta la mia roba, non fa per te -. Ed era vero; Mazzarò non sapeva che farsene, e non l'avrebbe pagato due baiocchi. Il barone gli dava ancora del tu, ma non gli dava più calci nel di dietro.
  
  - Questa è una bella cosa, d'avere la fortuna che ha Mazzarò! - diceva la gente; e non sapeva quel che ci era voluto ad acchiappare quella fortuna: quanti pensieri, quante fatiche, quante menzogne, quanti pericoli di andare in galera, e come quella testa che era un brillante avesse lavorato giorno e notte, meglio di una macina del mulino, per fare la roba; e se il proprietario di una chiusa limitrofa si ostinava a non cedergliela, e voleva prendere pel collo Mazzarò, dover trovare uno stratagemma per costringerlo a vendere, e farcelo cascare, malgrado la diffidenza contadinesca. Ei gli andava a vantare, per esempio, la fertilità di una tenuta la quale non produceva nemmeno lupini, e arrivava a fargliela credere una terra promessa, sinché il povero diavolo si lasciava indurre a prenderla in affitto, per specularci sopra, e ci perdeva poi il fitto, la casa e la chiusa, che Mazzarò se l'acchiappava - per un pezzo di pane. - E quante seccature Mazzarò doveva sopportare! - I mezzadri che venivano a lagnarsi delle malannate, i debitori che mandavano in processione le loro donne a strapparsi i capelli e picchiarsi il petto per scongiurarlo di non metterli in mezzo alla strada, col pigliarsi il mulo o l'asinello, che non avevano da mangiare.
  
  - Lo vedete quel che mangio io? - rispondeva lui, - pane e cipolla! e sì che ho i magazzini pieni zeppi, e sono il padrone di tutta questa roba -. E se gli domandavano un pugno di fave, di tutta quella roba, ei diceva: - Che, vi pare che l'abbia rubata? Non sapete quanto costano per seminarle, e zapparle, e raccoglierle? - E se gli domandavano un soldo rispondeva che non l'aveva.
  
  E non l'aveva davvero. Ché in tasca non teneva mai 12 tarì, tanti ce ne volevano per far fruttare tutta quella roba, e il denaro entrava ed usciva come un fiume dalla sua casa. Del resto a lui non gliene importava del denaro; diceva che non era roba, e appena metteva insieme una certa somma, comprava subito un pezzo di terra; perché voleva arrivare ad avere della terra quanta ne ha il re, ed esser meglio del re, ché il re non può ne venderla, né dire ch'è sua.
  Di una cosa sola gli doleva, che cominciasse a farsi vecchio, e la terra doveva lasciarla là dov'era. Questa è una ingiustizia di Dio, che dopo di essersi logorata la vita ad acquistare della roba, quando arrivate ad averla, che ne vorreste ancora, dovete lasciarla! E stava delle ore seduto sul corbello, col mento nelle mani, a guardare le sue vigne che gli verdeggiavano sotto gli occhi, e i campi che ondeggiavano di spighe come un mare, e gli oliveti che velavano la montagna come una nebbia, e se un ragazzo seminudo gli passava dinanzi, curvo sotto il peso come un asino stanco, gli lanciava il suo bastone fra le gambe, per invidia, e borbottava: - Guardate chi ha i giorni lunghi! costui che non ha niente! -
  
  Sicché quando gli dissero che era tempo di lasciare la sua roba, per pensare all'anima, uscì nel cortile come un pazzo, barcollando, e andava ammazzando a colpi di bastone le sue anitre e i suoi tacchini, e strillava: - Roba mia, vientene con me! -
\end{quote}

In questo testo Verga abbandona il mito di una società bassa e positiva:
anche in questa classe sociale così bassa, tutto è mosso da \emph{la
roba}.

\paragraph{La nuova direzione della ricerca verghiana}

La \textsc{roba}, insieme con le altre \textsc{Novelle rusticane}, rappresenta perfettamente la nuova direzione della ricerca verghiana dopo i \textsc{Malavoglia}, l'abbandono definitivo di ogni mitizzazione nostalgica e romantica del mondo rurale. Il polo positivo dei valori puri scompare e la realtà risulta tutta dominata dalla logica dell'interesse e della forza. La famiglia non è più il centro ideale di quei valori e la loro difesa dalle forze avverse: si pensi a Mazzarò che rimpiange i dodici tari spesi per il funerale della madre. Né si ha più un universo arcaico, regolato da un tempo ciclico, in cui tutto torna sempre identico: al centro della novella si pone il tema della dinamicità sociale che travolge tutti gli equilibri tradizionali, nella figura di un self-made man rurale, che dal nulla si crea una prodigiosa fortuna e la cui scalata sociale è inserita in un ben identificabile processo storico della modernità, la crisi della nobiltà di origine feudale e l'ascesa della borghesia.

Tranne che all'inizio, dove Mazzarò è visto dalla prospettiva di un ipotetico viandante di livello culturale "alto", che con le sue fantasie trasforma il personaggio in un essere favoloso, l'ottica narrativa è quella consueta al Verga verista, interna al mondo rappresentato, proveniente ``dal basso". Ma l'effetto dell'artificio della "regressione" è ben differente rispetto a \textsc{Rosso Malpelo}. Là l'eroe era un "diverso" rispetto all'ambiente, dotato di una statura intellettuale e morale infinitamente più alta; qui invece Mazzarò è perfettamente integrato nella logica della lotta per la vita. Quindi in \textsc{Rosso Malpelo} l'ottica del narratore, essendo estranea all'eroe, non era in grado di comprenderlo e stravolgeva malevolmente la sua figura, con un vistoso effetto di straniamento. Qui invece, poiché il narratore è in sintonia con l'eroe e la sua logica, si ha una celebrazione entusiastica, un vero panegirico dell'uomo che si è fatto dal nulla.

\paragraph{I temi ricorrenti della novella}

Dato questo modo di presentare Mazzarò, i temi che ricorrono costantemente nella novella sono:
\begin{enumerate}
\item l'ammirazione per la potenza dell'accumulo capitalistico, che riesce a creare ricchezze immense, un mondo di cose dalle proporzioni smisurate, epiche; la celebrazione impiega soprattutto la figura dell'iperbole (i mietitori sembrano un esercito di soldati, gli aratri sono numerosi come le lunghe file dei corvi, per la vendemmia accorrono villaggi interi alle vigne di Mazzarò...), ed assume le movenze ampie dell'inno, con cadenze musicali mae stosamente intonate;
\item le virtù eroiche del protagonista, l'intelligenza, l'energia infaticabile, ma soprattutto l'ascesi, la capacità di sacrificare tutto alla «roba», per cui Mazzarò appare quasi un santo martire dell'accumulo capitalistico;
\item il tendere inesausto sempre oltre gli obiettivi raggiunti, che fa di Mazzarò una sorta di eroe "faustiano", nel suo sogno di potenza senza limiti, che lo spinge a collocarsi in posi zione antagonistica addirittura rispetto alla suprema autorità in terra («voleva arrivare ad avere della terra quanta ne ha il re, ed esser meglio del re»),.
\end{enumerate}

\paragraph{Lo straniamento rovesciato}

La novella, grazie all'adozione di un punto di vista narrativo vicino al protagonista, presenta in tal modo la logica della «roba» in una luce epica, mitica, come qualcosa di sovrumano, titanico. Però vi è anche il rovescio della medaglia. Come avveniva per zio Crocifisso nel IV capitolo dei \textsc{Malavoglia}, questa celebrazione, proprio per il suo oltranzismo, produce l'effetto di straniamento "rovesciato": ciò che è strano, abnorme e ripugnante, l'avidità disumana e crudele di Mazzarò, che risalta con perfetta evidenza dall'oggettività dei fatti, appare normale, legittimo o addirittura meritorio della presentazione del narratore. Ciò mette crudamente in luce lo stravolgimento profondo di quel mondo che conosce solo l'interesse ed ignora ogni altro valore. Ne scaturisce una critica ferma della «religione della roba»; ma anche qui sono le cose che parlano da sé, senza che l'autore intervenga dall'esterno con giudizi e condanne.

\paragraph{La problematicità della visione verghiana}

L'impostazione della novella appare dunque intimamente problematica: Mazzarò ha veramente qualcosa di eroico e di epico nella sua dedizione ascetica al suo fine, nella sua potenza creatrice, nel suo tendere "faustiano" a mete sempre più alte; dall'altro lato però la logica dell'accumulo appare anche in tutta la sua disumana negatività. È quell'atteggiamento verso il «progresso» che si manifesta anche nella prefazione ai Vinti; e lo stesso atteggiamento ricomparirà presto dinanzi ad un altro eroe dell'accumulo capitalistico moderno, \textsc{Gesualdo}. Ma bisogna ancora tener conto della conclusione, che presenta un rovesciamento di prospettive. Nella sua tensione ad accrescere indefinita mente il possesso, Mazzarò non si scontra soltanto con avversari umani, con la società e le leggi economiche, ma con la natura stessa, col limite naturale della vita. Quella tensione va allora incontro al totale fallimento: e, in un gesto disperato e folle, Mazzarò tenta di uccidere le anatre e i tacchini, per portare con sé nella morte la «roba». Questa conclusione ha avuto interpretazioni diverse, che ora hanno insistito sulla comicità del gesto di Mazzarò, ora sul suo carattere tragico, terribile. L'oscillazione delle interpretazioni deriva probabilmente dalla problematicità del segmento narrativo finale, che rovescia i termini del resto del racconto. Se in precedenza Mazzarò appariva eroico nella prospettiva del narratore "basso", e meschino e abietto nella prospettiva morale dell'autore, ora il suo gesto di bastonare le anatre e i tacchini appare risibile nella prospettiva del narratore, che lo ritiene assurdo, non rispondente ad alcuna logica economica (Mazzarò per lui dovrebbe rassegnarsi e «pensare all'anima»), ma tragico nella prospettiva dell'autore, sensibile al dramma esistenziale dell'eroe, che ha posto la sua ragione di vita nell'accumulo infinito di «roba» ed è sconfitto dai limiti di natura. La duplicità di prospettive, pur rovesciate di segno, mette anche qui in evidenza la problematicità del personaggio.

\section{Mastro Don Gesualdo}

\begin{itemize}
\item
  \pagine{p. 280-282}
\end{itemize}

Questo testo esce nell'89, lo stesso anno in cui esce il \textsc{Piacere}
di D'Annunzio: D'Annunzio avrà tantissimo successo, mentre questo no.

Se nei \textsc{Malavoglia} l'ambientazione era immediatamente successiva
all'unificazione italiana, il \textsc{Mastro Don Gesualdo} è ambientato
attorno al '48

\subsection{Trama}

Mastro Don Gesualdo nasce povero, poi, grazie alla sua intelligenza e
alla sua spregiudicatezza, riuscirà ad arricchirsi. Nasce come un
muratore, e tenta la scalata sociale, esattamente come i Malavoglia.

Questo non rende Verga più ottimista, in quanto Mastro Don Gesualdo è
l'esempio di pessimismo più assoluto.

Egli è un personaggio né positivo né negativo: è generoso, ma è quasi
schiavista nei confronti dei suoi lavoranti, in quanto lui ha sempre
lavorato come un pazzo. Nel suo nome vi è un ossimoro: questo contrasto
sarà espresso nella sua figura stessa

Ha una serva (che è anche una concubina), ma non appena si presenta
l'occasione per fare il salto di società non si fa problemi ad
abbandonare la madre dei suoi figli, per sposare Bianca Trau, nobile. Ci
sarà una figlia, che però non sarà figlia di Mastro Don Gesualdo, in
quanto quel matrimonio era un matrimonio riparatore.

Non appena avviene questo salto di società, Verga è spietato nel
mostrare i lati negativi: la moglie lo disprezza, i familiari lo odiano,
la figlia si vergogna di lui.

La figlia si innamora di un cugino che la mette incinta: il padre la
costringerà a sposare un nobile; la figlia lo odia per questo. Il terzo
romanzo del ciclo dei vinti (\textsc{La Duchessa di Leyra}) sarebbe dovuto
essere basato su questa coppia.

Finirà i suoi giorni vedendo sua figlia sperperare tutto il suo
patrimonio.

Egli morirà di cancro allo stomaco: è molto emblematico di qualcosa che
lo consuma da dentro, ovvero il suo dissidio.

\subsection{Il protagonista}

Mentre Mazzarò (de \textsc{La Roba}) è completamente alienato dal mondo
reale, e prima di morire sta bene, Gesualdo è un personaggio
consapevole, e che vive questo dissidio interiore tra gli affetti e
l'interesse, nonostante prevalga l'interesse.

Il contrasto che nei Malavoglia era tra la società e Padron 'Ntoni, qui
è presente nello stesso personaggio, che è estremamente complesso.
Alcuni critici affermano che Verga si identifichi in questo personaggio,
ma dal momento che egli mostra la sua sconfitta, non è pensabile che
Verga si identifichi in lui.

\subsection{Narratore}

Il narratore è sullo stesso piano dei personaggi che appaiono in questo
testo: sono borghesi. Dal punto di vista linguistico, può sembrare che
lo scarto dalla cultura del narratore sia minimo, in quanto Verga
appartiene proprio alla borghesia. Il narratore è impersonale, in terza
persona, ma la focalizzazione è quasi esclusivamente di Mastro Don
Gesualdo.

Il narratore non racconta i fatti, ma ci pone le descrizioni come
pensieri del personaggio. Verga in quest'opera usa il discorso indiretto
libero, ovvero tramite l'eliminazione dei verbi dichiarativi e delle
proposizioni oggettive. Questa tecnica è usata proprio per il personaggio di
Mastro Don Gesualdo.

Lo straniamento prima era provocato dal fatto che molto spesso il
narratore, identificandosi con la società ma non condividendone gli
ideali, giudicava il protagonista in un determinato modo come in Rosso
Malpelo. Le persone che lavorano con Rosso Malpelo non condividono i
suoi ideali e di conseguenza il narratore diventa inattendibile,
facendoci sembrare strane cose normali e viceversa.

\subsubsection{Microsaggio: il discorso indiretto libero}

Partiamo da tre esempi:
\begin{enumerate}
\item Giovanni si ricordò di Mario e pensò: «Devo telefonargli>>.
\item Giovanni si ricordò di Mario e pensò che doveva telefonargli.
\item Giovanni si ricordò di Mario. Doveva telefonargli.
\end{enumerate}

L'esempio 1 propone un discorso diretto: il pensiero del personaggio è riportato testualmente, tra virgolette; l'esempio 2 propone un discorso indiretto: la proposizione che nel discorso diretto era indipendente qui diviene subordinata dichiarativa, retta da un verbo e dalla congiunzione "che"; l'esempio 3 propone un di discorso indiretto libero: è indiretto come l'esempio 2, ma non vi è nessuna marca grammaticale che segni l'inizio del pensiero del personaggio, né un verbo reg gente né una congiunzione; per questo viene chiamato "libero". Il procedimento è usato frequentemente a partire dalla narrativa del secondo Ottocento e consente agli scrittori grande scioltezza e duttilità nell'inserire discorsi o pensieri dei personaggi nel fluire della narrazione, senza apparenti soluzioni di continuità. Da un esempio fittizio ed elementare passiamo ad un esempio d'autore. Nel capitolo IV della parte I del \textsc{Mastro-don Gesualdo} di Verga, il protagonista rievoca la sua ascesa sociale con un lungo discorso indiretto libero.

Vediamone l'inizio:

\begin{quote}
Egli invece non aveva sonno. Si sentiva allargare il cuore. Gli venivano tanti ricordi piacevoli. Ne aveva portate delle pietre sulle spalle, prima di fabbricare quel magazzino! E ne aveva passati dei giorni senza pane, prima di possedere tutta quella roba!
\end{quote}

Sino alla frase: «Gli venivano tanti ricordi piacevoli» il discorso è del narratore, che descrive lo stato d'animo di Gesualdo. Subito dopo comincia invece il monologo in forma indiretta del personaggio; come si vede il passaggio è insensibile, non è segnato da un «Gesualdo pensava che ne aveva portate delle pietre». Il discorso indiretto libero è molto vicino al discorso diretto, ben più dell'indiretto "legato", tanto che conserva movenze e coloriture del parlato: si vedano qui le esclamazioni di Gesualdo. Più avanti sono riportati persino modi di dire caratteristici, di origine dialettale siciliana: «Santo che gli faceva mangiare i gomiti sin d'allora...>>.

Un metodo empirico ma infallibile per riconoscere un indiretto libero è trasformarlo in forma diretta, passando dalla terza alla prima persona e mutando i tempi dei verbi, i dimostrativi e gli avverbi di tempo e di luogo. Si provi col passo sopra citato: «Ne ho portate di pietre sulle spalle, prima di fabbricare questo magazzino! E ne ho passati di giorni senza pane, prima di possedere tutta questa roba!». Se è possibile questa trasformazione, si è sicuri di essere in presenza di un discorso del personaggio, espresso in forma indiretta libera.

\subsection{T: \textsc{La morte di mastro-don Gesualdo}}

Masto Don Gesualdo osserva quindi questo grande sperpero di denaro,
vedendo i servi che non si potrebbero permettere.

\begin{quote}
Parve a don Gesualdo d’entrare in un altro mondo, allorchè fu in casa della figliuola. Era un palazzone così vasto che ci si smarriva dentro. Da per tutto cortinaggi e tappeti che non si sapeva dove mettere i piedi — sin dallo scalone di marmo — e il portiere, un pezzo grosso addirittura, con tanto di barba e di soprabitone, vi squadrava dall’alto al basso, accigliato, se per disgrazia avevate una faccia che non lo persuadesse, e vi gridava dietro dal suo gabbione: — C’è lo stoino per pulirsi le scarpe! — Un esercito di mangiapane, staffieri e camerieri, che sbadigliavano a bocca chiusa, camminavano in punta di piedi, e vi servivano senza dire una parola o fare un passo di più, con tanta degnazione da farvene passar la voglia. Ogni cosa regolata a suon di campanello, con un cerimoniale di messa cantata — per avere un bicchier d’acqua, o per entrare nelle stanze della figliuola. Lo stesso duca, all’ora di pranzo, si vestiva come se andasse a nozze.

Il povero don Gesualdo, nei primi giorni, s’era fatto animo per contentare la figliuola, e s’era messo in gala anche lui per venire a tavola, legato e impastoiato, con un ronzìo nelle orecchie, le mani esitanti, l’occhio inquieto, le fauci strette da tutto quell’apparato, dal cameriere che gli contava i bocconi dietro le spalle, e di cui ogni momento vedevasi il guanto di cotone allungarsi a tradimento e togliervi la roba dinanzi. L’intimidiva pure la cravatta bianca del genero, le credenze alte e scintillanti come altari, e la tovaglia finissima, che s’aveva sempre paura di lasciarvi cadere qualche cosa. Tanto che macchinava di prendere a quattr’occhi la figliuola, e dirle il fatto suo. Il duca, per fortuna, lo tolse d’impiccio, dicendo ad Isabella, dopo il caffè, col sigaro in bocca e il capo appoggiato alla spalliera del seggiolone:

— Mia cara, d’oggi innanzi credo che sarebbe meglio far servire papà nelle sue stanze. Avrà le sue ore, le sue abitudini... Poi, col regime speciale che richiede il suo stato di salute...

— Certo, certo, — balbettò don Gesualdo. — Stavo per dirvelo... Sarei più contento anch’io... Non voglio essere d’incomodo...

— No. Non dico per questo. Voi ci fate a ogni modo piacere, caro mio.
Egli si mostrava proprio un buon figliuolo col suocero. Gli riempiva il bicchierino; lo incoraggiava a fumare un sigaro; lo assicurava infine che gli trovava miglior cera, da che era arrivato a Palermo, e il cambiamento d’aria e una buona cura l’avrebbero guarito del tutto. Poi gli toccò anche il tasto degli interessi. Mostravasi giudizioso; cercava il modo e la maniera d’avere il piacere di tenersi il suocero in casa un pezzo, senza timore che gli affari di lui andassero a rotta di collo... Una procura generale... una specie d’alter ego... Don Gesualdo si sentì morire il sorriso in bocca. Non c’era che fare. Il genero, nel viso, nelle parole, sin nel tono della voce, anche quando voleva fare l’amabile e pigliarvi bel bello, aveva qualcosa che vi respingeva indietro, e vi faceva cascar le braccia, uno che avesse voluto buttargliele al collo, proprio come a un figlio, e dirgli:

— Tè! per la buona parola, adesso! Pazienza il resto! Fai quello che vuoi!

Talchè don Gesualdo scendeva raramente dalla figliuola. Ci si sentiva a disagio col signor genero; temeva sempre che ripigliasse l’antifona dell’alter ego. Gli mancava l’aria, lì fra tutti quei ninnoli. Gli toccava chiedere quasi licenza al servitore che faceva la guardia in anticamera per poter vedere la sua figliuola, e scapparsene appena giungeva qualche visita. L’avevano collocato in un quartierino al pian di sopra poche stanze che chiamavano la foresteria, dove Isabella andava a vederlo ogni mattina, in veste da camera, spesso senza neppure mettersi a sedere, amorevole e premurosa, è vero, ma in certo modo che al pover’uomo sembrava d’essere davvero un forestiero. Essa alcune volte era pallida così che pareva non avesse chiuso occhio neppur lei. Aveva una certa ruga fra le ciglia, qualcosa negli occhi, che a lui, vecchio e pratico del mondo, non andavan punto a genio. Avrebbe voluto pigliarsi anche lei fra le braccia, stretta stretta, e chiederle piano in un orecchio: — Cos’hai?... dimmelo!... Confidati a me che dei guai ne ho passati tanti, e non posso tradirti!...

Ma anch’essa ritirava le corna come fa la lumaca. Stava chiusa, parlava di rado anche della mamma, quasi il chiodo le fosse rimasto lì, fisso... accusando lo stomaco peloso dei Trao, che vi chiudevano il rancore e la diffidenza, implacabili!

Perciò lui doveva ricacciare indietro le parole buone e anche le lagrime, che gli si gonfiavano grosse grosse dentro, e tenersi per sè i propri guai. Passava i giorni malinconici dietro l’invetriata, a veder strigliare i cavalli e lavare le carrozze, nella corte vasta quanto una piazza. Degli stallieri, in manica di camicia e coi piedi nudi negli zoccoli, cantavano, vociavano, barattavano delle chiacchiere e degli strambotti coi domestici, i quali perdevano il tempo alle finestre, col grembialone sino al collo, o in panciotto rosso, strascicando svogliatamente uno strofinaccio fra le mani ruvide, con le barzellette sguaiate, dei musi beffardi di mascalzoni ben rasi e ben pettinati che sembravano togliersi allora una maschera. I cocchieri poi, degli altri pezzi grossi, stavano a guardare, col sigaro in bocca e le mani nelle tasche delle giacchette attillate, discorrendo di tanto in tanto col guardaportone che veniva dal suo casotto a fare una fumatina, accennando con dei segni e dei versacci alle cameriere che si vedevano passare dietro le invetriate dei balconi, oppure facevano capolino provocanti, sfacciate, a buttar giù delle parolacce e delle risate di male femmine con certi visi da Madonna. Don Gesualdo pensava intanto quanti bei denari dovevano scorrere per quelle mani; tutta quella gente che mangiava e beveva alle spalle di sua figlia, sulla dote che egli le aveva dato, su l’Àlia e su Donninga, le belle terre che aveva covato cogli occhi tanto tempo, sera e mattina, e misurato col desiderio, e sognato la notte, e acquistato palmo a palmo, giorno per giorno, togliendosi il pane di bocca: le povere terre nude che bisognava arare e seminare; i mulini, le case, i magazzini che aveva fabbricato con tanti stenti, con tanti sacrifici, un sasso dopo l’altro. La Canziria, Mangalavite, la casa, tutto, tutto sarebbe passato per quelle mani. Chi avrebbe potuto difendere la sua roba dopo la sua morte, ahimè, povera roba! Chi sapeva quel che era costata? Il signor duca, lui, quando usciva di casa, a testa alta, col sigaro in bocca e il pomo del bastoncino nella tasca del pastrano, fermavasi appena a dare un’occhiata ai suoi cavalli, ossequiato come il Santissimo Sagramento, le finestre si chiudevano in fretta, ciascuno correva al suo posto, tutti a capo scoperto, il guardaportone col berretto gallonato in mano, ritto dinanzi alla sua vetrina, gli stallieri immobili accanto alla groppa delle loro bestie, colla striglia appoggiata all’anca, il cocchiere maggiore, un signorone, piegato in due a passare la rivista e prendere gli ordini: una commedia che durava cinque minuti. Dopo, appena lui voltava le spalle, ricominciava il chiasso e la baraonda, dalle finestre, dalle arcate del portico che metteva alle scuderie, dalla cucina che fumava e fiammeggiava sotto il tetto, piena di sguatteri vestiti di bianco, quasi il palazzo fosse abbandonato in mano a un’orda famelica, pagata apposta per scialarsela sino al tocco della campana che annunziava qualche visita - un’altra solennità anche quella. - La duchessa certi giorni si metteva in pompa magna ad aspettare le visite come un’anima di purgatorio. Arrivava di tanto in tanto una carrozza fiammante; passava come un lampo dinanzi al portinaio, che aveva appena il tempo di cacciare la pipa nella falda del soprabito e di appendersi alla campana; delle dame e degli staffieri in gala sguisciavano frettolosi sotto l’alto vestibolo, e dopo dieci minuti tornavano ad uscire per correre altrove a rompicollo; proprio della gente che sembrava presa a giornata per questo. Lui invece passava il tempo a contare le tegole dirimpetto, a calcolare, con l’amore e la sollecitudine del suo antico mestiere, quel che erano costate le finestre scolpite, i pilastri massicci, gli scalini di marmo, quei mobili sontuosi, quelle stoffe, quella gente, quei cavalli che mangiavano, e inghiottivano il denaro come la terra inghiottiva la semente, come beveva l’acqua, senza renderlo però, senza dar frutto, sempre più affamati, sempre più divoranti, simili a quel male che gli consumava le viscere. Quante cose si sarebbero potute fare con quel denaro! Quanti buoni colpi di zappa, quanto sudore di villani si sarebbero pagati! Delle fattorie, dei villaggi interi da fabbricare... delle terre da seminare, a perdita di vista... E un esercito di mietitori a giugno, del grano da raccogliere a montagne, del denaro a fiumi da intascare!... Allora gli si gonfiava il cuore al vedere i passeri che schiamazzavano su quelle tegole, il sole che moriva sul cornicione senza scendere mai giù sino alle finestre. Pensava alle strade polverose, ai bei campi dorati e verdi, al cinguettìo lungo le siepi, alle belle mattinate che facevano fumare i solchi!... Oramai!... oramai!...

Adesso era chiuso fra quattro mura, col brusìo incessante della città negli orecchi, lo scampanìo di tante chiese che gli martellava sul capo, consumato lentamente dalla febbre, roso dai dolori che gli facevano mordere il guanciale, a volte, per non seccare il domestico che sbadigliava nella stanza accanto. 

[Le condizioni di Gesuldo peggiorano sempre più e per lui sono solo inutili tormenti i consulti dei luminari della medicina. Quando sente che è giunta la sua ora vuole parlare con la figlia]

— \textit{Senti... Ho da parlarti... intanto che siamo soli}...

Ella gli si buttò addosso, disperata, piangendo, singhiozzando di no, di no, colle mani erranti che l’accarezzavano. L’accarezzò anche lui sui capelli, lentamente, senza dire una parola. Di lì a un po’ riprese:

— Ti dico di sì. Non sono un ragazzo... Non perdiamo tempo inutilmente. — Poi gli venne una tenerezza. — Ti dispiace, eh?... ti dispiace a te pure?...

La voce gli si era intenerita anch’essa, gli occhi, tristi, s’erano fatti più dolci, e qualcosa gli tremava sulle labbra. — Ti ho voluto bene... anch’io... quanto ho potuto... come ho potuto... Quando uno fa quello che può...

Allora l’attirò a sè lentamente, quasi esitando, guardandola fissa per vedere se voleva lei pure, e l’abbracciò stretta stretta, posando la guancia ispida su quei bei capelli fini.

— Non ti fo male, di’?... come quand’eri bambina?...

Gli vennero insieme delle altre cose sulle labbra, delle ondate di amarezza e di passione, quei sospetti odiosi che dei bricconi, nelle questioni d’interessi, avevano cercato di mettergli in capo. Si passò la mano sulla fronte, per ricacciarli indietro, e cambiò discorso.

— Parliamo dei nostri affari. Non ci perdiamo in chiacchiere, adesso...

Essa non voleva, smaniava per la stanza, si cacciava le mani nei capelli, diceva che gli lacerava il cuore, che gli pareva un malaugurio, quasi suo padre stesse per chiudere gli occhi.

— Ma no, parliamone! — insisteva lui. — Sono discorsi serii. Non ho tempo da perdere adesso. — Il viso gli si andava oscurando, il rancore antico gli corruscava negli occhi. — Allora vuol dire che non te ne importa nulla... come a tuo marito...

Vedendola poi rassegnata ad ascoltare, seduta a capo chino accanto al letto, cominciò a sfogarsi dei tanti crepacuori che gli avevano dati, lei e suo marito, con tutti quei debiti... Le raccomandava la sua roba, di proteggerla, di difenderla: — Piuttosto farti tagliare la mano, vedi!... quando tuo marito torna a proporti di firmare delle carte!... Lui non sa cosa vuol dire! — Spiegava quel che gli erano costati, quei poderi, l’Àlia, la Canziria, li passava tutti in rassegna amorosamente; rammentava come erano venuti a lui, uno dopo l’altro, a poco a poco, le terre seminative, i pascoli, le vigne; li descriveva minutamente, zolla per zolla, colle qualità buone o cattive. Gli tremava la voce, gli tremavano le mani, gli si accendeva tuttora il sangue in viso, gli spuntavano le lagrime agli occhi: — Mangalavite, sai... la conosci anche tu... ci sei stata con tua madre... Quaranta salme di terreni, tutti alberati!... ti rammenti... i belli aranci?... anche tua madre, poveretta, ci si rinfrescava la bocca, negli ultimi giorni!... 300 migliaia l’anno, ne davano! Circa 300 onze! E la Salonia... dei seminati d’oro... della terra che fa miracoli... benedetto sia tuo nonno che vi lasciò le ossa!...

Infine, per la tenerezza, si mise a piangere come un bambino.

\textit{— Basta, — disse poi. — Ho da dirti un’altra cosa... Senti...}

La guardò fissamente negli occhi pieni di lagrime per vedere l’effetto che avrebbe fatto la sua volontà. Le fece segno di accostarsi ancora, di chinarsi su lui supino che esitava e cercava le parole.

— Senti!... Ho degli scrupoli di coscienza... Vorrei lasciare qualche legato a delle persone verso cui ho degli obblighi... Poca cosa... Non sarà molto per te che sei ricca... Farai conto di essere una regalìa che tuo padre ti domanda... in punto di morte... se ho fatto qualcosa anch’io per te...

— Ah, babbo, babbo!... che parole! — singhiozzò Isabella.

— Lo farai, eh? lo farai?... anche se tuo marito non volesse...

Le prese le tempie fra le mani, e le sollevò il viso per leggerle negli occhi se l’avrebbe ubbidito, per farle intendere che gli premeva proprio, e che ci aveva quel segreto in cuore. E mentre la guardava, a quel modo, gli parve di scorgere anche lui quell’altro segreto, quell’altro cruccio nascosto, in fondo agli occhi della figliuola. E voleva dirle delle altre cose, voleva farle altre domande, in quel punto, aprirle il cuore come al confessore, e leggere nel suo. Ma ella chinava il capo, quasi avesse indovinato, colla ruga ostinata dei Trao fra le ciglia, tirandosi indietro, chiudendosi in sè, superba, coi suoi guai e il suo segreto. E lui allora sentì di tornare Motta, com’essa era Trao, diffidente, ostile, di un’altra pasta. Allentò le braccia, e non aggiunse altro.

— Ora fammi chiamare un prete, — terminò con un altro tono di voce. — Voglio fare i miei conti con Domeneddio.

Durò ancora qualche altro giorno così, fra alternative di meglio e di peggio. Sembrava anzi che cominciasse a riaversi un poco, quando a un tratto, una notte, peggiorò rapidamente. Il servitore che gli avevano messo a dormire nella stanza accanto l’udì agitarsi e smaniare prima dell’alba. Ma siccome era avvezzo a quei capricci, si voltò dall’altra parte, fingendo di non udire. Infine, seccato da quella canzone che non finiva più, andò sonnacchioso a vedere che c’era.

\textit{— Mia figlia! — borbottò don Gesualdo con una voce che non sembrava più la sua. — Chiamatemi mia figlia!}

— Ah, sissignore. Ora vado a chiamarla, — rispose il domestico, e tornò a coricarsi.
Ma non lo lasciava dormire quell’accidente! Un po’ erano sibili, e un po’ faceva peggio di un contrabbasso, nel russare. Appena il domestico chiudeva gli occhi udiva un rumore strano che lo faceva destare di soprassalto, dei guaiti rauchi, come uno che sbuffasse ed ansimasse, una specie di rantolo che dava noia e vi accapponava la pelle. Tanto che infine dovette tornare ad alzarsi, furibondo, masticando delle bestemmie e delle parolacce.

— Cos’è? Gli è venuto l’uzzolo adesso? Vuol passar mattana! Che cerca?

Don Gesualdo non rispondeva; continuava a sbuffare supino. Il servitore tolse il paralume, per vederlo in faccia. Allora si fregò bene gli occhi, e la voglia di tornare a dormire gli andò via a un tratto.

— Ohi! ohi! Che facciamo adesso? — balbettò grattandosi il capo.

Stette un momento a guardarlo così, col lume in mano, pensando se era meglio aspettare un po’, o scendere subito a svegliare la padrona e mettere la casa sottosopra. Don Gesualdo intanto andavasi calmando, col respiro più corto, preso da un tremito, facendo solo di tanto in tanto qualche boccaccia, cogli occhi sempre fissi e spalancati. A un tratto s’irrigidì e si chetò del tutto. La finestra cominciava a imbiancare. Suonavano le prime campane. Nella corte udivasi scalpitare dei cavalli, e picchiare di striglie sul selciato.

Il domestico andò a vestirsi, e poi tornò a rassettare la camera. Tirò le cortine del letto, spalancò le vetrate, e s’affacciò a prendere una boccata d’aria, fumando.

Lo stalliere che faceva passeggiare un cavallo malato, alzò il capo verso la finestra.

— Mattinata, eh, don Leopoldo?

— E nottata pure! — rispose il cameriere sbadigliando. — M’è toccato a me questo regalo!

L’altro scosse il capo, come a chiedere che c’era di nuovo, e don Leopoldo fece segno che il vecchio se n’era andato, grazie a Dio.

— Ah... così.... alla chetichella?... — osservò il portinaio che strascicava la scopa e le ciabatte per l’androne.

Degli altri domestici s’erano affacciati intanto, e vollero andare a vedere. Di lì a un po’ la camera del morto si riempì di gente in manica di camicia e colla pipa in bocca. La guardarobiera vedendo tutti quegli uomini alla finestra dirimpetto venne anche lei a far capolino nella stanza accanto.

— Quanto onore, donna Carmelina! Entrate pure; non vi mangiamo mica.... E neanche lui.... non vi mette più le mani addosso di sicuro....

— Zitto, scomunicato!... No, ho paura, poveretto.... Ha cessato di penare.

— Ed io pure, — soggiunse don Leopoldo.

Così, nel crocchio, narrava le noie che gli aveva date quel cristiano — uno che faceva della notte giorno, e non si sapeva come pigliarlo, e non era contento mai. — Pazienza servire quelli che realmente son nati meglio di noi.... Basta, dei morti non si parla.

— Si vede com’era nato.... — osservò gravemente il cocchiere maggiore. — Guardate che mani!

— Già, son le mani che hanno fatto la pappa!... Vedete cos’è nascer fortunati.... Intanto vi muore nella battista come un principe!...

— Allora, — disse il portinaio, — devo andare a chiudere il portone?

— Sicuro, eh! È roba di famiglia. Adesso bisogna avvertire la cameriera della signora duchessa.
\end{quote}

Verga comunica anche attraverso le scelte formali. La focalizzazione è
diventata quella del servo, emblema del fallimento più totale; negli
ultimi giorni Mastro Don Gesualdo ha sofferto a vedere i suoi soldi
sperperati, e nella morte viene addirittura ridicolizzato, in quanto i
servi non hanno alcun riguardo, disprezzandolo apertamente.

\paragraph{Gli ultimi giorni di mastro-don Gesualdo}

Colpito dalla malattia, Gesualdo è accolto a Palermo nel palazzo nobiliare del genero e della figlia: qui trascorre i suoi ultimi giorni come un intruso, relegato in disparte, circondato da una servitù che lo disprezza per le sue umili origini e angosciato dinanzi allo sperpero del palazzo, che dissipa le ricchezze da lui accumulate a prezzo di tante fatiche e sacrifici. Dopo aver tentato disperatamente di non arrendersi al male, ormai preda di rimorsi e nostalgie, egli muore solo, sotto lo sguardo infastidito e sprezzan te di un servo.

\paragraph{L'ottica straniante di Gesualdo}

Nella prima parte del passo domina il punto di vista di Gesualdo, che da spettatore osserva la vita del palazzo nobiliare. Da questa focalizzazione interna risalta soprattutto la sua estraneità a quell'ambiente: egli è un forestiero, ed è ben consapevole di essere tale. La sua ottica estranea però diviene straniante e gli consente di ergersi a giudice della realtà aristocratica, di coglierne con particolare acutezza gli aspetti negativi.

\paragraph{Valori borghesi e sperpero aristocratico}

Risalta allora la contrapposizione radicale di due visioni del mondo opposte, di due modi di vivere incompatibili. Gesualdo giudica secondo i valori borghesi, che si fondano sulla laboriosità, sulla capacità di far fruttare le ricchezze e di produrre altri beni, in un processo senza fine. Ciò che la sua prospettiva mette impietosa mente in luce nella vita aristocratica è invece la sterilità, l'improduttività, lo sperpero di beni senza alcun frutto, che ai suoi occhi ha qualcosa di insensato e sacrilego.

\paragraph{Il disprezzo per il ceto parassitario dei servi}

All'angoscia per lo spreco si unisce il disprezzo del borghese produttivo per il ceto dei servitori, ai suoi occhi solo una massa di parassiti che ozia e si ingrassa di quella ricchezza sperperata. Questo disordine, questo ozio sono anch'essi la negazione dell'ideale di lavoro accanito, ostinatamente indirizzato a un fine, che è proprio del borghese.

\paragraph{L'inautenticità della vita del palazzo}

L'occhio acuto di Gesualdo coglie inoltre la falsità, l'inautenticità di quel mondo e dei suoi rituali, che sono impostati su una recita di facciata, come quando i servi corrono al loro posto per ossequiare il duca che esce di casa, salvo poi far ricominciare il caos e la baraonda appena egli volta le spalle.

\paragraph{L'incomunicabilità fra padre e figlia}

Un altro tema fondamentale di questo passo è il rapporto di Gesualdo con la figlia. Gesualdo ha il culto dei valori familiari, dei sentimenti autentici e dei moti generosi. Amerebbe trovare nella figlia affetto e intimità, e da parte sua vorrebbe entrare nell'animo di lei, conoscerne i segreti per alleviare le pene che intuisce dal suo aspetto turbato, ma trova dinanzi a sé un muro impenetrabile. Questa incomunicabilità fra padre e figlia è il segno del fallimento di chi, pur credendo fondamentalmente nella fami glia e negli affetti, ha finalizzato tutta la sua vita alla conquista della «roba», finendo per trascurare proprio la famiglia.

Anzi, per preservare la «roba», Gesualdo ha condannato la figlia all'infelicità, impedendole di sposare l'uomo amato e obbligandola a unirsi con il duca de Leyra, che le ha assicurato il titolo nobiliare prestigioso. Così, nella freddezza e nel distacco della figlia, finisce per raccogliere ciò che egli stesso ha seminato. Nell'ottica di Gesualdo questa incomunicabilità è ricondotta a un dato di natura: è l'incomunicabilità fra il sangue plebeo dei Motta e quello aristocratico dei Trao (anche se il personaggio cerca di scacciare con fastidio il pensiero che Isabella non sia veramente sua figlia).

\paragraph{Gesualdo morente visto dall'esterno}

Nell'ultima pagina il punto di vista muta e il protagonista è visto dall'esterno, da una prospettiva estranea. Questa svolta è un tocco geniale da grande narratore: affidando la morte di Gesualdo all'occhio indifferente e annoiato di un servo, insensibile sino alla disumanità, Verga trova il modo più efficace per far sentire il fallimento umano di Gesualdo, che muore in assoluta solitudine, e più in generale per esprimere una visione desolatamente pessi mistica sulla possibilità dei rapporti umani.

\paragraph{Gesualdo «vinto»}

Pur essendo materialmente un vincitore nella lotta per la vita», Gesualdo è in realtà un «vinto»: è vero che la sua figura è caratte rizzata dalla dimensione grandiosamente epica dell'accumulo di ricchezza e dall'eroica ascesi ad esso consacrata (ancora ben visibile in queste pagine), ma quella scelta di vita ha portato inesorabilmente al fallimento sul piano umano. Dalla conclusione del romanzo si può ricavare l'atteggiamento di Verga verso la realtà del moderno capitalismo.

%sono arrivato qui il 7 mar 2021 - 17:40

\chapter{Decadentismo}

\section{Introduzione}

\begin{itemize}
\item
  \pagine{pp. 326-327}
\end{itemize}

Il \evidenziatore{termine} , in senso più stretto, ha una accezione
prevalentemente negativa. I protagonisti stessi si chiamano così, e
mettono in luce gli aspetti negativi. Questo termine appare nella prima
volta nel testo \textsc{Il languore}.

In \evidenziatore{senso più ampio} , critici successivi fanno
riferimento con questo termine ad una serie di tematiche e di poeti
propri della seconda metà dell'ottocento e che perdurano per la prima
parte del novecento. Utilizzando il termine in una accezione così ampia,

Il modello è \evidenziatore{\textbf{Baudelaire}} , ma la corrente vera e
propria risale agli anni 80 del 1800, fino ai primi decenni del '900.
Alcune tematiche, le più emblematiche, sono presenti ancora oggi.

Il decadentismo (in senso stretto) prende le mosse da
\evidenziatore{Verlaine} , che esprime una serie di temi che si rifanno
alla poesia di Baudelaire. Verlaine esprime ed elenca i poeti
``maledetti'', e indica Baudelaire come una sorta di modello.

È probabile che l'influenza della città di Parigi abbia giocato un ruolo
decisivo nello sviluppo dei temi decadentisti in
\evidenziatore{Baudelaire} , prima di quanto succedesse altrove.

\section{Decadentismo e Romanticismo}

\begin{itemize}
\item
  \pagine{pp. 335-338}
\end{itemize}

Bisogna evitare di creare delle \evidenziatore{cesure nette}: dobbiamo
renderci conto che sul piano di determinate tematiche il decadentismo
riprende passo passo le tematiche e i motivi che sono stati tipici del
romanticismo, che a sua volta varia molto da luogo a luogo.

Tutte le \evidenziatore{tematiche negative} del romanticismo sono anche
elementi tipici del decadentismo. L'atteggiamento però è diverso, e
cambiano alcuni aspetti del pensiero filosofico: un conto è opporsi al
positivismo e alla mentalità scientifica, durante il suo sviluppo; un
altro è opporsi e descriverlo quando questo è già in atto, e quando sono
evidenti le conseguenze negative.

Con le prime espressioni del romanticismo il
\evidenziatore{ruolo del letterato} entra in crisi; adesso
l'intellettuale vive l'impossibilità di adeguarsi alla società. Il tema
dell'artista incompreso è già presente nel periodo romantico. Ora questo
tema sembra essere riproposto dai poeti decadent, visto che
l'intellettuale si vede totalmente fuori luogo in una società in cui
regna l'interesse e il profitto. Mentre l'intellettuale romantico, però,
lottava e si dimenava con passione contro questa situazione, ora il
poeta cede la mano.

Vi è una sorta di \evidenziatore{stanchezza} che spinge alcuni autori a
cambiare atteggiamento, che è un po' il tratto distintivo della
corrente. È la stanchezza tipica della fine di un'epoca

\section{T: \textsc{Languore}}

Questa poesia, oltre ad aver proposto per la prima volta il termine
decadenza, contiene tutti i temi, anche di tipo poetico, che sono
presenti nella letteratura decadente.

\begin{verse}
\poemlines{5}
Sono l’impero alla fine della decadenza,\\
che guarda passare i grandi Barbari bianchi\\
componendo acrostici indolenti3 dove danza\\
il languore del sole in uno stile aureo.\\!
Soletta l’anima soffre di noia densa al cuore.\\
Laggiù, si dice, infuriano lunghe battaglie cruente.\\
O non potervi, debole e così lento ai propositi,\\
non volervi far fiorire un po’ quest’esistenza!\\!
O non volervi, o non potervi un po’ morire!\\
Ah! Tutto è bevuto! Non ridi più, Batillo?\\
Tutto è bevuto, tutto è mangiato6! Niente più da dire!\\!
Solo, un poema un po’ fatuo che si getta alle fiamme,\\
solo, uno schiavo un po’ frivolo che vi dimentica,\\
solo, un tedio d’un non so che attaccato all’anima!
\end{verse}

\finepoesia

Il soggetto del primo periodo è ``Io''.

In questi poeti sono presi in considerazione dei periodi storici tipici
dei mondi alla fine. Esempio tipico è il periodo alessandrino con i suoi
regni ellenistici, così come la caduta dell'Impero Romano d'Occidente.

Ci sono dei riferimenti alla \pagine{poetica}: il poeta, vedendo
sfilare le orde barbariche, non fa niente \emph{per indolenza}, ma sa lì
a comporre acrostici indolenti: fa riferimento alle pratiche poetiche di
puro virtuosismo. La poesia sarà molto selettiva; il poeta si rivolge ad
altri poeti che come lui siano in grado di capire: si usano dei termini
che diventano talvolta misteriosi e allusivi: la parola non è più
portatrice di un significato letterale, ma spesso è evocatrice.

Il languore fa riferimento alla stanchezza: anche il sole, simbolo di
energia e potenza, sta languendo.

L'immagine densa di nebbia riporta l'immagine alla poesia di Carducci.

È presente il tema della rinuncia: il poeta non è in grado di agire, è
un \textbf{inetto}; non ha la volontà di agire. La reazione è opposta a
quella romantica, dove il titanismo sovraneggiava.

Nell'ultima parte del sonetto è presente il tema dell'inutilità della
letteratura: manca la sostanza.

\paragraph{Il motivo della «decadenza»} Il sonetto è esemplare di una disposizione ideologica e sentimentale del Decadentismo, considerato come una letteratura della crisi, del riflusso, della «decadenza» appunto. Il poeta propone subito un'analogia con la «fine della decadenza» dell'Impero romano, identificandosi in esso (attraverso l'uso della prima persona) e assumendo quindi su di sé tutte le caratteristiche negative attribuite a questo periodo storico: la debolezza, la corruzione, la fuga nell'evasione e nel divertimento fatuo, l'incapacità di valutare e fronteggiare i pe ricoli della realtà. La corrispondenza tra un passato così lontano e il presente ha un valore polemico, in quanto implica un rifiuto dei miti del progresso ufficiale, le sue sicurezze ideologiche e la letteratura che ad esse si ispira; al loro trionfalismo si oppone un senso incombente di dissoluzione e di morte.

\paragraph{Il senso della fine} In \textsc{Languore} questa tematica viene sviluppata in alcuni dei suoi momenti essenziali: 
\begin{enumerate}
\item la passività nei confronti dei grandi e drammatici eventi della storia;
\item la poesia come segno di indolenza ed espressione di un «languore» profondo, che, privo di messaggi morali e sociali, si risolve in un puro esercizio formale (lo «stile d'oro»), raffinato e prezioso; 
\item la solitudine, che riprendendo la grande lezione baudelairiana – si trasforma in un senso opprimente di «noia»;
\item la sensazione della fine, che rende estenuati i piaceri materiali e spirituali, vuota l'attesa del futuro. L'idea che ormai ogni esperienza è stata provata («Tutto è bevuto, tutto è mangiato!» v. 11) finisce per vanificare ogni prospettiva di impegno intellettuale e spiritua le («Niente più da dire!» v. 11); è così ribadita l'inutilità della poesia, che, proponendosi come simbolo della crisi, viene presentata anch'essa come un valore «fatuo», effimero eperituro, da gettare «alle fiamme».
\end{enumerate}

\section{T: \textsc{L'albatro}}

Cronologicamente parlando, questa poesia si colloca almeno una
generazione prima dei poeti decadenti. L'autore, però, viene considerato
una sorta di modello, in quanto con la sua opera e la sua esperienza
esistenziale, egli anticipa molto l'esperienza decadentista.

In moltissime delle sue poesie, emergono tantissime delle sue tematiche
che saranno approfondite dai poeti decadenti: prima fra tutti il ruolo
dell'intellettuale.

Questa poesia è molto semplice: nell'ultima strofa è presente un
paragone chiarissimo tra poeta e albatro.

Baudelaire ci dice che il poeta finché vola alto, isolato, è compreso,
perché ci sono solo altri poeti. Quando scende sulla terra, è costretto
a misurarsi con i comuni esseri mortali e diventa ridicolo: è
sbeffeggiato.

L'intellettuale si trova in contrasto non solo con il mondo delle
persone comuni, che non possono intendere la poesia, ma si trova tra due
fuochi: in questo contesto da una parte vi è il proletariato urbano, e
dall'altra i padroni, gli imprenditori, in perenne
\evidenziatore{lotta di classe}. L'intellettuale si trova preso tra
queste due forze sociali, e non riesce a immedesimarsi in nessuna delle
due.

Egli quindi vive un doppio rifiuto, e
nell'\evidenziatore{immagine del gabbiano} goffo sulla nave, vi è
proprio la perfetta somiglianza con la situazione degli intellettuali.

I marinai si divertono alle spalle del gabbiano, e lo sbeffeggiano.
Nell'ultima strofa viene esplicitata la similitudine.

\begin{verse}
\poemlines{5}
Spesso, per divertirsi, i marinai \\
catturano degli albatri, grandi uccelli dei mari, \\
indolenti compagni di viaggio delle navi  \\
in lieve corsa sugli abissi amari.  \\!
L’hanno appena posato sulla tolda \\
e già il re dell’azzurro, maldestro e vergognoso, \\
pietosamente accanto a sé strascina \\
come fossero remi le grandi ali bianche.  \\!
Com’è fiacco e sinistro il viaggiatore alato! \\
E comico e brutto, lui prima cosi bello!  \\
Chi gli mette una pipa sotto il becco,  \\
chi imita, zoppicando, lo storpio che volava! \\!
Il Poeta è come lui, principe delle nubi  \\
che sta con l’uragano e ride degli arcieri; \\
esule in terra fra gli scherni, impediscono \\
che cammini le sue ali di gigante.
\end{verse}


\paragraph{Il conflitto fra l'intellettuale e la società} Nell'allegoria dell'albatro si trova l'enunciazione più compiuta della concezione romantica del poeta. L'albatro, con le sue ampie ali, signoreggia l'aria; ma, quando si posa sul suolo, proprio a causa delle ali non riesce a camminare ed appare goffo e ridicolo. Così il poeta ha le grandi ali della sua nobiltà spirituale, delle sue capacità intellettuali, del la sua sensibilità, della sua fantasia, che gli permettono di spaziare nei cieli della poesia e dell'ideale; ma, una volta mescolatosi agli uomini comuni, proprio il suo privilegio spirituale lo rende inadatto alla vita pratica e prosaica e lo trasforma in oggetto di scherno da parte della gente "normale".

Si delinea qui il conflitto tra l'intellettuale e il mondo borghese che è costitutivo della cultura ottocentesca. In una società che ha come valori fondamentali l'utile, l'interesse, la produttività, la razionalità calcolatrice, il senso pratico, e che trasforma anche l'opera d'arte in merce, l'artista, teso verso l'ideale e creatore di un valore disinteressato come la bellezza, appare un diverso, inadatto alla vita comune. La società, considerandolo un essere inutile e improduttivo, lo priva del prestigio quasi sacrale e dei privilegi materia li di cui godeva in età precedenti, lo relega ai margini, lo guarda con scherno e sospetto. Da questa diversità e inettitudine si sviluppa nell'artista un oscuro senso di colpa, che lo fa sentire come un reietto e un maledetto. Ma egli reagisce rovesciando il senso di colpa e assumendo la propria diversità come segno di superiorità e nobiltà; rifiuta quel mondo che non lo comprende e, rivendicando orgogliosamente il proprio privilegio spirituale, si isola egli stesso, sdegnosamente, dalla vita normale, disprezzando la gretta mediocrità borghese.

\paragraph{Lo stile} L'antitesi fra il volo libero dell'albatro nell'azzurro e la sua ridicola degradazione sulla tolda della nave si riproduce nello stile: da un lato ricorre una serie di preziose immagi ni liriche, «abissi amari» (v. 4), «re dell'azzurro» (v. 6), «principe dei nembi» (v. 13), «ride degli arcieri» (v. 14); dall'altro ad esse si contrappongono immagini basse e comi che, il marinaio che stuzzica il becco con la pipa (indicata con un termine gergale, brûl egueule, “brucia gola"; v. 11), quello che lo imita zoppicando. Si può verificare qui la tensione interna che è tipica dello stile baudelairiano. In contrasto con questa disarmonia, il verso è invece fluido e musicale.

\section{T: \textsc{Perdita d'aureola}}

\begin{itemize}
\item
  \pagine{p. 337}
\end{itemize}

Il poema fa parte dello \textsc{Spleen di Parigi}. Lo
\evidenziatore{\textbf{spleen}} è un atteggiamento malinconico, di noia, ed è
termine ed emblema dei poeti maledetti.

Gli intellettuali si trovano d'innanzi alla scelta tra lo scrivere
\evidenziatore{testi commerciali} o fare poesia ``pura'', perseguendo i
propri ideali: D'Annunzio, ad esempio, andrà incontro alle esigenze del
pubblico, per vendere più opere. Baudelaire, invece, insieme ai poeti
maledetti, assumono degli atteggiamenti e degli stili di vita
completamente opposti. Il poetastro, probabilmente, scrive poesia
commerciale

\begin{quote}
«Oh! Come! Voi qui, mio caro? Voi in questo luogo malfamato! Voi, il bevitor di
quintessenze! Voi, il mangiator d'ambrosia! Davvero, ne sono sorpreso!».

«Mio caro, vi è noto il mio terrore dei cavalli e delle carrozze. Poc'anzi, mentre
attraversavo il boulevard in gran fretta, e saltellavo nella mota, in mezzo a 
questo mobile caos, dove la morte arriva al galoppo da tutte le parti ad un
tempo, la mia aureola, ad un movimento brusco che ho fatto, m'è scivolata
giù dalla testa nel fango del selciato. Non ho avuto il coraggio di raccoglierla.
Ho giudicato meno sgradevole il perdere la mia insegna che non il farmi
fracassare le ossa. E poi, ho pensato, non tutto il male vien per nuocere. Ora
posso andare a zonzo in incognito, commettere delle bassezze e abbandonarmi
alla crapula come i semplici mortali. Ed eccomi qui, assolutamente simile a
voi, come vedete!».

«Dovreste almeno far affiggere che avete smarrita codesta aureola, o farla
reclamare dal commissario».

«No davvero! Qui sto bene. Voi solo mi avete ravvisato. D'altronde, la
grandezza m'annoia. E poi penso con gioia che qualche poetastro la raccatterà
e se la metterà in testa impudentemente. Render felice qualcuno, che piacere!
E soprattutto render felice uno che mi farà ridere! Pensate a X, o a Z!... Eh?
Che cosa buffa, sarà!...». 
\end{quote}

Il poemetto in prosa ha la forma di un dialogo tra il poeta e un amico che si meraviglia di trovarlo in un luogo malfamato. Il poeta spiega che può frequentare quei luoghi come i comuni mortali perché ha perso l'aureola che contrassegnava la sacralità, quasi la divinità del poeta nel passato.

Egli ha perso l'«aureola», cioè quella dignità sacrale, di sacerdote della Bellezza e della Poesia, che lo circondava nelle società aristocratiche del passato e gli garantiva una condizione privile giata e un forte prestigio sociale. La società borghese non assicura più al poeta questa dignità e questo prestigio, poiché altri sono in essa i valori dominanti. Per la coscienza comune, il poeta è divenuto un uomo come tutti gli altri. Il poeta finge ironicamente di accettare questa nuova condizione; in realtà si getta nel vizio proprio per accentuare la sua diversità dalla gente “normale", per negare polemicamente i valori perbenistici su cui si regge la società borghese. Così al posto del privilegio di un tempo si colloca una specie di privilegio negativo, rovesciato, quello del vizio e del male.

\section{Tra borghesia e proletariato}

\begin{itemize}

\item
  \pagine{p. 338}: Studiare molto bene
\end{itemize}

\section{Tematiche e motivi del decadentismo}

\begin{itemize}

\item
  \pagine{pp. 328-329}: La visione del mondo decadente
\item
  \pagine{pp. 329-331}: La Poetica del decadentismo
\item
  \pagine{p.332-334}: Temi e miti della letteratura decadente
\end{itemize}

Con il \pagine{positivismo} alcuni autori cercavano di esprimere a
livello teorico la funzione della letteratura: di tipo scientifico.
Esempio di questo è Zola. La letteratura doveva fornire una chiave di
lettura per la realtà. L'atteggiamento che fanno riferimento alla
cultura positivistica è della lettura della realtà in questi termini.

Il \evidenziatore{decadentismo}, invece, è completamente opposto. Il
gusto di questi poeti è la constatazione che la realtà non può essere
letta e interpretata con l'atteggiamento oggettivo, ma dietro ad una
facciata ci sono degli aspetti che non possono essere compresi in
termini razionali. Siamo nel periodo della \textbf{psicoanalisi}, con
Freud.

L'\evidenziatore{inconscio} si presenta come qualcosa che sfugge al
controllo razionale di un individuo. Freud cercherà di creare una
letteratura scientifica della psicoanalisi, ma intanto è stato scoperto
l'inconscio, una parte della realtà non spiegabile in termini
scientifici.

Questo crea uno sfasamento: quando si pensava di poter controllare tutto
con la scienza, nascono le paure. Pertanto, le tematiche fanno tutte
riferimento a strumenti di ricerca irrazionali: se nella cultura
positivista gli strumenti di analisi erano assolutamente razionali, ora
invece ci si affida a \evidenziatore{strumenti irrazionali}: intuito,
sogno, situazioni di ebrezza, malattia, oltre che ad altre categorie
quali l'estetismo, ovvero il bello come parametro per interpretare la
realtà e agire in essa.

Vi è una forte analogia tra
\evidenziatore{\textbf{simbolismo} e \textbf{decadentismo}}, al punto che talvolta
vengono usati come sinonimi. Ciò nonostante, il simbolismo è un'insieme
di strutture formali, differente dal movimento poetico vero e proprio.

Il poeta prende le \evidenziatore{distanze dal mondo}, non vuole più
essere capito. Crea un pubblico privilegiato, elitario, costituito da
altri poeti: la scrittura si fa più ermetica, le parole non sono più
termini che comunicano delle conoscenze oggettive e reali, ma sono
termini evocativi, preziosi, che rimandano ad altro e suscitano
emozioni.

Grande importanza viene data alla \evidenziatore{musicalità dei testi},
in quanto talvolta il suono è evocativo più del termine nudo e crudo con
il suo significato letterale.

In tal senso possiamo parlare di \evidenziatore{simbolismo}, in quanto
le immagini rimandano ad altro. Pascoli in questo è il poeta più
significativo. Nella sua poesia, infatti, spesso non si riesce neanche a
capire a cosa si faccia riferimento, ma sono semplicemente allusioni.

Il poeta è colui che è in grado di raccogliere le corrispondenze e i
significati nascosti, e diventa in tal modo \pagine{profeta}.

L'\evidenziatore{esteta} è colui che vede l'arte e il bello come unico
strumento conoscitivo, e come l'essenza che detta i parametri
comportamentali.

L'\evidenziatore{inetto}, protagonista dei romanzi di Svevo e
Pirandello, dimostra tutta la sua incapacità di relazionarsi nel
contesto attuale.

\chapter{Gabriele D’Annunzio}

\section{Vita}

\begin{itemize}
\item
  \pagine{p. 424-430}
\end{itemize}

D'Annunzio nasce nel 1863, e muore ne 1938. Nella sua vita sperimenta
tutti i generi letterari, fino al giornalismo, e affronta tutti gli
stimoli che gli vengono dalla letteratura, a partire da quella antica a
quella contemporanea, sul piano dell'imitazione.

Lui \evidenziatore{coglieva gli stimoli} che gli venivano da altri
autori, e le faceva sue; aveva una abilità straordinaria nel
verseggiare, e le prime raccolte di poesie risalgono ai suoi 16-17 anni.
Si ispira a modelli quali Carducci, che aveva cercato di riprodurre il
modo di verseggiare degli antichi; D'Annunzio ha una abilità estrema nel
riprodurre e rielaborare, non solo sul piano tecnico, ma anche
tematico/contenutistico. Ad esempio, il tema del superuomo, è una
rielaborazione discutibile dell'oltreuomo di Nietzsche.

D'Annunzio vive un cinquantennio di attività letteraria, e vive un'epoca
densissima di esperienze, in un modo tale che viene a crearsi, accanto
alla figura di D'Annunzio poeta e scrittore, anche il
\evidenziatore{Dannunzianesimo}. Egli infatti non lavorò solo sulle sue
opere, ma anche sulla sua vita: vivrà la sua vita come un'opera d'arte,
così come tipico del movimento \textbf{esteta}

Per \emph{Dannunzianesimo} si intende il lavoro di costruzione della
\evidenziatore{figura} da dare in pasto al pubblico, con determinati
scopi e per raggiungere determinati obiettivi; D'Annunzio non accettava
che l'intellettuale avesse perso quel ruolo prestigioso all'interno
della società.

Ciò lo distingue dai \evidenziatore{decadenti}? No, egli vive la stessa
atmosfera culturale, vive coscientemente l'idea che l'intellettuale non
rivesta più lo stesso ruolo prestigioso che rivestiva prima, ma non lo
accetta. Ciò non significa che riesca a ricostruire l'identità vincente
dell'intellettuale, ma che ci proverà con tutte le sue forze.

D'Annunzio si porrà una serie di obiettivi, ma ciò porterà ad una serie
di \evidenziatore{ambiguità} e di contraddizioni che non saranno mai
risolte. D'Annunzio ad esempio presenta personaggi estremamente vitali;
allo stesso tempo nelle sue opere i protagonisti vanno incontro a
delusione e fallimento, e ciò è un elemento tipico di molte sue opere.

Vi è questa sorta di \evidenziatore{\textbf{ossimoro}} nell'esistenza e
nell'opera di D'Annunzio. Quando egli crea la figura dell'*esteta-
(personaggio e scrittore), presenta un personaggio sprezzante del mondo
borghese e capitalista che stava prendendo il sopravvento; ma D'Annunzio
stesso cerca di accontentare quel pubblico che disprezza tanto.

La sua vita sembra essere \evidenziatore{fatta ad arte}, e ha sempre
come obiettivo primario quello di costruire un'immagine. D'Annunzio
dettò legge per il gusto di un'epoca: dopo di lui tutti gli autori non
potranno prescindere dalla sua presenza.

Egli nasce a Pescara; la sua \evidenziatore{terra d’origine} non è
distante dal meridione descritto da Verga, e non ne è molto diversa. In
\textsc{Terra vergine}, così come in \textsc{Canto Novo}, il contesto
contadino, povero, primordiale, fa la sua comparsa. Il modo in cui
D'Annunzio guarda a questo mondo contadino è molto diverso da quello di
Verga; anche l'interesse di denuncia è completamente assente in
D'Annunzio.

Studia in un liceo molto prestigioso; a 18 anni si trasferisce a Roma,
dove lascia l'università per dedicarsi al \evidenziatore{giornalismo}.
Si occupava di tutto, e dettava legge in termini di gusto.

Le sue \evidenziatore{avventure galanti} fanno parte della sua immagine
di uomo di successo che D'Annunzio voleva dare.

In corrispondenza della produzione letteraria, vanno formandosi delle
fasi con immagini differenti:

\section{Esteta}

Molto spesso è un artista dai gusti raffinati; è un individuo che pone
come criterio base per le sue scelte il bello e l'arte. D'Annunzio lo
persegue nella sua vita e nelle sue opere. *Il Piacere- è la
raffigurazione dell'esteta.

Sia nella vita che nell'opera di D'Annunzio sono molto importanti i
duelli.

L'esteta ha donne bellissime, e sia D'Annunzio che il protagonista de
\textsc{Il piacere} ne hanno.

È importante vedere come D'Annunzio guardi al protagonista de \textsc{Il
piacere}, Andrea Sperelli, perché è facile vedere l'ossimoro e il
contrasto sempre presente nella vita del poeta. Egli addirittura a
tratti e quasi critico nei confronti di questo personaggio, che è una
sorta di \emph{alter-ego} di D'Annunzio stesso.

\section{Bontà}

Questo periodo è legabile alla produzione del romanzo
l'\textsc{Innocente}: si può collocare tra gli anni 80 e 90; in questo
periodo egli subisce il fascino della letteratura russa, con il suo
scavo psicologico strenuante. È la storia di una vicenda omicida e del
pentimento.

\section{Superuomo}

\begin{itemize}
\item
  \pagine{p. 443-448}
\end{itemize}

Molti pensano che questa fase sia contrapposta a quella dell'esteta, ma
il superuomo è un uomo che si distingue, come l'esteta. La differenza
sostanziale è che l'esteta si isola, mentre il superuomo è colui che
vuole finalmente agire.

È una fase di fervente \evidenziatore{vitalismo}, perché l'esteta si
impone praticamente. Inizialmente è solo una produzione letteraria e
artistica, ma successivamente anche lui stesso vuole incarnare
l'immagine del superuomo. L'occasione si presenta con la prima guerra
mondiale.

Nel 1915 ritorna in Italia, dove si allea con gli interventisti. Entra
in \evidenziatore{guerra}, si arruola come volontario a 52 anni. Egli
trova nella guerra il contesto ideale per dare spazio alla figura del
superuomo. Va sugli aereoplani.

Sono famose alcune \evidenziatore{vicende} diventate
\evidenziatore{leggendarie}: la \textbf{beffa di Buccari} è una vicenda
che vede protagonista D'Annunzio e altri, che si recano con degli
autosiluranti in un golfo, bombardando gli austriaci; il \textbf{volo su
Vienna}: con qualche compagno si reca in volo su Vienna e lascia cadere
dall'aereoplano dei bigliettini con il tricolore Italiano; la
\textbf{vicenda di Fiume}: D'Annunzio autonomamente, dopo la guerra, fa
una marcia sulla città e si autoproclama sovrano della città di Fiume,
verrà arrestato. Negli anni successivi Mussolini lo metterà a tacere nel
Vittoriale.

\section{Opere}

\begin{itemize}
\item
  \pagine{p. 427-430 - L'estetismo e la sua crisi}
\end{itemize}

Sono tantissime, egli scrisse moltissimo, affrontando moltissimi generi:
dalla tragedia, alla poesia, alla scrittura autobiografica; anche perché
aveva proprio la curiosità di sperimentare il nuovo.

Le \evidenziatore{opere della giovinezza} guardano soprattutto a Verga
verista; ciò capita perché egli nasce a Pescara, un ambiente regionale,
povero, contadino, e quindi ci sono molti elementi che in qualche modo
permettono a D'Annunzio di riproporre le stesse ambientazioni delle
novelle e romanzi di Verga. Ci sono due opere dell'82, che si
corrispondono per le ambientazioni: \textsc{Terra Vergine} e \textsc{Canto
Novo}: la prima è una raccolta di racconti, un testo in prosa simile
alle raccolte di racconti di Verga; *Canto Novo- è una raccolta di
Poesie.

Il modo di riproporre queste \evidenziatore{ambientazioni} è totalmente
diverso da quello di Verga: quest'ultimo aveva una visione pessimista,
atta a denunciare la situazione. Lo sguardo di D'Annunzio, invece, è
profondamente diverso; queste ambientazioni rappresentano per lui un
grande stimolo e una grande attrattiva, in quanto li vede come ambienti
primordiali, dalle forti passioni, con personaggi sensuali e che vivono
sentimenti estremi. Non vi è alcun altro interesse. La tragedia \textsc{La
figlia di Iorio} è estremamente significativa; ne \textsc{I pastori} vi è un
ambiente pastorale, in cui emergono, attraverso oggetti simbolici,
questi elementi di nostalgia per questo mondo primitivo che esercita il
suo fascino su D'Annunzio.

Durante la fase dell'\evidenziatore{Estetismo}, D'Annunzio scrive
\textsc{Il piacere}. Il protagonista è l'emblema dell'esteta, ovvero colui
che immagina di concepire e di forgiare la propria vita come un'opera
d'arte, per cui l'unico valore in cui crede e che guida le sue azioni è
la bellezza. È un personaggio che prova disprezzo per la società
borghese capitalistica in cui vive, in quanto mette l'interesse al primo
posto. L'esteta sdegnosamente si separa da questa società. D'innanzi a
queste costruzioni, D'Annunzio decide che deve incarnare il modello e il
ruolo dell'esteta; di fatto però c'è sempre un sottofondo di qualcosa
che contrasta, dalla malattia alla donna che pone ostacoli; la figura
della donna è sempre molto importante, e addirittura nel \textsc{Piacere}
ce ne sono due; sarà l'amore nei confronti di uno delle due che non
consentirà al protagonista di raggiungere i suoi grandi obiettivi. I
personaggi vengono sempre a scontrarsi con un qualche tipo di realtà o
di sentimento che si erano proposti come obiettivi.


\section{T: \textsc{Un ritratto allo specchio}}

\begin{itemize}
\item
  \pagine{p. 431}
\end{itemize}

È molto interessante il punto di vista del
\evidenziatore{\textbf{narratore}}, che ci permette di capire come sia
problematica la realizzazione di questo personaggio: l'autore ha
costruito un personaggio come lo voleva concepire, ma è molto critico
sia nei suoi confronti che in quelli di Elena (la Femme Fatale).

La connotazione di molti dei termini utilizzati per
\evidenziatore{descrivere Elena} è negativa; l'aspetto dominante
negativo è quello della falsità e della doppiezza. Tutto ciò è
strettamente legato all'esigenza di incarnare il simbolo dell'esteta.

Nei programmi il personaggio vuole crearsi una identità di un certo
tipo, che nell'intenzione dovrebbe essere positiva, ma in realtà non è
così: assomiglia quasi all'inetto, vista l'incapacità di agire.

Elena è un doppio di Andrea Sperelli, e Andrea Sperelli è un doppio di
D'Annunzio. La focalizzazione è quella del protagonista, che si
identifica con l'autore.

Nel momento in cui l'esteta raggiunge il suo massimo livello espressivo
è già accompagnato da segni di decadenza.

\begin{quote}
Chi era ella mai?

Era uno spirito senza equilibrio in un corpo voluttuario. A similitudine di tutte le creature avide di piacere, ella aveva per fondamento del suo essere morale uno smisurato egoismo. La sua facoltà precipua, il suo asse intellettuale, per dir così, era l’imaginazione: una imaginazione romantica, nudrita di letture diverse, direttamente dipendente dalla matrice, continuamente stimolata dall’isterismo. Possedendo una certa intelligenza, essendo stata educata nel lusso d’una casa romana principesca, in quel lusso papale fatto di arte e di storia, ella erasi velata d’una vaga incipriatura estetica, aveva acquistato un gusto elegante; ed avendo anche compreso il carattere della sua bellezza, ella cercava, con finissime simulazioni e con una mimica sapiente, di accrescerne la spiritualità, irraggiando una capziosa luce d’ideale.

Ella portava quindi, nella comedia umana, elementi pericolosissimi; ed era occasion di ruina e di disordine più che s’ella facesse publica professione d’impudicizia.

Sotto l’ardore della imaginazione, ogni suo capriccio prendeva un’apparenza patetica. Ella era la donna delle passioni fulminee, degli incendii improvvisi. Ella copriva di fiamme eteree i bisogni erotici della sua carne e sapeva transformare in alto sentimento un basso appetito....

Così, in questo modo, con questa ferocia, Andrea giudicava la donna un tempo adorata. Procedeva, nel suo esame spietato, senza arrestarsi d’innanzi ad alcun ricordo più vivo. In fondo ad ogni atto, a ogni manifestazione dell’amor d’Elena trovava l’artifizio, lo studio, l’abilità, la mirabile disinvoltura nell’eseguire un tema di fantasia, nel recitare una parte dramatica, nel combinare una scena straordinaria

[...]

Ben però, in qualche punto, egli rimaneva perplesso, come se, penetrando nell’anima della donna, egli penetrasse nell’anima sua propria e ritrovasse la sua propria falsità nella falsità di lei; tanta era l’affinità delle due nature. E a poco a poco il disprezzo gli si mutò in una indulgenza ironica, poichè egli comprendeva. Comprendeva tutto ciò che ritrovava in sè medesimo.

Allora, con fredda chiarezza, definì il suo intendimento.

Tutte le particolarità del colloquio avvenuto nel giorno di San Silvestro, più d’una settimana innanzi, tutte gli tornarono alla memoria; ed egli si piacque a riconstruir la scena, con una specie di cinico sorriso interiore, senza più sdegno, senza concitazione alcuna, sorridendo di Elena, sorridendo di sè medesimo. ― Perchè ella era venuta? Era venuta perchè quel convegno inaspettato, con un antico amante, in un luogo noto, dopo due anni, le era parso strano, aveva tentato il suo spirito avido di commozioni rare, aveva tentata la sua fantasia e la sua curiosità. Ella voleva ora vedere a quali nuove situazioni e a quali nuove combinazioni di fatti l’avrebbe condotta questo giuoco singolare. L’attirava forse la novità di un amor platonico con la persona medesima ch’era già stata oggetto d’una passion sensuale. Come sempre, ella erasi messa con un certo ardore all’imaginazione d’un tal sentimento; e poteva anche darsi ch’ella credesse d’esser sincera e che da questa imaginata sincerità avesse tratto gli accenti di profonda tenerezza e le attitudini dolenti e le lacrime. Accadeva in lei un fenomeno a lui ben noto. Ella giungeva a creder verace e grave un moto dell’anima fittizio e fuggevole; ella aveva, per dir così, l’allucinazione sentimentale come altri ha l’allucinazione fisica. Perdeva la conscienza della sua menzogna; e non sapeva più se si trovasse nel vero o nel falso, nella finzione o nella sincerità.

Ora, questo a punto era lo stesso fenomeno morale che ripetevasi in lui di continuo. Egli dunque non poteva con giustizia accusarla. Ma, naturalmente, la scoperta toglieva a lui ogni speranza d’altro piacere che non fosse carnale. Oramai la diffidenza gli impediva qualunque dolcezza d’abbandono, qualunque ebrezza dello spirito. Ingannare una donna sicura e fedele, riscaldarsi a una grande fiamma suscitata con un baglior fallace, dominare un’anima con l’artifizio, possederla tutta e farla vibrare come uno stromento, habere non haberi, può essere un alto diletto. Ma ingannare sapendo d’essere ingannato è una sciocca e sterile fatica, è un giuoco nojoso e inutile.
\end{quote}

\paragraph{I procedimenti narrativi} Nei primi paragrafi ci troviamo di fronte a un discorso interiore del personaggio, in forma indiretta libera. Da «Così, in questo modo» interviene invece il narratore che pronuncia espliciti giudizi sul personaggio («con questa ferocia», «esame spietato», «cinico sorriso interiore»). Il narratore dunque non lascia interamente la parola ad Andrea Sperelli, ma introduce la sua prospettiva per prendere da lui le distanze. Si manifesta in tal modo quell'atteggiamento critico dell'autore verso il suo eroe che è la struttura portante di tutto il romanzo. Però
anche Andrea è critico verso se stesso e vede con spietata lucidità dentro il proprio animo.

\paragraph{La critica all'estetismo} Ciò su cui si esercita l'analisi corrosiva di Andrea Sperelli (e dell'autore dietro di lui) è il nucleo centrale stesso del suo estetismo. Nei primi due libri del romanzo l'eroe è costantemente presentato nell’atto di sovrapporre alla vita le sue costruzioni estetiche: ogni cosa, paesaggi naturali e scorci cittadini, situazioni, volti, gesti, arredi, passando attraverso i suoi occhi richiama particolari di opere d'arte famose (ad esempio la bellezza di Elena Muti è accostata a quella della Danae dipinta dal Correggio, i vasi che adornano la stanza sono simili a quelli del tondo di Botticelli).

Ora Andrea arriva a mettere a nudo la menzogna che si cela dietro tali sublimazioni estetizzanti: anche in lui, come in Elena, le «fiamme eteree» mascherano semplicemente «bisogni erotici» della carne, gli impulsi sensuali più materiali e volgari. È il momento di massima consapevolezza dell'eroe (e dello scrittore stesso), forse la chiave migliore per penetrare l'intero romanzo. Si misura qui con perfetta chiarezza come l'immagine dell'e steta entri in crisi, e d'Annunzio, inquieto e insoddisfatto, voglia prendere le distanze da essa, denunciandone le mistificazioni e le intime debolezze. In realtà sappiamo che l'este tismo esercita ancora un forte fascino sullo scrittore, per cui alla durezza della critica si mescola anche un ambiguo vagheggiamento dell'eroe. Ciò è rivelato chiaramente nel passo che segue.

\section{T: \textsc{Il parricidio di Aligi}}

\begin{itemize}
\item
  \pagine{p. 465}
\end{itemize}

Quest'opera teatrale si distingue un po' dalle altre: l'ambientazione è
molto semplice, di pastori, e il protagonista è proprio un pastore
(differentemente dalla consuetudine delle tragedie).

\includegraphics[width=12cm]{parte1}

\includegraphics[width=12cm]{parte2}

\includegraphics[width=12cm]{parte3}

\includegraphics[width=12cm]{parte4}

Anche nell'antica Roma il parricidio era il delitto che veniva punito
nel peggior modo possibile: era una società patriarcale; D'Annunzio più
volte aveva inneggiato alla società romana, ma in questa tragedia il
personaggio positivo (egli è una sorta di artista, e qualcuno ha voluto
vedere in questa figura l'artista che si separa e si ritira in un mondo
altro) commette un parricidio. Sembra quasi quindi che D'Annunzio
condanni questo sistema a cui ha sempre inneggiato. I critici hanno
provato a dare delle interpretazioni, che però non risultano
convincenti:
\begin{itemize}
\item Aligi e Mila rappresentino la figura dell'artista che si
separa dal resto del mondo e che comunque andrà incontra ad una
sconfitta; 
\item un'altra interpretazione, di stampo psicoanalitico, vede
nella tragedia una sorta di complesso di Edipo di D'Annunzio nei
confronti del padre: avrebbe quindi riversato su Lazaro i suoi
sentimenti per suo padre.
\end{itemize}

\paragraph{Una società patriarcale} La «tragedia» (così denomina l'opera d'Annunzio stesso) è ambientata in un mondo agricolo-pastorale arcaico, barbarico, immobile. D'Annunzio sente un'attrazione fortissima per il primitivo e si compiace di registrare minuziosamente costumanze, comportamenti, credenze, rituali magico-religiosi, formule, scongiuri, proverbi. Ciò fa sì che il testo offra ricchi materiali ad una lettura sociologica ed antropologica.

Il rapporto tra Lazaro di Roio ed il figlio Aligi è tipico di una società patriarcale. Lazaro è il padre-padrone, che ha sui figli una potestà assoluta, senza limiti, la stessa che esercita sul podere, sugli animali, sugli strumenti di lavoro. Lo stesso potere illimitato il padre rivendica sulle donne: esse sono a sua disposizione, per questo ritiene di aver diritto al possesso su Mila, e la vuol prendere anche con la forza.

\paragraph{I "diversi"}

Ma questo assetto patriarcale, pur così immobile, non è privo di contraddizioni interne. Aligi, in quel mondo così compatto, è il "diverso", che si emargina dalla società agricola: è inetto ai lavori dei campi, e per questo sta solitario sui monti a guardare le pecore. È mite e pio, non ha l'aggressività animalesca, sensuale e violenta dei contadini (Lazaro, i mietitori di Norca che inseguono Mila); ha una sensibilità più elevata, ed ha subito pietà della perseguitata con la quale intreccia un rapporto del tutto casto. La sua posizione in quella società è quella dell"intellettuale", dell'artista: non solo si isola dal gruppo sociale in una totale solitudine, vivendo come trasognato in un'altra dimensione, ma crea anche primitivi prodotti artistici (intaglia un angelo nel legno e vuol portarlo in dono al papa a Roma). Mila ha una posizione simmetrica e affine a quella di Aligi. Anch'essa è diversa dal grup po, in quanto maga e prostituta vagante: proprio per questa sua diversità è disprezzata e perseguitata (oltre che temuta) da tutti. In realtà la sua diversità è superiorità spirituale:
innanzitutto accetta il legame casto e spirituale con Aligi; poi è disposta a sacrificarsi per ché Aligi possa rientrare nella famiglia, dalla madre e dalla sposa, dimostrandosi pronta anche al suicidio; infine si addossa la colpa dell'uccisione di Lazaro, facendosi giustiziare eroicamente al posto di Aligi. Aligi e Mila sono gli elementi refrattari che non accettano l'ordine ferreo della società agricolo-patriarcale, ne minano dall'interno la compattezza; per questo motivo quella società si coalizza contro di loro, e finisce per dominarli: Mila subisce il supplizio del rogo, Aligi viene reintegrato nell'ordine. L'ordine sociale conservatore trionfa di fatto, ma ideal mente trionfa l'eroina ribelle, sublime e tragica, con il suo sacrificio.

D'Annunzio da un lato è affascinato dall'ordine conservatore della società patriarcale, una società fortemente organica, governata da pochi valori certi e indubitabili: ed in questo si manifestano le sue posizioni reazionarie, tese a ricercare e a valorizzare il carattere originario della stirpe. Ma il suo estetismo individualistico possiede anche una carica di rifiuto e rivolta eversiva nei confronti degli assetti costituiti, e ciò lo spinge a vagheggiare con compiacimento la diversità dell'``artista” Aligi e dell'eroina sublime Mila, che contestano quell'assetto, opponendosi alla prepotenza del patriarca. La tragedia è tutta percorsa da questa ambivalenza: ma d'Annunzio è lo scrittore dell'ambiguità, e questo forse è ancora oggi uno degli aspetti più interessanti della sua opera. Il linguaggio mira a rispettare da un lato il primitivismo della materia, dall'altro la solennità sublime del tragico

\section{Le Laudi}

\begin{itemize}

\item
  \pagine{pp. 470-472}: Le Laudi
\end{itemize}

Non è la prima raccolta di poesie, ma \textsc{Le Laudi} rappresentano un
momento maturo della vita di D'Annunzio. Fanno parte di un grande
progetto poetico di D'Annunzio, che aveva immaginato un ciclo di
svariati libri che raccogliessero all'interno delle poesie che dovevano
essere legate da un filo conduttore.

Questo modo di approcciarsi alla materia non è nuovo: grandi progetti e
grandi cicli, che devono esprimere un disegno che il poeta ha in mente.

Le poesie del libro \textsc{Alcyone} è l'unica a non essere stata
criticata, in quanto sembra essere l'unica poesia \textbf{pura} di
D'Annunzio; nonostante tutti siano d'accordo nel giudicare il poeta come
un grande verseggiatore, la critica ha sempre stroncato il manierismo di
D'Annunzio, il suo messaggio puramente politico e propagandistico; la
poesia di \textsc{Alcyone} sembra essere l'unica depurata e sciolta dal
messaggio che imperversa nelle altre opere.

D'Annunzio, in Italia, come Pascoli, fa un primo passo verso il
simbolismo tanto caro in Francia. La Poesia comunica attraverso la sua
forma in maniera assoluta.

Il ciclo prevedeva molti più libri, ma come sempre dopo \textsc{Maia},
\textsc{Elettra} e \textsc{Alcyone} non se ne fece nulla. I libri che aveva
progettato non vennero scritti, e la raccolta sostanzialmente si
interrompe dopo i primi tre.

\textbf{Maia} è il primo libro, ed è un poemetto più che una raccolta di
poesie: abbiamo la trasfigurazione di un viaggio (simbolico), che fa
riferimento ad un viaggio che D'Annunzio fece in Grecia con Eleonora
Duse. Questo viaggio era un viaggio nel mondo classico, antico,
prezioso, più volte vagheggiato da D'Annunzio. Qualche anno prima egli
si era schierato apertamente contro la speculazione edilizia e la
modernità. In \textsc{Maia} il protagonista, quando torna a casa, si
imbatte in quella barbarie descritta sopra. Abbiamo un capovolgimento
però, in quanto il protagonista ribalta i suoi principi, ed inizia ad
apprezzare quegli aspetti della società moderna. Il protagonista è un
superuomo. D'Annunzio quindi esprime attraverso la poesia i suoi
cambiamenti.

\textsc{\textbf{Alcyone}} (\pagine{p. 482-483}) è la raccolta migliore di
D'Annunzio. È una raccolta di poesia con un filo conduttore, che è una
estate. È una sorta di cronaca diario di un'estate, trascorsa da
D'Annunzio insieme ad Eleonora Duse. Il destinatario di queste poesie è
quindi lei. L'inizio è una tarda primavera, con i suoi acquazzoni e il
tempo variabile, fino ad inizio settembre.

In questo libro si esprime al meglio il tema del \emph{panismo}, ovvero
la fusione dell'uomo nella natura. ``\emph{pas}'', ovvero la radice del
termine, in greco significa tutto; inoltre Pan è una divinità boschiva.

In queste poesie diventa importante il significante, così come le
suggestioni: solo l'animo sensibile e del poeta è in grado di
interpretarle; è una sorta di linguaggio segreto. Ci troviamo di fronte
ad una poesia semplice, senza messaggi reconditi.

\subsection{T: \textsc{La sera Fiesolana}}

Una delle prime del ciclo, siamo all'inizio dell'estate, stimolato da
una gita che D'Annunzio e la Duse avevano fatto ad Assisi, città mistica
con messaggi spirituali, giornata calda. La sera, tornando a Fiesole e
ammirando il crepuscolo i due sono avvolti da tutta una serie di
sentimenti e suggestioni, il poeta rende in parole tutte queste
sensazioni. L'inizio è un bozzetto paesaggistico, ma poi la poesia
rimane su questo tono, e questi bozzetti rappresentano i sentimenti che
l'autore vuole esprimere, attraverso diverse figure retoriche, come
allitterazioni e parole onomatopeiche.

\setcounter{mar}{0}

\subsubsection{Prima parte}

\begin{verse}
Fresche le mie parole ne la sera\\
ti sien come il fruscìo che fan le foglie\\
del gelso ne la man di chi le coglie\\
silenzioso e ancor s’attarda a l’opra lenta\\
su l’alta scala che s’annera\\
contro il fusto che s’inargenta\\
con le sue rame spoglie\\
mentre la Luna è prossima a le soglie\\
cerule e par che innanzi a sé distenda un velo\\
ove il nostro sogno si giace\\
e par che la campagna già si senta\\
da lei sommersa nel notturno gelo\\
e da lei beva la sperata pace\\
senza vederla.\\!
Laudata sii pel tuo viso di perla,\\
o Sera, e pe’ tuoi grandi umidi occhi ove si tace\\
l’acqua del cielo!\\!
\end{verse}

Le parole sono oggetto di fusione fra la natura e l'uomo, allitterazione
della f e onomatopea, che richiamano i suoni delle foglie, che fanno
questo rumore quando un contadino silenzioso le raccoglie. Abbiamo
l'immagine del contadino che appoggia la scala su un albero di gelso,
man mano che scende la sera la scala crea un'immagine sull'albero.
Apparizione della luna, motivo che piace molto anche a Pascoli, la luna
si preannuncia con il chiarore prima ancora di apparire, simil teofania,
ovvero apparizione della divinità. Attraverso ciò che noi vediamo del
paesaggio, sembra che la divinità si esprima attraverso le sue forme e
le sue immagini, gli elementi si fanno vivi e l'uomo spera che evochino
qualcosa o che ci confortino. La luna prima di apparire distende un velo
del cielo, che corrisponde al chiarore, i due amanti contemplano il
paesaggio come in un sogno. Il gelo non è il freddo, siamo in estate, è
un refrigerio che il poeta immagina di sera dopo una giornata calda. La
campagna già spera di godere della pace notturna accompagnata dalla
freschezza.

\subsubsection{Seconda parte}

\begin{verse}
\setverselinenums{18}{20}
Dolci le mie parole ne la sera\\
ti sien come la pioggia che \textit{bruiva}\mat{è un termine dotto, che piace anche a Pascoli e Montale, e
  a Verlaine, vuol dire ``che fa rumore''. È un saluto lacrimoso della
  primavera che se ne sta andando.}\\
tepida e fuggitiva,\\
commiato lacrimoso de la primavera,\\
su i gelsi e su gli olmi e su le viti\\
e su i pini dai novelli rosei diti\\
che giocano con l’aura che si perde,\\
e su ’l grano che non è biondo ancóra\mat{Immagine del grano che cambia colore durante la stagione,
  trascolorazione.}\\
e non è verde,\\
e su ’l fieno che già patì la falce\\
e trascolora,\\
e su gli olivi, su i \textit{fratelli}\mat{Epiteto francescano: l'olivo come simbolo di pace e ``fratello''} olivi\\
che fan di santità pallidi i clivi\\
e sorridenti.\\!
Laudata sii per le tue vesti \textit{aulenti}\mat{altro aggettivo che piace, significa profumato},\\
o Sera, e pel \textit{cinto}\mat{Similitudine, immagine suggestiva, il ``cinto'' è qualcosa che cinge
  la sera come il salice veniva usato per legare i covoni di fieno.Cinto: linea dell'orizzonte che circoscrive la sera} che ti cinge come il salce\\
il fien che odora!\\!
\end{verse}

Il ``laudata si'' fa riferimento a San Francesco, persona molto legata
alla natura e agli animali. Il viso di perla fa riferimento a Piccarda,
ci suggerisce una compenetrazione dell'uomo con la natura, elementi di
fusione e confusione, sono analogie che si colgono nei dettagli e nei
particolari che vengono accostate senza similitudini o metafore. La sera
è personificata, gli umidi occhi della sera dove si tace l'acqua del
cielo sono le pozzanghere, che danno anch'esse senso di refrigerio,
l'acqua e le lacrime hanno funzione purificatrice. La simbologia
dell'acqua è spesso utilizzata

\subsubsection{Terza parte}

\begin{verse}
\setverselinenums{35}{35}
Io ti dirò verso quali reami\\
d’amor ci chiami il fiume, le cui fonti\\
eterne a l’ombra de gli antichi rami\\
parlano nel mistero sacro dei monti;\\
e ti dirò per qual segreto\\
le colline su i limpidi orizzonti\\
s’incùrvino come labbra che un divieto\\
chiuda, e perché la volontà di dire\\
le faccia belle\\
oltre ogni uman desire\\
e nel silenzio lor sempre novelle\\
consolatrici, sì che pare\\
che ogni sera l’anima le possa amare\\
d’amor più forte.\\!
Laudata sii per la tua pura morte,\\
o Sera, e per l’attesa che in te fa palpitare\\
le prime stelle!
\end{verse}

Rimanda un po' a Leopardi, vago e indefinito, anche D'Annunzio usa
termini vaghi, immagine complessiva vaga e indefinita, non si deve dare
una spiegazione. L'immagine è quella delle colline che si stagliano
nella sera, suggestioni legate a credenze popolari legate al mondo
classico, il poeta ha la sensazione di vedere le colline come delle
labbra di donne che stanno per suggerire qualche mistero o segreto, però
non possono parlare. Osservandole resta la sensazione di serenità
propria del sentimento di quando si sta per assistere a qualcosa di
piacevole. Nel silenzio notturno si sente il rumore del fiume, evoca
l'amore e un'idea legata al mondo classico, dove le fonti erano
considerate luoghi sacri, molto spesso anche i luoghi naturali, come le
grotte. Il paesaggio assume caratteri antropomorfi e il poeta vede le
colline come labbra che devono svelare un segreto, ansiose di parlare,
ma un divieto divino glielo impedisce. La sensazione piacevole di attesa
rimane, considerate come elemento che possa dare consolazione. Ripresa
di ``laudata si''. La morte non ha accezione negativa, ma considerata
come momento di riposo, come il fresco della sera per la campagna.

\finepoesia

\paragraph{La prima strofa: l'apparizione della luna} Ogni strofa, svolgendo un suo motivo, è autonoma dalle altre e forma quasi una lirica a sé; tanto che, nella prima pubblicazione sulla “Nuova Antologia” nel novembre del 1899, ciascuna recava un titolo particolare. Quello della prima era \textsc{La natività della luna}: infatti in essa l'immagine centrale, intorno a cui si compongono tutte le altre, è appunto il sorgere della luna che assume il valore arcano di una teofania, cioè di un'apparizione della divinità. La luna è sempre stata vista come una divinità nelle religioni primitive ed in quelle antiche, e d'Annunzio nelle sue opere frequentemente si compiace di recuperare queste figurazioni antiche del passato.

Se la luna che nasce ha qualcosa di divino, solo la parola del poeta può evocarla: le sue parole, che risuonano «fresche» nel silenzio della sera, sono come la formula magico-liturgica che propizia l'apparizione della divinità. Ma non viene descritto il sorgere effettivo della luna: sarebbe un evento troppo concreto e preciso, che dissiperebbe l'aura di sospensione magica e arcana. Il poeta sceglie di evocare, con sottili suggestioni, l'attimo inafferrabile che precede il sorgere della luna. È uno di quei momenti indefiniti, ambigui, che sono prediletti da d'Annunzio.

La luna, non ancora rivelatasi, distende dinanzi a sé un velo luminoso, e già in esso la campagna si sente come sommersa dal gelo notturno, bevendo il refrigerio atteso prima ancora che la luna compaia. Simmetricamente rispetto all'inizio, in cui una sensazione uditiva, le parole del poeta e il fruscio delle foglie, si fonde con una sensazione tattile, la freschezza, anche nell'ultima parte della strofa si crea una rete di segrete analogie, fondate sulla sinestesia. Il tenue velo argenteo irradiato dalla luna, quindi una realtà visiva, è assimilato al «notturno gelo», dunque a sensazioni tattili. Al gelo si associa anche un'idea di liquidità: la campagna «beva» la pace, la luce argentea è come un liquido fresco che dà refrigerio alla campagna, riarsa dal sole e assetata.

La complessa rete di immagini, luce lunare-gelo-liquidità-pace, allude quindi all'azione miracolosa della luna-divinità: col suo apparire dà refrigerio, riporta la vita dove era l'aridità. Ma allora risulta un altro legame segreto, tra il «gelo» diffuso dalla luna e la «freschezza delle parole del poeta. La parola poetica e l'apparizione divina sono intimamente collegate. Le parole hanno le stesse prerogative divine dell'entità mitica che porta il refrigerio e la vita.

\paragraph{La personificazione della sera}

Il motivo dell'acqua che dà refrigerio alla terra è riproposto dalla "ripresa" inframmezzata alle strofe («pe' tuoi grandi umidi occhi ove si tace /l'acqua del cielo!», vv. 16-17). Anche qui al centro vi è una figurazione mitica, affine alla luna, la sera personificata in una divinità femminile. E numerosi sono i legami tematici con la strofa precedente. Il «viso di perla» della sera si presenta come ripresa in chiave metaforica del motivo del la luce lunare e si armonizza con la tonalità cromatica che ne scaturiva, il colore argenteo. Come la luna porta il refrigerio della sua fredda luce, che è assimilata ad una sostanza liquida, così la sera porta il refrigerio della pioggia (anticipando così anche il tema della strofa successiva).

Il carattere religioso della figurazione femminile, però, più che al mito antico rimanda qui ad una religiosità francescana, come denuncia l'eco del \textsc{Cantico delle creature} («Laudata sii [...]»). Anche il «viso di perla» della sera, attraverso la mediazione delle mistiche donne dello Stilnovismo, può evocare certe immagini della Vergine nella pittura duecentesca. D'Annunzio ama molto queste commistioni di sacro e profano, di sensualità e liturgia, ed ha un senso tutto estetizzante della religiosità cattolica.

\paragraph{La seconda strofa: partitura musicale e linguaggio analogico }

La seconda strofa è costruita su procedimenti apparentemente più semplici e lineari ma in realtà molto elaborati. In primo luogo prevale la partitura musicale: la parola tende a divenire puro suono, a dissolversi in musica, grazie alla modulazione degli accenti e delle rime, alla qualità timbrica dei suoni. Il gioco delle immagini ripropone la metafora dell'acqua già presente nella precedente "ripresa", la pioggia tiepida di giugno, con cui la primavera prende commiato (il titolo della strofa nella prima pubblicazione era appunto \textsc{La pioggia di giugno}).

La strofa si chiude con un'immagine ancora di sapore religioso, quella degli olivi, francescanamente chiamati «fratelli»: si istituisce quindi di nuovo un legame con la trama mitico-religiosa che percorre la prima strofa e la “ripresa”. L'immagine degli olivi insiste anch'essa su un gioco analogico («i fratelli olivi / che fan di santità pallidi i clivi», vv. 29-30): il verde argenteo delle foglie dell'olivo dà come una sfumatura di pallore ai clivi; al tempo stesso gli olivi, per tradizione, sono sempre stati considerati simbolo di umile santità. Il legame tra le due sfere di immagini è reso dal "pallore", che da un lato è un dato materiale, riferito al colore, dall'altro si collega metaforica mente all'idea di santità, evocando immagini di macerazione ascetica. Questi procedi menti analogici, così abilmente costruiti da d'Annunzio, saranno poi ripresi da tanta poesia nel Novecento.

\paragraph{La terza strofa: una sensualità panica}

La seconda "ripresa" segna il passaggio ad una tematica diversa, si può dir quasi contrastante, che dominerà poi nell'ultima strofa. Il nucleo centrale è il profumo della sera («le tue vesti aulenti», «il fien che odora»), un'immagine quindi più sensuale e voluttuosa. La tematica della terza strofa, infatti, ha il suo centro nel motivo amoroso. La poesia trapassa dal senso di sacralità arcana della prima strofa (la teofania della luna), alla musicalità languida e misticheggiante della seconda (gli olivi “francescani") ad una sensualità panica e naturalistica, con un voluto contrasto di toni. Anche qui non manca una sospensione mitico-religiosa. Le fonti «eterne» dei fiumi parlano nel «mistero sacro» dei monti, all'ombra degli «antichi» rami: c'è come un'eco del culto antico per le fonti e i boschi, abitati dalle divinità. Ma il messaggio arcano delle fonti allude a «reami / d'amor», quindi ad una forza erotica che pervade la natura ed in cui l'uomo si immedesima. Sensuale è anche la trasfigurazione delle colline in labbra, chiuse da un divieto ma ansiose di rivelare il loro segreto; che sarà un segreto, come si può intuire dai versi precedenti, di vita piena e gioiosa, di esperienze amorose sublimi, di bellezza ineffabile e oltreumana: non è del tutto assente quindi neppure nella \textsc{Sera fiesolana}, ritenuta una delle liriche più “pure" di \textsc{Alcyone}, la sostanza ideologica del superomismo.

\subsection{T: \textsc{La pioggia nel pineto}}

Siamo nel mezzo dell'estate, all'interno della raccolta di
\textsc{Alcyone}.

La \evidenziatore{scena immaginata} è quella di due amanti (D'Annunzio
ed Eleonora Duse) si trovano in una pineta ai margini di una spiaggia e
vengono colti da un temporale: il poeta mette in verse una serie di
suggestioni che questa situazione implica; i temporali estivi sono molto
veloci e violenti, e questa pioggia riversandosi su un terreno molto
secco e pieno di piante, si sprigiona una serie di profumi dal terreno,
e si può avvertire una sorta di musica provocata dal rumore della
pioggia sulle piante; si uniscono anche le voci di alcuni animali estivi
e il rumore del mare. Sentiamo quindi una melodia che non è composta di
parole umane, ma è composta dalla natura.

È la poesia in cui si realizza l'immagine del \pagine{panismo}; i due
amanti, colti da questa pioggia (l'acqua ha il valore di rinascita e
purificazione, con l'aggiunta della crescita dell'amore), si sciolgono e
si fondono nella natura (ulteriore valore dell'acqua). La donna viene
colpita dalla pioggia, e quindi i suoi capelli aderiscono al corpo, i
vestiti quasi si sciolgono. La metamorfosi però riguarda solo
\textbf{determinati individui}, i superuomini; è una esperienza
\emph{esclusiva} e \emph{totale}; l'uomo quasi ricava l'immortalità, in
quanto la natura è l'unico elemento immortale; con la fusione l'uomo
riesce a percepirlo.

Il poeta usa la \evidenziatore{musicalità}, prodotta dalla natura, per
esprimere la fusione: figure retoriche, versi brevi; questa poesia ha il
ritmo della pioggia: talvolta accelera, talvolta rallenta, provocando
rumori diversi; tutte queste suggestioni sono rese al meglio da
D'Annunzio.

In questa poesia sono molto presenti le rime di qualsiasi tipo.

\setcounter{mar}{0}

\begin{verse}
\textit{Taci}\mat{Rende evidente l’aspetto fonico di questa poesia; l’io lirico si sta riferendo alla donna in quanto donna e in quanto essere umano; egli vuole ascoltare il rumore della natura}. Su le soglie\\
del bosco non odo\\
parole che dici\\
umane; ma odo\\
parole \textit{più nuove}\mat{musica creata dalla natura stessa}\\
che parlano gocciole e foglie\mat{Non ci sono similitudini, né metafore. Qui c’è l’analogia; è forse la figura retorica emblematica del simbolismo, in quanto talvolta manca proprio il collegamento tra i due termini paragonati: collegamenti veloci e molto suggestivi}\\
\textit{lontane}\mat{richiamo a Leopardi con i suoi termini vaghi ed indefiniti}.\\!
Ascolta. Piove\\
dalle nuvole sparse.\\
Piove su le tamerici\\
salmastre ed arse,\\
piove su i pini\\
scagliosi ed irti,\\
piove su i \textit{mirti}\mat{pianta sacra a venere}\\
divini,\\
su le ginestre fulgenti\\
di fiori accolti,\\
su i ginepri folti\\
di coccole \textit{aulenti},\\
piove su i nostri vólti\\
\textit{silvani}\mat{Significativo perché fa riferimento al termine latino “selva”: inizia il panismo},\\
piove su le nostre mani\\
ignude,\\
su i nostri vestimenti\\
leggieri,\\
su i freschi pensieri\\
che l’anima schiude\\
\textit{novella}\mat{è novella per la funzione purificatrice},\\
su la \textit{favola bella}\mat{storia d'amore}\\
che ieri\\
t’illuse, che oggi m’\textit{illude}\mat{L’etimologia latina fa riferimento al termine “gioco”, e il fatto che utilizzi questo verbo per fare riferimento all’amore ci dice che è un gioco destinato a finire.},\\
o \textit{Ermione}\mat{Nome importante, figlia di Elena e Menelao}.\\!
Odi? La pioggia cade\\
su la \textit{solitaria}\mat{senza presenza umana}\\
verdura\\
con un crepitìo che dura\\
e varia nell’aria\\
secondo le fronde\\
più rade, men rade.\mat{provocano un suono diverso quando colpite dalla pioggia}\\
Ascolta. Risponde\\
al \textit{pianto}\mat{la pioggia è paragonata al pianto del cielo} il canto\\
delle \textit{cicale}\mat{iniziano ad uscire le voci degli animali}\\
che il pianto australe\\
non impaura,\\
né il ciel cinerino.\\
E il pino\\
ha un suono, e il mirto\\
altro suono, e il ginepro\\
altro ancóra, stromenti\\
diversi\\
sotto innumerevoli \textit{dita}\mat{le dita sono goccie}.\\
E immersi\\
noi siam nello spirto\\
silvestre,\\
d’arborea vita viventi;\\
e il tuo vólto \textit{ebro}\mat{la donna è felice}\\
è \textit{molle}\mat{bagnata} di pioggia\\
come una foglia,\mat{si paragonano elementi della donna con la natura: realizzazione del \textbf{panismo}}\\
e le tue chiome\\
auliscono come\\
le chiare ginestre,\\
o creatura terrestre\\
che hai nome\\
Ermione.\\!
Ascolta, ascolta. L’accordo\\
delle \textit{aeree}\mat{stanno in alto, e sono contrapposte alle rane} cicale\\
a poco a poco\\
più sordo\\
si fa sotto il \textit{pianto}\mat{pioggia}\\
che cresce;\\
ma un \textit{canto}\mat{canto delle rane} vi si mesce\\
più roco\\
che di laggiù sale,\\
dall’umida ombra remota.\\!
Più sordo e più fioco\\
s’allenta, si spegne.\\
Sola una nota\\
ancor trema, si spegne,\\
risorge, trema, si spegne.\mat{tutti aggettivi che fanno riferimento all'intensità della musica}\\
Non s’ode voce del mare.\\
Or s’ode su tutta la fronda\\
crosciare\\
l’argentea \textit{pioggia}\mat{sinestesia}\\
che \textit{monda}\mat{purifica},\\
il croscio che varia\\
secondo la fronda\\
più folta, men folta.\\!
Ascolta.\\
La \textit{figlia dell’aria}\mat{cicala}\\
è muta; ma la \textit{figlia}\\
\textit{del limo}\mat{rana} lontana,\\
la rana,\\
canta nell’ombra più fonda,\\
chi sa dove, chi sa dove!\\
E piove su le tue ciglia,\\
Ermione.\\!
Piove\mat{anadiplosi: ripresa degli ultimi versi di una strofa} su le tue ciglia nere\\
sì che par tu pianga\\
ma di piacere; non bianca\\
ma quasi fatta \textit{virente}\mat{la donna, da sempre connotata come bianca, diventa ``virente'', verde: \textbf{panismo}},\\
par da scorza tu esca.\\
E tutta la vita è in noi fresca\\
aulente,\\
il cuor nel petto è come pèsca\\
intatta,\\
tra le pàlpebre gli occhi\\
son come polle tra l’erbe,\\
i denti negli alvèoli\\
son come mandorle acerbe.\mat{paragoni tra le parti della donna con parti della natura: \textbf{panismo}}\\!
E andiam di fratta in fratta,\\
or congiunti or disciolti\\
(e il verde vigor rude\\
ci allaccia i mallèoli\\
c’intrica i ginocchi)\mat{la metamorfosi è completa: \textbf{panismo}}\\
chi sa dove, chi sa dove!\\
E piove su i nostri vólti\\
silvani,\\
piove su le nostre mani\\
ignude,\\
su i nostri vestimenti\\
leggieri,\\
su i freschi pensieri\mat{chiusa circolare}\\
che l’anima schiude\\
novella,\\
su la favola bella\\
che ieri\\
m’illuse, che oggi t’illude,\\
o Ermione.
\end{verse}

Questa poesia si esaurisce in sé stessa, non c’è un fine ultimo, ma ci dice qualcosa solo nel momento in cui lo leggiamo; alcuni non sono d’accordo, in quanto è presente il panismo.
Nel momento in cui la leggiamo percepiamo il suo valore assoluto, con le sue percezioni e suggestioni.

\paragraph{La struttura musicale} La poesia ha un'evidente struttura musicale: le quattro strofe sono organizzate come i movimenti di una sinfonia. Nella sinfonia generale della pioggia il poeta distingue il suono diverso di varie voci, il rumore delle gocce a seconda delle foglie più o meno rade, il canto delle cicale, il canto roco delle rane, che sono come strumenti solisti che si alternano al "pieno" dell'orchestra. Grazie al suo straordinario virtuosismo verbale, d'Annunzio mira a trasformare la parola in musica, in linea con i dettami fondamentali della poetica decadente (Verlaine, \textsc{Arte poetica}, v. 1: «La musica, più d'ogni cosa [...]»). Ma la partitura musicale della lirica vuole essere a sua volta la riproduzione, o la traduzione in linguaggio umano, di un'altra musica, quella composta dalla pioggia. È una situazione affine a quella di \textsc{Lungo l'Affrico}, in cui la «Grazia del ciel» deve effondersi come «musica» nel canto del poeta, o quella della \textsc{Sera fiesolana}, in cui le sue parole sono «fresche» come il «fruscio» delle foglie del gelso. Vi è, secondo la metafisica decadente, una rispondenza profonda tra la parola poetica e la realtà oggettiva; la parola è collegata con l'essenza stessa, misteriosa e segreta, delle cose, è come la formula magica in grado di svelarla.

\paragraph{Il tema panico} Al centro di tutto il discorso si pone il tema panico dell'identificazione del soggetto umano con la vita vegetale, che torna insistente, sviluppato con numerose variazioni. Il poeta e la donna sono viventi «d'arborea vita», il volto della donna è «molle di pioggia / come una foglia» (vv. 57-58), i capelli profumano come ginestre, Ermione è una creatura «terrestre», che scaturisce dalla terra come la vegetazione, oppure sembra uscire dalla scorza degli alberi, come le ninfe della mitologia antica. L'identificazione culmina nell'ultima strofa: il cuore delle due creature umane è «come pèsca / intatta», gli occhi sono «come polle tra l'erbe», i denti «come mandorle acerbe».

\paragraph{Gli strumenti formali} La partitura musicale della poesia è costruita con strumenti formali sofisticatissimi. Innanzitutto la metrica, che è estremamente libera, non soggetta ad alcuno schema tradizionale. Si succedono versi brevi, senari, settenari, ottonari, novenari, ma compaiono persino versi trisillabi, composti da una sola parola («lontane», «divini», «silvani», «leggieri»...). Questa estrema frammentazione dei versi tende a riprodurre la pluralità innumerevole di presenze e di voci che si affollano nella pineta sotto le fitte gocce di pioggia.

Altro strumento per eccellenza del virtuosismo musicale di d'Annunzio sono le rime, che ricorrono anch'esse molto liberamente, senza alcuno schema fisso, e inoltre la modulazione fonica, data dalla variazione di timbri chiari delle /a/ e di quelli cupi delle /o/ in sillaba accentata. Alla qualità musicale del discorso poetico dà un contributo fondamentale anche la modulazione fonica. Basti osservare la variazione tra i toni chiari delle /a/ e i toni cupi delle /o/ toniche in questi versi: «e varia nell'aria / secondo le fronde / più rade, men rade», che pare quasi avere un'intenzione mimetica della varietà dei suoni delle gocce sulle foglie. Ancora il canto limpido delle cicale che si diffonde nella vastità dell'aria è reso con la predominanza della vocale /a/ nelle sedi toniche: «al pianto il canto / delle cicale/ che il pianto australe...»; mentre il suono più cupo e roco del canto delle rane è reso con il predominio delle più oscure vocali /o/ e /u/: «dall'umida ombra remota», «nell'ombra più fonda». Così il tremolio del canto delle cicale in diminuendo è reso con la frequenza della vibrante /r/: «ancor trema [...] risorge». A questo virtuosismo metrico e timbrico si unisce anche l'uso scaltrito di numerosi procedimenti retorici. In primo luogo l'anafora, come la serie insistita dei «piove» nella prima strofa; ma anche l'epifora, come, nel celebre "calando" del canto delle cicale, la triplice ripetizione a fine verso della clausola «si spegne»; inoltre allitterazioni («ciel cinerino», «spirto / silve stre», «vita viventi», «limo lontana», «verde vigor»), paronomasie («ombra» / «fronda»); vi è poi la ripetizione a distanza di certe clausole con minime variazioni, con un gusto da ritornello di canzone popolare, come si notava («secondo le fronde / più rade, men rade», «secondo la fronda / più folta, men folta»); il ritardo del nome della donna sempre collocato al termine del periodo sintattico e della strofa: «che oggi m'illude, / o Ermione», «che hai nome / Ermione», «E piove su le tue ciglia, / Ermione», «che oggi t'illude, / o Ermione». Si può notare infine la serie di mosse interlocutorie verso la destinataria del discorso («Taci», «Ascolta», «Odi?», «Ascolta, ascolta»), che suonano come l'invito a partecipare ad un mistero iniziatico, il mistero della fusione panica con la natura vegetale, officiato dalla pioggia che purifica. Ma anche qui insistere solo sullo straordinario virtuosismo del gioco verbale e metrico potrebbe risultare fuorviante. Come si è già notato per altre poesie di \textsc{Alcyone}, anche in questo caso non si tratta di poesia "pura", come si ama spesso affermare; l'ideologia ha infatti nel discorso complessivo un peso determinante. Essa si manifesta sia nella volontà di cogliere il messaggio di una vita oltreumana mediante le parole «più nuove» della pioggia, sia nella convinzione che la propria parola sia lo strumento privilegiato per afferrare e restituire tale linguaggio arcano, sia nell'invito ad una trasfigurazione attraverso la fusione con la condizione vegetale. Che il motivo ideologico sia toccato con maggior leggerezza che altrove (\textsc{Meriggio}, ad esempio) non deve far dimenticare che esiste.

\section{Il periodo notturno}

\begin{itemize}
\item
  \pagine{p. 511}
\end{itemize}

\subsection{T: \textsc{La prosa ``notturna''}}

\begin{itemize}
\item
  \pagine{p. 512}
\end{itemize}

\end{document}
