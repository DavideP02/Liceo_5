\documentclass[a4paper, twoside, titlepage]{book}

\usepackage[T1]{fontenc}
\usepackage[utf8]{inputenc}
\usepackage[italian]{babel}

\usepackage{microtype}
\usepackage{xcolor}
\usepackage{titlesec}
\usepackage{xargs}
\usepackage{multicol}
\usepackage{amsmath}
\usepackage{amssymb}
\usepackage{quoting}
\quotingsetup{font=small}

\usepackage{ulem}
\usepackage{framed}
\usepackage{amsmath}
\usepackage{graphicx}

\DeclareUnicodeCharacter{200B}{{\hskip 0pt}}

%definizioni particolari
\newcommand{\straniero}[1]{\textit{#1}} %parole straniere
\newcommand{\titolo}[1]{\textsc{#1}} %titoli
\newcommand{\evid}[1]{\textbf{\textcolor{blue}{#1}}} %parole evidenziate
\newcommand{\salt}{\hspace{1em}[...]} %comando per la creazione dei puntini di sospensione tra quadre
\newcommand{\elenco}[1]{%
\begin{itemize}
#1
\end{itemize}}
\newcommand{\elencon}[1]{%
\begin{enumerate}
#1
\end{enumerate}}
%

%citazioni
\newcommand{\citazione}[1]{%
  \begin{quotation}
  \noindent #1
  \end{quotation}}
%

%definire header e footer
\usepackage{fancyhdr}
\pagestyle{fancy}

\fancyhf{}
\fancyhead[LE,RO]{\scshape\thepage}
\fancyhead[LO]{\scshape\footnotesize\nouppercase{\rightmark}}
\fancyhead[RE]{\scshape\footnotesize\nouppercase{\leftmark}}
%

%rimuovere header e footer dalle pagine vuote
\makeatletter
\def\cleardoublepage{\clearpage\if@twoside \ifodd\c@page\else
    \hbox{}
    \vspace*{\fill}
    \vspace{\fill}
    \thispagestyle{empty}
    \newpage
    \if@twocolumn\hbox{}\newpage\fi\fi\fi}
\makeatother
%

\renewcommand{\emph}[1]{\textcolor{blue}{#1}}

%%%%%% HYPERREF VA CARICATO SEMPRE PER ULTIMO, DEMENTE!

\usepackage[colorlinks, pdftex, pdfauthor={Davide Peccioli},
            pdftitle={Appunti Letteratura},
            pdfsubject={Svevo, Pirandello e Nietzsche}]{hyperref}
\definecolor{RoyalBlue}{rgb}{0.0, 0.14, 0.4}
\hypersetup{
     colorlinks=true,
     linkcolor=blue,
     filecolor=blue,
     citecolor = black,
     urlcolor=cyan,}

\begin{document}

\begin{titlepage} % pagina del titolo
\begin{center}
    \null
    \vfill
    {\huge \textsc{Appunti di Letteratura}}\\
    \vspace{2em}
    {\Large Pascoli, Svevo e Pirandello}\\
    \vspace{3em}
    {\large \textbf{Davide Peccioli}}\\
    \vspace{1em}
    {\large Classe 5\textsuperscript{a}\\\vspace{0.4em} 21 maggio 2021}
    \fancyfoot[C]{}
    \vfill
\end{center}
\end{titlepage}
\tableofcontents

\part{Pascoli}

\chapter{Introduzione e caratteri generali}

Pascoli è passato alla storia come il poeta delle piccole cose: in Pascoli abbiamo molte poesie, brevi, graziose, sembrano dei bozzettini impressionistici con immagini di campagnia, e quindi talvolta i termini possono apparire semplici.
Aveva una vera e propria passione per la botanica: abbiamo una serie di immagini e nomi di fiori, scene di campagna e della natura.

L'immagine di Pascoli come poeta delle piccole cose è piuttosto vecchia, e in tempi recenti è stata parecchio modificata, riconsegnandoi un Pascoli decisamente più interessante: le piccole cose che molto spesso troviamo nelle sue poesie sono \textbf{simboli}; Pascoli sembra essere giunto in modo autonomo a quelle soluzioni tipiche della poesia simbolista che si stavano sviluppando in Europa, soprattutto nella letteratura francese.

Tutte le scelte foniche, linguistiche, sintattiche e lessicali di Pascoli rimandano a qualcosa \textbf{d'altro}: conoscendo la personalità di Pascoli, la sua sintassi frantumata, i suoi suoni onomatopeici (figure retoriche apparentemente comunissime), spesso rimandano a qualcos'altro di molto più profondo:
\elenco{\item la \textbf{sintassi frantumata} rimanda ad un mondo che non trova più un ordine: i poeti non si ritrovano più, non si riconoscono più nella realtà che li circonda, e a tutto questo si aggiunge il vissuto personale di Pascoli, che specie nell'ultima parte della sua vita è afflitto da timori}
Pascoli quindi non è più visto come il poeta delle piccole cose: indubbiamente sono quelli gli strumenti che egli utilizza, ma al giorno d'oggi egli è visto come un poeta dalla psicologia non disturbata, ma sicuramente particolare: queste situazioni estreme consentono di avere una sensibilità tale da cogliere i segnali dal mondo.

In Pascoli, il poeta si identifica con il \textbf{fanciullino}, il poeta è colui che ha mantenuto quel tipo di atteggiamento un po' ingenuo nei confronti della realtà, con quella sensibilità distintiva tipica del bambino, che è in grado di cogliere quei segni della natura, quei segni sotterranei nascosti: attraverso la poesia Pascoli li restituisce al lettore.

La figura retorica dell'\textbf{analogia} diventa dominante: l'analogia funziona come accostamento tra due elementi, ma è un accostamento che non si basa più su nulla di esplicito: vengono accostati elementi distantissimi tra di loro, e questo accostamento inaspettato inusuale colpisce più profondamente.
L'analogia diventa una figura dominante, così come avviene in generale nella poesia simbolista


\section{Biografia}

Nasce nel 1855, in Romagna, in una famiglia appartenente ad una borghesia media benestante: molto patriarcale e numerosa. Il padre era fattore della tenuta dei Principi Tolonia.

Egli poteva permettersi il mantenimento più che onorevole di questa numerosa famiglia.
Fin da piccolissimo Pascoli entra a studiare in collegio, come i suoi fratelli.

Nel 1867 muore il padre, avvenuta attraverso omicidio: il 10 agosto del 1867, il padre di Pascoli viene ucciso a fucilate. Fu sconvolgente per la famiglia: omicidio crudele fatto per gelosia; la famiglia conosceva i mandanti.

La giustizia non giunse mai a nessuna conclusione: questo fu molto straziante per il poeta; non era soltanto la perdita del padre, ma anche il senso di profonda ingiustizia che il poeta vive in età molto tenera; questo diventa il tema dominante di quasi tutte le sue poesie, nonché ragione di una crescita psicologia particolare del poeta. La psiche del poeta si è fermata in quell'anno, a 12 anni.
Non a caso Pascoli, parlando della sua poetica, parlerà del \textbf{fanciullino}

Il fanciullino è colui che guarda la realtà con gli occhi del bambino, e riesce a cogliere degli elementi che fuggono all'adulto.

Le conseguenze di questo episodio sono vissute anche a livello economico: l'anno successivo alla morte del padre, con una frequenza spaventosa, inizieranno a morire molti altri elementi della famiglia: nei tre anni successivi la madre, la sorella, e altri fratelli, tra cui Luigi e il fratello più grande, che aveva preso le redini finanziarie.
Questo porta ad un altro tema della sua poesia, ovvero le difficoltà economiche: riuscì a studiare solo grazie a borse di studio e alla benevolenza dei suoi professori, che riconoscevano le doti di questo scrittore.

Divenne classicista e grechista, e ottenne la cattedra che era stata di Carducci. Partecipò a molti concorsi di poesia in lingue antiche.

Nella sua vita non capita quasi nulla, se non un episodio in età giovanile, quando studiava all'università a Bologna, nel 1879: Pascoli si avvicina alla vita politica militante: come giovane studente universitario si avvicina al socialismo ed inizia a partecipare ad alcune manifestazioni; ad una di queste viene catturato, e fece alcuni mesi di carcere. Questa fu una esperienza fu per lui traumatica, tanto che abbandonò la politica militante per sempre, ritirandosi in una vita in campagna, lontano dai riflettori.

Egli amava tantissimo la campagna, i fiori, nonché i riti della campagna; si racconta che amava frequentare i contadini, verso sera si ritirava molto spesso con loro a bere, giocando a carte. La sua dimensione ideale era quella della campagna.


\section{T: \textit{Biglietto per Ida}}

Vediamo sotto i nomi dei due fratelli, Giovanni e Mariù, incrociati sul fondo.
Attorno alla scritta ci sono foglie e cuori.
In alto c'è una stella, che è Ida (che si stava sposando).
C'è un nido con due uccellini che piangono, che rappresentano lui e la sorella Maria.

Gli unici fratelli che sono rimasti uniti fra di loro sono Giovanni, Maria e Ida.
È un nucleo famigliare che ha rifiutato la possibilità, dopo la tragedia dell'infanzia, di ricostruire una nuova famiglia, ma anzi ha sempre cercato di ricostruire quel nucleo finito molto tempo addietro.

Quando Ida si sposa, i due fratelli, che hanno un rapporto quasi morboso e inquietante, che ritorna prepotentemente nelle poesie, per gli altri due è come se nuovamente fosse venuto qualcosa a distruggere quel nido famigliare.

Pascoli ha avuto la possibilità di sposare una cugina, ma la reazione di Mariù è così terribile da farlo desistere. I carteggi tra Giovanni e la fidanzata hanno del terribile.

\section{Temi}

Sono tutti comportamenti anomali che ci mostrano lo stato psicologico di questo poeta.

Egli è quasi l'unico poeta che non ha scritto poesie d'amore. La tematica dell'amore, che è la più frequente in poesia, è trattato molto poco, e viene sempre visto come qualcosa di misterioso e torbido.

Questo deve essere una chiave di lettura per la poesia di Pascoli: non troviamo una ideologia politica, ma è \textbf{poesia pura}, con un grande uso degli aspetti musicali e fonici.

Con Pascoli abbiamo soluzioni pari a quelle del simbolismo francese, a cui il poeta giunge attraverso soluzioni personali.

Risulta molto difficile, di fronte ad un testo poetico, rispondere alla domanda "di cosa parla?"

Alcuni temi ricorrenti sono il \textbf{nido}, che rappresenta la famiglia che è stata distrutta, e che lui cerca di ricostruire in maniera \textit{non naturale}, in quanto cerca di farlo con gli stessi elementi del passato. Le sorelle diventano figure materne, e lui è un fanciullino di 12 anni.

Il tema della \textbf{morte} è onnipresente nella poesia di Pascoli. Si presenta con delle suggestioni che il poeta coglie e cerca di trasmetterci

Nella poesia di Pascoli è assente una parte filosofia espressa, come avveniva per Leopardi: è tutto lasciato a suggestioni, immagini e suoni.


\section{Pascoli e il positivismo}

Cronologicamente gli anni in cui Pascoli si forma sono gli anni del positivismo (anni '70), e del resto vediamo la sua mania di usare i termini scientifici per appellarsi ai fiori.

Per certi aspetti quindi Pascoli si colloca ancora in questo clima. Allo stesso tempo fa parte di questo clima decadente che caratterizza la fine del secolo.

\chapter{\textit{Myricae}}
\elenco{\item \emph{p. 553}}

Myricae è un termine latino, che significa tamerici (un arbusto).

È una raccolta poetica di Pascoli, iniziata nel 1891. All'inizio ci sono pochissimi testi, ma nelle edizioni successive vengono aggiunti nuovi componimenti: nell'edizione definitiva ci sono più di 100 componimenti.

È forse la raccolta di poesie più famosa di Pascoli.

Il titolo è una citazione di Virgilio, colta dal poeta con il significato contrario rispetto al termine usato da Virgilio.
Virgilio dice che non a tutti piacciono le myricae (ovvero le piante semplici), e che quindi per venire incontro a tutti lui avrebbe innalzato lo stile.
Pascoli ne fa il titolo della sua opera: indica con questo titolo che le sue poesie sono cose semplici (e per questo a lungo Pascoli era stato preso come il poeta delle piccole cose). La forza di queste poesie è però nel loro significato simbolico

\section{T: \textit{Arano}}
\elenco{\item \emph{p. 553}}

Il soggetto del titolo è "i contadini". In Pascoli il titolo diventa un elemento fondamentale per capire la poesia.

Abbiamo un paesaggio di campagna.

Non è un sonetto, è formato da due terzine e una quartina.

Sono molto presenti gli enjambement, che spezza e unisce allo stesso tempo. Si allunga il tempo della lettura: crea una attenzione particolare del lettore rispetto ai termini in questione.
Siccome si tratta di una spezzatura, in Pascoli molto spesso provoca e va a sostenere il senso di solitudine, di essere separato e strappato.

Fa parte di una sezione che si intitola “ultima passeggiata”, in cui lascia tutta una serie di poesie, piccoli bozzetti: rimandano ad una concezione realistica e coloristica, al punto che si definiscono quasi “impressionistiche”
Il paesaggio assume un significato simbolico.
Questi sono paesaggi che il poeta vede prima di recarsi in città.
Questo pendolarismo è effettivamente nella sua biografia

Pascoli ha ben presente la lezione di Esiodo: lo lesse, e in questa poesia ci trasmette la \textbf{fatica del lavoro}, tema molto presente in Esiodo.

Ciò che il poeta ci vuole trasmettere lo percepiamo in parte da ciò che ci dice, ma soprattutto lo percepiamo dall’uso di alcune forme retoriche, all’uso di termini che rimandano a colori.


\section{T: \textit{Lavandare}}
\elenco{\item P. 555}

Questa poesia fa sempre parte della sezione “l’ultima passeggiata”, e l’Io lirico passeggiando sente il canto delle lavandaie.
Nella seconda parte della poesia sono riportati dei versi di alcune canzoni popolari realmente esistenti, che si ricollegano al tema del suono già presente nella prima parte della poesia.

Questa poesia comunica un senso di abbandono e di solitudine. La poesia non ha un significato, un concetto, un tema: non vuole comunicare una idea.
Lo stesso pascoli diceva che la poesia non serviva a niente, non aveva finalità. La finalità sbucava fuori da sola: proprio per essere priva di finalità essa è apprezzabile solo per il piacere che può comunicare; la finalità diventa quindi quella di creare qualcosa di bello per l’uomo.

\section{T: \textit{X agosto}}
\elenco{\item P. 557}

Giorno della morte del padre.
Quando il padre venne colpito dalla fucilate, la cavalla che lo portava giunse comunque a casa (portando il cadavere).
C’è una poesia in cui Pascoli immagina la madre interrogare la cavalla a riguardo dell’omicidio.

Il 10 agosto è anche il giorno di San Lorenzo, in cui ci sono le stelle cadenti, che esprimono i desideri.
Pascoli interpreta questo suo dolore personale, questo suo lutto, come un lutto che investe tutta l’umanità: quindi interpreta le stelle cadenti come le lacrime del cielo


\section{T: \textit{Assiuolo}}
\elenco{\item \emph{p. 560}}

Il tema della morte arriva attraverso suggestioni.
La struttura è fantastica, tutto funziona alla perfezione, e le figure retoriche sono dosate in modo simmetrico.

C'è un esempio significativo dell'analogia.

L'assiuolo è un uccello notturno, e nel testo viene utilizzato il termine \textit{chiu}, come veniva chiamato in campagna; il suono riproduce il suo verso, e in campagna si pensava che questo verso portasse male.
Il suono è lugubre, ed evocano le figure dei morti.

L'immagine che si ha quello di figure di morti evocate dall'ambiente lunare e notturno, che bussano cercando di mettersi in contatto con il poeta: bussano a delle porte che non si possono più aprire: non si possono mettere in contatto con il poeta.
Questa disillusione è denunciata dalle parole della poesia, dallo schema delle figure retoriche: c'è un climax che si gioca sull'interezza della poesia (con i tre termini alla fine di ogni strofa).

C'è una struttura metrica assolutamente regolare, che però attraverso forzature, sintassi usate in modo estremo, uso delle parentesi, è cambiata e distrutta. All'interno di questa struttura regolare ci sono le forzature di pascoli, che mettono in evidenza qualcosa.

\section{T: \textit{Temporale}}
\elenco{\item \emph{p. 564}}

Non riusciamo a capire questa poesia se non guardando il titolo. Sono quasi ermetiche, fatte di suggestioni molto personali.

Il termine \textbf{impressionismo} viene utilizzato molto spesso nell'analizzare la poesia di Pascoli.

La poesia all'apparenza potrebbe sembrare un bozzetto impressionistico, con note di colore qua e là, ma in realtà non è così.
In letteratura l'esperienza del temporale ha sembra avuto valenza simbolica: da Boccaccio, per cui rappresenta un'insidia per la vita dell'uomo.
In Pascoli la tempesta ha un significato ancora più profondo: c'è un aspetto della sua biografica che è una sorta di punto fermo nella sua vita, un grande dolore che Pascoli ha vissuto da bambino: il temporale suscita paura.
Quello che ci vuole esprimere Pascoli non è tanto la funzione simbolica, lui ci vuole proprio far sentire la paura che si vive durante il temporale.

Si può notare la sintassi frantumata, sono tutti accostamenti, è difficile fare l'analisi del periodo: non c'è una sintassi compiuta. La sintassi frantumata allude a quella realtà che non ha più un ordine, che è frantumata per il soggetto, ma non solo: le esperienze personali del poeta diventano esperienze dell'umanità.

Guardando la poesia senza titolo potremmo anche non sapere quale sia il soggetto della poesia: questo aspetto sarà poi tipico della poesia ermetica, in cui il titolo diventa parte integrante del testo stesso, perché molto spesso noi non sapremmo neanche di cosa si stia parlando senza il titolo.

\elenco{\item \textit{bubbolio} è una onomatomea: è spesso usata in Pascoli; è una figura retorica che o riproduce il suono di qualcosa, oppure è un termine che riproduce nell'accostamento dei suoni il suono della cosa a cui si allude; \textit{bubbolio lontano} fa riferimento al rumore rombante di quando si avvicina il temporale
\item \textit{rosseggia l'orizzonte}: note di colore; colori e suoni sono sempre molto presenti nelle poesie di Pascoli, perché sono queli elementi che trasmettono sensazioni: queste note di colore sono quelle che lo fanno definire "impressionista"
\item \textit{nero di pece a monte}: molto spesso Pascoli utilizza queste espressioni: con questa scelta sembra di toccare più concretamente questo concetto: abbiamo spesso questo meccanismo: un sostantivo, il "di", e poi una cosa materiale
\item \textit{stracci di nubi chiare}: ecco un esempio di analogia; stracci fa allusione a qualcosa di stracciato, spezzato; stesso meccanismo di prima.
}


\section{T: \textit{Novembre}}
\elenco{\item \emph{p. 566}}

Questa lirica sarà molto apprezzata da Carducci.
Abbiamo una strofa saffica.
Pascoli usa una forma che si lega alla tradizione classica, a cui opera delle distorsioni e delle inarcature che rendono evidenti alcuni temi; sono scelte che stupiscono in questo tipo di struttura.

Solit bozzetto paesaggistico e impressionistico, che è simbolo di qualcosa. Novembre è simbolo di una stagione dell'anima.
In questa poesia abbiamo ben chiaro il sentimento di disillusione.

Novembre è uno dei mesi più bui e tristi: l'inverno sta iniziando ed è ancora molto lungo. Inoltre è la stagione dei morti.

Abbiamo una atmosfera ingannevole, in quanto l'atmosfera è gemmea, e ci sono segnali che ci fanno sembrare sia un giorno di primavera. Siamo disillusi al \textbf{verso 5}, con quel \textit{Ma secco}, quando il tema diventa quello dei morti.

È difficile dire di cosa parli la poesia: parla di sentimenti, sensazioni che durano qualche istante



\section{T: \textit{Il lampo}}
\elenco{\item \emph{p. 569}}

Da una prefazione in edita la terza edizione di \textit{Myricae} apprendiamo che fu concepito come metafora degli ultimi momenti del padre morente: "i pensieri che tu, o padre mio benedetto, facesti in quel momento, in quel batter d'ala - Il momento fu rapido [...] ma i pensieri non furono brevi e pochi. Quale intensità di passione! Come un lampo in una notte buia: dura un attimo e ti rivela tutto un cielo pezzato, lastricato, squarciato, affannato, tragico". A questa tematica si lega l'immagine finale dell'occhio aperto e subito richiuso, che richiama l'ultimo sguardo di un moribondo.

Fa parte della trilogia: \textit{Il temporale}, \textit{Il lampo} e \textit{Il tuono}

Ci trasmette una sensazione di paura, e nello spavento generale una visione: il lampo per un istante illumina a giorno la scena; la luce anziché confortare crea uno sgomento ancora maggiore, ed è lo stesso sgomento che avrebbe provato il padre sbarrando gli occhi prima di morire.

In questa poesia non si può più parlare di impressionismo di Pascoli, ma quasi di espressivismo: il poeta vuole esasperare determinati aspetti.

A differenza di altri componimenti di \textit{Myricae}, qui le impressioni visive non hanno neppure l'apparenza di oggettività impressionistica, sono immediatamente connotate da un valore simbolico e dotate di una forte carica espressionistica, portate ad un estremo di tensione deformante: si noti la terra che all'improvviso chiarore del lampo appare "ansante, livida, in sussulto" (\textbf{v. 2}), il cielo "ingombro, tragico, disfatto" (\textbf{v. 3}), la bianca casa paragonata all'occhio che si apre "largo, esterrefatto" per chiudersi subito dopo.


\chapter{I \textit{Poemetti}}

\elenco{\item \emph{p. 573}}

I \textit{Poemetti} vengono composti in diverse tappe, dal 1897 al 1909: con l'ultima edizione se ne aggiungono sempre di più rispetto alle precedenti.
Il tempo di composizione corrisponde a quello di \textit{Myricae}.
Le due opere sono profondamente diverse: i poemetti hanno una struttura prevalentemente narrativa, quindi anche se sono in versi sono composizioni molto più lunghe rispetto a quelle di \textit{Myricae}

Questa serie di poemetti va quasi a formulare un racconto; molti di essi sono collegati a due personaggi, Rigo e Rosa, collocati nel contesto campestre.
Infatti molti hanno riscontrato una forte analogia con le \textit{Georgiche} di Virgilio: l'ambientazione è la campagna, e nei vari poemetti sono descritti tutti i ritmi della natura e del lavoro in campagna. Oltre a questa lezione di Virgilio, è presente anche la lezione di Esiodo, con la sua opera \textit{Le opere e i giorni}.

L'opera viene intesa da molti come una sorta utopia regressiva, in quanto Pascoli riflette il suo ideale di vita della campagna (positivo, perché rappresenta una sorta di protezione nei confronti del mondo attuale, della politica e della violenza), però proiettato nel passato e non nel futuro. Pascoli non sembra cogliere tutti quegli aspetti negativi che erano invece presente in Verga.

\section{Posizioni politiche}

Non coglie la lotte di classe caratteristica di Verga: il suo atteggiamento è quello di una sorta di filantropia umanitaria, di tipo cristiano, che consiste in una sorta di benevolenza ed empatia nei confronti degli altri uomini; questo resta del suo socialismo.

Negli ultimi anni della sua vita Pascoli addirittura abbraccia posizioni nazionalistiche, e di favore nei confronti delle guerre per la fondazione di nuove colonie: nel 1911 appoggia con entusiasmo la guerra di Libia. In realtà c'è una spiegazione logica a questo atteggiamento, che si cala in un periodo estremamente negativo per il poeta: l'ultima parte della sua vita è rattristata e angustiata dalla malattia e da una serie di timori che serpeggiavano in europa e in Italia e che sfocieranno nella Prima Guerra Mondiale.

La sua approvazione nei confronti della guerra di Libia aveva all'origine una sorta di sentimento di Pietà, di empatia, di preoccupazione per tutti gli emigranti italiani come ad esempio Italy. Questa odissea dei migranti era vissuta con preoccupazione e tristezza che lo porta a vedere nella guerra Per le coloni una sorta di ricompensa è una sorta di soluzione. E come se questi migranti che sono stati sradicati dal nido avrebbero potuto ritrovare sostanzialmente una patria, un compenso, anche se a spese di altre popolazioni.

\section{T: \textit{Digitale purpurea}}

\elenco{\item \emph{p. 579}}

Come filo portante di tutti i puoi metti ce la vita di campagna, i lavori dei campi. La vicenda non è ricostruita, si può immaginare che ci sia qualcosa dietro questo scambio di battute tra due donne. 
\textbf{Digitale purpurea} era il nome di un fiore: ricordiamo che i fiori avevano un grande fascino nella letteratura decadente, ed in particolare alcuni fiori simbolici; la digitale purpurea ha dei petali che possono sembrare delle dita, di un rosso molto intenso, abbinato alla passione amorosa. Pascoli quando parla dell'amore ne parla sempre attraverso immagini simboliche, quasi sempre tratte dal mondo della natura.

Sono rappresentate due donne che si incontrano e parlano di un avvenimento che risale ai tempi del collegio. La sorella di Pascoli, Maria, si è pronunciata a distanza di anni su questo poemetto: una delle due donne del poemetto si chiama proprio Maria, e rappresenta lei stessa, mentre non sa a chi alludesse il fratello con la figura delle altra donna, rachele.
Molti pensano che rachele rappresenti simbolicamente Ida, l'altra sorella di Pascoli, che si è sposata.
Rachele rappresenta sostanzialmente l'attrazione per l'eros, e quindi potrebbe fare riferimento a Ida, che a differenza di Pascoli e della sorella Maria avrebbe affrontato il grande passo di cercare di ricostruire un nuovo nido.

Il poemetto si apre con una sorta di flashback, un ricordo della loro fanciullezza.

\subsection*{I}

\elenco{\item \textbf{verso 1-5}: ecco i due tipi di donna, la donna angelica e la \textit{femmé fatale}; per farci capire quanto queste due donne siano diverse, "\textit{l'altra...}" è spostata nel verso successivo; gli occhi ardenti della femme fatale alla fine di questa poesia vengono ripresi
\item \textbf{verso 10}: \textit{piccoli anni}: sono piccole loro, non gli anni
\item \textbf{versi 12-14}: la precisione con cui vengono nominate le piante è tipica di Pascoli; \textit{zirlano}: è il verso del tordo;
\item \textbf{verso 14}: \textit{quel segreto canto}: sta per cantuccio
\item \textbf{verso 15}: \textit{fior di...?}: enjambement, sostenuto dai punti di sospensione e dal punto interrogativo
\item \textbf{verso 15-16}: \textit{fior di morte}: fa riferimento proprio alla digitale purpurea: questo fiore rappresenta un divieto (le bambine del collegio avevano il divieto di avvicinarsi a questo fiore)
}
La poesia è piena di gesti, di immagini, sguardi, piuttosto inquietanti

\elenco{\item \textbf{versi 16-18}: Maria ha veramente paura di quello che si dice del fiore
}

\subsection*{II}

Inizia l'immagine del passato.

Abbiamo una serie di espressioni, ricordi, immagini che si affollano nella mente delle due donne: sono ricordi estremamente lontani, e quindi riappaiono nella mente delle donne sottoforma di flash.

\elenco{\item \textbf{verso 4}: \textit{si profuma il lor pensiero}: la suggestione olfattiva va a sovrapporsi a quella visiva
\item \textbf{versi 10-11}: \textit{Oh! quale vi sorrise oggi, alle grate, ospite caro?}: questo verso ha dato origine a due interpretazioni (\emph{nota 15}):
  \elenco{\item alle grate le bambine magari possono ricevere degli ospiti
  \item potrebbe far riferimento a Cristo}
\item \textbf{verso 18}: il bianco è quello della divisa, ma è simbolo di purezza
\item \textbf{verso 22}: \textit{In disparte}: nel cantuccio di cui si faceva riferimento all'inizio.}

L'immagine simbolica dell'amore, il fiore, è accomunato dall'immagine della morte.

\subsection*{III}

Si ritorna al presente: ci sono gesti un po' inquietanti: lo stringersi le mani le fa capire vicendevolmente, segno di un linguaggio loro, misterioso.

\elenco{\item \textbf{verso 5}: \textit{triste e pio}: citazione Dantesca, canto di Paolo e Francesca}

Ciò che Rachele ha comunicato con quella stretta di mano è che lei è venuta meno al divieto. Rachele non guarda negli occhi Maria.
Non spiega cosa è avvenuto, ma spiega lo stato d'animo in cui si trovava: la fanciulla sentiva ancora dentro di sé l'agitazione di un sogno fatto la notte; è tutto molto vago e indefinito, l'amore è vissuto da Pascoli come una sorta di tabù, come qualcosa di morboso e proibito.

\elenco{\item \textbf{versi 23-25}: vengono ripresi gli occhi ardenti descritti all'inizio; sono indizio della malattia che consuma Rachele: è allusivamente indicata com econseguenza della trasgressione}

\chapter{T: \textit{Il gelsomino notturno}}
\elenco{\item \emph{p. 605}}

Fa parte della raccolta dei canti di Castelvecchio.
È un epitalamio, tipo di poesia scritta per le nozze.

Il tema dell'amore è molto raro in Pascoli, e così anche nelle uniche poesie d'amore riesce ad essere presente anche la tematica della morte.
Soprattutto in ambito decadente, eros e Thanatos erano molto spesso accomunati. Sebbene Pascoli possa essere inserito in questo contesto culturale, la sua soluzione è assolutamente personale.

La critica psicanalitica si è scatenata, in primis per la presenza del tema della morte, e poi perché notiamo una certa reticenza di Pascoli nell'affrontare il tema della passione amorosa o il tema del concepimento; vediamo una certa difficoltà ad affrontare in altro modo il tema, se non attraverso un atteggiamento per cui l'atto amoroso è concepito come qualcosa da cui lui è escluso.
Il fatto che l'atto del concepimento è sempre sostituito nella poesia da un elemento della natura: negli ultimi versi, in cui si preannuncia la nascita di questo bambino, ecco che Pascoli sostituisce l'immagine del fiore.

Ci sono immagine allusive di un sentimento torbido, perché questo aspetto della vita viene vissuto in modo tormentato da Pascoli. Molti critici motivano questo atteggiamento con l'esperienza biografica: è come se il bambino Pascoli sia rimasto bloccato all'anno della morte del Padre, e che per tutta la vita, anziché cercare di costruire un proprio nido, lui abbia bloccato la sua crescita e cercato di ricostruire quel nido originario, impedendosi di creare un novo nido e una nuova vita.

È come se P. avesse sviluppato un senso di colpa, dettato da un giuramento di assoluta fedeltà nei confronti della famiglia originaria: per ciò il pensiero di una donna con cui creare una nuova famiglia lo fa sentire in colpa.

\emph{Non fare paragrafo 11. Fino a pagina 506}

\part{Svevo}

\chapter{Introduzione e caratteri generali}

\elenco{\item \emph{p. 760}}

Nasce nel 1861, e la sua opera più importante (\textit{La coscienza di Zeno}) è del 1923.
Ha vissuto diverse esperienze significativo.

È un romanziere decisamente moderno. Insieme a Pirandello è il primo scrittore ad affrontare in modo sistematico il romanzo psicologico.

Ciò non significa che prima ci siano romanzi con uno scavo psicologico significativo, ma quando si parla di romanzo psicologico parliamo di un romanzo che si concentra sull'interiorità, in modo sistematico, e alla luce degli studi legati all'invenzione della psicoanalisi (sono anni di Freud). Anche dal punto di vista degli strumenti narrativi, si serve degli strumenti dati dalla psicoanalisi.

L'\textbf{inetto} è prima di tutto il disgregarsi dell'immagine maschile che corrisponde all'uomo forte, con certezze, all'eroe di tanti romanzi precedenti.
È un anti eroe, ovvero un personaggio che non ha alcuna certezze: è una forte allusione alla perdita di certezze dello scrittore e dell'intellettuale.
Non a caso i personaggi di Svevo sono spesso artisti mancati, o sono vicini al mondo artistico letterario.

L'inettitudine è un tipo di atteggiamento generale, che significa \textit{In aptus}, inadatto, incapace.
Questo concetto è molto ambiguo nella \textit{Coscienza di Zeno}, perché Zeno paradossalmente si rivelerà essere un uomo di grande successo.

\textit{La Coscienza di Zeno} è, nella finzione, la pubblicazione dello scritto-diario del protagonista, Zeno stesso.

Anche in passato, nella letteratura romantica, i personaggi sono finiti miseramente, con un suicidio. Tutti questi personaggi però avevano aspirazioni titaniche, e poi hanno fallito; l'inetto non prova nulla di tutto ciò

Per la prima volta ci troviamo di fronte ad un autore che non poggia la sua opera su basi riferite alla tradizione classica. La formazione di Zeno è \textit{sui generis}, in quanto fu avviato a degli studi tecnici dal padre.
A Trieste vi era una ricca e forte borghesia, e questa era la strada immaginata dal padre per Svevo.

Nel corso della sua vita, Svevo leggerà i classici italiani, però non c'è traccia di ciò nella sua opera. Alcuni critici ritengono che la lingua di Svevo sia imperfetta, in quanto Trieste era una città in cui si parlavano due lingue, il tedesco e l'italiano: Svevo molto probabilmente si forma maggiormente su scrittori Tedeschi, e studia l'italiano quasi come una lingua straniera; questo aspetto si ripercuote sulla lingua di Svevo, che molti ritengono imperfetta.
In realtà questa imperfezione è una scelta ben precisa

Il narratore qui è significativo: sostanzialmente il passaggio dal narratore che parla in terza persona ad un narratore che parla in prima persona. Essendo il romanzo psicologico lo studio dell'interiorità del personaggio, solo il personaggio stesso può descrivere la propria coscienza.

Il narratore però è inattendibile, perché soprattutto in Svevo egli cerca di ricomporre quegli aspetti legati alla sua coscienza che lo portano a contraddizioni, inganni. Nella \textit{Coscienza di Zeno} il dottor S ci dice proprio che Zeno è un bugiardo. Non abbiamo però punti di riferimento per capire quando il protagonista mente, etc etc etc. L'immagine che ne nasce è quella di un uomo che non ha più riferimenti nella realtà.

\section{Biografia}

La biografia dell'autore è molto importante: non succedono cose straordinarie, ma sono importanti perché ci permettono di capire il suo rapporto con la letteratura.

Abbiamo visto molti intellettuali che vivono un disagio perché non si riconoscono nei valori della società borghese. Qui l'autore, ad un certo punto della sua vita, incarna proprio la figura sociale del borghese arrivato economicamente; è come guardarsi dal di fuori, in quanto anche Svevo stesso vive il disagio dell'intellettuale.
Ad un certo punto Svevo dirà addio alla letteratura, per "risolvere il conflitto".

Svevo nasce nel 1861, da genitori ebrei, e appartiene ad una famiglia borghese. Il padre lo avvia agli studi tecnici, che gli permettano di entrare subito nel mondo del lavoro.
Nel giovane Svevo nasce subito una passione per la letteratura: per quanto non abbia fondato la sua formazione culturale sui classici, e attraverso studi canonici, egli si avvicinò autonomamente, soprattutto a scrittori tedeschi, iniziando a scrivere qualcosa.

Negli anni '80, quando Svevo aveva 18 anni, c'è il \textbf{fallimento del padre}: implica una serie di scelte, come quella di andare a lavorare in una banca, e soprattutto il \textit{declassamento sociale}, e di conseguenza economico; questo è un elemento fondamentale, che avrà delle conseguenze estremamente gravi.

Dopodiché, nel '92 Svevo scrive il suo primo romanzo: \textit{Una vita}.
Ci sono delle figure significative, in quanto il protagonista è un inetto.
Il romanzo non ha successo, e a posteriori possiamo dire che l'idea di Svevo di abbandonare la letteratura, nel 1902, sia data dal fallimento di tutte le sue opere.

Inizialmente l'attività in banca lo distoglie dagli interessi letterari: ci lavora circa 20 anni, di un lavoro arido, che egli vive con una certa indifferenza.
Nel '96 avviene il matrimonio con una lontana parente (si conoscono ad un funerale), che appartiene ad una famiglia estremamente facoltosa.
Il matrimonio è il momento della pienezza nella cultura borghese, oltrepiù significò anche poter riacquistare nella società un peso economico importante.

Grazie alla ditta dei suoceri, egli dovette viaggiare molto, e imparare l'inglese. Il suo maestro sarà proprio \textbf{James Joyce}, che diventerà suo amico e che farà conoscere il suo romanzo \textit{La coscienza di Zeno}.

Egli abbandonerà la scrittura nel 1902:
\citazione{Io, a quest'ora e definitivamente, ho eliminato dalla mia vita quella ridicola e dannosa cosa che si chiama letteratura}

Sicuramente l'insuccesso delle sue opere lo porterà lontano dalla letteratura per parecchio tempo.

Insieme alla conoscenza di Joyce, Svevo fece anche un'altra conoscenza, ovvero quella delle idee di Freud e delle sue opere.
In molti incontri in cui Svevo prese la parola, egli si espresse proprio nei confronti della psicoanalisi, che probabilmente conosceva bene.

In termini medici Svevo non era convinto dalla psicoanalisi, ma quello che Freud aveva offerto agli scrittori era ineguagliabile: la psicoanalisi per lui fu utilissima agli scrittori, perché forniva una serie di strumenti di indagine preziosissimi per scrivere romanzi psicologici.

Lo stesso Pirandello dice di non conoscere Freud, ma chiaramente le sue idee ebbero influenza su di lui.

Svevo sarà molto influenzato da alcuni pensatori, come Darwin, Nieztche, Schopenauer, Marx, spesso in antitesi tra di loro, ma prendendo da ciascuno di essi dei determinati aspetti utili alla sua letteratura.

Nel 1923 egli pubblicherà \textit{La Coscienza di Zeno}. Soprattutto in Italia ci fu una certa diffidenza nei confronti di Svevo. Fu poi Montale che scrisse una presentazione gratificante, nella rivista che dirigeva, che permise la conoscenza e l'apprezzamento di Svevo in Italia.

All'estero ebbe un grandissimo successo, soprattutto in Francia, dove l'opera fu presentata da Joyce.

Egli muore nel 1928: egli stava ritornando da una vacanza, ed ebbero un incidente di scarsa identità. Lui si ruppe in femore, e il giorno dopo morì per un enfisema polmonare, dovuto soprattutto alle sue 60 sigarette al giorno.

Egli in una nota definisce la sua vita come "non bella". Questo significa che Svevo biograficamente prende le distanze da quelli che sono stati gli interessi principali di D'Annunzio.
Egli non avrà alcun affanno nel costruire una immagine di sé, sarà distantissimo dal superuomo e dall'esteta, distantissimo da qualsiasi altro tipo di intellettuale precedenti, come la \textit{Bohemiene}. Questo si riflette sui personaggi, che sono individui assolutamente normali, alle prese con vicende normali.

Svevo era nato a \textbf{Trieste}: le sue posizioni in relazione alla posizione di Trieste furono quelle di irredentismo: all'epoca in cui Svevo visse a Trieste, questa era una città molto particolare: era multietnica, sotto il dominio austriaco, e vi era la presenza slava e italiana. Lo stesso "Italo Svevo" era uno pseudonimo per \textbf{Aron Hector Schmitz}, e ciò ci mostra la sua volontà di giungere ad una unione tra le sue due appartenenze italiana e tedesca

Si parla di cultura mittleeuropea, in quanto Trieste era di cultura molto aperta. Era stato inaugurato un porto che rappresentava per gli austriaci l'unico sbocco sul mare.

\section{Psicoanalisi}

\elenco{\item approfondimento \emph{p. 844}: Svevo e la psicoanalisi; leggere e confrontare con gli appunti}

È la scoperta dell’inconscio, qualcosa di ancora più nascosto dell’interiorità.
Quando si parla di letteratura fantastica, si dice che essa nasca quando l’uomo si rende conto che non tutto si può spiegare con la scienza. Si mettono in luce tutte quelle zone d’ombra della psiche.
Freud va a studiare queste cose con approccio scientifico, egli infatti era laureato in medicina.
L’inconscio è la parte interna dell’individuo che ha vita propria, che non può essere controllata da nessuno. Questo è destabilizzante, la crisi della personalità nasce da qui.
Il momento in cui la nostra ragione non riesce a controllare le nostre pulsioni è nel sonno.
I suoi studi parte dalla ricerca per la cura alla nevrosi. Si stabilisce un rapporto tra la malattia e la possibilità dell’individuo di descriverla e capirne le origini, la malattia diventa uno strumento conoscitivo:

\elenco{\item \textbf{Lapsus freudiano}
\item \textbf{Es}: impulsi, nel bambino e negli animali troviamo la completa espressione di questa parte di inconscio. Il bambino non se ne vergogna.
\item \textbf{Superio}: serie di norme comunemente accettate e non, come decenza, etica, morale, leggi, che vanno a controllare l’Es
\item \textbf{Io}: sta al centro tra i due, serve a bilanciarli. l’individuo è in salute ed equilibrato se l’io “fa bene il suo lavoro”, differentemente si va in nevrosi, ed è qui che entra in scena lo psicanalista.
}

La scoperta della psicoanalisi è un approccio di tipo scientifico per spiegare qualcosa che sembra sfuggire alla ragione. La psicoanalisi ha portato alla scoperta dell'inconscio, che va a creare una crisi di personalità dell'individuo, in quanto quando egli scopre che c'è una parte di sé su cui non ha alcun controllo emerge un aspetto inquietante.

L'intervento della psicoanalisi serve a riportare alla luce alcuni elementi che possono essere definiti parte dell'inconscio, come desideri e traumi, che sono stati messi a tacere.

Insieme a questo tentativo di portare in equilibrio l'individuo, definito nevrotico se l'\textit{es} e il \textit{superio} non è equilibrato correttamente dall'\textit{io}: a questo punto ha bisogno della psicoanalisi

Svevo non aveva fiducia nella psicoanalisi come terapia medica, ma proprio dalla psicoanalisi egli riesce ad ottenere gli strumenti di indagine che gli permettono di scavare nei suoi personaggi.

Nella terapia psicoanalitica un elemento fondamentale è il sogno è quel momento in cui non c'è nessun controllo della ragione, per cui una sua eventuale analisi ci permette di scoprire qualcosa dell'inconscio.

Ci sono molti altri elementi, come i lapsus, gli atti mancati, la gestualità del corpo, in cui l'inconscio prende il sopravvento.

Un esempio di atto mancato ne \textit{La Coscienza di Zeno} è quando lui si dimentica di andare al funerale del cognato.
L'atto mancato è un qualcosa che viene meno rispetto la volontà dell'individuo. Nel caso di questo episodio Zeno si dimentica di andare al funerale del cognato. Per Freud questo è un chiaro atto mancato, che va spiegato in questi termini: l'inizio del rapporto tra Zeno e Guido è conflittuale, in quanto quest'ultimo è un protagonista forte e vitale, che tende a contrapporsi all'inetto; poi diventano soci in affari e grandi amici, però probabilmente nell'inconscio di Zeno questa antipatia e questa avversione per Guido resta, tanto che quando quest'ultimo muore Zeno si dimentica di andare al funerale.

Svevo dice
\citazione{Grande uomo quel nostro Freud, ma più per i romanzieri che per gli ammalati. \textit{Letterariamente} certo Freud è più interessante. Magari avessi fatto io una cura con lui. Il mio romanzo sarebbe risultato più intero. E perché voler curare la nostra malattia? Davvero dobbiamo togliere all'umanità quello ch'essa ha di meglio? Io credo sinceramente che il vero successo che mi ha dato la pace è consistito in questa convinzione.}

Svevo inizia ad avvicinarsi alla psicoanalisi nel 1908, che è abbastanza presto per l'epoca. Insieme ad un suo nipote egli cerca di tradurre i primi scritti di Freud sul sogno.
Egli aveva avuto rapporti con un collaboratore di Freud, nel 1910-11.

Il cognato di Svevo, molto più giovane di lui, era un giovane omosessuale tossicodipendente, che i genitori avevano affidato alle cure di Freud. Ci sono delle lettere in cui Svevo racconta questa esperienza, con l'assoluta mancanza di riscontro della terapia, in quanto il cognato venne distrutto alla fine della terapia.

Nel linguaggio di Freud, l'esperienza di Svevo è una sorta di resistenza: la resistenza del paziente a volte avviene: il paziente a volte è intenzionato a difendere la propria malattia.
Questo problema non è neanche così distante da Zeno, nei confronti del fumo, che fino alla fine continuerà a fumare.
Questo vizio attanglierà anche l'autore stesso

Il fumo, all'interno del romanzo, viene talvolta interpretato come una sorta di autoinganno: lui è un inetto, e considerando che inizia la sua terapia con il tentativo di smettere di fumare, è probabile che egli si senta un inetto a causa della malattia del fumo: se poi, guarito da questa malattia, egli avesse scoperto di essere ancora un inetto, non avrebbe più potuto giustificare la sua inettitudine; di conseguenza, sembra quasi che Zeno \textit{voglia} continuare a fumare.

Svevo dice:
\citazione{E perché voler curare la nostra malattia? Davvero dobbiamo togliere all'umanità quello che essa ha di meglio?}

La presa di coscienza rispetto alla malattia è uno strumento conoscitivo.

\subsection{Malattia}

\elenco{\item approfondimento \emph{\href{https://www.corriere.it/cultura/eventi/2013/scala/notizie/malattia-destino-ineluttabile-cosi-tisi-consumo-l-ottocento-42b66182-5e85-11e3-aee7-1683485977a2.shtml}{malattia}\footnote{https://www.corriere.it/cultura/eventi/2013/scala/notizie/malattia-destino-ineluttabile-cosi-tisi-consumo-l-ottocento-42b66182-5e85-11e3-aee7-1683485977a2.shtml}}
}
Nell’800 fa la sua comparsa in letteratura, non che prima fosse bandita, ne è un’esempio la peste che si sentì la prima volta nell’Illiade, quando la peste va a colpire l’esercito dei Greci, poi la peste di Atene, e quella del Boccaccio.

Dall’800 però arriva una malattia che fece tantissimi morti, la tubercolosi, come nella signora delle camelie, ma anche tanti autori ne furono colpiti. 
Divenrta così importante perché poco  apoco divenne metafora del malessere iteriore. La malattia fisica porta con sé molte immagini evocative, come il pallore della donna ammalata che ci riporta al pallore delle donne cantate partendo dalla lirica provenzale, la tisi porta anche altri colori come quello del sangue, il rosso ha grande valore figurativo in letteratura.

Lo troviamo anche nella scapigliatura, come la Fosca, ma anche nel naturalismo e nei personaggi di Zola. Altra malattia è l’alcolismo che miete parecchie vittime, compisce le eroine come anche i poeti poveri, vedi i bohemian. 
Col decadentismo la malattia diventa l’espressione del “cupio (desidero) dissolvio (distruzione)”, è quindi il desiderio dell’autodistruzione. 
La malattia di Zeno è il disadattamento, ci sono anche riscontri fisici, la malattia dell’anima diventa malattia fisica. Questo è il filo portante della coscienza di Zeno


\chapter{\textit{Una vita}}
\elenco{\item \emph{p. 770-773}}

È la storia di un giovane, Alfonso Nitti, che abbandona il paese e la madre per venire a lavorare a Trieste, dopo che la morte del padre, medico condotto, ha lasciato la famiglia in ristrettezze. Si impiega presso la banca Maller, ma il lavoro gli appare arido e mortificante. Il giovane, imbevuto di letteratura, orgoglioso della sua cultura umanistica, evade costruendosi «sogni da megalomane» e vagheggiando la gloria letteraria.
L'occasione per un riscatto dalla sua vita vuota e solitaria, riempita solo dalle avide letture presso la biblioteca comunale, gli è offerta da un invito a casa del padrone della banca, Maller. Alfonso conosce così Macario, un giovane brillante e sicuro di sé, e stringe con lui una forma di amicizia. In Macario l'eroe, nella sua provinciale goffaggine e timidezza, trova una sorta di appoggio e di modello. La figlia di Maller, Annetta, ha anch'essa ambizioni letterarie e sceglie Alfonso come collaboratore nella stesura di un romanzo. Alfonso, pur senza amare Annetta, la seduce e la possiede. A questo punto l'eroe avrebbe la possibilità di trasformare radicalmente la propria vita, sposando la ricca ereditiera. A tale soluzione è spinto insistentemente dalla signorina Francesca, istitutrice in casa di Maller e sua amante, che aspira anch'essa al salto di classe attraverso il matrimonio con il padrone. Alfonso invece, preso da un'inspiegabile paura, fugge da Annetta e da Trieste, adducendo come pretesto una malattia della madre.

Tornato al paese, trova effettivamente la madre gravemente ammalata. Dopo la sua morte ritorna di nuovo a Trieste, deciso a rinunciare alla crudele «lotta per la vita che domina nell'ambiente in cui vive, credendo di aver scoperto nella rinuncia e nella contemplazione la sua vera natura (si rivela in questo l'influenza di Schopenhauer, il filosofo amato da Svevo, \textit{I maestri di pensiero: Schopenhauer, Nietzsche, Darwin}, \emph{p. 766}). Ma la realtà smentisce le belle teorie e i nobili programmi. Alfonso credeva di aver interamente superato le passioni, invece, all'apprendere che Annetta, sdegnata con lui, si è fidanzata con Macario, è invaso da una dolorosa gelosia; riteneva di non curarsi più del giudizio degli altri, ed invece si sente ferito dal disprezzo e dall'odio che lo circonda nella banca.

Trasferito ad un compito di minore importanza, affronta indignato il signor Maller, ma nell'emozione si lascia sfuggire frasi che vengono interpretate come ricatti. Da questo momento commette errori irreparabili: scrive ad Annetta per chiederle che cessing le persecuzioni nei suoi confronti, ma di nuovo il suo gesto è avvertito dai Maller come ricattatorio. All'appuntamento che egli ha chiesto alla ragazza, per una definitiva spie gazione, si presenta il fratello, che lo sfida a duello, Alfonso, sentendosi «incapace alla vita», decide di cercare nella morte una via di scampo, il mezzo per divenire «superiore ai sospetti e agli odi», distruggendo la fonte della sua infelicità, il suo organismo «che non conosceva la pace».

\section{T: \textit{Le ali del gabbiano}}
\elenco{\item \emph{p. 773}}

Da questo brano evinciamo con estrema chiarezza il carattere dell'inetto, che poi vedremo con delle varianti nei romanzi successivi.
\elenco{\item \textbf{riga 29}: paura per sé e per gli altri
\item \textbf{riga 39}: l'inetto non è capace, vorrebbe mostrare sangue freddo ma non riesce, tanto che suscita una reazione ironica da parte dell'antagonista 
\item \textbf{riga 45}: fa dei tentativi, non riusciti
\item gli antagonisti lo prendono in giro
\item \textbf{riga 63}: malattia}

Questo aspetto, cioè da una parte il suo volere e dall'altra non riuscire ad attendere questa volontà, è significativo del fatto che alla base ci sia la teoria Darwiniana e di Schopenauer

Macario, dinnanzi all'atteggiamento di Alfonso, gli spiega teoricamente qual è il problema. Svevo prende spunto da molti filosofi, tra cui Darwin e Schopenauer.

Svevo ama molto la filosofia, studia i filosofi tedeschi e non solo: ne fa un uso particolare, in quanto i filosofi sono spesso in contrasto tra di loro. Talvolta sembra difficile trovare un contatto tra alcuni di essi, perché l'atteggiamento di Svevo è "strumentale": egli trae molto spesso dai filosofi degli strumenti di analisi che gli servono per costruire i suoi personaggi.
Per esempio, per quel che riguarda Marx, a Svevo interessa non il pensiero, quanto la scoperta della lotta tra le classi, ripresa in Svevo, e che gli serve per dare una connotazione ai suoi personaggi.
Svevo stesso è andato "avanti e indietro" tra le classi sociali.

Da Schopenauer, che spesso è visto in parallelo a Leopardi, Svevo prende la teoria sugli istinti inconsci dell'essere umano, che secondo Schopenauer devono essere soppressi; gli suggeriscono il modo di tratteggiare i suoi personaggi, che sono spesso dei bugiardi: nella \textit{Coscienza di Zeno} è il dottor S che dice proprio che Zeno è un bugiardo. Il protagonista quindi è \textbf{inattendibile}. Solo il narratore, in un romanzo, può smascherare gli inganni del protagonista, sempre che questi non coincidano. Ne \textit{La Coscienza di Zeno}, coincidendo queste due figure, non è possibile smascherare le bugie. Il lettore è senza parametri per poter capire le finzioni di Zeno, per cui non si hanno più certezze. Sia in \textit{Una Vita} che in \textit{Senilità} il narratore è esterno.

In questo brano, Macario spiega la sua filosofia ad Alfonso: gli dice che uno è vincente o perdente per natura, e la lotta per la vita fa sì che i più forti vincano e i più deboli perdano. Si usa la metafora con il gabbiano, vincente per natura. Ci sono continui riferimenti al cervello, chiamato \textit{essere inutile}: dietro agli inetti di Svevo ci sono sempre intellettuali, e questo è significativo del modo in cui Svevo denuncia il malessere dell'intellettuale (situazione che vive anche sulla sua persona, in quanto egli abbandona la letteratura). Viene irriso il protagonista, in quanto pur studiando è incapace.
Si tratta di capire se c'è possibilità di attuare la volontà: no, in quanto è una \textbf{legge naturale}.
\citazione{Chi non sa per natura piombare a tempo debito sulla preda non lo imparerà giammai e inutilmente starà a guardare come fanno gli altri, non li saprà imitare. Si muore precisamente nello stato in cui si nasce, le mani organi per afferrare o anche inabili a tenere.}

L'ultima battuta è un attacco alla passione di Alfonso per la poesia.

Ancora anni più tardi in una lettera del 1927, Pirandello ribadirà, citando Schopenauer, che il contemplatore è un prodotto della natura, finito quanto il lottatore (cioè il contemplatore non è a uno stadio imperfetto della formazione umana, ma un prodotto di natura già perfettamente compiuto come lo è il lottatore). Quindi è probabile che sia d'accordo sul fatto che l'inettitudine di Alfonso sia un dato di natura

Le teorie di Macario sono giuste, in quanto Alfonso alla fine si suicida: è probabile che l'idea di Svevo sia uguale a quella di Macario.
Non si capisce perché egli affidi a Macario, personaggio negativo, la sua stessa idea.


\chapter{\textit{Senilità}}
\elenco{\item \emph{p. 777}: trama \textit{Senilità}}

Siamo alle prese con un inetto, pauroso, che si è costruito un nucleo famigliare piuttosto rassicurante, non ha rapporti con il mondo esterno, ha rapporti di amicizia con un pittore, che è l'esatto contrario di lui. Secondo alcuni, l'artista presente nel romanzo sarebbe Umberto Veruda (raffigurato a \emph{p. 778}). Questo personaggio è un po' ambiguo.

Ad un certo punto il protagonista sente la necessità di vivere la vita, perché quella che lui vive è una sorta di malattia.
Entra in gioco la seconda protagonista femminile, Angiolina, che rappresenta la vita e la salute.
Il protagonista è attratto da questa vita, ma succede che anche il suo rapporto con la donna è fallimentare, perché Angelina non è un personaggio positivo in toto: è giovane, opportunista, ha altri amanti, e il protagonista si innamora di lei, e siccome la donna non corrisponde nella realtà alla fanciulla che ama, ne costruisce un'altra ideale, completamente differente da Angiolina: si costruisce una sorta di donna angelica strutturata secondo il motivo della lirica antica.

Emilio è un inetto che cerca tutta una serie di scusanti, autoinganni, rispetto al suo modo di agire, e anche qui abbiamo un narratore esterno che svela al lettore tutti questi autoinganni del protagonista. Il protagonista mente, e non si capisce se lo fa apposta o se è un modo di autodifendersi e di trovare delle scusanti.

Il narratore è esterno, e la focalizzazione per la maggior parte è quella del protagonista, però a volte il narratore fa come il narratore di \textit{Una Vita}, mettendo a nudo gli autoinganni del protagonista. Non sempre lo fa, e a volte non succede che il narratore intervenga, però il lettore riesce lo stesso a capire che la posizione dell'autore è di distanza rispetto al protagonista: abbiamo l'\textbf{ironia oggettiva} o \textbf{implicita}: il narratore non dice niente, ma il protagonista è smentito dai fatti.
La verità viene svelata dal linguaggio utilizzato dal protagonista, stereotipato, che gli fa perdere credibilità.
\newpage

\section{T: \textit{Il ritratto dell'inetto}}
\elenco{\item \emph{p. 782}}

In questo brano si evincono quali sono le funzioni e le intenzioni del narratore: è esterno, parla in terza persona. La focalizzazione è quasi sempre quella del protagonista, ma il narratore interviene a chiarire e a smascherare gli autoinganni del protagonista. Ora il narratore interviene con termini che vanno in qualche modo a limitare l'idea di carriera.

Il narratore è onnisciente, cin quanto interviene senza correggere quello che ha detto il protagonista. Il narratore interviene sul non detto, ironizzando sul fatto che la famiglia sia una sorella \textit{non ingombrante}.

Narratore e protagonista fino ad ora hanno parlato di carriera.
Emerge dalla pagina iniziale in senso della “senilità”: Emilio ha paura di affrontare la vita che gli appare piena di pericoli e perciò rinuncia a vivere. Emilio si crea quindi una maschera superomistica, non ha la lucidità di vedersi nella sua effettiva mediocrità di romanziere fallito e sterile.

L’idea di Emilio (inetto) è che questa fase piuttosto grigia sia solamente un periodo di preparazione quando in realtà non è così.
Della psicoanalisi Svevo rifiuta l’aspetto medico e accetta tutti quegli strumenti di analisi che gli sono utili per descrivere i suoi personaggi. Angiolina non è un inetto, è una ragazza povera ma non così tanto da dover soffrire la fame. La salute caratterizza il personaggio di Angelina che non è inetto. A fare contrappeso alla figura dell’inetto abbiamo sempre una figura “illuminata”.

La narrazione è \textbf{eterodiegetica} ma il narratore non si eclissa. Il narratore rappresenta l’alternativa di una prospettiva superiore, più lucida e consapevole e si pone come proposito quello di smascherare impietosamente i suoi autoinganni.
Angelina è la possibilità intravista dal protagonista di creare un diversivo nella sua vita. Il protagonista non parla di carriera, ma di un impieguccio in banca, o di una carriera letteraria.
Questo ci riporta alla distinzione che abbiamo fatto tra Darwin e Shopenhauer, quando nella gita in barca di Alfonso si è parlato della differenza tra colui che vive la vita da spettatore e al lottatore, che invece si fa largo nella vita, questo \textbf{per natura}.
L’idea invece di emilio, che è inetto, è che questa fase piuttosto grigia sia solo un periodo di preparazione, ma non è così, infatti anche questo personaggio finirà male.
Qui ci sta un  confronto con l’inetto della coscienza di Zeno, è un uomo che ha successo nella vita, è possibile che svevo abbia rivalutato la figura dell’inetto?  Oppure che ne abbia colto qualcosa di positivo?

\elenco{\item \textbf{riga 38} Descrive il personaggio femminile, ragazza povera ma non troppo, si chiama Angiolina
La salute caratterizza il personaggio che non è inetto, a fare da contrappeso alla figura dell’inetto in ogni romanzo c’è sempre una figura “illuminata”, nella coscienza di Zeno è la moglie.
}

Quando finalmente Emilio riesce a conquistarla e possederla, non gli piace cosa vede, ha anti amanti, un’arrivista, da lui scaturisce una gelosia che avrà conseguenze nel suo comportamento. Lui quindi ne crea un’altra, più confacente alle sue aspettative, fedele, rappresentazione della donna angelo.


\chapter{\textit{Coscienza di Zeno}}

\elenco{\item leggere tutte le analisi del testo sul libro}

Qui a differenza del testo precedente non abbiamo più un narratore interno, è onnisciente perché è il protagonista che parla di se stesso, ma non sa tutto degli altri protagonisti, oltre che essere inattendibile, bugiardo
La malattia di Zeno è il disadattamento, ci sono anche riscontri fisici, la malattia dell’anima diventa malattia fisica. Questo è il filo portante della coscienza di Zeno

È pubblicata nel 1923, e moltissimi anni lo separano dall'ultimo libro pubblicato.
La prospettiva dell'autore cambia, in quanto è passata una guerra mondiale dal precedente...

Questo romanzo probabilmente non avrebbe avuto successo se non fosse che Joyce e Montale ci credettero molto, e gli garantirono un certo successo, nonostante qualche resistenza in Italia: infatti, in Italia alcuni considerano strana e brutta la lingua di Svevo. Alcuni esempi sono
\citazione{"Ero riuscito di fare accettare"\\
"Cosa ha da fare"}

\section{Giacomo Debenedetti: \textit{Una scrittura che morde le cose}}
Si è molto parlato, a proposito di Svevo, del suo presunto "scrivere male". Svevo scrive male se si assume come modello la scrittura di D'Annunzio o comunque la scrittura letteraria della nostra più alta tradizione classicista (da Alfieri a Foscolo a Leopardi). Ma non scrive male se si guarda alla funzionalità delle idee che intende comunicare. Il linguaggio sveviano è uno strumento che, per quanto inelegante, funziona in modo efficace. È questa la tesi del critio letterario e saggista \textbf{Giacomo Debenedetti} (1901-67).

La scrittura sveviana è bruttissima senza dubbio, quando venga messa a confronto con le inflessioni più naturali e mature, a cui i secoli hanno avvezzato la prosa italiana. Ma che, dopo esserne stati sconcertati, si comincia a sentire pronta, precisa a suo modo e ricca di risorse, per finire forse con l'amarla, come si ama l'accento sia pir vizioso di persona che ci tenga incatenati con la sola consistenza del discorso e il calore della comunicativa. Si prova anzi uno specifico e raro piacere a constatare come l'elocuzione e la sintassi di Svevo, malgrado tutti gli arbitrii e le cacofonie esterne e interne, arrivino ad attaccare e mordere le cose; e come da quelle disuguaglianze e incertezze formali balzi l'evidenza di un ritratto, la netta figura di una situazione o di un movimento. Ed è, direi, il piacere di assistere al funzionamento di un utensile efficace, per quanto inelegante

Probabilmente alcune scelte formali erano estremamente adatte a tratteggiare il personaggio dell'inetto.

\section{T: \textit{Prefazione}}

È il dottor S che parla
Abbiamo parlato di una certa resistenza di Svevo alla psicoanalisi, ma questo non significa che Svevo rifiutasse completamente l'analisi: in alcuni brani Svevo accenna ad un'auto analisi, a cui crede molto: è un sistema "casalingo" di Svevo, e possiamo immaginare che questo romanzo, con forti riferimenti autobiografici, sia una sorta di autoanalisi da parte dell'autore, casalinga, bonaria, e anche un po' ironica.

Quest'opera nasce dal rifiuto di Zeno di continuare la cura del dottor S, che per vendicarsi pubblica tutto ciò che il paziente ha scritto fino a quel momento.
Ci mette in guardia della sua inaffidabilità.

Il motivo del manoscritto ritrovato è un topos letterario utilizzato per introdurre il romanzo. Qui è qualcosa di molto simile.
Se negli altri casi il ritrovamento del manoscritto serviva a dare veridicità alla vicenda, qui succede che il dottor S ci mette in guardia rispetto alle tante bugie dette dall'autore.

Il narratore è interno, che parla in prima persona (è lo stesso Zeno), ed è inattendibile: entrano in gioco gli autoinganni e tutte quelle costruzioni che il protagonista fa per giustificarsi, e noi non abbiamo più nessun parametro per capire se si tratta di verità oppure no. È il relativismo più assoluto.

È come se Svevo ci dicesse che non ci sono più riferimenti oggettivi, ed è la maniera in cui Svevo ci mette di fronte alla fine di quell'atteggiamento di fiducia tipico del positivismo nella ragione e nei dati oggettivi.

Ci sono una serie di elementi estremamente significativi in questo senso: esempio è l'uso del tempo: non c'è un tempo oggettivo, e l'opera non si svolge in senso diacronico, ma abbiamo un tempo misto, ovvero una sovrapposizione di piani temporali che avviene nello stesso brano e nello stesso momento; è un tentativo di riprodurre i momenti veloci della nostra mente, così come lo è il flusso di coscienza: è una tecnica narrativa che serve a riprodurre il lavorio della nostra mente e del nostro inconscio.

Zeno giustifica la sua menzogna nello scrivere
\citazione{Con ogni nostra parola toscana noi mentiamo! Se egli sapesse come raccontiamo con predilezione tutte le cose per le quali abbiamo pronta la frase e come evitiamo quelle che ci obbligherebbero di ricorrere al vocabolario! È proprio così che scegliamo dalla nostra vita gli episodi da notarsi. Si capisce come la nostra vita avrebbe tutt'altro aspetto se fosse detta nel nostro dialetto.}

\section{T: \textit{Il fumo}}
\elenco{\item \emph{p. 806}}

Una novità rispetto al romanzo tradizionale è che non si proceda più attraverso una narrazione diacronica: ci sono delle sovrapposizione temporali, che vengono definite tempo misto;
i capitoli sono raggruppati in ordine tematico, non cronologico; alla fine del romanzo si può ricostruire una sorta di fabula, che non è la cosa più importante.

Forse Zeno non voleva effettivamente guarire, perché il fumo gli serviva a giustificare la sua inettitudine

\subsection*{Righe 1-6}

Qui siamo al presente, ovvero il momento in cui Zeno sta scrivendo. La narrazione inizia con un momento X, che è quello dello Zeno anziano che scrive

\subsection*{Righe 7-9}

Se identifichiamo Zeno con Svevo, negli anni '70 Zeno era molto giovane. In ogni caso anche Zeno, da quello che si dice dopo, era giovane a quell'epoca.

Senza nessuna analessi siamo passati al passato. Ciò che rende plausibile questi passaggi è il fatto che egli stia scrivendo in base ai suoi ricordi, in modo naturale, come se affiorassero i ricordi.

Qui abbiamo il passato, che si va a sovrapporre al tempo presente in cui scrive.

\subsection*{Righe 9-13}

Il ricordo nasce intorno alle sigarette, gli riaffiorano alla mente i ricordi dei volti delle persone che lo avevano accompagnato nelle sue prime scorribande.

\subsection*{Righe 14-18}

L'episodio raccontato dopo è quando ruba le sigarette al padre, che lo scopre: mentirà al padre. Da quel momento non ruberà mai più

\subsection*{Righe 23-25}

Ricordiamo che la psicoanalisi serve a recuperare il rimosso, e in qualche modo mette l'individuo in relazione con la sua malattia e con la capacità di riuscire a darne una spiegazione: nel tentativo di ricordare quando ha iniziato a fumare, Zeno si mette in relazione con la possibilità di spiegare l'origine di quella abitudine.

\subsection*{Righe 45-47}

È rievocato un episodio, sempre legato al fumo, di quando Zeno, un giorno, ancora bambino, tornato da una gita è sul divano, e fa finta di dormire per sentire cosa dicono i grandi.

Mentre è nel dormiveglia egli sente dal padre che si lamenta con la madre di non trovare più i suoi mozziconi (Zeno glieli rubava). La madre, quando sta per parlare, gli fa cenno di parlare sottovoce, perché Zeno stava dormendo.

Zeno è contento che il padre sia stato redarguito dalla madre in rispetto di lui che sta dormendo

\subsection*{Righe 70-77}

Non è una prolessi, ma abbiamo il futuro che si stende naturalmente così, sul passato e sul presente.

\section{T: \textit{La morte del padre}}
\elenco{\item \emph{p. 811}}

Siamo portati a pensare che Zeno menta, soprattutto riguardo all'affetto che cerca di dimostrare nei confronti del padre.

Zeno sta raccontando gli ultimi giorni di vita del padre, e poi l'ultimo gesto. Il padre tenta di alzarsi, Zeno lo spinge giù nel letto: il padre prima di morire ha le braccia sollevate, e mentre muore la mano gli ricade sul volto di Zeno, come uno schiaffo.

Quello potrebbe benissimo non essere uno schiaffo, ma Zeno non ne è sicuro: ha dei sensi di colpa nei confronti del padre, e ha paura che l'ultimo gesto del padre sia stato per punirlo.

\subsection*{Riga 19-21}

Frecciatina ai signori dell'analisi.

\subsection*{Righe 71-76}

Anche la sera in cui la prima volta il padre si ammala, i due avevano avuto una discussione a tavola, sulla forma della terra. Il padre rappresenta quell'ideale borghese, anche magari limitato (ha le sue certezze e non accetta altre verità). Il padre sostiene che la terra è immobile, mentre Zeno vuole fargli notare che non è così.

Vediamo un perfetto borghese, l'antagonista dell'inetto, che viene descritto in termini non completamente positivi: è come se Zeno, inetto, nel tentativo di spiegare il modo di comportarsi degli altri, nell'incapacità di spiegare le loro ragioni, ne mettesse in luce i limiti.

Succede anche per la descrizione di Augusta, la moglie. È brutta, ma molto sana e molto borghese. Mentre la descrive lui mette in luce alcuni aspetti, che a sua opinione sono molto positivi, ma che in realtà non lo sono tanto: "ignorava che quando a questo mondo ci si univa, l'unione sarebbe durata poco". Alla fine Zeno dice che, analizzando la sua salute, la trasforma in malattia, ma che egli crede nonostante questo che egli sia molto sana. (\emph{p. 822-823})

Possiamo quindi vedere che in quest'ultimo romanzo l'inetto è meno negativo come personaggio

Questo inetto cos'ha in più rispetto agli inetti precedenti e rispetto ai sani? Ha consapevolezza del suo essere incompleto: i sani e i forti hanno coscienza della loro forza e della loro salute, e hanno la coscienza dell'essere perfetti, e quindi non sono aperti al cambiamento.
L'inetto, invece, sa di essere imperfetto, e quindi è più aperto al miglioramento. Questo è uno spiraglio positivo nella valutazione dell'inetto.

\subsection*{Righe 218-242}

In una prima analisi quello è uno schiaffi, e lui non ha neanche capito che il padre è morto.

\section{T: \textit{La profezia di un'apocalisse cosmica}}
\elenco{\item \emph{p. 848}}

\subsection*{Righe 11-21}

Zeno esprime la teoria secondo cui l'uomo sfugge alla selezione naturale di Darwin.

L'uomo è il soggetto principale del progresso, ma non quello fisico: Zeno vuole dire che il progresso dell'uomo non è dovuto a qualche miglioramento del suo organismo, ma per qualche ordigno

\subsection*{Righe 29-31}

Una visione apocalittica, il cui senso preciso ci sfugge. Il brano è scritto dopo la prima guerra mondiale, che può aver avuto la sua influenza.

\part{Pirandello}

\chapter{Introduzione e caratteri generali}

\elenco{\item \emph{p. 868}}

Con Pirandello vediamo la fine delle certezze tipiche della realtà disegnata dal pensiero positivista.
Un ulteriore e definitivo passo avanti: la realtà oggettiva non esiste ed è anche impossibile descriverla ed evocarla attraverso la letteratura: non esistono uomini eccezionali, non esiste il superuomo, e non esiste alcun poeta o veggente in grado di cogliere quelle corrispondenze di cui parlano tanti poeti decadenti (e anche Pascoli).

Arte e letteratura saranno molto particolari: troveranno il loro compimento soprattutto con il teatro con Pirandello. Le prime rappresentazioni teatrali finirono con il pubblico che usciva inviperito, con un clamoroso insuccesso.

Non c'è più alcuna realtà oggettiva, né da vivere né da descrivere.

\section{Biografia}

Pirandello nasce nel 1867, in Sicilia come Verga, tanto che alcuni racconti giovanili sono stati paragonati alle novelle di Verga.
Nasce a Girgenti, da un padre proprietario di una miniera di zolfo (condizione piuttosto agiata).

Compie gli studi prima a Palermo, e poi a Roma. Studia lettere.
Per degli screzi finisce il suo percorso di studi a Bon, in Germania, dove si laurea con una tesi in filologia sul dialetto di Girgenti.

Inizia la sua produzione nel 1901: esce il primo romanzo, \textit{L'esclusa}. Era stato messo in quartiere alla fine dell'Ottocento

Pirandello è uno dei pochi che ha toccato tutti i generi letterari, dalla poesia, al romanzo, al dramma, al saggio. Si occupa anche di cinema: quando è morto si stava occupando della realizzazione cinematografica de \textit{Il fu Mattia Pascal}.

Nel 1903 c'è un avvenimento parecchio destabilizzante nella vita di Pirandello: la miniera del padre viene distrutta, e Pirandello subisce il declassamento: conosce la povertà. Nel frattempo si era sposato, e i soldi della dote della moglie erano stati investiti in queste miniere.
La moglie, che aveva una personalità particolare, ebbe un peggioramento incredibile nelle sue crisi nervose.

Nel 1904 esce \textit{Il fu Mattia Pascal}. Pirandello scrive il suo romanzo nel periodo in cui si acutizza la malattia nervosa della moglie: il tema della pazzia acquisisce un ruolo preponderante nella sua narrazione.
Sebbene non abbia approfondito l'opera di Freud, sicuramente aver assistito la moglie ha influito moltissimo sulle tematiche affrontate da Pirandello. Molti suoi personaggi finiscono pazzi e ricoverati in cliniche e ospizi: è quasi il tema base.

Nel 1915 scoppia la guerra, e inizialmente è schierato dalla parte degli interventisti: la guerra porterà solo dolore, perché il figlio maggiore sarà fatto prigioniero dagli austriaci, e questo provoca un peggioramento nelle condizioni della moglie.
La moglie sarà ricoverata dopo questo episodio, e non uscirà mai più dall'ospedale psichiatrico.

Nel 1910 Pirandello si era avvicinato al mondo del teatro.
Qualche anno dopo, nel 1914, dopo il delitto Matteotti, si iscrive al partito fascista.

Pirandello, in nome delle sue posizioni patriottiche, aveva visto con favore l'intervento in guerra, considerandolo come una sorta di fase del processo risorgimentale.

Il rapporto con Mussolini è molto ambiguo. Nel 1915 Pirandello diventa il direttore del teatro d'arte di Roma: lo stato investa 250 mila lire per far partire questo progetto, ma 50 mila lire furono sborsati direttamente da Mussolini; è probabile che l'adesione al fascismo di Pirandello sia molto opportunistica, perché Mussolini finanziava il suo progetto teatrale, a cui teneva molto.

Negli anni in cui inizia ad occuparsi di teatro in modo continuativo conosce Abba Marta, una attrice molto giovane. Fu la sua musa ispiratrice.
La loro relazione fu molto strana, platonica. Abba Marta lo chiamava "Il Maestro".

Pirandello era capo comico, oltre che autore. Le sue didascalie sono molto lunghe e dettagliate, durano anche più di due pagine. egli seguiva da vicino la messa in scena dei suoi testi teatrali.

Nel 1934 Pirandello riceve il premio Nobel, e si può notare che a differenza di tutti coloro che vincono il premio Nobel, egli si rifiutò di fare un discorso.
Un premio ritirato nel '34 avrebbe richiesto un discorso che conteneva un elogio al duce.

Nel 1936 Pirandello muore. L'ultima parte della sua vita fu dedicata alla produzione teatrale, e l'ultima opera, lasciata incompiuta, è il \textit{Gigante della montagna}.

\section{Psicoanalisi}

A differenza di Svevo, che si è interessato subito all'opera di Freud, Pirandello disse di non aver mai letto Freud, anche se conobbe Vinet, un intellettuale interessato alla psicoanalisi, ma che l'ha affrontata differentemente rispetto a Freud: egli ha estrapolato la teoria della \textbf{confederazione delle anime}.

Questa teoria affascinava molto Pirandello.

\section{Pirandello e la malattia mentale}

A differenza di Italo Svevo, che lesse alcune opere di Freud, Pirandello ha sempre dichiarato di non conoscere direttamente le teorie freudiane. Egli ebbe però, ancora più direttamente dello scrittore triestino, un contatto doloroso con il mondo della malattia mentale, a seguito della lunga e penosa vicenda dei disturbi psichici di cui soffrì la mogli. Precoce risulta il suo interesse per questo settore di studi, testimoniato in particolare dalla lettura del saggio \textit{Les altérations de la personnalité} (\textit{Le alterazioni della personalità}, 1892) dello psicologo francese \textbf{Alfred Binet}. Pirandello elogia espressamente "quella rassegna di meravigliosi esperimenti psico-fisiologici, dai quali, com'è noto, si argomenta che la presunta \textbf{unità del nostro io} non è altro in fondo che un \textbf{aggregamento temporaneo scindibile e modificabile di vari strati di coscienza} più o meno chiari" (dal saggio \textit{Arte e scienza}, 1908)

\subsection{L'io come "confederazione di anime"}

Ciò che suggestiona e poi inciderà profodnamente sul pensiero pirandelliano è la possibilità di penetrare nei territori bui della coscienza individuale, di svelare la natura dissociata della personalità umana. Quest'ultima viene rappresentata negli studi di Binet come una sorta di "confederazione di anime" dominate da un io egemone, che tiene sotto controllo una vita psichica caotica e brulicante di fantasmi, normalmente celata sotto la soglia della coscienza. Pirandello trova così conferma alle sue intuizioni sullo \textbf{sdoppiamento della personalità}, che ricorrono in una lettera del 1884 indirizzata alla futura consorte Antonietta:

\citazione{In me son quasi due persone. Tu già ne conosci una; l'altra neppure la conosco bene io stesso. Voglio dire che \textbf{consto di un \textit{gran me} e d'un \textit{piccolo me}}: questi due signori sono sempre \textbf{in guerra tra di loro}; l'uno è spesso allaltro sommamente antipatico. Il primo è taciturno e assorto continuamente in pensieri, il secondo parla facilmente, scherza e non è alieno dal ridere e dal far ridere. Quando questi ne dice qualcuna un po' scema, quegli va allo specchio e se lo bacia. Io sono \textbf{perpetuamente diviso tra queste due persone}: ora impera l'una, ora l'altra. Io tengo naturalmente molto più alla prima, voglio dire al mio gran me; mi adatto e compatisco la seconda, che è in fondo un essere come tutti gli altri, coi suoi pregi comuni e coi comuni difetti}

Si coglie già qui la propensione a temi quali la \textbf{scomposizione dell'io}, il \textbf{doppio} e la \textbf{maschera}, che avranno un ruolo fondamentale nella definizione dei personaggi pirandelliani sia nella narrativa sia nel teatro, e che troveranno nel saggio sull'\textit{Umorismo} del 1908 una loro chiara teorizzazione:

\citazione{Ciasuno si racconcia la maschera come può - la maschera esteriore. Perché dentro poi c'è l'altra, che spesso non s'accorda con quella di fuori. E niente è vero! Vero il mare, sì, vera la montagna; vero il sasso; vero un filo d'erba; ma l'uomo? Sempre mascherato, senza volerlo, senza saperlo.}

Le anime, come spiritelli, stanno dentro di noi, in quella sfera che è l'inconscio. L'individuo non sa di averle. Prevale un io egemone, e le altre stanno sotto al livello della coscienza, e possono uscire in qualsiasi momento.

\section{Pensiero}

La sua idea di fondo resta invariata per tutta la sua vita. Addirittura, nonostante i primi testi siano spesso paragonati a quelli di Verga, gli aspetti legati al pensiero sono molto diversi dal tipo di approccio di Verga, e che quindi anche in questi primi testi, apparentemente legati al verismo, si vede il pensiero di fondo di Pirandello.

Il pensiero di Pirandello si articola intorno ad alcuni punti fondamentali.

Il primo è la \textbf{contrapposizione tra vita e forma}: secondo Pirandello anche l'uomo, che fa parte della natura, è inserito in una realtà e in una vita in movimento. Non si può parlare di una realtà oggettiva, fissa e descrivibile (certezze del positivismo). La realtà che vede Pirandello è fluida, dove non vi sono certezze né forme cristallizzate che possono essere descritte. Anche l'uomo ne fa parte.

L'individuo cerca, apparentemente in modo contraddittorio, di darsi una forma. Ci destabilizza vedere che qualcosa dentro di noi non è in linea con la personalità che crediamo di avere, o addirittura con le tante personalità: una per ogni persona che ci vede.
Le personalità non sono solo quella che ci diamo noi, ma sono tante quante sono le persone con cui veniamo in contatto. Dipendono dal contesto.
Il fatto che noi tendenzialmente ci vogliamo dare una forma, e vogliamo corrisponere ad una personalità, e il fatto che la società, l'educazione e le regole ci impongano delle personalità, o delle maschere (la maschera teatrale, come quella del teatro greco, rigida, di terra cotta, è qualcosa di rigido che nasconde i nostri sentimenti) ci fissa in un tipo, e non ci permette di esprimere i nostri sentimenti personali. Ci costringono nella forma.

Questa è una contraddizione, perché l'uomo per natura è pulsione: la vita è flusso, e quindi cambiamento, istinto: necessariamente, quando entra in conflitto con la forma c'è dolore e fissità; non c'è più sentimento, non c'è più vita.

Pirandello è estremamente tragico nella sua visione della vita.

Molto spesso Pirandello ci illustra una società borghese, con le industrie, le macchine, gli impiegati, gli uffici, ed è la realtà che vive: diventa una sorta di paradigma, un microcosmo che è immagine di tutti gli individui.

In termini ordinari non ci sono soluzioni a questo stato di cose. Le soluzioni sono discutibili e temporanee, che sono il tirarsi fuori: l'esperienza personale di Pirandello probabilmente ha giocato un ruolo fondamentale.
Pirandello deve aver vissuto l'esperienza di volersi tirare fuori, e assumere una posizione estraniata.
Questo è possibile o in una situazione come quella di Mattia Pascal, o molto più spesso per mezzo della pazzia.

La \textbf{pazzia} è un sistema per vivere al di fuori della maschera e al di fuori della forma. Molti pazzi di Pirandello vivono in una sorta di simbiosi con la natura (come succede in \textit{Uno, nessuno e Centomila})

Ci sono dei momenti, come ne \textit{Il treno ha fischiato}, di perdita della lucidità, dei momenti di pazzia in cui il protagonista si rifugia per poter tollerare la forma di tutti i giorni.

In questo contesto dove la realtà non è oggettiva, la letteratura non è in grado di descriverla: ecco perché parliamo di \textbf{relativismo conoscitivo}, di pliralità prospettica.
Questo è esattamente ciò che fa Pirandello nei suoi primi drammi: le sue prime opere furono un fiasco incredibile: la gente che andava a teatro usciva arrabbiata, perché Pirandello gettava tutti in una confuzione incredibile; non c'era più il cattivo che era stato punti, il bravo che usciva vittorioso, e questo era destabilizzante; questa era l'unica realtà descrivibile, una realtà di caos.

Caos era fra l'altro il nome della borgata in cui è nato Pirandello: è un nome in qualche modo illuminante per quello che poi sarebbe stato il suo pensiero.

\section{Pirandello all'università}

Gabriele d'Annunzio frequenta poco le aule universitarie, ma negli aristocratici saloti romani in cui, ammirato dalle nobil-donne, recita il suo ruolo di istrione, magnifica le doti di un professore di cui dice di non perdere una lezione. Il suo nome è Onorato Occioni, titolare della cattedra di Letteratura latina dell'Università "La Sapienza", nonché rettore dell'ateneo. In effetti, tra gli studiosi di Filolofia latina, Occioni ha fama di oratore d'eccezione: un affabulatore capace di ammaliare, ma che in realtà - si dice - conosce poco la lingua di Cicerone e di Virgilio.

Qualche anno dopo, il professore ha tra i suoi allievi un altro futuro protagonista della letteratura italiana, Luigi Pirandello. Un giorno - siamo nel 1889 - nel tradurre in aula un brano di una commedia di Plauto, il \textit{Miles gloriosus}, Occioni commette un errore grossolano, e un giovane sacerdote che siede accanto a Pirandello ride e dà di gomito al compagno. Il latinista se ne accorge, e va su tutte le furie. Il sacerdote si scusa, ma Pirandello rincara la dose, mettendo alla berlina l'irascibile professore. Mal gliene incoglie: Occioni, forte della sua autorità, riunisce d'urgenza il Consiglio di facoltà, che suggerisce all'incauto studente di lasciare l'ateneo. Meglio, a questo punto, evitare ritorsioni: poche settimane dopo, Pirandello è a Bonn.


\chapter{T: \textit{Ciàula scopre la luna}}
\elenco{\item \emph{p. 894} - fino a riga 231}

Questo testo in qualche modo ha fatto pensare a molti critici che Pirandello stesse ripercorrendo la strada di Verga: in questa novella troviamo un protagonista poveraccio.
La parte iniziale della novella descrive il lavoro nella miniera, e ha subito fatto pensare a Rosso Malpelo.

Il protagonista è un ragazzo di trent'anni che lavora nella miniera. I due personaggi (il protagonista e Rosso Malpelo) un po' si assomigliano: sono dei vinti, dei personaggi maltrattati dalla società, che fanno una vita estremamente grama lavorando giorno e notte.
C'è però una differenza tra Ciaula e Rosso malpelo: Rosso Malpelo ad un certo punto esprime a sua filosofia, egli aveva un suo stile di vita, lucidissimo; Ciaula invece è un ragazzo ritardato, che non si rende conto: anche il narratore ci dice che non si rende conto: in questa sua forma di alienazione, egli riesce a godere di qualche attimo di felicià quando scopre la luna.

Questa novella è stata accomunata alle novelle di stampo Verghiano.
Questo in realtà è soltanto apparente. Sicuramente l'ambiente, l'uso dei soprannomi e delle forme dialettali ci riconducono ad un clima verista, ma ben presto vediamo che gli aspetti che interessano Pirandello sono differenti.

\elenco{\item \textbf{riga 213}: descrizione del volto di zi' Scarda:  il lettore riceve una impressione di dolore, che rimane fissato per tutta la durata della novella.
\item \textbf{riga 216}: novità rispetto a Verga: si sente la voce dell'autore; il narratore commenta, e cerca di interpretare l'espressione del personaggio. Ciò non significa che siamo di fronte ad un narratore alla Manzoni, onnisciente, che guida il lettore: il narratore qui non ci da alcuna certezza, ma anzi entra per insinuare dubbi e farci riflettere.
\item \textbf{riga 222}: In questa miniera, anni prima, era scoppiata una mina, e zi' Scarda aveva person un occhio e un figlio; egli, nonostante fosse vecchio, è tenuto in miniera come cortesia, in quanto deve gestire e mantenere l'intera famiglia.
\item \textbf{riga 224}: \textit{la grossa lagrima}: c'è l'immagine della lacrima, che diventa simbolo di questo dolore calato in questa realtà. È simbolo del dolore, universale, di tutti gli uomini: non è proprio dell'ambiente, a differenza con Verga: non sono più denunce, ma simboli
\item \textbf{riga 229}: \textit{zi' Scarda aveva sempre la bocca arsa}
\item \textbf{riga 232}: inizia una contrapposizione luce-buio che andrà avanti tutta la novella
\item \textbf{riga 243}: \textit{Calicchio}: si tratta di Calogero, il figlio morto
}

Ciaula, il protagonista, ha un nome che significa cornacchia. C'è una differenza sostanziale con Rosso Malpelo: è uno scarto della società ma Rosso ha una sua filosofia di vita. Questo protagonista invece è un essere alienato mentalmente dalla società.

Ciaula ha il compito di portare sulle spalle il carico, fino alla superficie. Ha il vantaggio di vedere la luce del sole.
Il protagonista è un alienato mentale: ci introduce in questa novella ai personaggi tipici di Pirandello, e al tema della \textbf{pazzia}.

Non c'è via di scampo da questa realtà, e le uniche soluzioni che Pirandello vede sono lo straniamento o la pazzia. Lo straniamento della pazzia viene messo in campo molte volte dai personaggi di Pirandello.

Questo personaggio mantiene tutta l'ingenuità tipica delle persone deboli di mente, con delle paure.

Egli non ha paura del buio della miniera (Pirandello lo descriverà con termini simili a quelli per l'utero materno), ma ha paura del buio vero (della notte).

L'immagine di Ciaula è un'immagine grottesca, espressionistica, e questo è un essere ingenuo.
La sua descrizione è quasi commovente. Lui sembra non capire l'ingiustizia e il dolore che lo contraddistinguono, gli scivolano via: è una posizione estraniata rispetto alla realtà, e questa è una forma di difesa.

\elenco{\item \textbf{righe 297-302}: \textit{Cosa strana}: Ciaula non aveva paura del buio della miniera: ci era abituato e ci stava \textit{come dentro il suo alvo materno} (\textbf{riga 302}). Nella sua ingenuità lui è più vicino agli elementi della natura (come il protagonista di \textit{Uno, nessuno e centomila} alla fine del romanzo).
\item \textbf{riga 20 (?)} si profila l'apparizione della luna; questa luna è diversa da quelle precedenti della letteratura: la luna va incontro a Ciaula. L'uscita di Ciaula dalla montagna può essere intesa come nascita o come rinascita. L'apparizione della luna è una sorta di \textbf{epifania} (rivelazione).
}

La luna simboleggia le fasi di morte e rinascita, che è un po' quello che è successo a Ciaula.

\chapter{Saggio: \textit{L'umorismo}}
\elenco{\item \emph{p. 878}}

Scritto intorno al 1908, anche se pubblicato nel 1920.

Si divide in due parti.

Nella \textbf{prima parte} Pirandello da delle indicazioni sul termine \textit{umorismo}, e l'umoristica è un'arte che in letteratura si trova in ogni epoca.

Nella \textbf{seconda parte} egli ci spiega cos'è l'umorismo.

La realtà non è più una realtà oggettiva. La realtà è molto differente da ciò che sta sotto, e che non appare.

L'arte non corrisponde più a canoni di bellezza, di perfezione formale. L'arte avrà il compito di oltrepassare l'apparenza per poter capire. L'arte deve insinuare il dubbio, e solo sul dubbio si può iniziare a ragionare.

Nel saggio c'è un paragone con una situazione che serve a far capire l'atteggiamento umoristico (che non è comico): fa l'esempio della donna anziana, truccata in modo esagerata e volgare; vedendo una donna così conciata si può avvertire un senso di comicità: si avverte che quello che noi vediamo è esattamente il contrario di quello che dovrebbe essere. Se però andiamo oltre all'apparenza, e scopriamo che lei si concia così per mantenere l'attenzione del marito molto più giovane di lei, e che ha paura di perdere, ecco che noi andiamo a scoprire la verità di quell'individuo, che è molto dolorosa.

Pirandello riporta un esempio letterario: Don Chisciotte: oltre la superficie è una persona alienata, che non è come vorrebbe essere, che soffre e finisce sempre perdente.

\section{T: \textit{Un'arte che scompone il reale}}
\elenco{\item \emph{p. 879}}

\subsection*{Righe 26-38}

\emph{Leggere}

\subsection*{Righe 65-94}

L'individuo vive di questa vita, ma ha necessità di una forma, ed è lui a costruirsi la forma entro cui soffoca: la forma come una maschera ci chiude e ci fissa; la forma sono le trappole (matrimonio, situazione economica, leggi, situazioni, vincoli sociali): l'uomo ha bisogno di rispondere ad una forma.
La prima che ci danno, quando nasciamo, è il nome. L'unico personaggio di Pirandello che riesce a liberarsi anche del nome è il protagonista di \textit{Uno, nessuno e Centomila}, che in questa situazione di pazzia delle ultime pagine del libro si trova addirittura senza nome.

Anche quando Mattia Pascal resta privo di questa forma, a poco a poco si ricostruisce questa forma: pensa ad un nuovo nome, pensa ad uno stato sociale: l'uomo ha necessità, e gli altri hanno necessità di creare su quell'individuo delle personalità, che sono le varie maschere che indossiamo nei vari ambiti in cui ci troviamo.

La letteratura non può fissare con i canoni classici di perfezione di bellezza: non può che essere paradossale.

\subsection*{Righe 95-108}

\emph{Leggere da soli}


\chapter{T: \textit{Il treno ha fischiato}}
\elenco{\item \emph{p. 901}}

Il termine \textit{masca} in piemontese significa "strega".

Il protagonista è un impiegato. Molti letterati hanno vissuto questa esperienza dell'impiegato (\emph{aggiungere approfondimento impiegato}).

I termini iniziali indica il fallimento della mania positivistica di voler dare una spiegazione scientifica a tutto: abbiamo una serie di termini scientifici, all'inizio della novella (in \textit{medias res}).
Nonostante questo nessuno riesce a darsi spiegazioni di quello che abbia fatto il protagonista, soprattutto una spiegazione scientifica.

Belluca è il protagonista della novella.

C'è un narratore che ci guida un po'.

Il protagonista è un individuo, che ha una maschera: quella di un uomo \textit{mansueto, metodico e paziente}; i colleghi e il capoufficio lo conoscono così

Ci sono degli aggettivi, quali "specialissime", "naturalissimo": naturalissimo è la routine in cui si muove questo personaggio, che è di un certo tipo. In questa routine il personaggio ha questa maschera. La sua reazione è la cosiddetta \textbf{coda del mostro}: si prende per assurdo un comportamento, ma solo perché si vede solo una parte, e non anche quello che sta dietro (la coda appunto).

\elenco{\item \textbf{riga 54}: La rivelazione e l'epifania c'è già stata, ma gli altri non lo sanno. Il sorriso è un segno esteriore di questa rivelazione.
\item \textbf{riga 56}: Anche la sua espressione non gli si confà: lui ha avuto la sua epifania, e tutti i suoi atteggiamenti fanno crollare le certezze degli altri: non ci sono nessi causali, non c'è una realtà oggettiva. Si dissolve la maschera di Belluca (che hanno costruito gli altri): c'è una pluralità prospettica.
}

Il fatto epifanico è quell'avvenimento che giunge improvviso: sono delle azioni e dei fatti che sono completamente insignificanti, a volte un po' strani. Per esempio, il treno ha fischiato (\textbf{riga 60}) per Belluca
Per gli altri, questo è sintomo di pazzia.

Il fatto epifanico è qualcosa che squarcia il grigiore della trappola in cui si trova Belluca: nel suo caso la trappola è costituita dalla famiglia, e da tutte quelle regole che ci sono in ufficio. L'evento squarcia questa realtà, squarcia la forma.

Pensando ad un individuo che ha avuto la vita limitata sempre da una forma, che ad un certo punto viene a mancare, si può immaginare che la vita sgorghi tutta improvvisamente: ci sono degli eccessi.
È difficile regolarsi dopo aver perduto la forma, così come è difficile rientrare nella forma, quasi doloroso

\elenco{\item \textbf{riga 67}: \textit{giù risate da pazzi}: la prima reazione, come ci dice ne \textit{L'umorismo}, è quella di ilarità; dopo, si avverte il sentimento del contrario, e a questo punto subentra dolore; tra poco il narratore ci spiegherà la trappola di Belluca, e noi capiremo come non ci sia nulla da ridere della sua situazione
\item \textbf{righe 73-74}: dopo essersi liberati della forma è difficile sottostargli di nuovo
\item \textbf{righe 88-90}: Questa cosa è assurda per tutto coloro che non conoscono la vera vita di Belluca. Il narratore, abbastanza attendibile, conosce la situazione di Belluca, è il meglio informato sui fatti, e lui non si stupisce.
}

L'assurdità della vita funziona come un contesto in cui quello che si inserisce ha un significato.
La \textbf{coda del mostro} è mostruosa se la separiamo dal resto del mostro, ma se la riattacchiamo al corpo questa coda non è mostruosa, ma naturale.
La vita assurda che conduce Belluca è il contesto in cui quello che è capitato non appare più come assurdo.

\elenco{\item \textbf{riga 91}: il narratore non ride, c'è il sentimento del contrario, si va oltre al comico.
\item \textbf{righe 109-122}: è già scattato il sentimento del contrario, con la descrizione della situazione. LE tre cieche vanno a contrapporsi alla luce epifanica di Belluca
\item \textbf{righe 151-178}: flusso di coscienza: è l'immagine lampante di quel sentimento e di quella vita che sta balzando fuori dalla forma, un flusso che non si può controllare e che si esprime verbalmente attraverso il flusso di coscienza (non tiene conto delle regole della sintassi tradizionale).
\item \textbf{riga 183}: si fa fatica a rientrare nella forma, \textit{ha ecceduto}
}

Il finale esprime una sorta di ottimismo.

Tutte le novelle di Pirandello sono raccolte nel testo \textit{Novelle per un anno}

\chapter{Altri romanzi}

\section{\textit{L'esclusa}}
\elenco{\item \emph{p. 908}
}

È il primo romanzo di Pirandello, pubblicato nel 1901.

È una vicenda che si svolge in Sicilia; a prima vista sembra che l'ambientazione sia un po' quella delle novelle Verghiane. In realtà non è così.

Questo romanzo, per ambientazione, personaggi e alcune tematiche fa pensare alle ambientazioni veriste: ci troviamo in Sicilia, in un contesto modesto; di fatto la vicenda va in tutt'altra direzione: non abbiamo l'esigenza dell'autore di mostrarci una realtà in cui i deboli vengono sopraffatti, ma all'autore interessa l'\textbf{assurdità della vita}

La vicenda è paradossale: Marta è una giovane donna, sposata, maestra: viene accusata di adulterio dal marito, adulterio che lei non ha commesso. Nel momento in cui il marito l'accusa di adulterio, per tutti è adulterio, anche per i familiari (tranne la madre e la sorella).

L'\emph{ambientazione} ha una sua importanza, ma non è questo che interessa a Pirandello: è molto più importante \textbf{ciò che appare} rispetto a ciò che è

Ci sono diverse peripezie: perde il figlio che aveva in grembo, è costretta a cambiare città, e riesce a rifarsi una vita.
Incontra l'uomo che si pensava fosse il suo amante, e lo diventa veramente.
Ad un certo punto la suocera è in fin di vita, e lei corre al suo capezzale. La suocera è emblematica, perché anche lei era stata accusata di adulterio.
Al capezzale della suocera, il marito di Marta le chiede di tornare insieme, proprio quando lei è veramente colpevole.

Quando era innocente per la società era colpevole, mentre quando è davvero colpevole viene perdonata.

\section{\textit{Il turno}}
\elenco{\item \emph{p. 909}}

Il gioco del caso è ancora ripreso nel breve romanzo successivo, \textit{Il turno} (1895), dove un innamorato deve aspettare il suo "turno" per sposare la donna amata, dopo la morte di altri due mariti. Il tema è però impostato a livello di minore responsabilità intellettuale, come divertimento comico, dai risvolti bizzarri, grotteschi, quasi marionettistici.

\section{\textit{I vecchi e i giovani}}
\elenco{\item \emph{p. 910}}

È un romanzo sociale di ambientazione siciliana. È la Sicilia dei sanguinosi moti dei Fasci siciliani del 1893, sconvolta dalle lotte di classe, con i clericali da un lato, tesi ad impedire il consolidamento del nuovo regime liberale, e la classe dirigente dall'altro, che disperde nel disordine morale i sacrifici e i meriti acquisiti.

Più che casi individuali, i personaggi del romanzo interpretano i diversi aspetti della complessa situazione storica che stanno vivendo.

Così il principe don Ippolito di Colimbetra, fedele suddito borbonico; don Flaminio Salvo, esponente della nuova borghesia capitalista; Roberto Auriti, glorioso garibaldino che si spegne in un'esistenza amorfa; il giovane principe Gerlando di Colimbetra, sostenitore delle nuove idee e per questo costretto all'esilio.

I personaggi rappresentano un contrasto di concezioni e di ideali che si risolve nel contrasto tra due generazioni: quella che ha fatto l'Unità e che vede perduta l'eredità del Risorgimento, e quella più giovane, che nel gretto conservatorismo dei padri scorge solo la difesa di interessi reazionari.

Ne I vecchi e i giovani l'autore esprime un giudizio storico molto severo sul processo di riunificazione dell'Italia e dello stato nato da essa. Non a caso Carlo Salinari, analizzando questo romanzo, parla di tre “fallimenti collettivi” riferendosi al Risorgimento, come mancato moto generale di rinnovamento dell'Italia; all'unità, come fallito strumento di liberazione e di sviluppo delle zone più arretrate e in particolare della Sicilia e dell'Italia meridionale; e al socialismo, che avrebbe potuto essere la ripresa del movimento risorgimentale. Questi fallimenti si sovrappongono poi a quelli “individuali” «dei vecchi che non hanno saputo passare dagli ideali alla realtà e si trovano a essere responsabili degli scandali, della corruzione e del malgoverno dei giovani».

Nell'ultimo capitolo della seconda parte del romanzo, don Cosmo, il fratello intellettuale di Ippolito, fornisce la chiave di lettura degli avvenimenti e il punto di vista di Pirandello, nel corso della conversazione con Gerlando:

\citazione{«Una cosa è triste, cari miei: aver capito il gioco! Dico il gioco di questo demoniaccio beffardo che ciascuno di noi ha dentro e che si spassa a rappresentarci di fuori, come realtà, ciò che poco dopo egli stesso ci scopre come una nostra illusione, deridendoci degli affanni che per essa ci siamo dati, e deridendoci anche, come avviene a me, del non averci saputo illudere, poiché fuori di queste illusioni non c'è più altra realtà...»}

Questo romanzo riprende i temi giovanili, ed è un romanzo storico.

Mostra due generazioni a confronto, e riprende le idee del periodo giovanile di Pirandello, con forti ideali patriottici.

In questo romanzo mette a confronto due generazioni: i vecchi sono quelli che hanno fatto il risorgimento italiano, e i giovani sono quelli che vanno a deludere le aspettative del risorgimento; rappresentano la disillusione che segue a questo processo di unità


\chapter{\textit{Il fu Mattia Pascal}}
\elenco{\item \emph{p. 909-910}
\item \emph{p. 914-917}}

La società presenta delle trappole: Mattia Pascal ha delle trappole ben definite: oltre alle convenzioni, ha una trappola costituita dal contesto familiare (moglie e suocera): la famiglia, tanto agognata da Pascoli, è vissuta da Pirandello come una trappola, come qualcosa che imprigiona.
Così, insieme alla famiglia, c'è anche il contesto economico di Mattia Pascal (i suoi debiti).

Mattia Pascal ad un certo punto si libera da queste trappole, privato di forma: lui scappa di casa, e va a Montecarlo: qui vive una avventura incredibile: va a giocare al casinò e vince tantissimi soldi, da star bene tutta la vita.
Lui decide di tornare a casa, e mentre torna a casa legge la notizia della sua morte suicida (riconosciuto da Moglie e Suocera).

Lui appena letta la notizia si arrabbia: ad un certo punto è illuminato da una intuizione: lui in quel momento ha solo la vita e non la forma; Mattia Pascal per la legge non esiste più: lui è al di fuori, ha una posizione estraniata rispetto alla sua vecchia vita. Non ritornerà a casa.

Pirandello ci dice che l'uomo ha bisogno della Forma, e quindi questo è proprio quello che succederà a Mattia Pascal.

Il romanzo è pubblicato nel 1904: questo è un periodo molto particolare, che segue il disastro economico che si cala sulla famiglia, e che fa peggiorare la situazione mentale della moglie.
Il libro è scritto proprio mentre Pirandello è al capezzale della moglie.

L'opera risente moltissimo di questo contesto, e tutta l'opera di Pirandello è influenzata dalla tematica della \textbf{pazzia} (Vedasi \textit{Il treno ha fischiato})

Il tema della pazzia in questo romanzo non è rappresentato direttamente, ma è rappresentata come qualcosa che aliena e che estranea, e che forse anche preserva.
Mattia Pascal è un \emph{forestiere della vita}

È la storia paradossale di un piccolo borghese, imprigionato come sempre nella trappola di una famiglia insopportabile e di una misera condizione sociale, che, per un caso fortuito, si trova improvvisamente libero e padrone di sé: diviene economicamente autosufficiente grazie ad una cospicua vincita al casino di Montecarlo e apprende di essere ufficialmente morto, in quanto la moglie e la suocera lo hanno riconosciuto nel cadavere di un annegato. Però, invece di approfittare della liberazione dalla forma sociale per vivere immerso nel fluire della vita, senza più assumere maschere, Mattia Pascal si sforza di costruirsi un'identità nuova. In lui resta insuperabile l'attaccamento alla vita sociale, alla trappola, quindi egli soffre perché la sua identita falsa lo costringe all'esclusione dalla vita degli altri. Decide pertanto di rientrare nella sua vecchia identità, tornando in famiglia, ma scopre che la moglie si è risposata ed ha avuto una figlia da un altro. Non gli resta dunque che adattarsi alla sua condizione sospesa di "forestiere della vita", che contempla gli altri dall'esterno, consapevole di non essere più "nessuno".

Quando la forma si rompe, ne sgorga la vita. Mattia Pascal decide di godersi la vita: inizia a viaggiare.
Mattia Pascal incorre in un errore: non è ancora filosofo, come il protagonista di \textit{Uno, Nessuno e Centomila}; liberatosi della forma, ad un certo punto ne sente la necessità. Si crea una \textbf{forma fittizia}.
Inizia con una serie di modifiche che partono dall'aspetto esteriore: si taglia la barba, etc etc.
Poi si costruisce una nuova identità: si chiamerà Adriano Meis e si costruisce un passato fittizio.

Inizia una nuova vita, e vuole una casa: non vuole più stare in una camera d'albergo, senza oggetti suoi.

Si rende conto che anche la vita senza una forma non può essere vita vera. Sente quindi la \textbf{necessità di questa forma}.
Si reca a Roma, va da un affittacamere, che è un po' un filosofo, ed è fissato dalle sedute spiritiche.
C'è la figlia dell'affittacamere, l'ombra della sorella di lei, e il marito vedovo di quest'ultima.

Lui si innamora della figlia, \emph{Adriana}, ma col passare del tempo si rende conto che non può vivere senza la forma. Un passaggio significativo è il seguente: è per strada, vede la sua ombra, e tutti possono calpestare la sua \textbf{ombra}: si rende conto che lui è un'ombra, e tutti lo possono calpestare.

Inscena quindi un \emph{suicidio} e decide di ritornare alla sua vecchia vita. Pensa di poter ritornare nella sua vecchia forma.

Anche quella che era considerata una trappola non c'è più: la moglie si è risposata, ha avuto altri figli, e gli è solo consentito di ritornare nella dimensione del lavoro.


Il finale di questo testo ci mostra quello che il personaggio definisce il \textbf{forestiere della vita}: Mattia Pascal è in una posizione straniata, vede la vita dal di fuori. È lo stesso tipo di straniamento di Belluca, ma è forzato.
Lui ha fatto l'errore di voler tornare nella forma, e viene soffocato prima dalla forma fittizia e poi dall'impossibilità di tornare nella forma originale.

\section{T: \textit{La costruzione della nuova identità e la sua crisi}}
\elenco{\item \emph{p. 917}}

\subsection*{Righe 10-62}

Sia all'inizio del romanzo, quando Mattia Pascal si trova pieno di soldi e morto per lo stato.

Egli cade subito nell'errore di volersi ricostruire una nuova forma: dapprima riguarda l'aspetto fisico, ma poi riguarda il nome e l'identità.

\subsection*{Righe 71-74}

\subsection*{Righe 75-99}

\subsection*{Righe 136-144}

\subsection*{Righe 202-215}

\section{T: \textit{Lo strappo nel cielo di carta}}
\elenco{\item \emph{p. 926}
\item lette le righe segnate + il resto da soli}

Mattia ha appena fatto l’operazione all’occhio, ed è cieco. Questo è significativo.

Il signor Paleari gli parla di un teatrino che sta per essere messo in atto, l’Oreste di Sofocle.

Oreste è figlio di Agamennone, ed è piccolo quando avviene l’omicidio del padre. Va via dalla reggia, e ritorna a 18 anni per vendicare la morte del padre, con l’aiuto di Elettra, sua sorella. Uccidono Egisto e Clitemnestra, assassini di Agamennone

Il signor Paleari spiega la differenza tra Oreste e Amleto. Oreste è un eroe tragico, senza ripensamento, mentre Amleto è un eroe moderno, con molti ripensamenti.

Il signor Paleari descrive come sarà il teatrino delle marionette e dice "se ci fosse uno strappo in questo cielo di carta le marionette non saprebbero più cosa fare". Il teatrino è un mondo fittizio,  ma le marionette non lo sanno, per le marionette è vero.
Questa immagine serve al Paleari per spiegare la sua teoria della \textbf{lanterninosofia}; la lanterna costituisce i grandi ideali come religione, patria, legge... se questi ideali vanno in crisi, quindi se lanterna si spegne, l'individuo non ha più punti di riferimento e non sa più cosa fare.
Secondo Pirandello il primo momento in cui ogni ideale va in crisi nella storia dell'uomo è la scoperta copernicana, gli uomini non hanno più punti di riferimento certi e quindi vanno in crisi.

\elenco{\item \textbf{righe 1-19}:  L’identità è falsa, ma è il parametro che ci consente di sapere cosa dobbiamo fare e cosa dobbiamo essere. Se si crea lo strappo nel cielo di carta (che è un po’ come il treno del treno ha fischiato) si rompe la forma, e l’individuo non sa più come fare.
Nel mondo moderno alcune teorie hanno messo in crisi l’uomo moderno, distruggendo le sue certezze
\item \textbf{righe 44-58}: Il lanternino è un ideale in miniatura, quello della persona, individuali.
Pirandello ci parla di un’ombra nera che fa paura, che potrebbe essere la morte o lo stato che c’è prima di nascere.
Se si spegne il lanternino, noi abbiamo paura di questo nero, ma non dovrebbe essere così, perché noi il nero lo vediamo solo perché abbiamo acceso il lanternino.
\item \textbf{righe 82-83}}

\chapter{\textit{I quaderni di serafino Gubbio operatore}}
\elenco{\item \emph{p. 938}}

Questo romanzo è molto particolare.

Da un lato abbiamo l’operatore, Serafino, che è una sorta di cameraman, che filma e che è dietro una macchina e osserva. Vive al di fuori, un forestiere del mondo.
È un osservatore.

Dall’altro lui decide di scrivere questa storia dopo che è successo un qualcosa, che riguarda altri personaggi: una donna (femme fatale), un uomo innamorato di lei, e una fanciulla che è il tratto di unione tra le vicende degli attori e di Serafino (Serafino è innamorato di lei).

Dopo le vicende Serafino sarà muto.

Prima riusciva ad estraniarsi, a vivere da forestiere della vita. Poi si innamora della fanciulla, e questo fa “saltare” i suoi piani: è trascinato anche lui nella vita.

Il fatto che ha fatto diventare muto Serafino: l’uomo spara per davvero (sul palco) alla femme fatale, e la tigre presente sul set mangia l’uomo.
Questa vicenda è molto forzata.

\section{T: \textit{Viva la macchina che ... la vita}}

\elenco{\item \emph{p. 940} - fino a riga 55}

Pirandello considera il cinema in maniera molto critica: questo brano ci consente di ripensare alla conclusione della coscienza di Zeno, dove Svevo immagina uno scenario di distruzione, in cui la macchina è l’unica evoluzione dell’uomo.

La modernità ha i suoi rumori.

\section{T: \textit{L'automobile e la carrozzella: la modernità e il passato}}

\elenco{\item \emph{p. 945}: fare da soli}


\chapter{\textit{Uno, Nessuno e Centomila}}
\elenco{\item \emph{p. 913}
\item \emph{p. 948}}

Ultimo romanzo di Pirandello

È il romanzo maturo di Pirandello, dopodiché egli continuerà a scrivere prevalentemente drammi teatrali.

Sembra che Vitangelo Moscarda abbia compiuto un passo in avanti rispetto a Mattia Pascal, verso quella scelta che possa consentire una soluzione tra quel contrasto tra vita e forma, che non sembra poterne avere una.
Sembra che egli sia riuscito a rompere il vincolo ultimo (il primo a sancire l’essere formale dell’individuo), cioè il nome e il cognome.
Per quanto Mattia Pascal si è ridotto a forestiere della vita, di fatto lui era lì, vicino.
Fisicamente, invece, Vitangelo Moscarda, si separa dal consorzio degli uomini, e vive in questa sorta di ospizio.

Per un caso Vitangelo Moscarda scopre di vivere in tante maschere. È figlio di un banchiere (usuraio) che l’ha lasciato molto ricco, e non ha alcuna preoccupazione al mondo.
Si guarda allo specchio, e scopre dalla moglie di avere il naso storto: capisce che la moglie lo vede diversamente da come si vede lui.
Inizia una sorta di indagine, e si rende conto che lo vedono in tanti modi diversi.

Il titolo del romanzo è emblematico. \textbf{Centomila} è il numero di persone che lo vedono, e quindi di maschere che ha.

Lui si rende conto che nel suo paese la forma più presente è quella dell’usuraio, che lui odia.
Decide di distruggere tutte queste maschere, attraverso atti, atteggiamenti e azioni che possano distruggere quelle convenzioni tipiche della società.

Per distruggere l’immagine dell’usuraio va da un poveraccio, e caccia un poveretto che viveva in una catapecchia. Tutta la società è inorridita, e alla fine lui ricompensa il poveretto con un sacco di soldi, stupendo tutti quando, distruggendo così la sua forma di usuraio.

Entra in conflitto con tutti coloro che gli hanno costruito addosso una maschera.

Ad un certo punto tutti credono che lui sia l’amante di un’amica di sua moglie, e tutti lo considerano un adultero. Lui non può proprio sopportare questa maschera.
Alla fine lo troviamo in un ospizio, che aveva finanziato lui stesso, dove vive seguendo unicamente il flusso della vita.

Lo ritroviamo nel momento in cui è presente ad un processo, il processo che viene fatto alla donna che gli ha sparato, e lui viene chiamato a testimoniare.
Quando lo chiamano lui non si riconosce al suo nome, perché nella vita che sta vivendo lui ha rotto completamente tutti i legami con la società.

La storia è scritta a posteriori da lui stesso.

Lui rappresenta lo stadio ultimo, e forse più perfetto, rispetto a Mattia Pascal, perché lui è riuscito a distruggere completamente una forma: Pascal ci era andato vicino a distruggerla, ma poi ne ha creata una nuova.
Vitangelo Moscarda invece va fino in fondo, ed è felice: forse è la felicità dei folli che non si rendono conto, ma è felice; vive tra gli alberi, nella natura.
Non sembra voler tornare indietro.

I personaggi si muovono senza una conseguenza logica, come nel suo teatro

\section{T: \textit{“Nessun nome”}}
\elenco{\item \emph{p. 949}}

È l’ultima pagina del romanzo

\elenco{\item \textbf{riga 10}: Lui stesso si definisce “povero svanito”
\item \textbf{riga 19}: \textit{Io sono vivo e non concludo}: il finale mostra come il personaggio sia filosofo; è riuscito a raggiungere il suo obiettivo; è l’unico personaggio che dice che lui è \textbf{vivo}
\item \textbf{riga 38}: \textit{La città è lontana}: lui è lontano dal consorzio umano, anche fisicamente}

L’abito a pagina 950 è molto significativo: l’abito è la nostra forma, è il segno della maschera. 
Nei \textit{Sei personaggi in cerca d’autore} i personaggi arrivano dalla platea, non da dietro alle quinte, e nelle didascalie (che per Pirandello sono lunghissime) è descritto il fatto che debbano indossare dei vestiti \textbf{rigidi}, e si deve percepire dal di fuori: i vestiti devono rappresentare la trappola della forma in cui i personaggi (vivi dentro) si muovono.


\chapter{Teatro}
\elenco{\item \emph{p. 959}}

Pirandello aveva iniziato a comporre qualcosa nei primissimi anni del ‘900, però dal 1915 egli inizia ad occuparsi con continuità al teatro, producendo anche altro, e nell’ultima fase della sua vita si dedica unicamente a quello.

Egli era presente durante l’allestimento del teatro.

Si è iniziato a occupare anche di cinema, e nell’anno della sua morte stava lavorando alla produzione cinematografica de \textit{Il fu Mattia Pascal}.

L’ultima opera, \textit{I giganti della montagna} non è mai stata completata.
Gli ultimi testi sono un po’ particolari.

Il teatro di Pirandello non offriva certezze, ma metteva in scena l’assurdità della realtà.

Ci sono tre fasi del teatro Pirandelliano:
\begin{enumerate}
\item \textbf{grottesco}
\item \textbf{meta teatro}
\item \textbf{ultima fase} (che noi tralasciamo)
\end{enumerate}

\section{Grottesco}

La fase del \textbf{grottesco} caratterizza l’opera di Pirandello, e spesso lasciava il pubblico arrabbiato e insoddisfatto. 
In quell’epoca andava di moda il \textbf{dramma borghese}, in cui i protagonisti rappresentano la borghesia, anche nei temi: matrimonio, situazione finanziaria, triangolo amoroso, dramma familiare. I canoni seguiti erano quelli del naturalismo, e si raggiunse un livello di verosimiglianza, anche negli arredi, incredibili.

I personaggi di Pirandello si comportano in modo \textbf{non coerente}. Ad esempio, ne \textit{Il gioco delle parti} (che appartiene alla fase del grottesco) abbiamo un triangolo amoroso (marito, moglie e amante), tipico di tanto dramma borghese.
La moglie e l’amante vivono insieme, e il marito sembra aver assorbito bene il colpo, essendo molto gentile con la moglie e l’amante.
Il finale lascia spiazzati: la donna viene offesa e viene diffamata, cosa molto grave all’epoca. L’unico modo era di sfidare a duello il giovane che l’aveva offeso (compito che spetta al marito).
Il giorno dell’incontro, il marito chiama l’amante e gli dice che deve andare lui a duellare: l’amante finirà ucciso.

Nei drammi di Pirandello, il serio e il tragico con il comico si mescolavano. E quello che Pirandello ci spiega nel saggio sull’umorismo.

Si chiama \textbf{teatro del grottesco} perché questi personaggi sembrano delle caricature, sono esagerati, iperbolici, e non hanno alcuna coerenza.
Quando Pirandello scopre la pluralità prospettica nella realtà, quando scopre che non sempre c’è un effetto determinato da un’azione, ci dice che neanche l’arte può esprimere un mondo oggettivo.
Ecco che quindi mette in scena questi avvenimenti e questi fatti paradossali, dove sembra e poi invece è diversamente.

\textit{Così è se vi pare} è la storia di una famiglia costituita da un marito, una moglie ed una madre.

Pirandello utilizza ancora una forma teatrale che può ricollegarsi al dramma borghese. 
In molte opere ci sono ancora le stesse vicende tipiche del dramma borghese di fine ottocento: Pirandello però è come se le svuotasse di quei contenuti e ci mettesse davanti agli occhi l’assurdità di queste relazioni e di questi personaggi; i personaggi quindi si muovono in maniera irrazionale: egli voleva che i personaggi recitassero in maniera non naturale.

% @import "piero.png" {width="300px"}
Questa fotografia è estremamente significativa: i personaggi stanno formando un triangolo (famoso triangolo amoroso). Nessuno dei tre guarda nel volto l’altro: i personaggi sono sconnessi.

Questo è un filo rosso dei drammi di Pirandello: i personaggi non riescono a comunicare tra di loro; lo si vedrà perfettamente nella trilogia del \textbf{metateatro}, e in particolare nei \textit{Sei personaggi in cerca di autore}.

\subsection{\textit{Il gioco delle parti}}

Nel \textit{Gioco delle parti}, abbiamo questo personaggio filosofo: il marito cornuto fa il superiore, è messo da parte, osserva il rapporto tra la moglie e l’amante.
Sembra un filosofo, ma la conclusione del dramma è di tutt’altro stampo: il vero filosofo è il protagonista di \textit{Uno, Nessuno e Centomila}, che è riuscito ad uscire dalla forma, mentre il protagonista de \textit{Il gioco delle parti} non è riuscito a separarsene completamente (andando a duello).

\subsection{\textit{Pensaci Giacomino}}

Altro dramma significativo è \textit{Pensaci Giacomino}. 
Giacomino è un uomo anziano, impiegato statale da una vita, pagato talmente poco da non essere riuscito a farsi una famiglia.
Sembra un forestiere dalla vita, ma in realtà vuole vendicarsi dello stato.
Si sposa, ormai vecchissimo, con una donna giovanissima, così che lo stato debba pagare per un sacco di anni alla moglie la sua pensione dopo che lui morirà

\subsection{\textit{Così è se vi pare}}

\textit{Così è se vi pare} è la storia di un gruppo famigliare composto da tre persone: marito, moglie e suocera. Costoro sono terremotati, e hanno perso tutto, anche i documenti. Non hanno una identità ufficiale.
Arrivano in una cittadina, e qui il loro comportamento inizia a destare la curiosità di tutto: l’uomo tiene la moglie e la suocera distanti tra di loro: la figlia non può andare a trovare la madre.
Per comunicare le due sono costrette ad usare un secchio per scambiarsi bigliettini e oggetti.
La moglie non parla mai.
Ad un certo punto in una casa del paese giungono una volta la suocera e una volta il marito, e ognuno offre una spiegazione. 

L'uoom afferma che si tratta in realtà della seconda moglie, essendo la prima, la figlia della suocera, morta in un terremoto; l'anziana donna è pazza, sostiene sempre il genero, ed è convinta che si tratti ancora di sua figlia.
A sua volta la suocera afferma che è pazzo il genero, e che la donna relegata in casa è davvero la figlia, che si finge una seconda moglie per assecondare il marito.

Si crea un’attesa incredibile per sapere qual è la verità. Nell’ultima scena compare la moglie, che si presenta davanti alla folla curiosissima, e dice che lei è “chi volete che io sia”.
Questo è l’emblema della pluralità prospettica.

\section{Metateatro}
\elenco{\item \emph{p. 985}}

L’opera più importante è il \textit{Sei personaggi in cerca di autore}, ma ci sono anche \textit{Ciascuno a suo modo}, \textit{Questa sera si recita a soggetto} (\emph{leggere solo le trame})

C’è una quarta opera, che spesso viene considerato \textbf{metateatro}, ovvero \textit{Enrico IV}.
Quest’ultimo è un dramma, la cui trama è estremamente affascinante, che ci riporta all’eroe estraniato tipico di Pirandello.

\subsection{\textit{Enrico IV}}

In tutto il dramma non si dice il nome del protagonista di questa opera.

% ![Foto 28 apr 2021, 131216](/assets/Foto%2028%20apr%202021,%20131216.jpg)
% ![Foto 28 apr 2021, 131222](/assets/Foto%2028%20apr%202021,%20131222.jpg)

“Enrico IV” è un eroe antico, un personaggio che ha trovato quello spazio al di fuori della forma che aveva già trovato Vitangelo Moscarda, ma è un personaggio tragico, perché ha sentito l’esigenza di ricollegarsi alla vita, e sente la mancanza e il bisogno di quelle pulsioni tipiche della sua vita.
È un personaggio, come Mattia Pascal, che non è riuscito a vivere completamente la vita al di fuori della forma.

L’opera è difficile da seguire a teatro, e risulta difficile capire dove finisce la finzione.

\subsection{\textit{Sei personaggi in cerca di autore}}

Quest’opera si distingue completamente dal tema del dramma borghese (che Pirandello vuole scansare), in quanto si rompono tutti gli schemi.
I sei personaggi sono dei personaggi che avrebbero una loro storia, e infatti durante la rappresentazione questa storia viene un po’ fuori, e questa storia è un tipico dramma borghese.

% ![Foto 28 apr 2021, 131945](/assets/Foto%2028%20apr%202021,%20131945.jpg)
L’autore ha rifiutato questo dramma, e non lo vuole scrivere. Quindi questi sei personaggi, che avrebbero la storia, sono in cerca di un autore che la scriva. L’autore \textbf{rifiuta il dramma borghese}.

A teatro non è presente la quarta parete, e si vedono gli attori che fanno una prova de \textit{Il gioco delle parti}.

% ![Foto 28 apr 2021, 132340](/assets/Foto%2028%20apr%202021,%20132340.jpg)

Pirandello nel metateatro mette in scena l’\textbf{incomunicabilità}, a tutti i livelli: i sei personaggi rappresentano dei pezzi della loro storia, per fare capire agli attori la storia, e puntualmente gli attori non riescono a reinterpretarla a dovere.
Egli è fermamente convinto che ogni rappresentazione scenica rappresenta un tradimento nei confronti dell’autore, perché quello che l’autore ha in testa viene tradito da caratteristiche intrinseche degli autori, non modificabili.

La storia dei sei personaggi emerge a pezzettini, con continui interruzioni, mischiata alla storia della rappresentazione.

Questo escamotage dei personaggi che cercano un autore è già stato usato in \textit{Sostiene Pereira}.

La rottura della quarta parete è uno degli aspetti fondamentali del metateatro. Infatti i sei personaggi arrivano dalla platea. Ogni finzione è rotta.

\subsubsection{T: \textit{La rappresentazione teatrale tradisce il protagonista}}
\elenco{\item \emph{p. 991}}

Commento sul testo:
\elenco{\item \textbf{righe 130-136}: durante la rappresentazione della storia da parte degli attori i personaggi non si riconoscono; ci sono più punti all’interno del testo in cui vengono messe in luce queste dinamiche di contrasto: sono contrasti a tutti i livelli; sono momenti in cui Pirandello va a mettere in luce la \textbf{comunicazione che viene meno}; l’incomunicabilità è uno degli aspetti fondamentali del testo
\item \textbf{righe 164-171}: Pirandello ritiene la messa in scena di un testo una sorta di tradimento nei confronti dell’autore, e qui viene fuori; qui l’autore non c’è, ma ci sono i suoi personaggi, tanto che il capocomico ad un certo punto sbotta, dicendo che è sempre stato faticoso mettere in scena lo spettacolo di fronte all’autore; Pirandello era proprio uno di quegli autori.}

Questo testo è fondamentale perché da una parte riprende tutte le tematiche tipiche di Pirandello (pluralità prospettica, incomuncabilità), ma soprattutto perché mette in luce tutte le caratteristiche proprie del teatro: è teatro che parla di teatro. Sottolinea i rapporti tra l’opera è il suo autore.

Leggendo tra le righe si può notare anche il rapporto tra il personaggio e la persona: dopo aver letto questo testo ci si chiede quale tra i due (la persona, che è l’attore, e il personaggio che interpreta) è più \textit{vero}.
Nell’ottica Pirandelliana è più vero il personaggio, perché ha la sua forma ben precisa, immutabile.

Gli spettacoli di Pirandello gettavano tantissime domande sul pubblico.

\end{document}